% !TeX spellcheck = it_IT
\newpage
\section{Grafi}
Per la definizione di grafo si rimanda agli appunti di Fondamenti dell'Informatica. Ricordiamo solo le caratteristiche principali:
\begin{itemize}
	\item \textbf{V} rappresenta l'insieme di vertici
	\item \textbf{E} rappresenta l'insieme di archi
	\item Un grafo si dice \textbf{denso} quando $\lvert E \rvert \approx \lvert V \rvert^2$
	\item Un grafo si dice \textbf{sparso} quando $\lvert E \rvert \approx \lvert V \rvert$
	\item Un grafo può essere \textbf{orientato} o \textbf{non orientato}
	\item Il grafo è \textbf{pesato} quando viene associato un \emph{peso} agli archi o ai nodi
\end{itemize}
\subsection{Rappresentazione}
Dato un grafo con $n$ vertici tale che $V = \{1, 2, \ldots, n\}$ abbiamo due possibili modi di rappresentarlo.
\subsubsection{Matrice di adiacenza}
Una \textbf{matrice di adiacenza} rappresenta il grafo come una \textbf{matrice} $A$ di dimensione $n \times n$.  In questa matrice se un arco $(i, j)$ esiste allora l'elemento $A[i, j]$ sarà valorizzato a $1$, altrimenti sarà $0$. Lo \emph{spazio occupato} da questa matrice è $\lvert V \rvert ^2$ il che la rende efficiente solo per grafi piccoli.
\begin{note}
	Si noti che se il grafo è \textbf{non orientato} serve solo metà matrice, in quanto $(i, j) = (j, i)$.
\end{note}
\subsubsection{Lista di adiacenza}
Una \textbf{lista di adiacenza} rappresenta il grafo come un array di liste la cui dimensione dipende dal numero di archi presenti. Occupa $n+m$ spazio.

\subsection{Ricerca in un grafo}
Per cercare un elemento in un grafo dobbiamo esplorare ogni vertice e arco a partire da una sorgente $s$. Come risultato otteniamo un \textbf{albero} basato sul grafo iniziale che ha la sorgente come radice (o una foresta se il grafo non è connesso).

\begin{definition}[Percorso minimo]
	Un percorso minimo in un grafo $G$ tra due vertici $s, v$ è un percorso da $s$ a $v$  che contiene il minimo numero di archi. La sua lunghezza è detta \textbf{distanza minima} e viene indicata come $\delta(s, v)$. \\
	Una proprietà importante dato un arco tra due nodi $(u,v)$ e un nodo $v$ è la seguente:
	\begin{equation*}
		\delta(s, v) \leq \delta(s, u) + 1
	\end{equation*}
\end{definition}
\subsubsection{Breadth-First Search}
Questo algoritmo esplora il grafo un vertice alla volta espandendo la frontiera dei vertici esplorati in \textbf{ampiezza}. Per evitare di attraversare più volte nodi già visitati associamo alcuni "colori" ad ogni nodo:
\begin{itemize}
	\item \textbf{Bianco}: vertici non ancora visitati
	\item \textbf{Grigio}: vertici visitati ma non ancora esplorati
	\item  \textbf{Nero}: vertici completamente esplorati
\end{itemize}
Il concetto quindi è di esplorare i vertici scorrendo le \emph{liste di adiacenza} dei vertici grigi. Ogni vertice da esplorare viene aggiunto in una \textbf{queue} (coda) e appena vengono terminati e diventano neri vengono rimossi. Di seguito una possibile implementazione dell'algoritmo:
\begin{lstlisting}[language=C, caption=Implementazione di BFS, mathescape=true]
BFS(G, s) {
	inizializza vertici; // Vertici inizializzati ad $\infty$
	Q = {s}; // Q coda inizializzata con s e 0
	while (Q non vuota) {
		u = RemoveTop(Q);
		for each v $\in$ u->adj {
			if (v->d == infinito)
			v->d = u->d + 1;
			v->p = u;
			Enqueue(Q, v);
		}
	}
}
\end{lstlisting}
La \textbf{complessità} in tempo di questo algoritmo è $O(V + E)$ mentre quella in spazio è $O(V)$. \\
L'algoritmo in questione genera un \textbf{albero breadth-first}, dove i cammini verso la radice rappresentano i cammini minimi nel grafo $G$.
\begin{definition}[Albero breadth-first]
	Dato un grafo $G = \langle V, E \rangle$ con sorgente $s$, un albero breadth-first è un albero $G' = \langle V', E' \rangle, V' \subseteq V, E' \subseteq E$  tale che:
	\begin{itemize}
		\item $G'$ è un \textbf{sottografo} di $G$
		\item Un vertice $v$ appartiene all'albero se e solo se quel vertice è \textbf{raggiungibile} dalla sorgente nel grafo $G$
		\item Per ogni vertice dell'albero il percorso dalla sorgente è \textbf{minimo}
	\end{itemize}
\end{definition}
\subsubsection{Depth-First Search}