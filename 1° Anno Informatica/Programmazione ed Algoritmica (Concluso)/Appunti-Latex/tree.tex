% !TeX spellcheck = it_IT
\newpage
\section{Alberi}
%TODO Integrare robe ovvie sugli alberi
\subsection{Rappresentazione}
\subsubsection{Array}
Per rappresentare un albero \emph{binario} di profondità $d$ possiamo utilizzare un \textbf{array} di dimensione $2^{d+1}-1$. Questa scelta può portare a dei vantaggi e svantaggi:
\begin{itemize}
	\item \textbf{Vantaggi}:
	\begin{itemize}
		\item Accesso diretto ai nodi
	\end{itemize}
	\item \textbf{Svantaggi}:
	\begin{itemize}
		\item L'\emph{altezza} dell'albero deve essere nota
		\item Spreco di memoria
		\item \emph{Inserimento} e \emph{cancellazione} sono operazioni complicate
	\end{itemize}
\end{itemize}
Per questi motivi gli array si usano raramente per la rappresentazione di alberi.
\subsubsection{Liste}
Il modo più usato per la rappresentazione di alberi è quello delle \textbf{liste}, codificandoli come segue:
\begin{lstlisting}[language=C, caption=Alberi con liste, mathescape=true]
	struct node {
		int data;
		struct node *left;
		struct node *right;
	}
\end{lstlisting}
Questa scelta ci porta a vantaggi e svantaggi:
\begin{itemize}
	\item \textbf{Vantaggi}:
	\begin{itemize}
		\item L'\emph{altezza} dell'albero non deve essere nota
		\item Nessuno spreco di memoria
		\item \emph{Inserimento} e \emph{cancellazione} sono operazioni facili
	\end{itemize}
	\item \textbf{Svantaggi}:
	\begin{itemize}
		\item Mancanza di accesso diretto ai nodi
		\item Memoria aggiuntiva per memorizzare figlio destro e sinistro
	\end{itemize}
\end{itemize}

\subsection{Visitare}
Possiamo effettuare l'operazione di \textbf{visita} su un albero binario in tre modi diversi. Tutti questi algoritmi avranno \textbf{complessità} $O(n)$.
\subsubsection{Anticipata}
\begin{lstlisting}[language=C, caption=Alberi con liste, mathescape=true]
	Anticipata(x):
		if x != NULL
			print(x.key)
			Anticipata(x.left)
			Anticipata(x.right)
\end{lstlisting}
\begin{example}
	%TODO Da finire
\end{example}
\subsubsection{Posticipata}
\begin{lstlisting}[language=C, caption=Alberi con liste, mathescape=true]
	Posticipata(x):
		if x != NULL
			Posticipata(x.left)
			Posticipata(x.right)
			print(x.key)
\end{lstlisting}
\begin{example}
	%TODO Da finire
\end{example}
\subsubsection{Simmetrica}
\begin{lstlisting}[language=C, caption=Alberi con liste, mathescape=true]
	Simmetrica(x):
		if x != NULL
			Simmetrica(x.left)
			print(x.key)
			Simmetrica(x.right)
\end{lstlisting}
\begin{example}
	%TODO Da finire
\end{example}
\subsection{Albero binario di ricerca}
Un caso particolare di albero binario è un albero binario di ricerca. Questo rispecchia le seguenti proprietà, dato un nodo $x$, applicate ricorsivamente:
\begin{itemize}
	\item $x.left.key \leq x.key$, ovvero tutti i nodi alla sinistra sono i minori di $x$
	\item $x.right.key \geq x.key$, ovvero tutti i nodi alla destra sono i maggiori di $x$
\end{itemize}
\begin{note}
	Per effettuare la stampa \textbf{ordinata} degli elementi dobbiamo utilizzare la visita \textbf{simmetrica}.
\end{note}
\subsubsection{Ricerca}
Algoritmo ricorsivo per la ricerca di un elemento:
\begin{lstlisting}[language=C, caption=Ricerca ricorsiva, mathescape=true]
	RicercaABR_R(x, k)
		if x == NULL OR k == x.key
			return x
		if k < x.key
			return RicercaABR_R(x.left, k)
		else return RicercaABR_R(x.right, k)
\end{lstlisting}
Algoritmi per la ricerca del valore minimo e massimo di un albero:
\begin{lstlisting}[language=C, caption=Ricerca del minimo, mathescape=true]
	RicercaMIN_I(x)
		while x.left != NULL
			x = x.left
		return x
\end{lstlisting}
\begin{lstlisting}[language=C, caption=Ricerca del massimo, mathescape=true]
	RicercaMAX_I(x)
		while x.right != NULL
			x = x.right
		return x
\end{lstlisting}