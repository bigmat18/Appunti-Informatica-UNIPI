\newpage
\section{Relazioni}
\begin{definition}[Relazione]
	\label{def:relazione}
	Prendiamo in considerazione un prodotto cartesiano con due insieme $A, B$ che sia $A \times B = \{(a,b) \: | \: a \in A, b \in B\} = $ U (Universo). 
	\\
	Una relazione è $R \subseteq U$ dove, come scritto sopra, $U = A \times B$.
	\\
	In una relazione A è detto \textbf{insieme di partenza} e B è detto \textbf{insieme di arrivo}.
\end{definition}

%TODO Magari si può rimuovere questo hspace
\hspace{-15pt}
In sintesi possiamo definire una relazione come un sottoinsieme del prodotto cartesiano fra due insiemi.
\\
L'insieme $rel(A,B)$ è l'inseme di tutte le possibili relazioni fra $A$ e $B$.
\begin{example}
    Esempio relazione.\\
    Prendiamo due insiemi e facciamo il prodotto cartesiano.
    \begin{itemize}
    	\item Insieme degli \textbf{studenti} $S = \{luca, mario, angela, gino, maria\}$
    	\item Insieme dei \textbf{corsi} $C = \{PA, LAB, FI, AN\}$
    \end{itemize}
    Il prodotto cartesiano fra $S$ e $C$ è uguale a tutte le possibile coppie ordinate che si possono formare fra i due insieme, quindi:
    \begin{equation}
    	S \times C = \{(luca, PA), (luca, LAB), ..., (marica, AN)\}
    \end{equation}
    In questo caso possiamo creare una relazione del prodotto cartesiano andando appunto a prendere un sottoinsieme di $A \times B$ e stabilendo una regola o condizione per scegliere quali coppie ordinate vogliamo, ad esempio $R \subseteq A \times B$ è una relazione che specifica quali esami sono stati sostenuti dai vari studenti.
\end{example}

\subsection{Identità}
Quando l'insieme di partenza e quello di arrivo coincidono, si ottengono alcuni casi particolari di \hyperref[def:relazione]{relazioni}. Un esempio è il seguente:
\begin{example}
	Sia $\mathbb{U}$ l'insieme di tutti gli esseri umani, e consideriamo le seguenti relazioni di parentela:
	\begin{itemize}
		\item $Madre = {(x, y) \in \mathbb{U} \times \mathbb{U} \implies x \text{ madre di } y}$
		\item $Padre = {(x, y) \in \mathbb{U} \times \mathbb{U} \implies x \text{ padre di } y}$
		\item $Figlia = {(x, y) \in \mathbb{U} \times \mathbb{U} \implies x \text{ figlio di } y}$
		\item $Figlio = {(x, y) \in \mathbb{U} \times \mathbb{U} \implies x \text{ figlia di } y}$
	\end{itemize}
\end{example}
\begin{example}[Identità]
Preso un insieme A, l'identità su A è una relazione con se stessa, e si scrive $Id_A \subseteq$ A X A. Si può quindi vedere come nell'identità di un insieme ogni elementi è identico a se stesso. Inoltre l'identità di un insieme si definisce come:
\end{example}
\vspace{-10pt}
\begin{equation}
    Id_A = \{(a,a) \mid a \in A\}
\end{equation}
\vspace{-20pt}
\subsection{Composizione}
\begin{definition}[Composizione]
Siano $R: A \rightarrow B$ e $S: B \rightarrow C$. La \textbf{composizione} di R con S è la relazione $R;S: A \rightarrow C$ così definita:
\end{definition}
%TODO Aggiungere esempio fatto a lezione con sorella e zia (vedi slide)
%TODO Aggiungere immagini
\begin{equation}
    R;S = \{(x, z) \in A \times C \mid \exists b \in B . (x, y) \in R \land (y,z) \in S\}\footnote{Il simbolo "." indica \emph{tale che}. Ad esempio $\exists x.P$ indica che esiste un x per cui vale la proprietà P.}
\end{equation}

\begin{note}
	L'insieme di arrivo di R deve essere uguale all'insieme di partenza di S per effettuare l'operazione di composizione.
\end{note}

%TODO Fare l'esempio basandosi su padre/madre ma con nonni (vedi dispensa)
\begin{example}
\end{example}

\subsection{Relazione opposta}
\begin{definition}[Relazione opposta]
Sia $R:A\rightarrow B$ una relazione. La \textbf{relazione opposta} di $R$ è la relazione $R^{op}:B\rightarrow A$ definita come: 
\begin{equation}
	R^{op} = \{(y, x) \in B \times A \mid (x, y) \in R\}
\end{equation}
\end{definition}
%TODO OSS: non posso fare Rop;Sop in quanto l'insieme di partenza di Sop non è l'insieme di arrivo di Rop. posso però fare Sop;Rop. Inserisci grafico preso da slide

\begin{example}
    A = $\{$pagine web$\}$ \hspace{.5cm} B = $\{$parole del vocabolario$\}$ \hspace{.5cm} $R \subseteq$ A X B\\
    Si associa ciascuna pagina web con le parole in essa contenuta:
    \begin{itemize}
        \item $(x, y) \in \mathbb{R}$ dice che nella pagina web "x" è contenuta la parola "y".
        \item $R^{op}$ ($(y, x) \in \mathbb{R}$) dice per ogni parola "y" quali sono le pagine web che le contengono.
    \end{itemize}
\end{example}

\subsection{Leggi}
Come per gli insiemi, anche per le relazioni esistono delle leggi che regolano il comportamento delle varie operazioni. \\ \\
Per tutti gli insiemi A, B, C, D e per tutte le relazioni $R \subseteq A \times B$, $S \subseteq B \times C$, $T \subseteq C \times D$, valgono le leggi scritte nelle tabelle \ref{tab:leggi-composizione}, \ref{tab:leggi-relazioni-opposte}, \ref{tab:leggi-distributività}, una volta preso $A \times B$ come universo.
\begin{table}[h!]
    \setlength{\tabcolsep}{8pt}
    \renewcommand{\arraystretch}{2}
    \centering
    \begin{tabular}{|c|c|}
    \hline
        \textbf{Associatività} & (R;S);T = R;(S;T) \\
        \textbf{Unità} & $Id_A$;R = R;$Id_B$ = R \\
        \textbf{Assorbimento} & $\O$;S = S;$\O$ = $\O$ \\ \hline
    \end{tabular}
    \caption{Leggi composizione}
    \label{tab:leggi-composizione}
\end{table}
\begin{table}[h!]
    \vspace{-10pt}
    \centering
    \setlength{\tabcolsep}{8pt}
    \renewcommand{\arraystretch}{2}
    \begin{tabular}{|c|c|}
        \hline
        \textbf{Convoluzione} & $(R^{op})^{op} = R$ \\
        \textbf{Opposto-identità} & $(Id)^{op} = Id$ \\
        \textbf{Opposto-complemento} & $($A X B$)^{op}$ = (B X A) \\
        \textbf{Opposto-vuoto} & $(\O)^{op} = \O$ \\ \hline
    \end{tabular}
    \caption{Leggi relazioni opposte}
    \label{tab:leggi-relazioni-opposte}
\end{table}
\begin{table}[h!]
    \vspace{-10pt}
    \centering
    \setlength{\tabcolsep}{8pt}
    \renewcommand{\arraystretch}{2}
    \begin{tabular}{|c|c|c|}
        \hline
        \textbf{Distributività composizione} & R;(S $\cup$ T) = (R;S) $\cup$ (R;T) & (S $\cup$ T); R = (S;R) $\cup$ (T;R) \\
        \textbf{Distributività opposto} & $(R \cup S)^{op}$ = $S^{op} \cup R^{op}$ & $(R \cap S)^{op}$ = $S^{op} \cap R^{op}$ \\
        \textbf{Distributività opposto su negazione} & $(\overline{R})^{op} = (\overline{R^{op}})$ & \\
        \textbf{Distributività opposto su composizione} & $($R;S$)^{op}$ = $S^{op}$;$R^{op}$ & \\ \hline
    \end{tabular}
    \caption{Leggi distributività}
    \label{tab:leggi-distributività}
\end{table}
\\
%TODO Sistemare la spiegazione delle leggi citando la definizione e usando maybe le subsections
\textbf{Spiegazione Associatività:}\\
R;S $\subseteq$ A X C, facendo la composizione con T la prima parte dell'uguaglianza (R;S);T $\subseteq$ A X D. A sua volta, analizzando la seconda parte dell'uguaglianza, S;T $\subseteq$ B X D che poi se andiamo a comporre con R risulta che R;(S;T) $\subseteq$ A X D. Possiamo così vedere che l'uguaglianza è verificata. Per la dimostrazione discorsiva completa vedere in seguito. \\ \\
\textbf{Spiegazione Unità:}\\
Essendo che $Id_A = A \times A$ e $Id_B = B \times B$ vediamo che la prima parte dell'uguaglianza $Id_A;R = A \times B$ e la seconda è $R;Id_B = A \times B$, quindi la prima uguaglianza è verificata. La seconda uguaglianza si verifica in automatico visto che $A \times B = R$.

%TODO OSS: Dimostrare che data R:A->B allora R;(BxC)=AxB è falso (vedi slide)

\subsubsection{Dimostrazione proprietà associativa}
\begin{demostration}
    Facciamo una dimostrazione discorsiva della proprietà associativa in tabella \ref{tab:leggi-composizione}.\\
    Proprietà associativa: (R;S);T = R;(S;T) \hspace{.3cm} Con: $R \subseteq$ A X B, \: \: $S \subseteq$ B X C, \: \: $T \subseteq$ C X D.\\ \\
    \textit{Innanzitutto ricordiamo la proprietà per cui dati 2 insiemi X, Y $X = Y \Longleftrightarrow X \subseteq Y \land Y \subseteq X$.} \\
    Utilizziamo la proprietà sopra scritto andando a sostituire alla X "(R;S);T" e alla Y "R;(S;T)", troviamo così due condizioni che, per l'operatore logico $\land$, devono essere vere entrambe per far valere l'uguaglianza:
    \begin{itemize}
        \item (R;S);T $\subseteq$ R;(S;T) - \underline{Dimostrazione 1°}.
        \item R;(S;T) $\subseteq$ (R;S);T - \underline{Dimostrazione 2°}.
    \end{itemize}
        \underline{Dimostrazione 1°}: (R;S);T $\subseteq$ R;(S;T) \\ 
        \textit{Come prima cosa ricordiamo che un insieme $W \subseteq Z \: \: \forall \: w \in W \land z \in Z$.} \\
        Sostituendo "(R;S);T" a W e "R;(S;T)" a Z troviamo che, per fare in modo che la condizione che un insieme sia sottoinsieme di un altro $\forall \: (a,d) \in$ (R;S);T $ \land \: (a,d) \in$ R;(S;T). \footnote{Ricordati che (R;S);T ha al suo interno coppie (a,d) $\subseteq$ A X D per le operazioni di composizione}\\ \\
        Noi dobbiamo dimostrare che $\forall \: (a,d) \in$ R;(S;T) sia vera:\\
        Prendiamo come prima cosa una generica coppia di valori (a,d) $\in$ (R;S);T. Perché esista questa coppia deve esistere per forza un valore "c" che faccia da ponte fra "(R;S)" e "T" (ricordiamo che R;S $\subseteq$ A X C e T $\subseteq$ C X D). Possiamo scrivere quindi (Colore \textbf{nero} nella rappresentazione [\ref{fig:rappresentazioni-dim-associtiva}]):
        \begin{equation}
            (a,d) \in (R;S);T \Longrightarrow \: \exists \: c \in C \: \bullet (a,c) \in R;S \land (c,d) \in T \text{ (Nel disegno freccia NERA)}
            \label{eq:dim1-associtiva-1}
        \end{equation}
        Ora nella forma scritta sopra [\ref{eq:dim1-associtiva-1}] abbiamo una parte (a,c) $\in$ R;S che deve essere "scomposta" in maniera specifica per verificare che tutta la composizione di partenza sia vera, infatti deve esistere un valore b che colleghi l'insieme R ed S nell'operazione di composizione R;S (ricordiamolo che R $\subseteq$ A X B e R $\subseteq$ B X S). Possiamo scrivere quindi (Colore \textbf{blu} nella rappresentazione [\ref{fig:rappresentazioni-dim-associtiva}]):
        \begin{equation}
            (a,d) \in (R;S);T \Longrightarrow \: \exists \: b \in B, \: c \in C \: \bullet (a,b) \in R \land (b,c) \in S \land (c,d) \in T \text{ (Nel disegno freccia BLU)}
            \label{eq:dim1-associtiva-2}
        \end{equation}
        Ora per arrivare alla forma che dobbiamo dimostrare, R;(S;T), l'ultima forma [\ref{eq:dim1-associtiva-2}] è troppo estesa, infatti la parte $(b,c) \in S \land (c,d) \in T$ deve essere racchiusa per arrivare alla forma S;T. Quindi andiamo a scrivere che (Colore \textbf{rosso} nella rappresentazione [\ref{fig:rappresentazioni-dim-associtiva}]):
        \begin{equation}
            (a,d) \in R;(S;T) \Longrightarrow \: \exists \: b \in B \bullet (a,b) \in R \land (b,d) \in S;T \text{ (Nel disegno freccia ROSSA)}
            \label{eq:dim1-associtiva-3}
        \end{equation}
        L'ultima forma raggiunta [\ref{eq:dim1-associtiva-3}] dimostra che $\forall \: (a,d) \in$ R;(S;T) è vera e quindi che (R;S);T $\subseteq$ R;(S;T) è verificata. $\blacksquare$
        \begin{figure}[h!]
            \centering
            \includegraphics[width=10cm]{dimostrazione-associtiva.png}
            \vspace{-10pt}
            \caption{Rappresentazione dimostrazione 1°}
            \label{fig:rappresentazioni-dim-associtiva}
        \end{figure}
        \\\\
        \underline{Dimostrazione 2°}: R;(S;T) $\subseteq$ (R;S);T\\
        La seconda dimostrazione ha uno svolgimento analoga alla prima. Pure qui andiamo a considerare che: per fare in modo che la prima parte, cioè "R;(S;T)", sia sottoinsieme della seconda, la seconda deve contenere tutti gli elementi della prima. quindi $\forall (a,d) \in R;(S;T) \: \land \:(a,d) \in (R;S);T$. \\ \\Per fare in modo che ciò scritto sia vero dobbiamo dimostrare che $\forall \: (a,d) \in (R;S);T$ sia vero:\\
        Come prima cosa dobbiamo capire in che casi i punti appartengono a "(R;S);T", questo avviene quando esiste un punto "c" che collega "R;S" e "T" (R;S è uguale a A X C e T è C X D), quindi possiamo scrivere che:
        \begin{equation}
            (a,d) \in (R;S);T \Longrightarrow \forall c \in C \: \bullet \: (a,c) \in R;S \land (c,d) \in T
            \label{eq:dim2-associtiva-1}
        \end{equation}
        Da questo punto procediamo come nella dimostrazione 1° quindi andiamo a scomporre ulteriormente la forma [\ref{eq:dim2-associtiva-1}] in $(a,c) \in R;S$ per verificare i casi in cui la composizione esista:
        \begin{equation}
            (a,d) \in (R;S);T \Longrightarrow \: \exists \: b \in B, \: c \in C \: \bullet (a,b) \in R \land (b,c) \in S \land (c,d) \in T
            \label{eq:dim2-associtiva-2}
        \end{equation}
        L'ultimo passaggio è trasformare la forma [\ref{eq:dim2-associtiva-2}] in una versione che possa validare $\forall \: (a,d) \in (R;S);T$, ed essa sarebbe:
        \begin{equation}
            (a,d) \in R;(S;T) \Longrightarrow \: \exists \: c \in C \bullet (a,c) \in R;S \land (c,d) \in T
            \label{eq:dim2-associtiva-3}
        \end{equation}
        L'ultima forma travata [\ref{eq:dim2-associtiva-3}] verifica che $\forall \: (a,d) \in (R;S);T$ sia vero, di conseguenza pure R;(S;T) $\subseteq$ (R;S);T è verificato. $\blacksquare$\\ \\
        Dato che siamo riusciti a dimostrare entrambe le dimostrazioni la proprietà associativa ((R;S);T = R;(S;T)) con cui siamo partiti è verificata. $\blacksquare$
\end{demostration}


\subsection{Proprietà fondamentali}
Prendendo in considerazioni due insiemi $A$ e $B$ ed una relazione R, dove R $\subseteq$ A X B, valgono le seguenti proprietà.\\ \\
%TODO Inserire definizione ed esempi con relazioni già visti (vedi slide)
\textbf{Totale}: $\forall a \in A . (\exists b \in B . (a, b) \in R)$
\hfill
\textbf{Surgettiva}: $\forall \: \: b \in B . (\exists a \in A . (a, b) \in R)$
%TODO La totale e la surgettiva è simmetica
%TODO Acronimo TUSI
\begin{figure}[h!]
    \vspace{-8pt}
    \begin{subfigure}{.3\textwidth}
        \centering
        \includegraphics[width=6cm]{totale.png}
        \caption{Ogni elem. di A è collegato ad almeno uno di B}
    \end{subfigure}
    \hspace{4.3cm}
    \begin{subfigure}{.3\textwidth}
        \centering
        \includegraphics[width=6cm]{surgettiva.png}
        \caption{Ogni elem. di B ha almeno un entrata da A}
    \end{subfigure}
\end{figure}
\\
\textbf{Univalente}: \hspace{6.3cm} \textbf{Iniettiva}:
%TODO Sistemare layout
%TODO Integrare la definizione logica di "al più"
%TODO Aggiungere esempi da relazioni già viste
\\
$\forall a \in A . \exists$ al più un $b \in B . (a, b) \in R$
\hspace{2.4cm} 
$\forall b \in B . \exists$ al più un $a \in A . (a, b) \in R$
\begin{figure}[h!]
    \vspace{-7pt}
    \begin{subfigure}{.3\textwidth}
        \centering
        \includegraphics[width=6.2cm]{univalente.png}
        \caption{Ogni elem. di A deve avere al massimo 1 collegamento con B}
    \end{subfigure}
    \hspace{4.3cm}
    \begin{subfigure}{.3\textwidth}
        \centering
        \includegraphics[width=6cm]{iniettiva.png}
        \caption{Ogni elem. di B deve avere al massimo un entrante da A}
    \end{subfigure}
\end{figure}
\\
Le proprietà fra di loro sono legate da un rapporto di \textbf{dualità}:
%TODO inserire link alla proprietà (e.g. totale)
\begin{itemize}
    \item R $\subseteq A \times B$ è totale $\Longleftrightarrow$ $R^{op} \subseteq B \times A$ è surgettiva.
    \item R $\subseteq A \times B$ è surgettiva $\Longleftrightarrow$ $R^{op} \subseteq B \times A$ è totale.
    \item R $\subseteq A \times B$ è univalente $\Longleftrightarrow$ $R^{op} \subseteq B \times A$ è iniettiva.
    \item R $\subseteq A \times B$ è iniettiva $\Longleftrightarrow$ $R^{op} \subseteq B \times A$ è univalente.
\end{itemize}

\subsubsection{Teorema di caratterizzazione}
\label{teorema-caratterizzazione}
%TODO Sistemare le label
Prendendo due insiemi $A$ e $B$ ed una relazione $R$ tale che $R \subseteq A \times B$. Possiamo vedere come ogni proprietà fondamentale vale solo se sono soddisfatte determinate condizioni:
%TODO Aggiungere esempio grafico (slide)
\begin{itemize}
    \item \textbf{Totale} $\Longleftrightarrow Id_A \subseteq R;R^{op}$\\
    La congiunzione fra $R$ che è $A \times B$ e il suo opposto che è $B \times A$ torna un insieme $A \times A$, per questo l'identità di $A$ ($Id_A$), che sarebbe un insieme $A \times A$, è un sottoinsieme di $R;R^{op}$.\\
    Se $R$ non fosse totale vorrebbe dire che alcuni elementi di $R(A)$ non sono collegati.
    
    \item \textbf{Univalente} $\Longleftrightarrow R^{op};R \subseteq Id_B$\\
    La congiunzione fra $R^{op}$ che sarebbe $B \times A$ e $R$, che è $A \times B$, forma un insieme $B \times B$ che è quindi sottoinsieme di $Id_B$, che sarebbe $B \times B$.
    
    \item \textbf{Surgettiva} $\Longleftrightarrow Id_B \subseteq R^{op};R$\\
    La congiunzione fra $R^{op}$ ed $R$, che sono rispettivamente $B \times A$ e $A \times B$, torna una relazione $B \times B$, quindi $Id_B$, che è $B \times B$, è sottoinsieme.
    
    \item \textbf{Iniettiva} $\Longleftrightarrow R;R^{op} \subseteq Id_A$\\
    La congiunzione fra $R$ ed $R^{op}$ torna $A \times A$, essendo che $R$ è $A \times B$ e l'opposto è $B \times A$, che è sottoinsieme di $Id_A$ che è $A \times A$.
\end{itemize}

%TODO Inserire le 2 dimostrazioni del teorema di caratterizzazione (slide) e specificare una nota per "se e solo se" che indica che va dimostrato sia che se A allora B ma anche che se NON A allora NON B

\subsubsection{Proprietà di chiusura per composizione}
Per tutti gli insiemi $A, B, C$ e per tutte le relazioni, $R \subseteq A \times B$ e $S \subseteq B \times C$ vale che:
\begin{enumerate}
    \item Se $R$ ed $S$ sono totali allora la loro composizione $R;S$ è totale.
    \item Se $R$ ed $S$ sono univalenti allora la loro composizione $R;S$ è univalente.
    \item Se $R$ ed $S$ sono surgettive allora la loro composizione $R;S$ è surgettiva.
    \item Se $R$ ed $S$ sono iniettive allora la loro composizione $R;S$ è iniettiva.
\end{enumerate}

\subsection{Funzione}
Una funzione può essere definita utilizzando le proprietà fondamentali delle relazioni.
%TODO Esempi grafici
\begin{definition}[Funzione]
Dati due insiemi $A, B$ ed una relazione $R \subseteq A \times B$, tale relazioni si definisce funzione quando rispetta la proprietà \textbf{totale} e \textbf{univalente} quindi tutti gli elementi dell'insieme $A$ hanno uno ed un solo corrispettivo in $B$.
\end{definition}
\begin{example}
	Ad esempio ne lcaso dei booleani, dato l'insieme $B=\{$true, false$\}$:
	\begin{itemize}
		\item $\neg : B \rightarrow B$ $B = {(t, f), (f, t)}$
		\item $\wedge : B \times B \rightarrow B$
	\end{itemize}
\end{example}

\begin{definition}[Funzione parziale]
    Una funzione si dice parziale se rispetta solamente la proprietà univalente.
\end{definition}
\begin{example}
	Un esempio è la funzione $f:\frac{1}{x}$ con $f: \mathbf{R} \rightarrow \mathbf{R}$
\end{example}

\begin{definition}[Funzioni iniettive e surgettive]
	
\end{definition}

\begin{definition}[Biezione]
Dati due insiemi $A, B$ ed una relazione $R \subseteq A \times B$, tale relazioni si definisce funzione biettiva quando rispetta tutte e 4 le proprietà. Quindi:
\begin{itemize}
	\item $ \forall a \in A$ esiste \textbf{esattamente un} $b \in B . (a, b) \in \mathbf{R}$
	\item $ \forall b \in B$ esiste \textbf{esattamente un} $a \in A . (a, b) \in \mathbf{R}$
\end{itemize}
\end{definition}

\textbf{Domanda:} Se dati due insiemi A, B dove $|A| \neq |B|$ la loro relazione R $\subseteq$ A X B può essere una biezione?\\
La risposta è NO visto che avendo cardinalità diverse esisterà sempre un elemento in B che o non ha corrispettivo o ne ha 2.

\subsubsection{Composizione di funzioni}
\begin{proposition}
	Per tutti gli insiemi $A, B, C$ e tutte le funzioni $f:A\rightarrow B$ e $g:B\rightarrow C$, la relazione $f;g$ è una funzione, $f;g:A \rightarrow C$.
\end{proposition}
\begin{example}
    Dati due funzioni $f$ e $g$ dove:\\
	$f: A \rightarrow B$ \hspace{.5cm} $g: B \rightarrow C$ \footnote{Scrivere una funzione "$f$" nella forma $f: A \rightarrow B$  equivale a scrivere f $\subseteq A \times B$}\\ 
    La compozione si scrive come $f;g$ e sarebbe $f;g \subseteq A \times C$ \footnote{É possibile scrivere la composizione di funzioni anche come $g \cdot f$ oppure $g(f())$}
\end{example}
\begin{note}
	Indichiamo con $fun(A,B) = \{f \mid f: A \rightarrow B\}$ l'insieme di tutte le funzioni che vanno da $A$ a $B$. Di conseguenza $fun(A, B) \subseteq rel(A, B)$.
\end{note}

\subsubsection{Proprietà di chiusura per funzioni}
Per tutti gli insiemi $A, B, C$ e per tutte le relazioni, funzioni, $i: A \rightarrow B$ e $j: B \rightarrow C$ valgono le seguenti proprietà:
\begin{enumerate}
    \item $Id_A$ è una biezione, essendo una relazione con se stesso
    \item Se prendiamo $i: A \rightarrow B$ e $j: B \rightarrow C$, dove entrambe le funzioni sono biezioni, la loro composizione, $i;j: A \rightarrow C$ è a sua volta una biezione
    \item Se prendiamo la funzione $i: A \rightarrow B$ biettiva, il suo opposto $i^{op}: B \rightarrow A$ è a sua volta biettiva
\end{enumerate}

\subsubsection{Caratterizzazione in biezione}\label{caratterizzazione-biezione}
Per tutti gli insiemi $A, B$, per tutte le relazioni $R: A\leftrightarrow B$ vale che:
\begin{itemize}
	\item $R$ è una biiezione se e solo se $Id_A = R;R^{op}$ e $Id_B = R^{op};R$
	\item Una relazione $S:B\leftrightarrow A$ è l'inversa di $R$ se $Id_A = R;S$ e $Id_B = S;R$
	\item $R:A\leftrightarrow B$
\end{itemize}

\textbf{Spiegazione:} Per definire la proprietà di caratterizzazione per una relazione biettiva bisogna partire da cos'è una relazione biettiva: una relazione è biezione quando è contemporaneamente totale, univalente, surgettiva ed iniettiva. \\Da qui capiamo che se una relazione ha tutte e 4 le proprietà a suo volta dovrà rispettare per ciascuna di esse la caratterizzazione associata vista nel paragrafo \ref{teorema-caratterizzazione}. \\Quindi semplicemente riscriviamo questa quattro proprietà semplificando $Id_A \subseteq R;R^{op}$ con $R;R^{op} \subseteq Id_A$ in $Id_A = R;R^{op}$ e $Id_B \subseteq R^{op};$ con $R^{op};R \subseteq Id_B$ con $Id_B = R^{op};R$ (possiamo fare questa "semplificazione" perché due insiemi sono uguali quando uno è sottoinsieme dell'altro, come in questo caso).
\begin{example}
    Come si usa la caratterizzazione:\\
    Dati due insiemi A, B ed una relazione R $\subseteq$ A X B. Se riusciamo a trovare una relazione S $\subseteq$ B X A tale che venga soddisfatta la definizione sopra scritta per cui $Id_A = R;S \land S;R = Id_B$ è equivalente a dimostrare che la relazione R è una biezione.
\end{example}

\subsubsection{Insiemi di biezione}
\begin{definition}[Insiemi di biezione]
Dati due insiemi A e B, essi sono in biezione
\footnote{puoi chiamare due insiemi in biezione anche in corrispondenza, 1-1 o in una relazione biunivoca}
se esiste una biezione $i: A \rightarrow B$, e si scrive come $A \cong B$
\footnote{attenzione: usare il simbolo $\cong$ invece che un semplice = vuole dire che non per forza le due parti devono essere ugualia}
\end{definition}
\begin{example}
    Esempi insiemi di biezione:
    \begin{itemize}
        \item Dati gli insiemi $2 = \{0,1\}$ e bool = $\{true, false\}$ \hspace{.3cm} l'insieme di biezione è $2 \cong$ bool
        \item Dati A e B \hspace{.3cm} l'insieme di biezione è A $\times$ B $\cong$ B $\times$ A \\
        Altri modi per scriverlo sono: \hspace{.3cm} i:A $\times$ B $\longrightarrow$ B $\times$ A \: - \: i((a,b))=(b,a)\footnote{questa forma vuol dire che se diamo in input alla funzione i una coppia di valori (a,b) restituirà una coppia (b,a)}\\
        \textbf{NOTA:} A $\times$ B = B $\times$ A sarebbe farlo perché uno crea coppie (a,b) e l'altro coppie (b,a).
        \item Dati gli insieme $1 = \{0\}$ ed A \hspace{.3cm} l'insieme di biezione è A $\times$ 1 $\cong$ A \\
        Altri modi per scriverlo sono: \hspace{.3cm} i:A $\times$ 1 $\longrightarrow$ A \hspace{.3cm} i((a,0)) = a
        \item fun(A $\times$ B, C) $\cong$ fun(A, (fun(B,C))\\
        \textbf{Spiegazione esempio:} la prima funzione data una coppia (a,b) restituisce un valore c mentre la seconda funzione dando un valore a restituisce una nuova funzione, dove a sua volta se inseriamo un valore b restituisce c. Quindi f((a,b)) = c e f(a)(b)=c.
    \end{itemize}
\end{example}
\begin{example} Esempio particolare con dimostrazione \\
        Per tutti gli insiemi $A, B, C$ vale che:
        \begin{equation}
            (A \times B) \times C \cong A \times (B \times C)
        \end{equation}
        \begin{demostration}
            Innanzitutto scriviamo questa biezione sotto la seguente forma \\i((a,b),c) = (a,(b,c)) dove la funzione "i" prende in input una coppia di valori ((a,b),c) e restituisce (a,(b,c)).\\
            Utilizziamo la proprietà di caratterizzazione scritta nel paragrafo \ref{caratterizzazione-biezione} che dice che: \\ $R \Longleftrightarrow Id_A = R;R^{op} \land Id_B = R^{op};R$.\\
            Sfruttiamola applicandola al nostro caso quindi (consideriamo i come una relazione fra due insiemi W e K dove W è $((A \times B) \times C)$ e K è $(A \times (B \times C)))$:
            \begin{equation}
                i \: \: biettiva \Longleftrightarrow Id_W = i;i^{op} \land i^{op};i = Id_K
            \end{equation}
            Il nostro obbiettivo è quindi trovare $i^{op}$ che soddisfi le due condizioni determinate da $\land$ sopra:
            \begin{itemize}
                \item $Id_W = i;i^{op}$ - \underline{Dimostrazione 1°}
                \item $Id_K = i^{op};i$ - \underline{Dimostrazione 2°}
            \end{itemize}
                \underline{Dimostrazione 1°} - $Id_W = i;i^{op}$
                Ricordiamo che l'opposto di "i" è $i^{op}: A \times (B \times C) \longrightarrow (A \times B) \times C$ che quindi possiamo vedere come una funzione che prende in ingresso una coppia di valori (a, (b,c)) e restituisce ((a,b),c).\\
                Ora rappresentiamo la composizione fra i e $i^{op}$ come unione fra funzioni come funzione di una funzione quindi: $i;i^{op} = i^{op}(i(x))$\\
                Ora semplicemente riscriviamo la composizione di funzioni scrivendo i parametri di input ed output:
                \begin{equation}
                    i^{op}(i((a,b),c) = i^{op}(a, (b,c)) \: \: che \: \: restituisce \: \: ((a,b),c)
                \end{equation}
                Vediamo così che l'opposto di i restituisce lo stesso valore che restituisce $Id_W$ ($Id_W$ è una relazione fra W e W, visto che W è (A $\times$ B) $\times$ C possiamo scrivere che $Id_W((a,b),c) = ((a, b),c)$).
                \item $\mathcal{P}(A) \cong$ fun(A,2)\footnote{Questo insieme è quello dei numeri binari}. Questo caso è dimostrato. $\blacksquare$\\ \\
                \underline{Dimostrazione 2°} - $Id_K = i^{op};i$
                Procediamo in maniera analoga alla dimostrazione 1° quindi andiamo a rappresentare la composizione fra l'opposto $i^{op}$ ed i come funzioni di funzione $i(i^{op})$ andando poi ad inserire i parametri i input ed output:
                \begin{equation}
                    i(i^{op}(a,(b,c)) = i((a,b),c) \: \: che \: \: restituisce \: \: (a, (b,c))
                \end{equation}
                Pure in questo caso possiamo vedere che l'opposto di K $Id_K$ restituisce gli stessi valori scritti sopra (anche in questo caso tieni a mente che l'opposto di $Id_K$ è una relazione che associa K con K quindi, ricordando che K è A $\times$ (B $\times$ C), se la scriviamo sotto forma di funzione è $Id_K(a,(b,c)) = (a,(b,c)$). Anche questo caso è dimostrato. $\blacksquare$\\ \\
                Essendo che entrambi le casistiche sono state dimostrate possiamo concludere che l'insieme di biezione (A $\times$ B) $\times$ C $\cong$ A $\times$ (B $\times$ C) è dimostrato. $\blacksquare$
        \end{demostration}
\end{example}

\subsubsection{Proprietà insiemi di biezione}
Per tutti gli insiemi $A, B, C$ valgono le proprietà scritte nella tabella \ref{tab:proprietà-insiemi-biezione}.
\begin{table}[h!]
    \centering
    \setlength{\tabcolsep}{8pt}
    \renewcommand{\arraystretch}{2}
    \begin{tabular}{|c|c|}
    \hline
        \textbf{Riflessiva} & A $\cong$ A  \\
        \textbf{Simmetrica} & A $\cong B \Longrightarrow B \cong A$ \\
        \textbf{Transitiva} & $A \cong B$, $B \cong C \Longrightarrow A \cong C$\\ \hline
    \end{tabular}
    \caption{Proprietà insiemi di biezione}
    \label{tab:proprietà-insiemi-biezione}
\end{table}

\subsection{N-upla}
\begin{definition}[N-upla]
Sia $A$ un insieme e $n \in \mathbf{N}$. Una sequenza su $A$ di lunghezza $n$ è una n-upla ($a_0, a_1, \ldots, a_{n-1}$) dove $a_i \in A$ per ogni indice $i \in \{0,1,\ldots,n-1\}$. Definiamo quindi l'insieme $A^n$ du tutte le sequenze come:
\begin{equation}
    A^n = \{(a_0,a_1, ..., a_{n+1}) \:|\: \forall \: i \in \{0,...,n-1\}\:.\: a_i \in A\}
\end{equation}
\end{definition}

\begin{definition}[Sequenza su A di lunghezza arbitraria]
Una sequenza su A di lunghezza arbitraria è una sequenza di lunghezza n per qualsiasi numero naturale $n \in \mathbb{N}$. L'insieme di tutte le sequenze su A di lunghezza arbitraria:
\[A^* = \bigcup_{n\in \mathbb{N}} A^n\]
\end{definition}