\documentclass[a4paper,10pt]{article}
\usepackage[utf8]{inputenc}

% ----  Useful packages % ---- 
\usepackage{amsmath}
\usepackage{graphicx}
\usepackage{amsfonts}
\usepackage{amsthm}
\usepackage{amssymb}
\usepackage{mathtools}
\usepackage{enumitem}
% ----  Useful packages % ---- 

\usepackage{wrapfig}
\usepackage{caption}
\usepackage{subcaption}
\usepackage{hyperref}
\hypersetup{
    colorlinks,
    citecolor=black,
    filecolor=black,
    linkcolor=black,
    urlcolor=black
}

\graphicspath{ {./images/} }

% ---- Set page size and margins replace ------
\usepackage[letterpaper,top=2cm,bottom=2cm,left=3cm,right=3cm,marginparwidth=1.75cm]{geometry}
% ---- Set page size and margins replace ------

% ------- NOTA ------
\theoremstyle{remark}
\newtheorem{note}{Note}[subsection]
% ------- NOTA ------

% ------- OSSERVAZIONE ------
\theoremstyle{definition}
\newtheorem{observation}{Osservazione}[subsection]
% ------- OSSERVAZIONE ------

% ------- DEFINIZIONE ------
\theoremstyle{plain}
\newtheorem{definition}{Definizione}[subsection]
% ------- DEFINIZIONE ------

% ------- ESEMPIO ------
\theoremstyle{definition}
\newtheorem{example}{Esempio}[subsection]
% ------- ESEMPIO ------

% ------- DIMOSTRAZIONE ------
\theoremstyle{definition}
\newtheorem{demostration}{Dimostrazione}[subsection]
% ------- DIMOSTRAZIONE ------

% ------- TEOREMA ------
\theoremstyle{definition}
\newtheorem{theorem}{Teorema}[subsection]
% ------- TEOREMA ------

% ------- COROLLARIO ------
\theoremstyle{plain}
\newtheorem{corollary}{Corollario}[theorem]
% ------- COROLLARIO ------

% ------- PROPOSIZIONE ------
\theoremstyle{plain}
\newtheorem{proposition}{Proposizione}[subsection]
% ------- PROPOSIZIONE ------

% ---- Footer and header ---- 
\usepackage{fancyhdr}
\pagestyle{fancy}
\fancyhf{}
\fancyhead[LE,RO]{A.A 2022-2023}
\fancyhead[RE,LO]{Algebra Lineare}
\fancyfoot[RE,LO]{\rightmark}
\fancyfoot[LE,RO]{\thepage}

\renewcommand{\headrulewidth}{.5pt}
\renewcommand{\footrulewidth}{.5pt}
% ---- Footer and header ---- 

% ----  Language setting ---- 
\usepackage[italian, english]{babel}
% ----  Language setting ---- 

\title{\textbf{Algebra Lineare}}
\author{Realizzato da: Giuntoni Matteo}
\date{A.A. 2022-2023}

\begin{document}
\begin{titlepage} %crea l'enviroment
	\begin{figure}[t] %inserisce le figure
		\centering\includegraphics[width=0.98\textwidth]{marchio_unipi_pant541.png}
	\end{figure}
	\vspace{20mm}
	
	\begin{Large}
		\begin{center}
			\textbf{Dipartimento di Informatica\\ Corso di Laurea Triennale in Informatica\\}
			\vspace{20mm}
			{\LARGE{Corso 3° anno - 6 CFU}}\\
			\vspace{10mm}
			{\huge{\bf Ingegneria del Software}}\\
		\end{center}
	\end{Large}
	
	
	\vspace{36mm}
	%minipage divide la pagina in due sezioni settabili
	\begin{minipage}[t]{0.47\textwidth}
		{\large{\bf Professore:}\\ \large{Prof. Jacopo Soldani}}
	\end{minipage}
	\hfill
	\begin{minipage}[t]{0.47\textwidth}\raggedleft
		{\large{\bf Autore:}\\ \large{Filippo Ghirardini}}
	\end{minipage}
	
	\vspace{25mm}
	
	\hrulefill
	
	\vspace{5mm}
	
	\centering{\large{\bf Anno Accademico 2024/2025}}
	
\end{titlepage}
\tableofcontents
\newpage
\maketitle
\begin{center}
    \vspace{-20pt}
    \rule{11cm}{.1pt} 
\end{center}
% !TeX spellcheck = it_IT
\section{Introduzione}
\subsection{Sistemi di equazioni}
L'algebra lineare è lo studio delle soluzioni di sistemi di equazioni lineari utilizzando spazi vettoriali.
\begin{example}
Un esempio di sistemi di equazioni:
\begin{enumerate}
    \item
    $\begin{rcases*}
    	E_1: x + y = 5 \\ E_2: x + 2y = 6
    \end{rcases*}
	\Rightarrow E_2 - E_1$ (sostituzione):
	$\begin{cases}
		y = 5 - 3 = 2 \\ x = 3 - 2 = 1
	\end{cases}$ 
	Un unica soluzione.
    \item 
    $\begin{rcases*}
    	E_1: x + y = 3 \\ E_2: 2x + 2y = 6
    \end{rcases*}
	\Rightarrow E_2 - 2E_1$: 0 = 0.\\\\
    Infatti $E_2 = 2E_1 \Rightarrow$ hanno le stesse soluzioni $\Rightarrow$ $\exists \infty$ soluzioni.
    \item 
    $\begin{rcases*}
    	E_1: x + y = 3 \\ E_2: 2x + 2y = 5
    \end{rcases*}
	\Rightarrow E_2 - 2E_1: 0 = -1$ è impossibile infatti $\nexists$ soluzioni comuni.
\end{enumerate}
Possiamo vedere da questi esempi che abbiamo tre possibili risultati: 1 soluzione, $\infty$ e 0.
\end{example}

\subsection{Interpretazioni geometrica}
In ogni caso le equazioni $E_1$ ed $E_2$ rappresentano rette su un piano a 2 dimensioni. Le soluzioni comuni sono i punti di intersezione delle rette. \\Nel caso specifico dell'esempio 1.1.1 abbiamo che:
\begin{figure}[h!]
    \centering
    \begin{subfigure}{.3\textwidth}
        \centering
        \includegraphics[width=3cm]{images/rette-incidenti.png}
        \caption{1° hanno un punto in comune P=(1,2)}
    \end{subfigure}
    \hfill
    \begin{subfigure}{.3\textwidth}
        \centering
        \includegraphics[width=3cm]{images/rette-coincidenti.png}
        \caption{2° coincidono $\Rightarrow \infty$ punti in comune}
    \end{subfigure}
    \hfill
    \begin{subfigure}{.3\textwidth}
        \centering
        \includegraphics[width=3cm]{images/rette-parallele.png}
        \caption{3° sono parallele  $\Rightarrow \nexists$ punti in comune}
    \end{subfigure}
\end{figure}
\newpage
\subsection{Equazioni a 3 variabili}
Un esempio di equazione a 3 variabili è $x + 2y + 3z = 4$. Ciò crea, invece di una retta, un piano nello spazio 3-dimensionale.
Se adesso consideriamo le equazioni viste sopra $E_1$ ed $E_2$ come equazioni a 3 variabili possiamo vedere che esse corrispondono a 2 piani nello spazio ed i punti in comune formano una retta.\\
\begin{figure}[h!]
    \centering
    \begin{subfigure}{.3\textwidth}
        \centering
        \includegraphics[width=2.5cm]{images/piani-incidenti.png}
        \caption{1° forma una retta}
    \end{subfigure}
    \begin{subfigure}{.3\textwidth}
        \centering
        \includegraphics[width=2.5cm]{images/piani-coincidenti.png}
        \caption{2° i due piani coincidono}
    \end{subfigure}
    \begin{subfigure}{.3\textwidth}
        \centering
        \includegraphics[width=2.5cm]{images/piani-incidenti.png}
        \caption{3° i due piani sono paralleli}
    \end{subfigure}
\end{figure}
Se oltre a $E_1$ ed $E_2$ consideriamo una terza equazione $E_3$ essa corrisponde ad un terzo piano. \\
Possiamo vedere come esso si comporta intersecandolo con l'intersezione fra $E_1$ ed $E_2$, $E_1 \cap E_2$.
\begin{figure}[h!]
    \centering
    \begin{subfigure}{.3\textwidth}
        \centering
        \includegraphics[width=2.5cm]{piano-incotra-retta.png}
        \caption{$E_1 \cap E_2$ è una retta che, intersecata con $E_3$, crea un punto}
    \end{subfigure}
    \begin{subfigure}{.3\textwidth}
        \centering
        \includegraphics[width=2.5cm]{piano-coincide-retta.png}
        \caption{$E_1 \cap E_2$ può essere contenuto in $E_3$ quindi nuova retta}
    \end{subfigure}
    \begin{subfigure}{.3\textwidth}
        \centering
        \includegraphics[width=2.5cm]{piano-non-conindice-retta.png}
        \caption{$E_1 \cap E_2$ e $E_3$ possono non coincidere}
    \end{subfigure}
\end{figure}

\subsection{Caso generale}
Possiamo definire un sistema $(E)$ di $n$ equazioni a $m$ variabili con $n,m > 0$ e con $a_{nm}, b_{n} \in \mathbb{R}$ come:
\begin{flalign}\nonumber
&E_1: a_{11}x_1 + a_{12}x_2 + \ldots + a_{1m}x_m = b_1&&\\\nonumber
&E_2: a_{21}x_1 + a_{22}x_2 + \ldots+ a_{2m}x_m = b_2&&\\\nonumber
&\vdots&&\\
&E_n: a_{n1}x_1 + a_{n2}x_2 + \ldots + a_{nm}x_m = b_n&&\nonumber
\end{flalign}

\begin{definition}[Sistema omogeneo]
Il sistema $(E)$ è \textbf{omogeneo} se $b_1 = \ldots = b_n = 0$. In caso contrario possiamo considerare il sistema omogeneo associato ($E_{om}$) definito come:
\begin{flalign}\nonumber
&E_1: a_{11}x_1 + a_{12}x_2 + \ldots + a_{1m}x_m = 0&&\\\nonumber
&E_2: a_{21}x_1 + a_{22}x_2 + \ldots + a_{2m}x_m = 0&&\\\nonumber
&\vdots&&\\\nonumber
&E_n: a_{n1}x_1 + a_{n2}x_2 + \ldots + a_{nm}x_m = 0&&\nonumber
\end{flalign}
Se $(E)$ è \textbf{omogeneo}, $\exists$ sempre una soluzione comune del tipo $(x_1, \ldots, x_n) = (0_1, \ldots, 0_n)$.
\end{definition}

\begin{proposition}\label{prop-1}
Se $(c_1, ..., c_n)$ e $(d_1, ..., d_n)$ sono soluzioni di $(E)$ $\Longrightarrow$ $c_1 - d_1, ..., c_n - d_n$ è soluzione del sistema omogeneo.
\end{proposition}

\begin{demostration}
Se $(c_1, \ldots, c_m)$ è soluzione vuol dire che :\\
$E_1:a_{i1}c_1 + a_{i2}c_2 + a_{im}c_m = b_i$\\
$E_2:a_{i1}d_1 + a_{i2}d_2 + a_{im}d_m = b_i$\\
Quindi se sottraggo $E1 - E2$ e raccolgo viene:\\
$a_{i1}(c_1 - d_1) + a_{i2}(c_2 - d_m) + a_{im}(c_m - d_m)= 0\:\: \forall \: i,...,n$
\end{demostration}

\begin{theorem}
Se $(c_1,\ldots,c_m)$ è soluzione del sistema $(E)$ tutte le soluzioni $(E)$ sono della forma $(c_1 + e_1, c_2 + e_2, \ldots, c_m + e_m)$ dove $(e_1,\ldots,e_m)$ è soluzione di $E_{om}$.
\end{theorem}
In sinestesi si può semplificare questo teorema scrivendo:
\begin{equation}
    \text{"Soluzione generale" = "Soluzione particolare" + "Soluzione omogenea"}
\end{equation}

\begin{demostration}
La proposizione \ref{prop-1} dice che le soluzioni hanno questa forma. Viceversa se $(e_1,\ldots,e_m)$ sono soluzioni di ($E_{om}$) $\Longrightarrow$ $(c_1 + e_1, c_2 + e_2, \ldots, c_m + e_m)$ sono soluzioni di $(E)$.
\end{demostration}

\begin{example}
Prendiamo n=1 e m=2 e prendiamo come sistema di equazioni $(E): 2x + 3y = 5$ e come equazione omogenea $(E_{om}): 2x + 3y = 0$\\\\
Vediamo che le soluzioni particolari sono $x = y = 1$. Per calcolare le soluzioni omogenee si fa $2x = -3y$ e poi $x = -\frac{3}{2}y$, qui per ogni valore di y trovo un valore di x. \\
La soluzioni omogenea è $(-\frac{3}{2}p, p)$ dove p è un parametro che può essere qualsiasi valore.\\
Sappiamo che "sol. generale" = "sol. particolare" + "sol. omogenea" $\Rightarrow (1,1) + (-\frac{3}{2}t,t) = (1 - \frac{3}{2}t, 1 + t)$.
\end{example}

\begin{observation}
$(0,...,0)$ è sempre soluzione di $(E_{om})$. Quindi se (E) ammette una soluzione questo soluzione è unica $\Longleftrightarrow (0,...,0)$ è l'unica soluzione di $(E_{om})$.
\end{observation}

\subsection{Interpretazione geometrica caso generico}
L'interpretazione geometrica per ($E_{om}$) è un iperpiano attraverso l'origine", e la soluzione è traslazione di questo caso generale per un caso particolare.
\begin{enumerate}
    \item $n=1$, $m=2$ (E) $a_{1n}x_1 + a_{m2}x_2 = b_1$.\\
    Una soluzione $\Longleftrightarrow$ retta ($E_{om}$) $a_1x_1 + a_2x_2 = 0$ una soluzione a (E) $\Rightarrow$ retta attraverso (0,0).
    \item $n=1$, $m=2$, $a_{11}x_1 + a_{12}x_2 + a_{13}x_3 = a$ (E), punto attraverso (0,0,0).
\end{enumerate}

\subsection{Come trovare le soluzioni?}
Per trovare le soluzioni comuni di $(E)$ possiamo usare 3 operazioni per semplificare il sistema:
\begin{enumerate}
    \item Scambiare due equazioni.
    \item Moltiplicare $E_i$ per $\lambda \neq 0$ e fare la somma con $E_j$, $E_j = E_j + \lambda E_i$.
    \item Moltiplicare un'equazione $E_i$ per un costate $\lambda \neq 0$, $E_i \Rightarrow \lambda E_i$.
\end{enumerate}

\begin{observation}
Queste operazioni non cambiano l'insieme delle soluzioni di $(E)$.
\end{observation}

\begin{demostration}
Dimostriamo le 3 proprietà:
\begin{enumerate}
    \item La prima è ovvia quindi non ha bisogno di una dimostrazione.
    \item Se ($c_1, \ldots, c_n$) soluzioni di $E_i$ ed $E_j \Rightarrow$ è anche soluzione di $E_i + \lambda E_j$.\\
    Viceversa se ($c_1, \ldots, c_n$) soluzioni di $E_i$, $E_j + \lambda E_i \Rightarrow$ anche soluzione di ($E_j + \lambda E_i$) - $\lambda = E_j$.
    \item Se ($c_1, \ldots, c_n$) soluzioni di (E) $\Rightarrow$ anche di $\lambda E$ e viceversa.
\end{enumerate}
\end{demostration}


% !TeX spellcheck = it_IT
\newpage
\section{Algoritmo di Gauss}
\subsection{Matrice a scalini}
Utilizzando le proprietà (1), (2) e (3) viste sopra si ottiene un algoritmo per semplificare $(E)$.
In un primo momento si considera $b_1 = \ldots = b_n = 0$ e mettiamo i coefficienti in una matrice $n \times m$, [$a_{ij}$]:
\[
\begin{bmatrix}
a_{11} & a_{12} & \ldots & a_{1m}\\
a_{21} & a_{22} & \ldots & a_{2m}\\
& \vdots & & \\
a_{n1} & a_{n2} & \ldots & a_{nm}
\end{bmatrix}
\xRightarrow[]{\text{(Con operazioni)}}
\begin{bmatrix}
\ldots \\
a_{j1} + \lambda a_{i1} + a_{j2} + \lambda a_{i2} + \ldots + a_{jm} + \lambda a_{im}\\
\ldots
\end{bmatrix}
\]
Le operazioni di prima si traducono come:
\begin{enumerate}
    \item Scambiare due righe fra di loro.
    \item Sostituire la riga $R_j$ con la riga $R_j + \lambda R_i$.
    \item Moltiplicare una riga per $\lambda \neq 0$.
\end{enumerate}
Partendo da una matrice l'algoritmo produce, utilizzando le 2 operazioni una matrice detta \textbf{a forma di scalini}.

\begin{definition}[Matrice a forma a scalini]
Una matrice è a forma a scalini (per righe) se:
\begin{itemize}
    \item Le righe $(0,\ldots,0)$ sono "in fondo" alla matrice (partendo da sinistra).
    \item Il primo elemento di ogni riga (se esiste) è a destra del primo elemento diverso da 0 della riga precedente. Tale elemento si dice pivot.
\end{itemize}
\end{definition}

\begin{definition}[Pivot]
Il primo elemento diverso a 0 d ogni riga di una matrice (nella forma a scalini) si chiama pivot.
\end{definition}

\begin{example}
Esempio di matrici in forma a scalini e non:
\vspace{-10pt}
\begin{figure}[h!]
    \centering
    \begin{minipage}{.3\linewidth}
        \centering
        \[
            \begin{bmatrix}
            1 & 1 & 1\\
            1 & 1 & 0\\
            0 & 0 & 1
            \end{bmatrix}
        \]
        NO
    \end{minipage}
    \begin{minipage}{.3\linewidth}
        \centering
        \[
            \begin{bmatrix}
            1 & 1 & 1\\
            0 & 1 & 1\\
            0 & 0 & 1
            \end{bmatrix}
        \]
        SI
    \end{minipage}
    \begin{minipage}{.3\linewidth}
        \centering
        \[
            \begin{bmatrix}
            0 & 1 & 1\\
            1 & 1 & 0\\
            0 & 0 & 1
            \end{bmatrix}
        \]
        NO
     \end{minipage}
\end{figure}
\end{example}

\begin{observation}
	Quando ci troviamo davanti ad una matrice a scalini possiamo avere due casi principali:
	\begin{enumerate}
		\item Se abbiamo che ogni colonna ha un \textbf{pivot}, notiamo che partendo dal basso avremo $a \cdot X_m = 0 \Rightarrow X_m = 0$. Quindi sostituendo nella riga precedente avremo che $b \cdot X_{m-1} + a \cdot X_m = 0 \Rightarrow X_{m-1} = 0$ e così via fino ad arrivare ad una \textbf{soluzione banale}, ovvero $(0, \ldots, 0)$.
		\item Altrimenti per ogni colonna senza \textbf{pivot} avremo una variabile libera che dà $\infty$ soluzioni.
	\end{enumerate}
\end{observation}

\subsection{Algoritmo in un sistema omogeneo}
\begin{definition}[Algoritmo di Gauss]
Ogni matrice $n \times m$ si mette in forma a scalini (per righe) con operazioni del tipo 1 e 2.
\begin{enumerate}[start=0]
    \item Se la matrice è già in forma a scalini abbiamo finito.
    \item Si cerca il primo elemento diverso da $0$ della prima colonna diversa da $0$.
    \item Cambiando $n$ righe si può supporre che questo elemento è il pivot della prima riga. Se siamo in forma a scalini abbiamo finito, altrimenti procediamo.
    \item Si annullano tutti gli elementi della colonna del pivot sotto il pivot  con operazioni del tipo (B). Se siamo in forma a scalini abbiamo finito, altrimenti procediamo.
    \item Non consideriamo la prima riga e ricominciamo dal punto 1.
\end{enumerate}
\end{definition}

\newpage
\begin{example}
Prendiamo il seguente sistema di equazioni e scriviamolo su una matrice:
\[
\begin{array}{l}
x_1 + x_2 + 3x_4 = 0\\
3x_1 - x_2 + x_3 + 10x_4 = 0\\
x_1 + 5x_2 + 2x_3 + x_4 = 0
\end{array}
\hspace{.2cm}
\Rightarrow
\hspace{.2cm}
\begin{bmatrix}
1 & -1 & 0 & 3\\
3 & -1 & 1 & 10\\
1 & 5 & 2 & 1
\end{bmatrix}
\:\:
\parbox{4cm}{Da qui iniziamo ad applicare l'algoritmo}
\]
\end{example}
\begin{figure}[h!]
    \vspace{-20pt}
    \centering
    \begin{minipage}{.3\linewidth}
        \centering
        \[
            \begin{bmatrix}
            \underline{1} & -1 & 0 & 3\\
            3 & -1 & 1 & 10\\
            1 & 5 & 2 & 1
            \end{bmatrix}
        \]
        Applichiamo il passo (1) e troviamo 1 come pivot
    \end{minipage}
    \begin{minipage}{.3\linewidth}
        \centering
        \[
            \begin{bmatrix}
            1 & -1 & 0 & 3\\
            0 & 3 & 1 & 1\\
            0 & 6 & 2 & -1
            \end{bmatrix}
        \]
        Applichiamo il passo (3) e calcoliamo $R_2 = R_2 - 3R_1$ e $R_3 = R_3 - R_1$
    \end{minipage}
    \begin{minipage}{.3\linewidth}
        \centering
        \[
            \begin{bmatrix}
            0 & 3 & 1 & 1\\
            0 & 0 & -1 & -5
            \end{bmatrix}
        \]
        Applichiamo (4) e non consideriamo la riga $R_1$ per poi ripetere l'operazione (3) facendo $R_3 = R_3 - 3R_2$
    \end{minipage}
\end{figure}
\noindent Vediamo così che la matrice finale è in forma a scalini. Possiamo ora prendere i numeri nella matrice e andare a riscrivere il sistema di equazioni associato. Per il nostro esempio abbiamo:
\[
\begin{array}{l}
x_1 + x_2 + 3x_4 = 0\\
2x_2 + x_3 + x_4 = 0\\
-x_3 + 5x_4 = 0
\end{array}
\hspace{.2cm}
\xRightarrow[]{\text{(E la matrice)}}
\hspace{.2cm}
\begin{bmatrix}
1 & -1 & 0 & 3\\
0 & 3 & 1 & 1\\
0 & 0 & -1 & -5
\end{bmatrix}
\]
A questo punto se consideriamo $x_4 = t$ abbiamo che $x_3 = -5t$ e di conseguenza $2x_2 - 5t + t = 0$ e quindi $x_2 = 2t$ ed ancora abbiamo $x_1 = 2t -3t -t$. Possiamo dunque dire che in questo esempio $x_4$ essendo una colonna senza pivot è una "variabile libera" e quindi ci sono più soluzioni.\\
Potrebbe esserci anche il caso in cui la colonna contenga un pivot e quinci ci sarebbe un unica soluzione.

\subsection{Algoritmo in un sistema non omogeneo}
Se consideriamo invece un sistema non omogeneo formato aggiungendo una colonna con $b_1, ..., b_n$, è possibile utilizzare ugualmente l'algoritmo di Gauss aggiungendo una colonna alla matrice.

\begin{example}
Prendiamo il seguente sistema di equazioni e mettiamolo su una matrice:
\end{example}
\begin{figure}[h!]
    \vspace{-20pt}
    \centering
    \begin{minipage}{1\linewidth}
        \centering
        \[
            \begin{array}{l}
            x_1 + x_2 + x_3 + 2x_4 = 9\\
            x_1 + x_2 + 2x_3 + x_4 = 8\\
            x_1 + 2x_2 + x_3 + x_4 = 7\\
            2x_1 + x_2 + x_3 + x_4 = 6
            \end{array}
            \hspace{.2cm}
            \Rightarrow
            \hspace{.2cm}
            \begin{bmatrix}
            1 & 1 & 1 & 2 & 9\\
            1 & 1 & 2 & 1 & 8\\
            1 & 2 & 1 & 1 & 7\\
            2 & 1 & 1 & 1 & 6
            \end{bmatrix}
            \xRightarrow[]{\text{Uso algoritmo}}
            \begin{bmatrix}
            1 & 1 & 1 & 2 & 9\\
            1 & 1 & 2 & 1 & 8\\
            1 & 2 & 1 & 1 & 7\\
            2 & 1 & 1 & 1 & 6
            \end{bmatrix}
            \begin{array}{r}
            R_2 = R_2 - R_1\\
            R_3 = R_3 - R_1\\
            R_4 = R_4 - 2R_1
            \end{array}
        \]
        Il pivot è in $R_1$ ed è 1, da qui applichiamo (3) e poi (4).
    \end{minipage}
\end{figure}
\vspace{-15pt}
\begin{figure}[h!]
    \begin{minipage}{1\linewidth}
        \centering
        \[
            \begin{bmatrix}
            0 & 0 & 1 & 2 & -1\\
            0 & 1 & 0 & -1 & -2\\
            0 & -1 & -1 & -3 & -12
            \end{bmatrix}
            \Rightarrow
            \begin{bmatrix}
            0 & 1 & 0 & -1 & -2\\
            0 & 0 & 1 & -1 & -1\\
            0 & -1 & -1 & -3 & -12
            \end{bmatrix}
            \Rightarrow
            \begin{bmatrix}
            0 & 1 & 0 & -1 & -2\\
            0 & 0 & 1 & -1 & -1\\
            0 & 0 & -1 & -4 & -14
            \end{bmatrix}
        \]
        Il pivot ora è in $R_3$ e quindi applichiamo (2) per scambiare $R_2$ con $R_3$ e poi facciamo (3) con $R_4 = R_4 - R_2$.
    \end{minipage}
\end{figure}
\vspace{-15pt}
\begin{figure}[h!]
    \begin{minipage}{.45\linewidth}
        \centering
        \[
            \begin{bmatrix}
            0 & 0 & 1 & -1 & -1\\
            0 & 0 & -1 & -4 & -14
            \end{bmatrix}
            \Rightarrow
            \begin{bmatrix}
            0 & 0 & 1 & -1 & -1\\
            0 & 0 & 0 & -5 & -15
            \end{bmatrix}
        \]
        Usiamo (4) per eliminare $R_2$ e troviamo il pivot in $R_3$, applichiamo poi (4) con $R_4 = R_4 + R_4$
    \end{minipage}
    \hspace{1cm}
    $\Longrightarrow$
    \hspace{-1cm}
    \begin{minipage}{.45\linewidth}
        \centering
        \[
            \begin{bmatrix}
            1 & 1 & 1 & 2 & 9\\
            0 & 1 & 0 & -1 & -2\\
            0 & 0 & 1 & -1 & -1\\
            0 & 0 & 0 & -5 & -15
            \end{bmatrix}
        \]
        Abbiamo finito perché il risultato è a scalini    
\end{minipage}
\end{figure}
Il risultato finale corrisponde al seguente sistema:
\[
\begin{array}{l}
x_1 + x_2 + x_3 + 2x_4 = 9\\
x_2 - x_4 = -2\\
x_3 - x_4 = -1\\
-5x_4 = -15
\end{array}
\hspace{.5cm}
\parbox{5cm}{Se sostituiamo: \\$x_4 = 4, x_3 = 2, x_2 = 1, x_1 = 0$\\Quindi abbiamo un sistema con un unica soluzione}
\]
Da questo esempio possiamo vedere una caratteristica comune per questa tipologia di esercizi, cioè che il sistema di equazione ha un unica soluzione se ogni colonna contiene un pivot.

\begin{example}
Altro esempio di sistema di applicazione dell'algoritmo di gauss.
\[
    \begin{array}{l}
    x_1 + x_2 - 4x_3 = 1\\
    2x_1 + 3x_2 - 10x_3 = 2\\
    5x_1 + 3x_2 - 4x_3 = 5
    \end{array}
    \Longrightarrow
    \begin{bmatrix}
    1 & 1 & -4 & 1\\
    2 & 3 & -10 & 2\\
    5 & -3 & -4 & 1
    \end{bmatrix}
    \begin{array}{r}
    R_2 = R_2 - 2R_1\\
    R_3 = R_3 - 5R_1
    \end{array}
    \Longrightarrow
    \begin{bmatrix}
    0 & 1 & -2 & 0\\
    0 & -8 & 16 & 0
    \end{bmatrix}
    R_3 = R_3 + 8R_2
\]
\[
    \Longrightarrow
    \begin{bmatrix}
    1 & 1 & -4 & 1\\
    0 & 1 & -2 & 0\\
    0 & 0 & 0 & 0
    \end{bmatrix}
    \hspace{.3cm}
    \parbox{5cm}{Questa è una matrice in forma a scalini e la sua trasposizione in sistema di equazione è:}
    \hspace{.3cm}
    \begin{array}{l}
    x_1 + x_2 - 4x_3 = 1\\
    x_2 - 2x_3 = 0
    \end{array}
\]
Abbiamo quindi che $x_3$ è una variabile libera e quindi se poniamo $x_3 = t$ abbiamo $x_2 = 2t$ e $x_1 = 1+2t$.\\
Soluzione particolare: (1,0,0). Soluzione generale: $(1+2t, 2t, t)$. Sol. Omogenea: $(2t, 2t, t)$.
\end{example}
\hspace{-15pt}Da questo esempio vediamo invece che se c'è una colonna senza pivot all'ora esiste almeno una variabile libera e quindi ci sono $\infty$ soluzioni.

\begin{example}
Facciamo un ultimo esempio per vedere un ulteriore casistica per l'algoritmo di gauss.
\[
    \begin{array}{l}
    x_1 - x_2 - x_3 = 3\\
    3x_1 - 2 x_2 - 4x_3 = 3\\
    4x_1 + x_2 - 9x_3 = 7
    \end{array}
    \Longrightarrow
    \begin{bmatrix}
    1 & -1 & -1 & 3\\
    3 & -2 & -4 & 3\\
    4 & 1 & -9 & 7
    \end{bmatrix}
    \begin{array}{r}
    R_2 = R_2 - 3R_1\\
    R_3 = R_3 - 4R_1
    \end{array}
    \Longrightarrow
    \begin{bmatrix}
    0 & 1 & -1 & -6\\
    0 & 5 & -5 & -5
    \end{bmatrix}
    R_3 = R_3 -5R_2
\]
\[
    \Longrightarrow
    \begin{bmatrix}
    1 & -1 & -1 & 3\\
    0 & 1 & -1 & -6\\
    0 & 0 & 0 & 25
    \end{bmatrix}
    \hspace{.3cm}
    \parbox{5cm}{Questa è una matrice in forma a scalini e la sua trasposizione in sistema di equazione è:}
    \hspace{.3cm}
    \begin{array}{l}
    x_1 - x_2 - x_3 = 3\\
    x_2 - x_3 = -6
    0 = 25
    \end{array}
\]
Possiamo notare che l'equazione $0=25$ non ha senso quindi non c'è nessuna soluzione.
\end{example}
\hspace{-15pt}Anche in questo caso possiamo estendere l'esempio in un caso generale dicendo che se c'è un pivot nell'ultima colonna allora non esistono soluzioni particolari (il sistema omogeneo però ammette $\infty$ soluzioni).\\
In sintesi possiamo riassumere i 3 casi visti in questi esempi come di seguito:
\begin{itemize}
    \item Ogni colonna "non aggiunta" ha un pivot $\Longleftrightarrow$ unica soluzione.
    \item C'è un pivot nell'ultima colonna $\Longleftrightarrow \: \nexists$ soluzione.
    \item C'è una colonna "non aggiunta" senza pivot e l'ultima colonna non ne ha $\Longleftrightarrow \: \infty$ soluzioni.
\end{itemize}

\subsection{Algoritmo di Gauss-Jordan}
Questo algoritmo estende l'algoritmo di Gauss producendo una matrice ridotta a scalini.
\begin{definition}[Matrice ridotta]
Una matrice è in forma \textbf{ridotta} a scalini se:
\begin{itemize}
    \item E' in forma a scalini.
    \item Ogni pivot è uguale a 1.
    \item Ogni pivot è l'unico elemento $\neq 0$ nella sua colonna.
\end{itemize}
\end{definition}

\begin{example}
Esempio di matrici in forma a scalini ridotta e non.
\vspace{-10pt}
\begin{figure}[h!]
    \centering
    \begin{minipage}{.4\linewidth}
        \centering
        \[
            \begin{bmatrix}
            1 & 2 & 0 & 0\\
            0 & 0 & 1 & 0\\
            0 & 0 & 0 & 1\\
            \end{bmatrix}
        \]
        SI, è in forma a scalini ridotta
    \end{minipage}
    \hspace{.3cm}
    \begin{minipage}{.4\linewidth}
        \centering
        \[
            \begin{bmatrix}
            1 & 2 & 3 & 4\\
            0 & 0 & 1 & 2\\
            0 & 0 & 0 & 1
            \end{bmatrix}
        \]
        NO, questa è in forma scalini ma non ridotta.
    \end{minipage}
\end{figure}
\end{example}
\newpage
\begin{definition}[Algoritmo di Gauss-Jordan]
L'algoritmo sfrutta le 2 operazioni viste per l'algoritmo di gauss (A) e (B) ma aggiungendo anche l'operazione (C). Questo algoritmo, partendo da una matrice a scalini genera una matrice ridotta a scalini.
\begin{enumerate}
    \item Con l'algoritmo di Gauss portiamo la matrice in forma a scalini.
    \item In ogni riga si cerca il pivot (se esiste). Se il pivot è $\lambda \neq 1$, moltiplicare la riga per $\frac{1}{\lambda}$ (operazione (C)).
    \item Nella colonna dei pivot gli elementi sotto (e nella riga a sinistra) sono già uguali a 0. Annullare gli elementi sopra della colonna con operazioni del tipo (B).\\
    Questa operazione non cambia gli altri pivot perché sono o a sinistra o sotto.
\end{enumerate}
\end{definition}

\begin{example}
Proviamo ad applicare l'algoritmo di gauss-jordan con il seguente sistema di euqzioni
\[
    \begin{array}{l}
    2x_1 + x_2 - x_3 = -1\\
    3x_1 + 2x_2 - x_3 = 0\\
    4x_1 - 3x_2 + x_3 = -1\\
    5x_1 - 2x_2 + 2x_3 = 2
    \end{array}
    \Longrightarrow
    \begin{bmatrix}
    2 & 1 & -1 & -1\\
    3 & 2 & -1 & 0\\
    4 & -3 & 1 & -1\\
    5 & -2 & 2 & 2
    \end{bmatrix}
    \hspace{.3cm}
    \parbox{5cm}{Per semplificare la matrice usiamo l'operazione (C) e moltiplichiamo una riga per una costante:}
    \begin{array}{l}
    R_2 = 2R_2\\
    R_4 = 2R_4
    \end{array}
\]
\[          
    \Rightarrow
    \parbox{2cm}{Applichiamo l'algoritmo di gauss}
    \begin{bmatrix}
    2 & 1 & -1 & -1\\
    6 & 4 & -2 & 0\\
    4 & -3 & 1 & -1\\
    10 & -4 & 4 & 4
    \end{bmatrix}
    \begin{array}{l}
    R_2 = R_2 - 3R_1\\
    R_3 = R_3 - 2R_1\\
    R_4 = R_4 - 5R_1
    \end{array}
    \Rightarrow
    \begin{bmatrix}
    2 & 1 & -1 & -1\\
    0 & 1 & 1 & 3\\
    0 & -5 & 3 & 1\\
    0 & -9 & 9 & 9
    \end{bmatrix}
    \begin{array}{l}
    R_3 = R_3 + 5R_2\\
    R_4 = R_4 + 9R_2
    \end{array}
\]
\[
    \Rightarrow
    \begin{bmatrix}
    2 & 1 & -1 & -1\\
    0 & 1 & 1 & 3\\
    0 & 0 & 8 & 16\\
    0 & 0 & 18 & 36
    \end{bmatrix}
    \hspace{.3cm}
    \begin{array}{l}
    R_3 = \frac{1}{8}R_3\\
    R_4 = R_4 - \frac{18}{8}R_3
    \end{array}
    \hspace{.3cm}
    \parbox{4cm}{La matrice è in forma scalini quindi applichiamo il punto (2) dell'algoritmo di guass-jordan}
    \Rightarrow
    \begin{bmatrix}
    2 & 1 & -1 & -1\\
    0 & 1 & 1 & 3\\
    0 & 0 & 1 & 2\\
    0 & 0 & 0 & 0
    \end{bmatrix}
    R_1 = R_1 - R_2
\]
\[
\Rightarrow
    \begin{bmatrix}
    2 & 0 & -2 & -4\\
    0 & 1 & 1 & 3\\
    0 & 0 & 1 & 2\\
    0 & 0 & 0 & 0
    \end{bmatrix}
    \begin{array}{l}
    R_1 = R_1 + 2R_3\\
    R_2 = R_2 - R_3
    \end{array}
    \Rightarrow
    \begin{bmatrix}
    2 & 0 & 0 & 0\\
    0 & 1 & 0 & 1\\
    0 & 0 & 1 & 2\\
    0 & 0 & 0 & 0
    \end{bmatrix}
    R_1 = \frac{1}{2}R_1
    \Rightarrow
    \begin{bmatrix}
    1 & 0 & 0 & 0\\
    0 & 1 & 0 & 1\\
    0 & 0 & 1 & 2\\
    0 & 0 & 0 & 0
    \end{bmatrix}
    \begin{array}{l}
    x_1 = 0\\
    x_2 = 1\\
    x_3 = 2
    \end{array}
\]
\end{example}

\begin{example}
Vediamo ora un secondo esempio di questo algoritmo.
\[
    \begin{array}{l}
    2x_1 - 4x_2 + 3x_3 - x_4 = 3\\
    3x_1 - 6x_2 + x_3 + 9x_4 = 3\\
    4x_1 - 8x_2 + 5x_3 + x_4 = 7
    \end{array}
    \Rightarrow
    \begin{bmatrix}
    2 & -4 & 3 & -1 & 3\\
    3 & -6 & 1 & 9 & 8\\
    4 & -8 & 5 & 1 & 7
    \end{bmatrix}
    \hspace{.3cm}
    \parbox{4cm}{Anche in questo caso semplifichiamo la seconda riga con un operazione (C)}
    \hspace{.3cm}
    R_2 = 2R_2
\]
\[
    \Rightarrow
    \begin{bmatrix}
    2 & -4 & 3 & -1 & 3\\
    6 & -12 & 2 & 18 & 16\\
    4 & -8 & 5 & 1 & 7
    \end{bmatrix}
    \begin{array}{l}
    R_2 = R_2 - 3R_1\\
    R_3 = R_3 - 2R_1
    \end{array}
    \Rightarrow
    \begin{bmatrix}
    2 & -4 & 3 & -1 & 3\\
    0 & 0 & -7 & 21 & 7\\
    0 & 0 & -1 & 3 & 1
    \end{bmatrix}
    \begin{array}{l}
    R_2 = -\frac{1}{7}R_2\\
    R_3 = -R_3
    \end{array}
\]
\[
    \Rightarrow
    \begin{bmatrix}
    2 & -4 & 3 & -1 & 3\\
    0 & 0 & 1 & -3 & -1\\
    0 & 0 & 1 & -3 & -1
    \end{bmatrix}
    R_3 = R_3 + R_2
    \begin{bmatrix}
    2 & -4 & 3 & -1 & 3\\
    0 & 0 & 1 & -3 & -1\\
    0 & 0 & 0 & 0 & 0
    \end{bmatrix}
    R_1 = R_1 - 3R_2
\]
\[
    \Rightarrow
    \begin{bmatrix}
    2 & -4 & 0 & 8 & 6\\
    0 & 0 & 1 & -3 & -1\\
    0 & 0 & 0 & 0 & 0
    \end{bmatrix}
    R_1 = \frac{1}{2}R_1
    \begin{bmatrix}
    1 & -2 & 0 & 4 & 3\\
    0 & 0 & 1 & -3 & -1\\
    0 & 0 & 0 & 0 & 0
    \end{bmatrix}
    \begin{array}{l}
    x_2 = s\\
    x_4 = t
    \end{array}
    \begin{array}{l}
    x_1 = 3 + 2s - 4t\\
    x_3 = -1 + 3t
    \end{array}
\]
\end{example}
% !TeX spellcheck = it_IT
\newpage
\section{Spazi vettoriali}
\subsection{Spazio n-dimensionale}
\begin{definition}
	Uno spazio \textbf{n-dimensionale} standard su $\mathbf{R}$ si rappresenta come:
	\begin{equation*}	
		R^n = \Bigg\{ \begin{bmatrix}x_1\\x_2\\ \vdots \\ x_n\end{bmatrix} : x_i \in \mathbb{R} \Bigg\}
	\end{equation*}
\end{definition}
\noindent Geometricamente uno spazio n-dimensionale con $n=2$ sarà un punto sul piano cartesiano.\\
\begin{minipage}{.3\linewidth}
	\centering
	\[
	\begin{bmatrix}a_1\\a_2\end{bmatrix} \Longleftrightarrow Punto
	\] 
\end{minipage}
%TODO Inserire piano cartesiano con punto
\subsection{Operazioni}
Sugli spazi \emph{n-dimensionali} si possono effettuare alcune operazioni:
\begin{itemize}
    \item \textbf{Somma} $(x_1, x_2, x_3) + (x_1', x_2', x_3') = (x_1 + x_1', x_2 + x_2', x_3 + x_3')$.
    \item \textbf{Moltiplicazione} $\lambda(x_1, x_2, x_3) = (\lambda x_1, \lambda x_2, \lambda x_3)$.
\end{itemize}
\begin{figure}[h!]
    \vspace{-12pt}
    \centering
    \begin{minipage}{.3\linewidth}
    \centering
    \[
    \begin{bmatrix}x_1\\x_2\\ \vdots \\ x_n\end{bmatrix} + \begin{bmatrix}x_1'\\x_2'\\ \vdots \\ x_n'\end{bmatrix} = \begin{bmatrix}x_1 + x_1'\\x_2 + x_2'\\ \vdots \\ x_n + x_n'\end{bmatrix}
    \] 
    Somma
    \end{minipage}
    \begin{minipage}{.3\linewidth}
    \centering
    \[
    \lambda \cdot \begin{bmatrix}x_1\\x_2\\ \vdots \\ x_n\end{bmatrix} = \begin{bmatrix}\lambda x_1\\ \lambda x_2\\ \vdots \\ \lambda x_n\end{bmatrix}
    \]
    Moltiplicazione
    \end{minipage}
\end{figure}
%TODO Inserire rappresentazione geometrica di somma e prodotto

\subsection{Spazio vettoriale}
\begin{definition}[Spazio vettoriale]
Uno spazio vettoriale su $\mathbb{R}$ è un insieme V che ammette due tipi di operazioni:
\begin{itemize}
    \item Somma: dati $v_1, v_2 \in V \Longrightarrow v_1 + v_2 \in V$.
    \item Prodotto con $\lambda \in \mathbb{R}$: dato $v \in V \Longrightarrow \lambda \cdot v \in V$.
\end{itemize}
\end{definition}
\hspace{-15pt}Per queste operazioni esistono anche una serie di assiomi che devono essere rispettati:
\begin{table}[h!]
    \setlength{\tabcolsep}{5pt}
    \renewcommand{\arraystretch}{1.7}
    \centering
    \begin{tabular}{|c|c|}
        \hline
        Assiomi Somma & Assiomi Moltiplicazione\\
        \hline\hline
        $(v_1 + v_2) + v_3 = v_1 + (v_2 + v_3)$ & $(\lambda_1 + \lambda_2) v = \lambda_1 v + \lambda_2 v$\\
        $v_1 + v_2 = v_2 + v_1$ & $\lambda(v_1 + v_2) = \lambda v_1 + \lambda v_2$\\
        $\nexists\: 0 \in V \: : \: 0 + v = v + 0 = v \:\: \forall \:v$ & $(\lambda_1 \: \lambda_2)v = \lambda_1(\lambda_2 \: v)$\\
        $\forall \: v \nexists \: -v \in V \: : \: v + (-v) = (-v) + v = 0$ & $(1 \cdot v) = v$\\\hline
    \end{tabular}
    \caption{Assiomi somma e moltiplicazioni vettori}
\end{table}
\vspace{-10pt}
\begin{observation}
$R^n$ soddisfa tutti gli assiomi sopra scritti.
\end{observation}
\newpage
\begin{example}[Matrici]
Consideriamo una matrice $n \times m$ elementi reali $M_{n\times m}(\mathbb{R})$.
\begin{figure}[h!]
    \begin{minipage}{.2\linewidth}
    \vspace{-10pt}
    \centering
    \[
    \begin{bmatrix}
    a_{11} & \cdots & a_{1m}\\
    \vdots \\
    a_{n1} & \cdots & a_{nm}
    \end{bmatrix}
    \]
    \end{minipage}
    \begin{minipage}{.75\linewidth}
    $\mathbb{R}[x] = \{a_nx^n + a_{n-1}x^{n-1} + \cdots + a_0 \: : \: a_i \in \mathbb{R}, n \leq 0\}$\\\\
    Somma: se $A = [a_{ij}]$, $B = [b_{ij}] \in M_{n\times m}(\mathbb{R})$, $A + B = [a_{ij} + b_{ij}] \in M_{n\times m}(\mathbb{R})$\\\\
    Prodotto con $\lambda \in \mathbb{R}$: $\lambda A = [\lambda a_{ij}]$
    \end{minipage}
\end{figure}
\end{example}

\begin{example}[Polinomi]
	Presi dei polinomi a coefficienti reali del tipo:
	\begin{equation*}
		\mathbb{R}[x] = \{a_n \cdot x^n + a_{n-1} \cdot x^{n-1} + \ldots + a_0 \vert a_i \in \mathbb{R} \wedge n \geq 0\}
	\end{equation*}
	Possiamo eseguire entrambe le operazioni:
	\begin{itemize}
		\item \emph{Somma}: $(a_n \cdot x^n + a_{n-1} \cdot x^{n-1} + \ldots + a_0) + (b_m \cdot x^m + b_{m-1} \cdot x^{m-1} + \ldots + b_0)$ con $m \geq n$ è uguale a $b_m \cdot x^m + \ldots + (a_n + b_n) \cdot x^n + (a_{n-1} + b_{n-1}) \cdot x^{n-1} + \ldots + (a_0 + b_0)$
		\item  \emph{Prodotto con $\lambda$}: $\lambda \cdot (a_n \cdot x^n + \ldots + a_0) = \lambda \cdot a_n \cdot x^n + \ldots + \lambda \cdot a_0)$ 
	\end{itemize}
\end{example}

\begin{example}[Funzioni]
Prendiamo due funzioni continue $f, g: \mathbb{R}\to \mathbb{R}$. Possiamo effettuare le operazioni:
\begin{itemize}
	\item \emph{Somma}: $(f_1 + f_2)(x) = f_1(x) + f_2(x)$
	\item \emph{Prodotto con $\lambda$}: $(\lambda f)(x) = \lambda \cdot f(x)$
\end{itemize}
\end{example}

\subsection{Sottospazio}
Introduciamo ora il concetto di sottospazio vettoriale.
\begin{definition}[Sottospazio]
Sia V uno spazio vettoriale. Un \textbf{sottospazio} $W \subset V$ è un sottoinsieme tale che:
\begin{itemize}
    \item $v_1, v_2 \in W \Longrightarrow v_1 + v_2 \in W$.
    \item $v \in \mathbb{W} \Longrightarrow \lambda v \in W \:\forall \: \lambda$.
\end{itemize}
\end{definition}

\begin{proposition}
Un sottospazio $W \subset V$ è a sua volta uno spazio vettoriale.
\end{proposition}

\subsubsection{Interpretazione geometrica}
%TODO Sistema un po'
Un sottospazio vettoriale è una retta che passa per l'origine o un piano che \textbf{passa per l'origine}. 

%TODO Cerca di renderlo più leggibile e di capirci qualcosa
\begin{example}
Dato uno spazio vettoriale:
\begin{equation*}
	V = \mathbb{R}^n = \Bigg \{\begin{bmatrix}t_1\\ t_2\end{bmatrix}: t_i \in \mathbb{R}\Bigg\}
\end{equation*}
Prendiamo un sottospazio vettoriale di $V$:
\begin{equation*}
	\Bigg \{\begin{bmatrix}t_1\\ t_2\end{bmatrix} \in \mathbb{R}^n : t_1 = 0\Bigg\} \subset \mathbb{R}^n
\end{equation*}
Un elemento generale di questo sottospazio (sottospazio con $n=2$) è: $\begin{bmatrix}0\\x_2\end{bmatrix}(x_2 \in \mathbb{R})$\\
Se prendiamo $\begin{bmatrix}0\\x_2\end{bmatrix} + \begin{bmatrix}0\\y_2\end{bmatrix} = \begin{bmatrix}0\\x_1 + x_2\end{bmatrix} \in W$ e $\lambda \cdot \begin{bmatrix}0\\x_2\end{bmatrix} = \begin{bmatrix}0\\\lambda \cdot x_2\end{bmatrix} \in W$\\\\
Similmente se prendiamo $\Bigg \{\begin{bmatrix}t_1\\ t_2\end{bmatrix} \in \mathbb{R}^n \: : \: t_2 = 0\Bigg\} \subset \mathbb{R}^n$ vettori di forma  $\begin{bmatrix}x_1\\0\end{bmatrix}$ che è un sottospazio.\\\\
Se prendiamo invece $\Bigg \{\begin{bmatrix}t_1\\ t_2\end{bmatrix} \in \mathbb{R}^n \: : \: t_1 = 1\Bigg\}$ questo non è un sottospazio perché se prendiamo il caso con $n=2$ $\begin{bmatrix}1\\x_2\end{bmatrix} + \begin{bmatrix}1\\y_2\end{bmatrix} = \begin{bmatrix}2\\x_1 + x_2\end{bmatrix}$ che non è un sottospazio.\\\\
Prendiamo ora $\Bigg \{\begin{bmatrix}t_1\\ t_2\end{bmatrix} \in \mathbb{R}^n \: : \: t_1 = t_2\Bigg\} \subset \mathbb{R}$ questo è un sottospazio perché:
se facciamo $\begin{bmatrix}x_1\\x_2\end{bmatrix} + \begin{bmatrix}x_2'\\x_1'\end{bmatrix} = \begin{bmatrix}x_1 \cdot x_2'\\x_2 + x_1'\end{bmatrix}$ e $\lambda \cdot\begin{bmatrix}x_1 \\ x_2\end{bmatrix} = \begin{bmatrix}\lambda x_1 \\ \lambda x_2\end{bmatrix}$ quindi è un sottospazio.
\end{example}

\begin{example}
Facciamo un esempio differente, prendiamo $\Bigg\{\begin{bmatrix}a & b \\ c & d\end{bmatrix} \in M_{2\times 2}(\mathbb{R})\: :\: a= 0 \Bigg\}\subset M_{2 \times 2}(\mathbb{R})$ è un sottospazio. Ma nel caso ci ci fosse stato $a=1$ non sarebbe stato un sottospazio, perché non sarebbe passato passato per (0,0).
\end{example}

\begin{example}
Facciamo alcuni esempi prendendo delle funzioni all'interno degli spazi vettoriali.
\begin{itemize}
    \item Dato $\{f \in \mathbb{R}[x] : deg(f) \leq d\} \subset \mathbb{R}[x]$ con $d$ fisso $\geq 0$. Questo è un sottospazio perché:
    \begin{itemize}
    	\item Se $deg(f_1) \leq d$, $deg(f_2) \leq d \Longrightarrow deg(f_1 + f_2) \leq d$
    	\item  Se $deg(f) \leq d \Longrightarrow deg(\lambda \cdot f) \leq d$, $\forall \lambda$
    \end{itemize}
    \item $\{f \in \mathbb{R}[x] : deg(f) = d\} \subset \mathbb{R}[x]$ non è un sottospazio per diverse ragioni:
    \begin{itemize}
    	\item Se $d>0$ allora $0 \notin W_d$
    	\item Se $d=2$ abbiamo $f = x^2 + 3 \in W$, $g = -x^2 + x + 1 \in W$ ma $f + g = x + 4 \notin W$.
    \end{itemize}
    \item $\{f \in \mathbb{R} \: : \: f(0) = d\} \subset \mathbb{R}[x]$ invece è un sottospazio perché $f(0) = 0$, $g(0)=0 \Longrightarrow (f + g)(0) = 0$ e anche $(\lambda f)(0) = 0$.
    \item $\{f \in \mathbb{R} \: : \: f(0) = 1\}$ non è un sottospazio perché non contiene 0.
    \item $\{f \in \mathbb{R} \: : \: f(2022) = 0\}$ è un sottospazio.
\end{itemize}
\end{example}

\begin{example}
Dati $a_1, a_2 \in \mathbb{R}$ fissi e dato il seguente insieme vettoriale\\\\
$\Bigg \{ \begin{bmatrix}x_1 \\ x_2\end{bmatrix}\in \mathbb{R}^2 \: : \: a_1x_1 + a_2x_2 = 0\Bigg\} \subset \mathbb{R}^n$ è un sottospazio. Perché preso
$\begin{cases}a_1x_1 + a_2x_2 \\a_1y_1 + a_2y_2 \end{cases}$\\\\
vediamo che la somma $a_1(x_1 + y_1) + a_2(x_2 + y_2) = 0$ ed anche il prodotto con $\lambda$ fa $a_1(\lambda\: x_1) + a_2(\lambda \: x_2)=0$.
\subsubsection{Generalizzazione}
Possiamo dire che, dato $a_1, a_2, \cdots, a_m \in \mathbb{R}$ fissi:\\\\
$\Bigg \{ \begin{bmatrix}x_1 \\ \vdots \\ x_m\end{bmatrix}\in \mathbb{R}^2 \: : \: a_1x_1 + a_2x_2 + \cdots + a_mx_m = 0\Bigg\} \subset \mathbb{R}^n$ è un sottospazio.\\\\Vediamo dunque che le soluzioni di un equazioni lineari omogenee a n variabili definiscono un sottospazio di $\mathbb{R}^n$. Possiamo generalizzare ulteriormente:\\\\
$\Bigg \{ \begin{bmatrix}x_1 \\ \vdots \\ x_m\end{bmatrix}\in \mathbb{R}^2 \: : \: \begin{array}{l}
    a_{11}x_1 + a_{12}x_2 + \cdots + a_{1m}x_m = 0\\
    a_{21}x_1 + a_{22}x_2 + \cdots + a_{2m}x_m = 0\\
    \cdots\\
    a_{n1}x_1 + a_{n2}x_2 + \cdots + a_{nm}x_m = 0\\
\end{array}\Bigg\} \subset \mathbb{R}^n$ Quindi è un sottospazio.\\\\
Dunque che la soluzione di un sistema di questioni lineare omogenee definisce un sottospazio $\mathbb{R}^m$.
\end{example}
\newpage
\subsection{Combinazioni lineari}
\begin{definition}[Combinazione lineare e banale]
Sia V uno spazio vettoriale e $v_1, v_2, \ldots, v_m$ vettori in $V$. Una \textbf{combinazione lineare} di $v_1, \ldots, v_m$ è una somma $\lambda v_1+ \lambda v_2 + \ldots + \lambda_m v_m \in V$, dove $\lambda_1, \lambda_2, \ldots, \lambda_m \in \mathbb{R}$. La combinazione lineare è detta \textbf{banale} se $\lambda_1 = \ldots = \lambda_m = 0$. In questo caso $\lambda_1 v_1 + \ldots + \lambda v_m = 0$.
\end{definition}

\noindent Nota che una combinazione lineare può essere $0$ ma non banale, per esempio:\\
$V = \mathbb{R}^2$, \hspace{.2cm} $v_1 = \begin{bmatrix}1\\1\end{bmatrix}$, $v_2 = \begin{bmatrix}2\\2\end{bmatrix}$, \hspace{.2cm}allora $-2v_1 + 1v_2 = 0$.

\begin{definition}[Sottospazio generato]
Siano $v_1, \ldots, v_m \in V$ vettori. Il \textbf{sottospazio generato} da $v_1, \ldots, v_m$ è $Span(v_1, v_2, \ldots, v_m) = \{\lambda_1 v_1 + \lambda_2 v_2 + \ldots + \lambda_n v_m : \lambda_{1}, \ldots, \lambda_m \in \mathbb{R}\}$. Questo rappresenta l'insieme delle combinazioni lineari.
\end{definition}

\begin{proposition}
$Span(v_1, \cdots, v_m) \subset V$ è un sottospazio.
\end{proposition}

\begin{demostration}
Bisogna verificare che $v, w \in span \Longrightarrow v + w \in span$ e $\lambda v \in span \: \forall \: \lambda$.
\end{demostration}

\begin{example}
Prendiamo $\mathbb{R}^2 = span\Big\{\begin{bmatrix}0\\1\end{bmatrix},\begin{bmatrix}1\\0\end{bmatrix}\Big\}$. $span\Big\{\begin{bmatrix}0\\1\end{bmatrix}\Big\}$ e $span\Big\{\begin{bmatrix}1\\0\end{bmatrix}\Big\}$ sono due rette.\\
Se facciamo $span\Big\{\lambda_1 \begin{bmatrix}1\\0\end{bmatrix} + \lambda_2 \begin{bmatrix}0\\1\end{bmatrix}\Big\} = \Big\{\begin{bmatrix}\lambda_1\\\lambda_2\end{bmatrix}\Big\} = \mathbb{R}^2$
\end{example}

\begin{example}
Sia $W = \Big\{\begin{bmatrix}x_1\\x_2\\x_3\end{bmatrix} \in \mathbb{R}^3 \: : \: x_1 = 0\Big\}$ abbiamo allora che:
$W = span\Big\{\begin{bmatrix}0\\1\\0\end{bmatrix}, \begin{bmatrix}0\\0\\1\end{bmatrix} \Big\} \subset \mathbb{R}^3$ quindi: $\Big \{\lambda_1 \begin{bmatrix}0\\1\\0\end{bmatrix} + \lambda_2 \begin{bmatrix}0\\0\\1\end{bmatrix}\} = \{\begin{bmatrix}0\\\lambda_1 \\ \lambda_2\end{bmatrix}\} = \{\begin{bmatrix}x_1\\x_2\\x_3\end{bmatrix}\: : \: x_1=0\}$ è un sottospazio di $\mathbb{R}^3$ ma non è uguale a $\mathbb{R}^3$, ma è più piccolo essendo un piano attraverso l'origine.
\end{example}

\newpage
\subsection{Vettori lineamenti indipendenti}
\begin{definition}
I vettori $v_1, v_2, \cdots, v_m \in V$ sono \textbf{linearmente indipendenti} se l'unico caso in cui $\lambda_1 v_1 + \lambda_2 v_2 + \cdots + \lambda_m v_m = 0$ si ha quando $\lambda_1 = \cdots = \lambda_m = 0$.
\end{definition}
\noindent Questo vuol dire che se una combinazione lineare dei V è uguale a zero $\Longrightarrow$ la combinazione è banale.\\
Se $v_1, \cdots, v_n$ non sono indipendenti allora sono \textbf{linearmente dipendenti}.\\
$v_1, v_2, \cdots, v_m$ sono linearmente dipendenti $\Longleftrightarrow \: \exists \lambda_1, \lambda_2, \cdots, \lambda_m \in \mathbb{R}$ non tutti uguali a 0 tale che $\lambda_1 v_1 + \lambda v_2 + \cdots + \lambda_2 v_2 = 0$.

\begin{proposition}
$v_1, v_2, \cdots, v_m$ sono linearmente dipendenti $\Longleftrightarrow \: \exists \: 1 \leq i \leq n$ tale che $v_i$ è combinazione lineare dei $v_j$ per $j\neq i$.
\end{proposition}

\begin{demostration}
Se $v_1, \cdots, v_m$ sono dipendenti allora $\exists \: \lambda_1, \cdots, \lambda_m$ non tutti ugualia 0 tale che $\lambda_1 v_1 + \cdots + \lambda_n v_m = 0$. $\exists \: i \: : \: \lambda_i \neq 0$ che possiamo usare come dividendo: $\frac{\lambda_1}{\lambda_1} v_1 + \cdots + 1v_i + \cdots + \frac{\lambda_n}{\lambda_i} v_m = 0$
(mando tutti a destra) $v_i = -\frac{\lambda_1}{\lambda_i} v_1 + \cdots - \frac{\lambda_{i-1}}{\lambda_i}v_{i-1} - \frac{\lambda_{i+1}}{\lambda_i}v_{i+1} + -\frac{\lambda_n}{\lambda_i} v_m$.
Se $v_i = \lambda_1 v_1 + \cdots + \lambda_{i-1}v_{i-1} + \lambda_{i+1}v_{i+1} + \cdots + \lambda_m v_m$ allora $ \lambda_1 v_1 + \cdots + \lambda_{i-1}v_{i-1} - v_i + \lambda_{i+1}v_{i+1} + \cdots + \lambda_m v_m = 0$
\end{demostration}
\hspace{-15pt}In pratica per vedere se m vettori $v_1, \cdots, v_m \in \mathbb{R}^n$ sono linearmente indipendenti prendiamo innanzitutto m vettori:\\
$v_1 = \begin{bmatrix}a_{11} \\ a_{21} \\ \vdots \\ a_{n1}\end{bmatrix}$, $v_2 = \begin{bmatrix}a_{12}\\a_{22}\\\vdots\\ a_{n2}\end{bmatrix}$, $\cdots, v_m = \begin{bmatrix}a_{1m}\\a_{2m}\\\vdots\\ a_{nm}\end{bmatrix}$, questi sono vettori di $\mathbb{R}^m$.\\\\
Questi vettori sono linearmente indipendente, quindi l'equazione $\lambda_1v_1 + \lambda_2 v_2 + \cdots + \lambda_mv_m = 0$ vale, se e solo se ($\lambda_1, \cdots, \lambda_m$) è soluzione del sistema:
\[
\begin{array}{l}
     a_{11}x_1 + a_{12}x_2 + \cdots + a_{1m}x_m = 0\\
     a_{21}x_1 + a_{22}x_2 + \cdots + a_{2m}x_m = 0\\
     \vdots\\
     a_{n1}x_1 + a_{n2}x_2 + \cdots + a_{nm}x_m = 0\\
\end{array}
\hspace{.5cm}
\parbox{6cm}{Quindi $v_1, \cdots, v_m$ sono lin. indipendenti $\Longrightarrow$ il sistema sopra ammette solo la soluzione banale (0,$\ldots$,0)}
\]
\subsection{Interpretazioni geometrica}
Facciamo un interpretazione geometrica di quello visto sopra ponendo $n=2$. $V = \mathbb{R}^2$, $v_1, v_2 \in \mathbb{R}^2$ sono linearmente dipendenti $v_1, v_2 \neq 0$ oppure $\exists \: \lambda_1, \lambda_2 \: : \: \lambda_1 v_1 + \lambda_2 v_2 = 0$. Ad esempio $\lambda \neq 0$, $v_1 = -\frac{\lambda_1}{\lambda_2}v_2$ e se $v_2 = \begin{bmatrix}1\\0\end{bmatrix} \Longrightarrow v_1 = \begin{bmatrix}-\lambda_2\setminus\lambda_1\\0\end{bmatrix}$ e corrisponde un punto della retta $x_2 = 0$. \\
In generale se $v_1 = \begin{bmatrix}x_1\\x_2\end{bmatrix}$, $v_2$ deve essere $\begin{bmatrix}\lambda x_1 \\ \lambda x_2\end{bmatrix}$ quindi $v_1, v_2$ sono lin. dipendenti $\Longleftrightarrow$ i punti corrispondenti sono sulla tessa retta attraverso (0,0).
\begin{example}
Si decida se i seguenti vettori di $\mathbb{R}^2$ sono linearmente indipendenti:\\
$v_1 = \begin{bmatrix}1\\2\\3\end{bmatrix}$\hspace{.5cm}$v_2 = \begin{bmatrix}1\\0\\1\end{bmatrix}$\hspace{.5cm}$v_3 = \begin{bmatrix}0\\0\\1\end{bmatrix}$\hspace{.5cm}$v_4 = \begin{bmatrix}2\\2\\4\end{bmatrix}$.\\\\
Per faro dobbiamo cercare le soluzioni del sistema lineare omogeneo con la matrice associata.
\[
\begin{array}{l}
    x_1 + x_2 + 2x_4 = 0\\
    2x_1 + 2x_4 = 0\\
    3x_1 + x_2 + x_3 + 4x_4
\end{array}
\Rightarrow
\begin{bmatrix}
1 & 1 & 0 & 2\\
2 & 0 & 0 & 2\\
3 & 1 & 1 & 4
\end{bmatrix}
\begin{array}{l}
\text{Algoritmo di gauss}\\
R_2 = R_2 - 2R_1\\
R_3 = R_3 - 3R_1
\end{array}
\Rightarrow
\begin{bmatrix}
1 & 1 & 0 & 2\\
0 & -2 & 0 & -2\\
0 & -2 & 1 & -2
\end{bmatrix}
R_3 = R_3 - R_2
\]
\[
\Rightarrow
\begin{bmatrix}
1 & 1 & 0 & 2\\
0 & -2 & 0 & -2\\
0 & 0 & 1 & 0
\end{bmatrix}
\parbox{6cm}{In questo caso ci sono 3 pivot, una variabile libera $\Longrightarrow \infty$ soluzioni}
\]
Quindi il sistema ammette soluzioni non banali $\Longrightarrow$ i vettori sono lim. dipendenti.\\ 
Se si guardasse solo $v_1, v_2, v_3$ quello che risulterebbe sarebbe una matrice $3 \times 3$ con 3 pivot, in questo caso allora ci sarebbe solo la soluzione banale ed allora $v_1, v_2, v_3$ sarebbero lin. indipendenti.
\end{example}

\begin{proposition}
Se $v_1, v_2, \cdots, v_n \in V$ sono vettori tali che $v_n$ è combinazione lineare di allora: $span(v_1, v_2, ..., v_n) = span(v_1, v_2, \cdots, v_{n-1})$.
\end{proposition}

\subsection{Base di un sistema lineare}
\begin{definition}
Un sistema $v_i, \cdots, v_n$ di vettori è una \textbf{base} di V se i vettori $v_i, \cdots, v_n$:
\begin{itemize}
    \item Sono linearmente indipendenti.
    \item Lo $span(v_1, v_2, \cdots, v_n) = V$
\end{itemize}
\end{definition}

\begin{corollary}
Se $span(v_1, \cdots, v_n) ) C$ si può scegliere una base di V fra i $v_1, \cdots, v_n$.
\end{corollary}

\begin{example}
Vogliamo trovare la base standard di $\mathbb{R}^n$.\\
$e_1 = \begin{bmatrix}1\\0\\\vdots\\0\end{bmatrix}, e_2 = \begin{bmatrix}0\\1\\0\\\vdots\\0\end{bmatrix}, \cdots, e_n = \begin{bmatrix}0\\0\\\vdots\\0\\1\end{bmatrix}$
Possiamo osservare che $\begin{bmatrix}\lambda_1\\\lambda_2\\\vdots\\\lambda_n\end{bmatrix} = \lambda_1 e_1 + \lambda_2 \_2 + \cdots + \lambda_ne_n$,\\\\
dunque $span(e_1, \cdots, e_n) = \mathbb{R}^n$ e $\lambda_1e_1 + \cdots + \lambda_ne_n = 0$ se e solo se $\lambda_1= \cdots = \lambda_n = 0$. Ma questa non è l'unica base, c'è ne sono tante, ad esempio se prendiamo $n=2$.\\\\
$\begin{bmatrix}1\\0\end{bmatrix}, \begin{bmatrix}1\\1\end{bmatrix}$ è una base perché $\lambda_1\begin{bmatrix}1\\0\end{bmatrix} + \lambda_2\begin{bmatrix}1\\1\end{bmatrix} = \begin{bmatrix}\lambda_1 + \lambda_2\\\lambda_2\end{bmatrix} = \begin{bmatrix}0\\0\end{bmatrix}$ se e solo se $\lambda_1 + \lambda_2 = \lambda_2 = 0 \Longleftrightarrow \lambda_1 = \lambda_2 = 0$
\end{example}

\begin{example}
Troviamo la base standard di $M_{2\times2}(\mathbb{R})$.
\[\begin{bmatrix}1&0\\0&0\end{bmatrix}, \hspace{.5cm} \begin{bmatrix}0&1\\0&0\end{bmatrix}, \hspace{.5cm} \begin{bmatrix}0&0\\1&0\end{bmatrix}, \hspace{.5cm} \begin{bmatrix}0&0\\0&1\end{bmatrix}\]
Si applica lo stesso ragionamento visto sopra con $\mathbb{R}^n$.
\end{example}

\begin{example}
Base standard di $\mathbb{R}[x]_{\leq d} = \{f \in \mathbb{R}[x] \::\: deg(x) \leq d\}$ sarebbe $1, x, x^2, \cdots, x^s$. Infatti, $a_dx^d + a_{d-1}x^{d-1} + \cdots + a_0 = a_0 \cdot 1 + a_1 \cdot x + \cdots + a_d \cdot x^d$ è il sistema indipendente
\end{example}
\begin{example}
Prendiamo $\mathbb{R}[x]$ che non ammette di base finita. Infatti $\nexists f_1, \cdots, f_n \in \mathbb{R}[x] \::\: span(f_1, \cdots, f_n = \mathbb{R}[x]$ perché se $f \in span(f_1, \cdots, f_n)$ allora $deg(f) \leq max(deg(f_1), \cdots, deg(f_n))$. (Comunque è vero: $span(1,x,x^2, x^2, \cdots) = \mathbb{R}[x]$ e ogni sottoinsieme finito di $1,x, x^2, \cdots$ è lin. indipendente)
\end{example}

\begin{proposition}
	Sia $v_1, \ldots, v_n$ una base di V, e $v \in V$ un vettore. Allora $\nexists  \alpha_1, \ldots, \alpha_n \in \mathbb{R} \vert v = \alpha_1\cdot v_1 + \alpha_2\cdot v_2 + \ldots + \alpha_n\cdot v_n$. (Ogni vettore si scrive in modo unico come combinazione lineare degli elementi della base)
\end{proposition}

\begin{demostration}
	Scriviamo come $V = span(v_1, \ldots, v_n)$, l'esistenza degli $\alpha_i$ è chiaro. Se adesso $v = \alpha_1\cdot v_i + \ldots + \alpha_n\cdot v_n = \beta_1\cdot v_1 + \ldots + \beta_n\cdot v_n$ allora $0 = (\alpha_1 - \beta_1)\cdot v_1 + \ldots + (\alpha_n - \beta)\cdot v_n$ allora $\alpha_1 = \beta_1, \ldots, \alpha_1 = \beta_n$ perché i $v_i$ sono lin. indipendenti.
\end{demostration}

\subsection{Dimensione spazio vettoriale}
La dimensione di uno spazio vettoriale $V$ sarà definita come il numero degli elementi di una sua base. Questo numero sarà lo stesso per ogni base.
\begin{proposition}
	Sia V uno spazio vettoriale che ammette una base $e_1, e_2, \ldots, e_n$. Se $v_1, v_2, \ldots, v_n \in V$ e $r > n \Longrightarrow v_1, v_2, \ldots, v_n$ sono linearmente dipendenti.
\end{proposition}

\begin{demostration}
	Per $n=2$, la prima osservazione è che se la proposizione vale per $r =2$ vale per ogni $r > 2$. Infatti se $\lambda_1v_2 + \lambda_2v_2 + \lambda_3v_3 = 0$ è una combinazione lineare non banale allora $\lambda_1v_2 + \lambda_2v_2 + \lambda_3v_3 + 0 \cdot v_4 + 0 \cdot v_5 + \ldots + 0 \cdot v_r = 0$ è una combinazione non banale (perché $\lambda_1, \lambda_2$ o $\lambda_3$ è diverso da 0). Quindi siano $n=2, r=3$ e $e_1, e_2$ una base di V. Come $V=span(e_1,e_2)$, $v_1, v_2, v_3 \in span(e_1, e_2)$. Quindi:\\\\
	$\begin{array}{l}
		v_1 = a_{11}e_1 + a_{12}e_2\\
		v_2 = a_{21}e_1 + a_{22}e_2\\
		v_3 = a_{31}e_1 + a_{32}e_2
	\end{array}
	\hspace{.5cm}
	\parbox{8cm}{Dobbiamo trovare $\lambda_1, \lambda_2, \lambda_3$ non tutti $=0$ tali che $\lambda_1v_1 + \lambda_2v_2 + \lambda_3v_3 = 0$. Facciamo la sostituzione con il sistema a fianco.}$\\\\
	$\lambda_1(a_{11}e_1 + a_{12}e_2) + \lambda_2(a_{21}e_1+a_{22}e_2) + \lambda_3(a_{31}e_1 + a_{32}e_2) = 0$ che diventa $(\lambda_1a_{11} + \lambda_1a_{12})e_2 + (\lambda_2a_{21} + \lambda_2a_{22})e_2 + (\lambda_3a_{31} + \lambda_3a_{32})e_2 = 0$. Ma $e_1, e_2$ sono linearmente indipendenti, quindi:\\\\
	$\begin{array}{l}
		\lambda_1a_{11} + \lambda_2a_{21} + \lambda_3 a_{31} = 0\\
		\lambda_1a_{12} + \lambda_2a_{22} + \lambda_3 a_{32} = 0
	\end{array}
	\hspace{.3cm}
	\parbox{5cm}{Questo è un sistema omogeneo di equazioni per $\lambda_1, \lambda_2, \lambda_3$ con matrice di coefficienti}\hspace{.3cm}
	\begin{bmatrix}
		a_{11}&a_{21}&a_{31}\\
		a_{12}&a_{22}&a_{32}
	\end{bmatrix}$\\
	Se facciamo l'algoritmo di Gauss, ottengo un numero di pivot minore o uguale a 2 (perché ci sono solo due righe), allora ci sarà $\geq 1$ colonne senza pivot ed allora il sistema avrà $\infty$ soluzioni ed allora ci sarà una soluzione non banale $\lambda_1, \lambda_2, \lambda_2$. Ma se il sistema sopra ha una soluzione non banale ($\lambda_1, \lambda_2, \lambda_2$) allora anche $\lambda_1v_1 + \lambda_2 v_2 + \lambda_3 v_3 = 0$ sarà una combinazione non banale e quindi ci siamo.\\\\
	La dimostrazione per $n,r$ generale è la stessa, infatti alla fine ottengo un sistema lineare di n equazioni in $r > n$ variabili ed allora c'è sempre una soluzione non banale. $\blacksquare$
\end{demostration}

\begin{corollary}
	Sia $e_1,\ldots, e_n$ una base di $V$. Se $v_1,\ldots,v_n$ è un sistema linearmente indipendente $\Longrightarrow$ anche $v_1, \ldots, v_n$ è una base di $V$.
\end{corollary}

\begin{demostration}
	Dobbiamo dimostrare che $Span(v1,\ldots,v_n)=V$.\\
	Sia $v \in V$. Per la proposizione 3.9.1, il sistema di $n+1$ vettori $v_1,\ldots,v_n,v$ è linearmente dipendente. Quindi $\exists \lambda_{1},\ldots,\lambda_{n+1}$ non tutti $=0$ tali che $\lambda_{1} \cdot v_1 + \ldots + \lambda_n \cdot v_n + \lambda_{n+1} \cdot v = 0$. (*)\\
	Se $\lambda_{n+1}=0$ allora $\lambda_1 \cdot v_1 + \ldots + \lambda_n \cdot v_n = 0 \Longrightarrow \lambda_{1} = \ldots = \lambda_n = 0$ perché $v_1,\ldots,v_n$ sono \textbf{indipendenti} $\Longrightarrow $ con $\lambda_1, \ldots, \lambda_{n+1}$ non tutti $=0$.\\
	Quindi $\lambda_{n+1} \neq 0$. Ma allora  (*) mi dà $v=-\frac{\lambda_{1}}{\lambda{n+1}} + \ldots + (-\frac{\lambda_n}{\lambda_{n+1}}) \cdot v_n \Longrightarrow v \in Span(v_1, \ldots, v_n)$. \\
	Questo vale per ogni $v \in V \Longrightarrow V = Span(v_1, \ldots, v_n)$.
\end{demostration}

\begin{corollary}
	Se $v_1, \ldots, v_r$ ed $e_1, \ldots, e_n$ sono due basi di V allora $r=n$.
\end{corollary}

\begin{demostration}
	Se $r>n$, $v_1, \ldots, v_r$ è linearmente dipendente se $e_1, \ldots, e_n$ è una base (proposizione 3.8.1), quindi $r\leq n$. Se $r<n$ e $v_1, \ldots, v_n$ è una base allora $e_1, \ldots, e_n$ è linearmente dipendete e questa è una contraddizione. Dunque $r=n$. $\blacksquare$
\end{demostration}

\begin{definition}[Dimensione di un V]
Se V ammette una base $e_1, \ldots, e_n$ n è la dimensione di V. La dimensione di V si indica come $dim V = n$.
\end{definition}

\begin{corollary}
Se la dimensione di V è n e $v_1, \ldots, v_m$ sono vettori lin. indipendenti con $m<n \Longrightarrow \:\exists\: w_{m+1}, w_{m+2}, \cdots, w_n \vert v_1, \ldots, v_n, w_{m+1}, \cdots, w_n$ sono una base di V. 
\end{corollary}

\begin{demostration}
Dobbiamo verificare che $span(v_1, \ldots, v_m) = V$. Sappiamo che $span(v_1, \ldots, v_m)$ non può essere V, allora $\exists v \in V$ tale che $v \notin span(v_1, \ldots, v_m)$. Ma allora basta vedere che $v_1, v_2, \ldots, v_m$ sono linearmente indipendenti ed allora abbiamo una contraddizione.\\
Sia $\lambda_1 \cdot v_1 + \lambda_2 \cdot v_2 + \ldots + \lambda_m \cdot v_m + \lambda_0 \cdot v_0 = 0$ una combinazione lineare se $\lambda=0 \Longrightarrow \lambda_1 = \lambda_2 = \cdots = \lambda_n = 0$ perché $v_1, v_n \ldots, v_m$ lin. indipendenti allora $v_1, v_2, \ldots, v_m$ lin. indipendenti. Se prendiamo un $\lambda_{m+1}\neq 0$ possiamo fare $-\frac{\lambda_1}{\lambda \cdot }v_1 - \frac{\lambda_2}{\lambda} \cdot v_2 - \ldots - \frac{\lambda_n}{\lambda} \cdot v_n = V \in span(v_1, \ldots, v_n)$, ma questa è una contraddizione $v \notin span$. $\blacksquare$
\end{demostration}

\begin{example}
Decidiamo se $\begin{bmatrix}1\\0\\1\end{bmatrix}, \begin{bmatrix}2\\0\\0\end{bmatrix}, \begin{bmatrix}3\\2\\1\end{bmatrix}$ sono una base $\mathbb{R}^3$. \\Per faro dobbiamo solo decidere se sono indipendenti o meno, e per farlo usiamo Gauss.
\[
\begin{bmatrix}
1 & 2 & 3\\
0 & 0 & 1\\
1 & 0 & 1
\end{bmatrix}
R_3 - R_2 \Rightarrow
\begin{bmatrix}
1 & 2 & 3\\
0 & 0 & 1\\
0 & -2 & 1
\end{bmatrix}
\xRightarrow[]{\text{Scambio} R_2, R_3}
\begin{bmatrix}
1 & 2 & 3\\
0 & -2 & 1\\
0 & 0 & 1
\end{bmatrix}
\]
Vediamo dunque che ci sono 3 pivot e quindi i vettori sono lineamenti indipendenti e di conseguenza abbiamo una base.
\end{example}

\begin{example}
Prendiamo $V = \mathbb{R}[x]_{\leq 2}$. Abbiamo visto che $1, x, x^2$ sono una base questo vuol dire allora che $\mathbb{R}[x]_{< 2} = 2$. Vediamo se $1, 1+x, (1+x)^2$ forma una base. Per fare questo bisogna vedere se sono linearmente indipendenti. Supponiamo che:
$\lambda_11 + \lambda_2(1+x) + \lambda_3(1+x)^2 = 0 \Rightarrow \lambda_1 + \lambda_2 + \lambda_2x + \lambda_3x^2 + 2x\lambda_3 + \lambda_3 = 0 \Rightarrow (\lambda_1 + \lambda_2 + \lambda_3)1 + (\lambda_2 +\lambda_3)x + \lambda_3 + x^2 = 0$.\\
Visto che $1, x, x^2$ sono lin. indipendenti allora $\lambda_1 + \lambda_2 + \lambda_3 = 0$, $\lambda_2 + 2\lambda_3 = 0$, $\lambda_3 = 0$ sostituendo viene che $\lambda_2 = 0, \lambda_1 = 0$ allora $1, 1+x, (1+x)^2$ sono lin. indipendenti ed allora sono una base di $\mathbb{R}(x)_{<2}$.
\end{example}

\begin{example}
$W' = \{f \in \mathbb{R}[x]_{\leq 2} \::\: f(1) = f(2) = 0\}$, $W' \subset W$ (per esempio visto prima) e $dim(W) = 3 \Longrightarrow dim(W') \leq 2$. Ci sono due vettori indipendenti in $W$, $(x-1)(x-2), (x-1)(x-2)^2$ di gradi diversi $\Longrightarrow dim(W') = 2$.\\
$W' \subset W$ in base $W'$ è $(x-1)(x-2), (x-1)(x-2)^2$ e completiamo in una base di $W$ $(x-1), (x-1)(x-2), (x-1)(x-2)^2 \in W \setminus W'$ è una base di $W$, perché sono indipendenti:\\
$W \subset V$, $dim(V) = 4$, $dim(W) = 4$. $1, x-1, (x-1)(x-2), (x-1)(x-2)^2$ è una base di $V \setminus W$.
\end{example}

\begin{example}
$V = M_{3\times3}(\mathbb{R})$, la dimensione è $dim(V) = 9$. Mentre $W \subset \{ A \in M_{3\times 3}(\mathbb{R}): \text{ la somma di ogni riga è 0}\}$. Supponiamo $dim(W) < 9$ elementi linearmente indipendenti di W.
\[
\begin{bmatrix}
1 & -1 & 0\\
0 & 0 & 0\\
0 & 0 & 0
\end{bmatrix}
\begin{bmatrix}
1 & 0 & -1\\
0 & 0 & 0\\
0 & 0 & 0
\end{bmatrix}
\begin{bmatrix}
0 & 0 & 0\\
1 & -1 & 0\\
0 & 0 & 0
\end{bmatrix}
\begin{bmatrix}
0 & 0 & 0\\
1 & 0 & -1\\
0 & 0 & 0
\end{bmatrix}
\begin{bmatrix}
0 & 0 & 0\\
0 & 0 & 0\\
1 & -1 & 0
\end{bmatrix}
\begin{bmatrix}
0 & 0 & 0\\
0 & 0 & 0\\
1 & 0 & -1
\end{bmatrix}
\]
Essendo linearmente indipendente allora $dim(W) \geq 6$. Proviamo a dire che $dim(W) = 6$. L'idea:\\
$W_1 = \{A \in M_{3 \times 3}(\mathbb{R}) : \text{ la somma della prima riga = 0}\}$,\\ $W_1 = \{A \in M_{3 \times 3}(\mathbb{R}) : \text{ la somma della prima e della seconda riga = 0}\}$.\\
$W \subset W_2 \subset W_1 \subset M_{3\times3}(\mathbb{R})$, sapendo $dim(M_{3\times3}(\mathbb{R})) = 9, dim(W_1) = 6, dim(W_2) = 7, dim(W) = 7$.
\end{example}

\begin{example}
Sappiamo già: $v_1 = \begin{bmatrix}1\\2\\3\end{bmatrix}, v_2 = \begin{bmatrix}1\\0\\1\end{bmatrix}, v_3 = \begin{bmatrix}0\\0\\1\end{bmatrix}$ che sono una base di $\mathbb{R}^3$.
Troviamo le coordinate di $\begin{bmatrix}0\\2\\1\end{bmatrix}$ rispetto a queste basi.
$\alpha_1\begin{bmatrix}1\\2\\3\end{bmatrix} + \alpha_2\begin{bmatrix}1\\0\\1\end{bmatrix} + \alpha_3\begin{bmatrix}0\\0\\1\end{bmatrix} = \begin{bmatrix}0\\2\\1\end{bmatrix}$. Ed usiamo Gauss-Jordan.
\[
\begin{bmatrix}
1 & 1 & 0 & 0\\
2 & 0 & 0 & 2\\
3 & 1 & 1 & 1
\end{bmatrix}
\begin{array}{l}
    R_2 - 2R_1\\
    R_3 - 3R_1
\end{array}
\Rightarrow
\begin{bmatrix}
1 & 1 & 0 & 0\\
0 & -2 & 0 & 2\\
0 & -2 & 1 & 1
\end{bmatrix}
\begin{array}{l}
    R_3 - R_2\\
    R_2 = \frac{1}{2}R_1
\end{array}
\Rightarrow
\begin{bmatrix}
1 & 1 & 0 & 0\\
0 & 1 & 0 & -1\\
0 & 0 & 1 & -1
\end{bmatrix}
R_1 - R_2
\Rightarrow
\begin{bmatrix}
1 & 0 & 0 & 1\\
0 & 1 & 0 & -1\\
0 & 0 & 1 & -1
\end{bmatrix}
\]
Abbiamo quindi $\alpha_1 = 1, \alpha_2 = -1, \alpha_3 = -1$.
\end{example}

\begin{example}
Vedere se $\begin{bmatrix}1\\0\\1\end{bmatrix}, \begin{bmatrix}2\\0\\0\end{bmatrix}, \begin{bmatrix}3\\1\\1\end{bmatrix}$ è una base $\mathbb{R}^3$
e calcolare coordinate $\begin{bmatrix}2\\0\\4\end{bmatrix}$ rispetto a base. \\
Per calcolare le coordinate dobbiamo risolvere il sistema lineare che si crea con le 3 matrici:
\[
\begin{array}{l}
x_1 + 2x_2 + 3x_3 = 2\\
x_3 = 0\\
x_1 + x_3 = 4
\end{array}
\hspace{.3cm}
\begin{array}{r}
x_3 = 0\\
x_1 = 4\\
x_2 = -1
\end{array}
\hspace{.5cm}
\parbox{8cm}{Usiamo Gauss per verificare l'indipendenza perché se questi vettori sono indipendenti allora i coefficienti $4, -1, 0$ saranno le coordinate.}
\]
\[
\begin{bmatrix}
1 & 2 & 3\\
0 & 0 & 1\\
1 & 0 & 1
\end{bmatrix}
R_3 - R_1
\begin{bmatrix}
1 & 2 & 3\\
0 & 0 & 1\\
0 & -2 & -2
\end{bmatrix}
\text{Inverto } R_2, R_3
\begin{bmatrix}
1 & 2 & 3\\
0 & -2 & -2\\
0 & 0 & 1
\end{bmatrix}
\]
Torna $x_1 = 4, x_2 = -1, x_3 = 0$, inoltre abbiamo una forma a scalini con 3 pivot allora i vettori sono indipendenti e quindi sono una base.
\end{example}

\begin{proposition}
Se abbiamo uno spazio vettoriale $dim(V) = n$ ed abbiamo $v_1, v_2, \cdots, v_m$ vettori linearmente indipendenti di V con $m < n$ allora $\exists \: w_{m+1}, w_{m+2}, \cdots, w_n \::\: v_1, \cdots, v_m, w_{m+1}, \cdots, w_n$ sono una base di V.
\end{proposition}

\begin{demostration}
$Span(v_1 \ldots, v_m)$ non può essere V, perché se $span(v_1, \cdots, v_m) = V \Longrightarrow v_1, \cdots, v_m$ è una base ma $m<n$ è $dim(V) = n$ e questa è una contraddizione. Quindi $span(v_1,\cdots,v_m) \neq V \Longrightarrow \exists w_{m+1} \in V \::\: w_{m+1} \notin span(v_1, \cdots, v_m)$. Ma allora $v_1, \cdots, v_{m}, w_{m+1}$ sono linearmente indipendenti tale che se $\lambda_1v_1 + \cdots + \lambda_{m}v_m + \lambda_{m+1}w_{m+1} = 0$, $\lambda_{m+1} = 0 $ allora $\lambda_1 = \cdots = \lambda_m  = 0$. Se $\lambda_{m+1} \neq 0$ allora $v_{m+1} = (-\frac{\lambda_1}{\lambda_{m+1}})v_1 + \cdots + (\frac{-\lambda_m}{\lambda_{m+1}})w_m \in span(w_1, \cdots, v_m)$ che è una contraddizione.
\end{demostration}

\hspace{-15pt}Per ricapitolare se la $dim(V) = n$ e $v_1, \cdots, v_n$ sono vettori di V possiamo dire che:
\begin{itemize}
    \item Se $m > n$ allora i vettori sono linearmente dipendenti.
    \item Se $m = n$ e i vettori sono indipendenti allora si forma una base.
    \item Se $m < n$ e i vettori sono indipendenti allora si completa in una base di V.
\end{itemize} 

\begin{example}
Facciamo un esercizio che sarà suddiviso in due parti.
\begin{itemize}
    \item Decidiamo se i vettori $\begin{bmatrix}1\\-1\\1\end{bmatrix}, \begin{bmatrix}-1\\1\\-1\end{bmatrix}, \begin{bmatrix}-1\\-1\\1\end{bmatrix}$ sono una base di $\mathbb{R}^3$. $dim(\mathbb{R}^3) \Longrightarrow$ se sono indipendenti sono allora una base. Usiamo gauss.
    \[
    \begin{bmatrix}
    1 & -1 & -1\\
    -1 & 1 & -1 \\
    1 & -1 & 1
    \end{bmatrix}
    \begin{array}{l}
        R_2 + R_1\\
        R_3 - R_1
    \end{array}
    \Rightarrow
    \begin{bmatrix}
    1 & -1 & -1\\
    0 & 0 & -2 \\
    0 & 0 & 2
    \end{bmatrix}
    R_2 + R_3
    \begin{bmatrix}
    1 & -1 & -1\\
    0 & 0 & -2 \\
    0 & 0 & 0
    \end{bmatrix}
    \]
    Risulta avere 2 pivot e quindi i vettori sono dipendenti. I pivot però sono delle colonne 1 e 3 e quindi se escludiamo la colonna centrale abbiamo come risultato due vettori indipendenti che chiamiamo $v_1, v_2$.
    \item Ora come secondo punto dobbiamo completare $v_1, v_2$ in una base di $\mathbb{R}^3$. Per fare questo dobbiamo trovare un terzo vettore non contenente in $span(v_1, v_2)$.\\
    L'idea qui è che so che $v_1 = \begin{bmatrix}1\\0\\0\end{bmatrix}, e_2 = \begin{bmatrix}0\\1\\0\end{bmatrix}, e_3 = \begin{bmatrix}0\\0\\1\end{bmatrix}$ è la base standard. So anche che almeno uno di questi 3 vettori non è contenuto in $span(v_1, v_2)$ perché se $e_1, e_2, e_3 \in span(v_1, v_2) \Longrightarrow span(e_1, e_2, e_3) \subset span(v_1, v_2)$ ma $span(e_1, e_2, e_3) = \mathbb{R}^3$ e quindi abbiamo una contraddizione. A questo punto devo trovare quale dei 3 vettori non è in $span(v_1, v_2)$. Lo facciamo provando i vari vettori e trovano quello che utilizzando Guass faccia venire 3 pivots.
    \[
    \begin{bmatrix}
        1 & -1 & 0\\
        -1 & -1 & 0\\
        1 & 1 & 1
    \end{bmatrix}
    \begin{array}{l}
        R_2 + R_1\\
        R_3 - R_1
    \end{array}
    \Rightarrow
    \begin{bmatrix}
        1 & -1 & 1\\
        0 & -2 & 0\\
        0 & 2 & 1
    \end{bmatrix}
    R_3 - 2R_2
    \Rightarrow
    \begin{bmatrix}
        1 & -1 & 1\\
        0 & -2 & 1\\
        0 & 0 & 1
    \end{bmatrix}
    \]
    Il risultato in questo caso è 3 pivot quindi i vettori sono linearmente indipendenti e quindi è una base di $\mathbb{R}^3$.
\end{itemize}
\end{example}

\begin{proposition}
Sia $W \subset V$ un sottospazio. Allora:
\begin{enumerate}
    \item Abbiamo che $dim(W) \leq dim(v)$.
    \item Se $W \neq V$ allora $dim(W) < dim(v)$.
\end{enumerate}
\end{proposition}

\begin{demostration}
Per dimostrare questa proposizione bisogna andare a dimostrare i due punti separatamente.
\begin{enumerate}
    \item Se $r = dim(W)$ e $w_1, \ldots, w_r$ è una base di W allora se $r > n$ per una proposizione vista precedentemente $w_1, \ldots, w_r$ sarebbero linearmente dipendenti e questa è una contraddizione quindi $r\leq n$.
    \item Se $r = n$, $w_1, \ldots, w_r$ sono $n = r$ vettori lineamenti indipendenti di V ed allora sono una base di V quindi $span(w_1, \ldots, w_r) = V \Longrightarrow V = W$.
\end{enumerate}
\end{demostration}

\begin{example}
Sia $V = M_{2 \times 2}(\mathbb{R}), W = \{\begin{bmatrix}a & b \\ c & d\end{bmatrix} \in M_{2 \times 2}(\mathbb{R}) \::\: b = c\}$ (questa è definita anche matrice simmetrica).\\
Si calcoli da dimensione di W, $dim(W)$. Partiamo dal fatto che la $dim(M_{2 \times 2}(\mathbb{R})) = 4$ (basi standard). Mentre $V \neq M_{2 \times 2}(\mathbb{R})  \Longrightarrow dim(V) \leq 3$. Vediamo però che:
\[
\begin{bmatrix}1 & 0 \\ 0 & 0\end{bmatrix}
\begin{bmatrix}0 & 0 \\ 0 & 1\end{bmatrix}
\begin{bmatrix}0 & 1 \\ 1 & 0\end{bmatrix}
\hspace{.3cm}
\parbox{6cm}{Sono linearmente indipendenti quindi devono essere una base di W e quindi dim(W) = 3}
\]
\end{example}

\begin{example}
Sia $W = \{f \in \mathbb{R}[x\ \::\: deg(f) \leq 3, f(1) = 0\}$ sottospazio di $V = \mathbb{R}[x]_{\leq 3}$. Sappiamo che la $dim(V) = 4$ (1, x, $x^2, x_3$). La proposizione mi dice che $dim(W) \leq 3$ e se trovo 3 vettori indipendenti allora $dim(W) = 3$. Possiamo vedere che $x-1, x^2-1, x_3-1$ sono lin. indipendenti quindi concludiamo che $dim(W) = 3$.
\end{example}

\begin{observation}
Se V è un spazio, $V_1, V_2 \subset V$ sottospazi allora anche $V_1 \cup V_2$ è un sottospazio. Infatti se $v \in V_1 \cup V_2$ e $w \in V_1 \cap V_2 \Longrightarrow v+w V_1 \cap V_2$ perché $V_1$ sottospazio ed allora $v + w \in V_1$ ed allora in modo simile per $V_2 \Longrightarrow v + w \in V_2$ ed in modo simile $\lambda v \in V_1, \lambda v \in V_2 \Longrightarrow \lambda v \in V_1 \cap V_2 \forall \: \lambda \in \mathbb{R}$.
\end{observation}

\begin{example}
Sia $W_i \subset \mathbb{R}^n$ il sottospazio delle soluzioni dell'equazione omogenea $E_i \:: a_{i1}x_1 + \cdots + a_{in}x_n = 0$. Allora $W_1 \cap W_2 \cap \cdots \cap W_r$ è il sottospazio delle soluzioni comuni di $E_1, E_2, \cdots, E_r$.
\end{example}

\subsection{Formula di Grassman}
\begin{definition}[Somma fra sottospazi]
Siano $V_1, V_2 \subset V$ due sottospazi. La loro somma di $V_1, V_2$ è definita come:
\[V_1 + V_2 = \{v_1 + v_2 : v_1 \in V_1, v_2 \in V_2\}\]
\end{definition}

\begin{observation}
Si osservi che $V_1 + V_2 \subset V$ è un sottospazio a sua volta.
\end{observation}

\begin{demostration}
Questa definizione si somma fra sottospazi è vera perché se $v, w \in V_1 + V_2$ allora:
\[
\begin{rcases}
v = v_1 + v_2 & \text{ con } (v_i \in V_i)\\
w = w_1 + w_2 & \text{ con } (w_i \in V_i)
\end{rcases}
\Rightarrow v+w = (v_1 + w_1) + (v_2 + w_2) \in V_1 + V_2 \text{ perché }(v_1 + w_1) \in V_1, (v_2 + w_2) \in V_2
\]
Se $\lambda \in \mathbb{R}$, $\lambda v = \lambda(v_1 + v_2) = \lambda v_1 + \lambda v_2 \in V_1 + V_2$ con $\lambda v_1 \in V_1$ e $\lambda v_2 \in V_2$. $\blacksquare$
\end{demostration}

\begin{proposition}
Se $v_1, v_2, \cdots, v_n \in V_1 + V_2 \Longrightarrow span(v_1, \hdots, v_n) \subset V_1 + V_2$.
\end{proposition}

\begin{demostration}
La dimostrazione è abbastanza veloce, infatti basta vedere che se $V_1 + V_2$ è un sottospazio che contiene $v_1, \cdots, v_n$ allora contiene le loro combinazioni. lineari.
\end{demostration}

\begin{example}
Dato un $V_1 = \{\begin{bmatrix}a\\0\end{bmatrix}\::\: a \in \mathbb{R}\}, V_2 = \{\begin{bmatrix}0\\b\end{bmatrix}\} \::\: b \in \mathbb{R} \subset \mathbb{R}^2$ sottospazi.\\
Allora $V_1 + V_2 = \{\begin{bmatrix}a\\b\end{bmatrix}\::\: a,b \in \mathbb{R}\} = \mathbb{R}^2$
\end{example}

\begin{example}
Dati $V_1 = \{\begin{bmatrix}0\\a_2\\a_3\end{bmatrix}\::\: a_2, a_3 \in \mathbb{R}\}$, $V_2 = \{\begin{bmatrix}a_1\\a_2\\0\end{bmatrix}\::\: a_1, a_2 \in \mathbb{R}\} \subset \mathbb{R}^3$.\\
Allora $V_1 + V_2 = \{\begin{bmatrix}a_1\\a_2 + a_2\\a_3\end{bmatrix} \::\: a_1, a_2, a_3 \in \mathbb{R}\} = \mathbb{R^3}$ ma anche $V_1 \cap V_2 = \{\begin{bmatrix}0\\a_2\\0\end{bmatrix} \::\: a_2 \in \mathbb{R}\}$
\end{example}

\begin{theorem}[Formula di Grassman]
Sia $dim(V) < \infty$, $V_1, V_2 \subset V$ sottospazi allora:
\[dim(V_1 + V_2) = dim(V_1) + dim(V_2) - dim(V_1 \cap V_2)\]
\end{theorem}

\begin{demostration}
Per dimostrare questa formula sia $e_1, \cdots, e_4$ una base di $V_1 \cap V_2$. Si completa in una base $e_1, \cdots, e_r, v_{r+1}, \cdots, v_n$ di $V_1$ e $e_1, \cdots, e_r, w_{r+1}, \cdots, w_m$ di $V_2$. \\
Quindi abbiamo he $dim(V_1) = n, dim(V_2) = m$ e che $V_1 \cap V_2 = r$. A questo punto verifichiamo che $e_1, \cdots, e_r, v_{r+1}, \cdots, v_n, w_{r+1}, \cdots, w_m$ è una base di $V_1 + V_2$. Se fosse una base allora $dim(V_1 + V_2) = n + m - r$.
Per verificare se è una base verifichiamo se è lin indipendente, sia:\\
$\lambda_1 e_1 + \cdots + \lambda_r e_r + \mu_2 v_{r+1} + \cdots + \mu_{n-r}v_n + \nu_1 w_{r+1} + \cdots + \nu_{m-r}w_m = 0$. Tutti i coefficienti $\lambda_i, \mu_i, \nu_i \in \mathbb{R}$, dobbiamo ora vedere se sono tutti uguali a 0.\\\\
$\lambda_1 e_1 + \cdots + \lambda_r e_r + \mu_2 v_{r+1} + \cdots + \mu_{n-r}v_n = -\nu_1 w_{r+1} - \cdots - \nu_{m-r}w_m$. Vediamo che la parte $\lambda_1 e_1 + \cdots + \lambda_r e_r + \mu_2 v_{r+1} + \cdots + \mu_{n-r}v_n \in V_1$ mentre $-\nu_1 w_{r+1} - \cdots - \nu_{m-r}w_m \in V_2$, quindi $-\nu_1 w_{r+1} - \cdots - \nu_{m-r}w_m \in V_1 \cap V_2 \Longrightarrow$ come $e_1, \cdots, e_r$ è una base di $V_1 \cap V_2, \exists \: \alpha_1, \cdots, \alpha_r \::\: \alpha_1 e_1 + \cdots + \alpha_r e_r = -\nu w_{r+1} - \cdots - \nu_{m-r}w_m$. \\\\
Ma $e_1, \cdots, e_r, w_{r+1}, \cdots, w_m$ è base di $V_2$ ed allora è linearmente indipendente ed allora $\alpha_1 = \cdots = \alpha_r = \nu_1 = \cdots = \nu_{m-r} = 0$, ma allora $\lambda_1 e_1 + \cdots + \lambda_r e_r + \mu_1 v_{r+1} + \cdots + \nu_{n-r}v_n = 0 \Longrightarrow \lambda_1 = \cdots = \lambda_r = \mu_1 = \cdots = \mu_{n-r} = 0$ perché $e_1, \cdots, e_r, v_{r+1}, \cdots, v_n$ è una base di $V_1$.\\
Vediamo dunque che $span(e_1, \cdots, e_r, v_1, \cdots, v_{n-r}, w_1, \cdots, w_{m-r}) = V_1 + V_2$ se $v \in V_1 + V_2$, $v = v^1, v^2 \::\: v^1 \in V_1, v^2 \in V_2$. Ma allora $\exists \alpha_1, \cdots, \alpha_m, \beta_1, \cdots, \beta_m \in \mathbb{R} \::$\\
$v^2 = \alpha_1 e_1 + \cdots + \alpha_r e_r + \alpha_{r+1}v_1 + \cdots + \alpha_n v_{n-r}$ e $v_2 = \beta_1 e_1 + \cdots + \beta_r e_r + \beta_{r+1}w_1 + \cdots + \beta_m w_{m-r}$ perché $e_1, \cdots, e_r, v_1, \cdots, v_{n-r}$ è un base di $V_1$ e $e_1, \cdots, e_r, w_1, \cdots, w_{m-r}$ è una base di $V_2$. \\
Detto ciò allora abbiamo che $v_1 = v^1 + v^2 = (\alpha_1 + \beta_1)e_1 + \cdots + (\alpha_r + \beta_r)e_r + \alpha_{r+1}v_{r+1} + \cdots + \alpha_n v_n + \beta_{r+1}w_1 + \cdots + \beta_m w_m$. $\blacksquare$
\end{demostration}

\begin{example}
Consideriamo i due sottospazi in $\mathbb{R}^4$ seguenti:\\
$V= \Big\{\text{soluzioni di } \begin{array}{l}x_1 + 2x_2 + x_3 = 0\\-x_1 -x_2 + 3x_4 = 0\end{array}\Big\}, W = span\Bigg(w_1 = \begin{bmatrix}2\\0\\1\\1\end{bmatrix}, w_2 = \begin{bmatrix}3\\-2\\-2\\0\end{bmatrix}\Bigg)$\\
Calcoliamo $dim(V \cap W), dim(V + W)$. Sappiamo che $dim(W)=2$ perché $w_1 $ e $w_2$ sono \textbf{linearmente indipendenti}. Questo è ovvio in quanto abbiamo solo due vettori che non sono uno il multiplo dell'altro.\\
Bisogna dunque calcolare la $dim(V)$ tramite Gauss-Jordan:
\[
\begin{bmatrix}
1 & 2 & 1 & 0\\
-1 & -1 & 0 & 3
\end{bmatrix}
\xrightarrow{R_2 + R_1}
\begin{bmatrix}
1 & 2 & 1 & 0\\
0 & 1 & 1 & 3
\end{bmatrix}
\xrightarrow{R_1 - 2R_2}
\begin{bmatrix}
	1 & 0 & -1 & -6\\
	0 & 1 & 1 & 3
\end{bmatrix}
\]
Abbiamo dunque $x_3, x_4$ come variabili libere che fa si che $x_1 = x_3 + 6 x_4$ e $x_2 = -x_3 - 3x_4$, la soluzione generale è dunque:\\\\
$\begin{bmatrix}
	x_1 \\ x_2 \\ x_3 \\ x_4
\end{bmatrix}
= x_3
\begin{bmatrix}
	1\\-1\\1\\0
\end{bmatrix}
 + x_4
\begin{bmatrix}
	6\\-3\\0\\1
\end{bmatrix}$
\hspace{.5cm} con $v_1 = \begin{bmatrix}1\\-1\\1\\0\end{bmatrix}$ e $v_2 = \begin{bmatrix}6\\-3\\0\\1\end{bmatrix}$ \\\\
Quindi $dim(V) = 2$ e $v_1, v_2$ è una base. Cerchiamo ora $dim(V + W)$.
\[
\begin{bmatrix}
1 & 6 & 2 & 3\\
-1 & -3 & 0 & -2\\
1 & 0 & 1 & -2\\
0 & 1 & 1 & 0
\end{bmatrix}
\xrightarrow[R_2 + R_1]{R_3 - R_1}
\begin{bmatrix}
1 & 6 & 2 & 3\\
0 & 3 & 2 & 1\\
0 & -6 & -1 & -5\\
0 & 1 & 1 & 0
\end{bmatrix}
\xrightarrow{R_3 + 3R_1}
\begin{bmatrix}
1 & 6 & 2 & 3\\
0 & 3 & 2 & 1\\
0 & 0 & 3 & -3\\
0 & 1 & 1 & 0
\end{bmatrix}
\xrightarrow{\text{Inverto } R_2, R_4}
\begin{bmatrix}
1 & 6 & 2 & 3\\
0 & 1 & 1 & 0\\
0 & 0 & 3 & -3\\
0 & 3 & 2 & 1
\end{bmatrix}
\]
\[
\xrightarrow{R_4 - R_2}
\begin{bmatrix}
1 & 6 & 2 & 3\\
0 & 1 & 1 & 0\\
0 & 0 & 3 & -3\\
0 & 0 & -1 & 1
\end{bmatrix}
\xrightarrow{R_4 + \frac{1}{3}R_3}
\begin{bmatrix}
1 & 6 & 2 & 3\\
0 & 1 & 1 & 0\\
0 & 0 & 3 & -3\\
0 & 0 & 0 & 0
\end{bmatrix}
\]
Abbiamo dunque 3 pivots ed allora le prime 3 colonne sono indipendenti ma $v_1, v_2, w_1, w_2$ sono dipendenti. Questo fa si che $dim(V+W) =$3.\\
Utilizzando poi Grassman: $dim(V \cap W) = dim(V) + dim(W) - dim(V+W) = 2 + 2 - 3 = 1$
\end{example}

\begin{example}
Siano $V = span\Bigg(\begin{bmatrix}1\\1\\1\\1\end{bmatrix}, \begin{bmatrix}1\\-1\\1\\1\end{bmatrix}\Bigg), W = span\Bigg(\begin{bmatrix}1\\0\\1\\0\end{bmatrix}\begin{bmatrix}1\\2\\0\\2\end{bmatrix}\Bigg)$.\\\\
Chiamiamo i due vettori in V $v_1, v_2$ mentre i due in $W$ $w_1, w_2$. Trovare basi di $V + W, V \cap W$. Per $V+W$ facciamo:
\[
\begin{bmatrix}
1 & 1 & 1 & 1\\
1 & -1 & 0 & 2\\
1 & 1 & 1 & 0\\
1 & -1 & 0 & 2
\end{bmatrix}
\begin{array}{l}
    R_2 - R_1\\
    R_3 - R_1\\
    R_3 - R_1
\end{array}
\Rightarrow
\begin{bmatrix}
1 & 1 & 1 & 1\\
0 & -2 & -1 & 1\\
0 & 0 & 0 & -1\\
0 & -2 & -1 & 1
\end{bmatrix}
R_4 - R_2
\Rightarrow
\begin{bmatrix}
1 & 1 & 1 & 1\\
0 & -2 & -1 & 1\\
0 & 0 & 0 & -1\\
0 & 0 & 0 & 0
\end{bmatrix}
\]
Abbiamo dunque 3 pivtos ed allora $dim(V+W) = 3$, perché $v_1, v_2, w_2$ sono lin. indipendenti e quindi sono una base. Utilizzando allora Grassmann: $dim(V \cap W) = dim(V) + dim(W) - dim(V+W) = 2 + 2 - 3 = 1$.\\
In uno spazio di $dim = 1$ ogni vettore diverso da 0 è base, per trovarlo facciamo:\\
\[\begin{array}{l}
    x_1 + x_2 + x_3 + x_4 = 0\\
    -x_2 - x_3 - x_4 = 0\\
    -x_4 = 0
\end{array}
\hspace{.3cm}
\parbox{7cm}{Abbiamo dunque che $x_4 = 0, x_3$ = t sono variabili libere. Quindi $x_2 = -\frac{7}{2}, x_1 = -\frac{t}{2}$}
\]
La soluzione generale è dunque $(-\frac{t}{2}, -\frac{t}{2}, t, 0)$ e con $t=1$ abbiamo $(\frac{1}{2}, \frac{1}{2}, -1, 0)$ che fa si abbiamo $\frac{1}{2}v_1 + \frac{1}{2}v_2 - w_1 = 0$. Dunque $w_1 = \frac{1}{2}v_1 + \frac{1}{2}v_2 \in V \cap W$.\\
Quindi $w_1$ è una base di $V \cap W$ e questo perché so che questo spazio ha $dim = 1$ grazie a Grassman. Ogni volta che $V \cap W$ è della forma $t \cdot w_1$ allora ri dimostra che $dim(V \cap W) = 1$.
\end{example}
\begin{example}
	Dati $V = M_{3 \times 3}(R)$, $V_1 = \{\text{matrici diagonali}\}$ e $V_2 = \{\text{matrici dove la 1° riga = 2° riga}\}$.
	\[
	\text{Base di $V_1$: }
	\begin{bmatrix}
		1 & 0 & 0 \\
		0 & 0 & 0 \\
		0 & 0 & 0 
	\end{bmatrix}
	\text{,}
	\begin{bmatrix}
		0 & 0 & 0 \\
		0 & 1 & 0 \\
		0 & 0 & 0 
	\end{bmatrix}
	\text{,}
	\begin{bmatrix}
		0 & 0 & 0 \\
		0 & 0 & 0 \\
		0 & 0 & 1
	\end{bmatrix}
	\Rightarrow dim(V_1) = 3
	\]
	\[
	\text{Base di $V_2$: }
	\begin{bmatrix}
		1 & 0 & 0 \\
		1 & 0 & 0 \\
		0 & 0 & 0 
	\end{bmatrix}
	\text{,}
	\begin{bmatrix}
		0 & 1 & 0 \\
		0 & 1 & 0 \\
		0 & 0 & 0 
	\end{bmatrix}
	\text{,}
	\begin{bmatrix}
		0 & 0 & 1 \\
		0 & 0 & 1 \\
		0 & 0 & 0
	\end{bmatrix}
	\text{,}
	\begin{bmatrix}
		0 & 0 & 0 \\
		0 & 0 & 0 \\
		1 & 0 & 0
	\end{bmatrix}
	\text{,}
	\begin{bmatrix}
		0 & 0 & 0 \\
		0 & 0 & 0 \\
		0& 1 & 0
	\end{bmatrix}
	\text{,}
	\begin{bmatrix}
		0 & 0 & 0 \\
		0 & 0 & 0 \\
		0 & 0 & 1
	\end{bmatrix}
	\Rightarrow dim(V_1) = 6
	\]
	\[
	\text{Elemento generale di $V_1 \cap V_2$: }
	\begin{bmatrix}
		0 & 0 & 0 \\
		0 & 0 & 0 \\
		0 & 0 & 1
	\end{bmatrix}
	\Rightarrow dim(V_1 \cap V_2) = 1
	\]
	Grassmann: $dim(V_1 + V_2) = 3 + 6 - 1 = 8$
\end{example}
\newpage
\section{Applicazione lineare}
\begin{definition}[Applicazione lineare]
Siano $V_1, V_2$ spazi vettoriali su $\mathbb{R}$. Un'applicazione lineare (o mappa lineare) è una mappa $\varphi: V_1 \to V_2$ soddisfano:
\begin{enumerate}
    \item $\varphi(v_1 + v_2) = \varphi(v_1) + \varphi(v_2) \: \forall \: v_1, v_2 \in V_1$.
    \item $\lambda \varphi(v) = \varphi(\lambda v) \forall \: v \in V_1$.
\end{enumerate}
\end{definition}

\begin{example}
Alcuni esempi di applicazioni lineari:
\begin{itemize}
    \item $V_1 = \mathbb{R}^n$, $V_2 = \mathbb{R}$, $\varphi\Big(\begin{bmatrix}a_1\\\vdots\\a_n\end{bmatrix}\Big) = \lambda_1a_1 + \cdots + \lambda_n a_n$ con $\lambda_1 \cdots \lambda_n$ fisso.
    \item $V_1 = \mathbb{R}^n$, $V_2 = \mathbb{R}^2$, $\varphi\Bigg(\begin{bmatrix}a_1 \\ \vdots \\ a_n\end{bmatrix}\Bigg) = \Bigg(\begin{array}{c}\lambda_1 a_1 + \cdots + \lambda_n a_n \\ \mu_1 a_1 + \cdots + \mu_n a_n\end{array} \Bigg)$ con $\lambda_1, \cdots, \lambda_n$ e $\mu_1, \cdots, \mu_n$ fissi.
    \item $V_1 = \mathbb{R}^n, V_2 = \mathbb{R}^{n-1}$, $\varphi\Bigg(\begin{bmatrix}a_1 \\ \vdots \\ a_n\end{bmatrix}\Bigg) = \begin{bmatrix}a_1 \\ \vdots \\ a_{n-1}\end{bmatrix}$
    \item $V_1 = \mathbb{R}[x]_{\leq d}, V_2 = \mathbb{R}[x]_{\leq d-1}$ quindi è come scrivere $\varphi(f) = f'$. \\
    E questo va bene perché sono rispettate le proprietà (a) e (b) della definizione sopra.
    \item $V_1 = \{f:[0,1]\to \mathbb{R} \::\: f \text{ continua }, \int_0^1 f<\infty\}$, $V_2 = \mathbb{R}$. Vediamo che $\varphi (f) = \int_0^1 f$.\\
    Infatti, anche in questo caso, le proprietà (a) e (b) della definizione sono rispettate.
\end{itemize}
\end{example}

\hspace{-15pt}Sia $\varphi: V_1 \to V_2$ un'applicazione lineare e sia $e_1, \cdots, e_n$ una base di $V_1$ allora sia $v \in V_1$, $v = \lambda_1 e_1 + \lambda_2 e_2 + \cdots + \lambda_n e_n$. $\varphi(v) = \varphi(\lambda_1 e_1) + \varphi(\lambda_2 e_2) + \cdots + \varphi(\lambda_n e_n) = \lambda_1 \varphi(e_1) + \lambda_2 \varphi(e_2) + \cdots + \lambda_n \varphi(e_n)$.\\
In conclusione, conoscere $\varphi(v) \Longleftrightarrow$ conoscere $\varphi (e_1), \cdots, \varphi (e_n)$ e lineare coordinate di v rispetto $e_1, \cdots, e_n$. Viceversa se faccio $\varphi (e_1) = v_1, \varphi(e_2) = v_2, \cdots, \varphi(e_n) = v_n$ allora $\exists !$ applicazione lineare $\varphi v_1 - \varphi v_2$ con queste proprietà.

\begin{example}
$V_1 = \mathbb{R}^n$, $V_2 = \mathbb{R}^2$ e le basi standard sono $\begin{bmatrix}1\\0\end{bmatrix}, \begin{bmatrix}0\\1\end{bmatrix}$. Esiste una sola $\varphi: \mathbb{R}^2 \to \mathbb{R}^2$ tale che $\varphi \Big(\begin{bmatrix}1\\0\end{bmatrix}\Big) = \Big(\begin{bmatrix}0\\0\end{bmatrix}\Big)$, $\varphi \Big(\begin{bmatrix}0\\1\end{bmatrix}\Big) = \Big(\begin{bmatrix}0\\1\end{bmatrix}\Big)$. Infatti tale $\varphi$ è dato da $\varphi \Big(\begin{bmatrix}x_1\\x_2\end{bmatrix}\Big) = \Big(\begin{bmatrix}0\\x_2\end{bmatrix}\Big)$
\end{example}

\subsection{Nucleo e immagine}
\begin{definition}
Sia $\varphi: V_1 \to V_2$ un'applicazione lineare possiamo definire di $\varphi$:
\begin{itemize}
    \item \textbf{Il nucleo}: $Ker(\varphi) = \{v \in V_1 \::\: \varphi(v) = 0\} \subset V_1$ sottospazio. 
    \item \textbf{L'immagine}: $Im(\varphi) = \{v_2 \in V_2 \::\: \exists v_1 \in V_1 \::\: \varphi(v_1) = v_2\} \subset V_2$ sottospazio.
\end{itemize}
\end{definition}

\begin{example}
Alcuni esempi di nucleo ed immagine di un'applicazione lineare.
\begin{enumerate}
    \item Per $\varphi: \mathbb{R}^2 \to \mathbb{R}^2$, $\varphi\Bigg(\begin{bmatrix}x_1\\x_2\end{bmatrix}\Bigg) = \begin{bmatrix}0\\x_2\end{bmatrix}$.\\
    $Ker(\varphi) = \{\begin{bmatrix}x_1\\0\end{bmatrix} \::\: x_1 \in \mathbb{R}\} = span\Bigg(\begin{bmatrix}1\\0\end{bmatrix}\Bigg)$ \hspace{.3cm} $Im(\varphi) = \{\begin{bmatrix}0\\x_2\end{bmatrix} \::\: x_2 \in \mathbb{R}\} = span\Bigg(\begin{bmatrix}0\\1\end{bmatrix}\Bigg)$
    \item $\varphi: \mathbb{R}^2 \to \mathbb{R}^2$, $\varphi\Bigg(\begin{bmatrix}x_1\\x_2\end{bmatrix}\Bigg) = \begin{bmatrix}x_1\\0\end{bmatrix}$.\\
    $Ker(\varphi) = \{\begin{bmatrix}x_1\\0\end{bmatrix} \::\: x_1 \in \mathbb{R}\} = span\Bigg(\begin{bmatrix}1\\0\end{bmatrix}\Bigg)$ \hspace{.3cm} $Im(\varphi) = \{\begin{bmatrix}x_2\\0\end{bmatrix} \::\: x_2 \in \mathbb{R}\} = span\Bigg(\begin{bmatrix}1\\0\end{bmatrix}\Bigg)$
    \item $\varphi: \mathbb{R}[x]_{\leq d} \to \mathbb{R}[x]_{\leq d-1}$, $Ker(\varphi) = \{$ polinomi costanti $\} = span(1)$ \hspace{.3cm} $Im(\varphi) = \mathbb{R}[x]_{d-1}$
\end{enumerate}
\end{example}

\begin{theorem}
Sia $dim(V_1) < \infty$ e sia $\varphi: V_1 \to V_2$ un'applicazione lineare, allora vale che:
\[dim\: Ker(\varphi) + dim \: Im(\varphi) = dim \: V_1\]
\end{theorem}

\begin{demostration}
Per dimostrare il teorema sopra partiamo prendendo $v_1, \cdots, v_r$ una base di $Ker(\varphi)$ (quindi $dim\:Ker(\varphi) = r$), e $w_1, \cdots, w_s$ una base di $Im(\varphi)$ (quindi $dim\:Im(\varphi) = s$). \\\\
Siano poi $\overline{v_1}, \cdots, \overline{v_s} \in V_1$ tali che $\varphi(\overline{v_1}) = w_1, \cdots, \varphi(\overline{v_s}) = w_s$. Noi dobbiamo dimostrare che $v_1, \cdots, v_r, \overline{v_1}, \cdots, \overline{v_2}$ è una base di $V_1$ (in questo modo dimostriamo che $dim\:V_1 = r + s$ ed il teorema è verificato).\\\\
Verifichiamo l'indipendenza lineare. Supponiamo che $\lambda_1 v_1 + \cdots + \lambda_r v_r + \lambda_{r+1}\overline{v_1} + \cdots + \lambda_{r+s}\overline{v_s} = 0$. Applichiamo $\varphi$: $\lambda_1 v_1 + \cdots + \lambda_r v_r + \varphi(\lambda_{r+1}\overline{v_1} + \cdots + \lambda_{r+2}\overline{v_s}) = 0$ ($\varphi(v_1) = 0 \:\forall : i$). quindi $\lambda_{r+1}\varphi(\overline{v_1}) + \cdots + \lambda_{r+2}\varphi(\overline{v_s}) = 0$ che è come scrivere $\lambda_{r+1}w_1 + \cdots + \lambda_{r+s}w_s = 0 \Longrightarrow \lambda_{r+1} = \cdots = \lambda_{r+s} = 0$ perché $w_1, \cdots, w_s$ base.\\\\
Quindi $\lambda_1 v_1 + \cdots + \lambda_r v_r = 0$ ed allora $\lambda_1 = \cdots = \lambda_r = 0$ perché $v_1, \cdots, v_r$ è una base di $Ker(\varphi)$.\\
In fine $\lambda_1 = \cdots = \lambda_r = \lambda_{r+1} = \cdots = \lambda_{r+s} = 0$.\\
$span(v_1, \cdots, v_r, \overline{v_1}, \cdots, \overline{v_s}) = v_1$ tale che sia $v \i V_1$. $\varphi(v) \in Im(\varphi) \Longrightarrow \exists \: \overline{\lambda_1}, \cdots, \overline{\lambda_s}$ tale che $\varphi(v) = \overline{\lambda_1}w_1 + \cdots + \overline{\lambda_s}w_s$. Ma allora $\varphi(v - \overline{\lambda_1}\overline{v_1} - \cdots - \overline{\lambda_s}\overline{v_s}) = \varphi(v) - \overline{\lambda_1}\varphi(\overline{v_1}) - \cdots - \overline{\lambda_s}\varphi(\overline{v_s}) = 0$. Quindi $v - \overline{\lambda_1}\overline{v_1} - \cdots - \overline{\lambda_s}\overline{v_s} \in Ker(\varphi)$, ma allora $v - \overline{\lambda_1}\overline{v_1} - \cdots - \overline{\lambda_s}\overline{v_s} = \lambda_1 v_1 + \cdots + \lambda_r v_r$ $\forall \: \lambda_{1}, \cdots, \lambda_r$ perché $v_1, \cdots, v_r$ base di $Ker(\varphi)$. In somma $v = \lambda_1 v_1 + \cdots + \lambda_r v_r + \overline{\lambda_1}\overline{v_1} + \cdots + \overline{\lambda_s}\overline{v_s}$
\end{demostration}
% !TeX spellcheck = it_IT
\newpage
\section{Determinante}

\begin{definition}[Determinante]
	Il determinante $det(A)$ di una matrice $A \in M_{n \times n}(\mathbb{R})$ è uno scalare in $\mathbb{R}$.
	\begin{equation*}
		n=1 A=[a] det(A)=a
	\end{equation*}
	\begin{equation*}
		n=2 A=\begin{bmatrix}
			a & b\\
			c & d
		\end{bmatrix}
		det(A) = ad - bc
	\end{equation*}
	Si noti che $det(A)\neq0 \Longleftrightarrow$ le colonne di $A$ sono linearmente indipendenti.
\end{definition}
\begin{theorem}
	Se $n=2$, $A=\begin{bmatrix}
		a & b \\ c & d
	\end{bmatrix}$, $a,b,c,d \geq 0$ e $ad - bc \neq 0$ allora $det(A)$ corrisponde all'area del parallelogramma definita da $\begin{bmatrix}
		a \\ c
	\end{bmatrix}$. $\begin{bmatrix}
		b \\ d
	\end{bmatrix}$.
	%TODO Inserisci disegno area del parallelogramma
	\begin{example}
		Di seguito alcuni esempi del calcolo del determinante e della corrispondenza con l'area del parallelogramma.
		\begin{enumerate}
			\item $A=\begin{bmatrix}
				1 & 0 \\ 0 & 1
			\end{bmatrix}$, $det(A)=1 \cdot 1 - 0 \cdot 0 = 1$
			\item $A=\begin{bmatrix}
				1 & 1 \\ 0 & 1
			\end{bmatrix}$, $det(A)=1 \cdot 1 - 1 \cdot 0 = 1$
			\item $A=\begin{bmatrix}
				1 & 1 \\ 0 & 2
			\end{bmatrix}$, $det(A)=1 \cdot 2 - 1 \cdot 0 = 2$
			\item $A=\begin{bmatrix}
				1 & 1 \\ 1 & 2
			\end{bmatrix}$, $det(A)=1 \cdot 2 - 1 \cdot 1 = 1$
		\end{enumerate}
		%TODO Inserisci disegno area
	\end{example}
\end{theorem}

\begin{definition}[Determinante per induzione]
	Se $A \in M_{n \times m}(\mathbb{R})$, sia $A_{ij}$ una matrice ottenuta da $A$ cancellando la riga $i$ e la colonna $j$.
	\begin{equation*}
		A_{ij} \in M_{(n-1)(m-1)}(\mathbb{R})
	\end{equation*}
	Il determinante si può definire induttivamente come segue:
	\begin{itemize}
		\item \textbf{Ipotesi induttiva}: supponiamo che $det(A_{ij}) \in M_{(n-1)(m-1)}(\mathbb{R})$ sia già definito
		\item  \textbf{Passo induttivo}: $det(A)$ si definisce come 
		%TODO Ti sei perso e il pc si è scaricato
	\end{itemize}
\end{definition}
\begin{definition}[Formula di Cramer]
	Dati una matrice $A = [a_{ij}] \in M_{n \times m}(\mathbb{R})$, la \textbf{matrice aggiunta} $\tilde{A} = [\tilde{a}_{ij}] \in M_{n \times m}(\mathbb{R})$ e sia $\tilde{a}_{ij} = (-1)^{i+j} \cdot det(A_{ij})$, allora:
	\begin{equation*}
		A \cdot \tilde{A} = det(A) \cdot I
	\end{equation*}
\end{definition}
\begin{corollary}
	Se $det(A) \neq 0$, $A$ è \textbf{invertibile} e
	\begin{equation*}
		A^{-1} = \frac{1}{det(A)} \cdot \tilde{A}
	\end{equation*}
\end{corollary}
\begin{example}
	\begin{equation*}
		A = \begin{bmatrix}
			1 & 0 & 3 \\
			0 & 2 & 0 \\
			4 & 0 & 1
		\end{bmatrix}
	\end{equation*}
	\begin{equation*}
		det(A) = 2 \cdot \begin{bmatrix}
			1 & 3 \\
			4 & 1
		\end{bmatrix} = -22
	\end{equation*}
	\begin{equation*}
		\tilde{A} = \begin{bmatrix}
			2 & 0 & -6 \\
			0 & -11 & 0 \\ 
			-8 & 0 & 2
		\end{bmatrix} \Longrightarrow A^{-1} = -\frac{1}{22} \cdot \tilde{A}
	\end{equation*}
	\begin{equation}
		A^{-1} = \begin{bmatrix}
			-\frac{1}{11} & 0 & \frac{3}{11} \\
			0 & \frac{1}{2} & 0 \\
			\frac{4}{11} & 0 & -\frac{1}{11}
		\end{bmatrix}
	\end{equation}
	\begin{equation*}
		A \cdot A^{-1} = \begin{bmatrix}
			1 & 0 & 0 \\
			0 & 1 & 0 \\
			0 & 0 & 1
		\end{bmatrix}
	\end{equation*}
\end{example}

\begin{proposition}
	Se $A$ è invertibile allora $det(A) \neq 0$
\end{proposition}

\begin{theorem}[Teorema di Binet]
	Dati $A$, $B \in M_{n \times m}(\mathbb{R})$ vale che
	\begin{equation*}
		det(A \cdot B) = det(A) \cdot det(B)
	\end{equation*}
\end{theorem}
\begin{proposition}
	Sapendo che $\exists A^{-1} \Longrightarrow A \cdot A^{-1} = I$ allora:
	\begin{equation*}
		det(A) \cdot det(A^{-1}) = det(A \cdot A^{-1}) = det(I)
	\end{equation*}
\end{proposition}

\begin{theorem}
	Sia $A \in M_{n \times m}(\mathbb{R})$ allora sono equivalenti:
	\begin{enumerate}
		\item $A$ è \textbf{invertibile}
		\item $det(A) \neq 0$
		\item Le colonne di $A$ sono \textbf{linearmente indipendenti}
	\end{enumerate}
\end{theorem}

\begin{observation}
	Dati questi teoremi, facciamo alcune osservazioni:
	\begin{enumerate}
		\item Data una matrice $n=2$ $A = \begin{bmatrix}
			a& b \\
			c & d
		\end{bmatrix} $\\
		$\begin{bmatrix}
			a \\ c
		\end{bmatrix}$, $\begin{bmatrix}
			b \\ d
		\end{bmatrix}$ sono \textbf{linearmente indipendenti} $\Longleftrightarrow$ Non sono collineari \\
		$\Longleftrightarrow$ l'area del parallelogramma associato è diversa da $0$\\
		$\Longleftrightarrow det(A) \neq 0$
		
		\item (3) $\Longleftrightarrow rango(A) = n$
		\item Le condizioni sono equivalenti %TODO Non so cosa significhi
		\item Le righe di $A$ sono linearmente indipendenti
	\end{enumerate}
\end{observation}

\begin{definition}[Matrice trasposta]
	Se $A = [a_{ij}]$ la sua trasposta è la matrice $A^t = [a_{ji}]$, ovvero la riga $i$ di $A$ diventa la colonna $i$ di $A^t$.
\end{definition}
\begin{observation}
	\begin{equation*}
		det(A) = det(A^t)
	\end{equation*}
	Da questo deduciamo che (2) $\Longleftrightarrow det(A^t) \neq 0 \Longleftrightarrow $ le colonne di $A^t$ sono linearmente indipendenti $\Longleftrightarrow$ (4)
\end{observation}

\begin{proposition}
	Sia $\phi : V \to V$ un'applicazione lineare, $B$, $B'$ due basi di $V$ e $A=[\phi]^B_B$, $A' = [\phi]^{B'}_{B'}$. Allora $det(A) = det(A')$. Quindi $det(A)$ dipende solo da $\phi'$.
\end{proposition}

\begin{theorem}
	Sia $\phi: V \to V$ un'applicazione lineare, $B$ una qualsiasi base e $A = [\phi]^B_B$ allora è equivalente dire:
	\begin{enumerate}
		\item $\phi$ è un \textbf{isomorfismo}
		\item $det(A) \neq 0$
		\item $im(\phi) = V$
		\item $ker(\phi) = \{0\}$
	\end{enumerate}
\end{theorem}
% !TeX spellcheck = it_IT
\newpage
\section{Autovalori}
\begin{definition}[Autovalori]
	Sia $V$ uno spazio vettoriale su $\mathbb{R}$ ($dim(V) < \infty$) e $\phi: V \to V$. $\lambda \in \mathbb{R}$ è \textbf{autovalore} di $\phi$ se $\exists v \neq 0$ in $V$ tale che $\phi(V) = \lambda \cdot v$. In questo caso $v$ è \textbf{autovettore} di $\phi$ (associato a $\lambda$).
\end{definition}
\begin{observation}
	Alcune osservazioni su questa definizione:
	\begin{enumerate}
		\item $v$ può essere autovettore per un solo $\lambda$. Infatti, se $\begin{cases}
			\phi(v)=\lambda_{1} \cdot v \\
			\phi(v)=\lambda_{2} \cdot v
		\end{cases} \Longrightarrow (\lambda_{1} - \lambda_{2}) \cdot v = 0 \Longrightarrow \lambda_{1} = \lambda_{2}$
		\item In generale ci sono molti autovettori associati allo stesso $\lambda$
	\end{enumerate}
\end{observation}

\begin{definition}[Diagonalizzabile]
	$\phi$ è \textbf{diagonalizzabile} se $\exists$ base $B$ tale che $[\phi]^B_B$ è una matrice \textbf{diagonale}.
\end{definition}

\begin{proposition}
	$\phi$ è diagonalizzabile se e solo se $V$ ammette una base costituita da autovettori di $\phi$. \\
\end{proposition}

\begin{example}
	Se $\phi: \mathbb{R}^2 \to \mathbb{R}^2$, $v \mapsto A \cdot v$  dove $A = \begin{bmatrix}
		1 & 1 \\
		0  &1
	\end{bmatrix}$ $\Longrightarrow$ $\phi$ non è diagonalizzabile.  
\end{example}

\subsection{Come trovare gli autovalori?}
$\lambda$ è autovalore per $\phi$ se e solo se $\phi(v) = \lambda \cdot v$ per un $v \neq 0$. Quindi $\phi(v) - \lambda \cdot v = 0 \Longrightarrow (\phi - \lambda \cdot id) \cdot v = 0$.
%TODO Ti sei perso
\end{document}
