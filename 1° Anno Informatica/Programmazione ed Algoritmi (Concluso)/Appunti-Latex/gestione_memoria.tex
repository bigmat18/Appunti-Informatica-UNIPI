\newpage
\section{Gestione della memoria}
\subsection{Record di attivazione}
\begin{definition}
	è l'insieme di strutture dati e funzioni necessarie all'esecuzione dei programmi viene aggiunto al codice eseguibile dal compilatore.
\end{definition}
\begin{definition}[Activation record o stack frame]
	Contiene tutte le informazioni necessarie all'esecuzione del blocco o della funzione.
\end{definition}
\begin{definition}[Dynamic chain o call chain]
	Rappresenta la sequenza di chiamate e e serve a garantire il corretto ordine di esecuzione.
\end{definition}
\begin{definition}[Static chain]
	Implementa lo scoping statico e garantisce che i nomi siano referenziati rispettando la visibilità di variabili e funzioni.
\end{definition}
\begin{tabular} { |c|p{250px}|}
	\hline
	Puntatore catena dinamica  & Indirizzo del record di attivazione della funzione chiamante\\
	\hline
	Puntatore catena statica & Indirizzo del prossimo record di attivazione dove risolvere i nomi
		non presenti nel blocco corrente (implementazione dello scoping
		statico) \\
	\hline
	Indirizzo di ritorno & Indirizzo dell’istruzione da eseguire al termine della funzione/blocco corrente \\
	\hline
	Indirizzo risultato & Indirizzo nel record di attivazione del chiamante per memorizzare il risultato \\
	\hline
	Parametri & Spazio riservato alla associazione parametri formali - parametri attuali \\
	\hline
	Variabili locali & Spazio riservato alla allocazione delle variabili locali al blocco \\
	\hline
	Risultati temporanei & Spazio riservato alla allocazione delle variabili temporanee generate dal compilatore \\
	\hline
\end{tabular}
\subsection{Divisione della memoria}
\begin{figure}[h!]
	\centering
	\includegraphics[width=13cm]{images/gestione-memoria.png}
	\caption{Gestione della memoria di un programma}
\end{figure}
\begin{note}
	Partiamo dal presupposto che un \textbf{blocco} sia considerato come una funzione senza parametri.
\end{note}