\section{Complessità}
\subsection{Limiti inferiori}
\subsubsection{Complessità di un problema}
Per determinare il limite inferiore di un problema al caso pessimo, analizzo le seguenti cose:
\begin{itemize}
	\item \textbf{Dimensione dei dati}: Se la soluzione di un problema richiede l'esame di tutti i dati in input, allora $\Omega(n)$ è un limite inferiore. \emph{E.g. sommare tutti gli elementi di un array.}
	\item \textbf{Eventi contabili}: se la soluzione di un problema richiede la ripetizione di un certo evento, allora il numero di volte che l'evento si ripete (moltiplicato per il suo costo) è un limite inferiore.
	\item \textbf{Alberi di decisione}: sono alberi in cui
	\begin{itemize}
		\item ogni nodo non foglia effettua un test su un attributo
		\item ogni arco uscente da un nodo è un possibile valore dell'attributo
		\item ogni nodo foglia assegna una classificazione
	\end{itemize}
	Si applica a problemi risolubili attraverso sequenze di decisioni che via via riducono lo spazio delle soluzioni.
	%TODO inserire immagine di albero di decisione
	\begin{note}
		Alcune formule importanti per gli alberi:
		%TODO Inserisci formule
	\end{note}
	%TODO Esempio con ricerca binaria
\end{itemize}

\subsubsection{Ordinamento}
\subsubsection{Insertion sort}
\textbf{Proprietà}: al termine del passo j-esimo dell'algoritmo l'elemento j-esimo viene in inserito al posto giusto e i primi $j+1$ elementi sono ordinati.
\begin{lstlisting}[language=Javascript, caption=Algoritmo insertion sort, mathescape=true]
	insertionSort(A) =
	var j:Int = 0;
	var i:Int = 0;		$\Theta(1)$
	var k:int = 0;
	for (j=1; j<n; j++) {		$n-1$ volte
		k = A[j];
		i = j-1;		$\Theta(1)$ $n-1$ volte
		while(i >= 0 && A[i]>k) {
			A[i+1] = A[i];			$\Theta(1)$ $\sum\limits_{j=1}^{n-1} (t_j-1)$ volte
			i=i-1;
		}
		A[i+1] = k;		$\Theta(1)$ $n-1$ volte
	}
\end{lstlisting}
\begin{table}[h]
	\begin{tabular}{ |c|c|c|c|c|c| }
		\hline
		0 & 1 & 2 & 3 & 4 & 5 \\
		\hline
		5 & 2 & 4 & 6 & 1 & 3 \\
		\hline 
		5 & 2 & 4 & 6 & 1 & 3 \\
		\hline 
		5 & 5 & 4 & 6 & 1 & 3 \\
		\hline 
		2 & 5 & 4 & 6 & 1 & 3 \\
		\hline 
		2 & 5 & 4 & 6 & 1 & 3 \\
		\hline 
		2 & 5 & 5 & 6 & 1 & 3 \\
		\hline 
		2 & 4 & 5 & 6 & 1 & 3 \\
		\hline 
		2 & 4 & 5 & 6 & 1 & 3 \\
		\hline 
		2 & 4 & 5 & 6 & 1 & 3 \\
		\hline 
	\end{tabular}
	\begin{tabular} { |c|c|c|c|}
		\hline
		j & i & k & while \\
		\hline
		0 & 0 & 0 & no \\
		\hline
		1 & 0 & 2 & si \\
		\hline
		1 & -1 & 2 & no \\
		\hline
		1 & -1 & 2 & no \\
		\hline
		2 & 1 & 4 & si \\
		\hline
		2 & 0 & 4 & no \\
		\hline
		2 & 0 & 4 & no \\
		\hline
		3 & 2 & 6 & no \\
		\hline
		3 & 2 & 6 & no \\
		\hline
	\end{tabular}
	\caption{Esempio di esecuzione}
\end{table}
\textbf{Complessità} bla bla bla\\ %TODO Inserisci il calcolo della complessità
\textbf{Correttezza}:
\begin{itemize}
	\item dimostro l'\textbf{invariante di ciclo} per assicurarmi che la mia proprietà venga mantenuta durante tutta l'esecuzione. Lo faccio tramite \emph{induzione}:
	\begin{itemize}
		\item Caso base: per $j=1$
		\item Hp induttiva: per $j=n'$
		\item Passo induttivo: dimostro che vale anche per $j=n'+1$
	\end{itemize}
	\item verifico la \textbf{terminazione}: il \emph{for} è eseguito esattamente $n-1$ volte e il \emph{while} al più $j-1$ volte, quindi tutte le iterazioni sono finite e l'algoritmo termina.
\end{itemize}
\textbf{Memoria impiegata}: ordina in loco quindi non usa memoria aggiuntiva.

\subsubsection{Selection sort}
\textbf{Proprietà}: al termine del passo j-esimo dell'algoritmo i primi $j+1$ elementi di A sono ordinati e contengono i $j+1$ elementi più piccoli di A.
\begin{lstlisting}[language=Javascript, caption=Algoritmo selection sort, mathescape=true]
	insertionSort(A) =
	var j:Int = 0;
	var i:Int = 0;		$\Theta(1)$
	var min:int = 0;
	for (i=0; i<n-1; i++) {		$n-1$ volte
		min = i;		$\Theta(1)$ $n-1$ volte
		for(j=i+1; j<n; j++) {
			if A[j] < A[min] {min = j};			$\Theta(1)$ $\sum\limits_{j=1}^{n-1} (t_j-1)$ volte
		}
		swap(A[i],A[min]);		$\Theta(1)$ $n-1$ volte
	}
\end{lstlisting}
\begin{table}[h]
	\begin{tabular}{ |c|c|c|c|c|c| }
		\hline
		0 & 1 & 2 & 3 & 4 & 5 \\
		\hline
		5 & 2 & 4 & 6 & 1 & 3 \\
		\hline 
		1 & 2 & 4 & 6 & 5 & 3 \\
		\hline 
		1 & 2 & 4 & 6 & 5 & 3 \\
		\hline 
		1 & 2 & 3 & 6 & 5 & 4 \\
		\hline 
		1 & 2 & 3 & 4 & 5 & 6 \\
		\hline 
		1 & 2 & 3 & 4 & 5 & 6 \\
		\hline
	\end{tabular}
	\begin{tabular} { |c|c|c|}
		\hline
		j & i & min \\
		\hline
		0 & 0 & 0 \\
		\hline
		1 & 0 & 4 \\
		\hline
		2 & 1 & 1 \\
		\hline
		3 & 2 & 5 \\
		\hline
		4 & 3 & 3 \\
		\hline
		5 & 4 & 4 \\
		\hline
	\end{tabular}
	\caption{Esempio di esecuzione}
\end{table}
\textbf{Complessità}
%TODO Inserisci il calcolo della complessità
\begin{equation}
	\sum\limits_{j=1}^{n-1} j = \frac{n(n-1)}{2} \in O(n^2)
\end{equation}
\begin{itemize}
	\item Caso pessimo: $O(n^2)$
	\item Caso migliore: $O(n^2)$
	\item Caso medio: $O(n^2)$
\end{itemize}
\textbf{Correttezza}:
\begin{itemize}
	\item dimostro l'\textbf{invariante di ciclo} per assicurarmi che la mia proprietà venga mantenuta durante tutta l'esecuzione. Sempre tramite induzione.
	\item verifico la \textbf{terminazione} in maniera analoga all'insertion sort.
\end{itemize}
\textbf{Memoria impiegata}: ordina in loco quindi non usa memoria aggiuntiva.