\documentclass[a4paper,10pt]{article}
\usepackage[utf8]{inputenc}

% ----  Useful packages % ---- 
\usepackage{amsmath}
\usepackage{graphicx}
\usepackage{amsfonts}
\usepackage{amsthm}
\usepackage{amssymb}
% ----  Useful packages % ---- 

\usepackage{wrapfig}
\usepackage{caption}
\usepackage{subcaption}
\usepackage{hyperref}
\hypersetup{
    colorlinks,
    citecolor=black,
    filecolor=black,
    linkcolor=black,
    urlcolor=black
}

\graphicspath{ {./images/} }

% ---- Set page size and margins replace ------
\usepackage[letterpaper,top=2cm,bottom=2cm,left=3cm,right=3cm,marginparwidth=1.75cm]{geometry}
% ---- Set page size and margins replace ------

% ------- NOTA ------
\theoremstyle{remark}
\newtheorem{note}{Note}[subsection]
% ------- NOTA ------

% ------- OSSERVAZIONE ------
\theoremstyle{definition}
\newtheorem{observation}{Osservazione}[subsection]
% ------- OSSERVAZIONE ------

% ------- DEFINIZIONE ------
\theoremstyle{plain}
\newtheorem{definition}{Definizione}[subsection]
% ------- DEFINIZIONE ------

% ------- ESEMPIO ------
\theoremstyle{definition}
\newtheorem{example}{Esempio}[subsection]
% ------- ESEMPIO ------

% ------- DIMOSTRAZIONE ------
\theoremstyle{definition}
\newtheorem{demostration}{Dimotrazione}[subsection]
% ------- DIMOSTRAZIONE ------

% ------- TEOREMA ------
\theoremstyle{definition}
\newtheorem{theorem}{Teorema}[subsection]
% ------- TEOREMA ------

% ------- COROLLARIO ------
\theoremstyle{plain}
\newtheorem{corollaries}{Corollario}[theorem]
% ------- COROLLARIO ------

% ------- PROPOSIZIONE ------
\theoremstyle{plain}
\newtheorem{proposition}{Proposizione}[subsection]
% ------- PROPOSIZIONE ------

% ---- Footer and header ---- 
\usepackage{fancyhdr}
\pagestyle{fancy}
\fancyhf{}
\fancyhead[LE,RO]{A.A 2020-2021}
\fancyhead[RE,LO]{Analisi Matematica}
\fancyfoot[RE,LO]{\rightmark}
\fancyfoot[LE,RO]{\thepage}

\renewcommand{\headrulewidth}{.5pt}
\renewcommand{\footrulewidth}{.5pt}
% ---- Footer and header ---- 

% ----  Language setting ---- 
\usepackage[italian, english]{babel}
% ----  Language setting ---- 

\title{\textbf{Analisi Matematica}}
\author{Realizzato da: Giuntoni Matteo}
\date{A.A. 2021-2022}

\begin{document}
\tableofcontents
\newpage
\maketitle
\begin{center}
    \vspace{-20pt}
    \rule{11cm}{.1pt} 
\end{center}

% !TeX spellcheck = it_IT
\section{Introduzione}
\subsection{Sistemi di equazioni}
L'algebra lineare è lo studio delle soluzioni di sistemi di equazioni lineari utilizzando spazi vettoriali.
\begin{example}
Un esempio di sistemi di equazioni:
\begin{enumerate}
    \item
    $\begin{rcases*}
    	E_1: x + y = 5 \\ E_2: x + 2y = 6
    \end{rcases*}
	\Rightarrow E_2 - E_1$ (sostituzione):
	$\begin{cases}
		y = 5 - 3 = 2 \\ x = 3 - 2 = 1
	\end{cases}$ 
	Un unica soluzione.
    \item 
    $\begin{rcases*}
    	E_1: x + y = 3 \\ E_2: 2x + 2y = 6
    \end{rcases*}
	\Rightarrow E_2 - 2E_1$: 0 = 0.\\\\
    Infatti $E_2 = 2E_1 \Rightarrow$ hanno le stesse soluzioni $\Rightarrow$ $\exists \infty$ soluzioni.
    \item 
    $\begin{rcases*}
    	E_1: x + y = 3 \\ E_2: 2x + 2y = 5
    \end{rcases*}
	\Rightarrow E_2 - 2E_1: 0 = -1$ è impossibile infatti $\nexists$ soluzioni comuni.
\end{enumerate}
Possiamo vedere da questi esempi che abbiamo tre possibili risultati: 1 soluzione, $\infty$ e 0.
\end{example}

\subsection{Interpretazioni geometrica}
In ogni caso le equazioni $E_1$ ed $E_2$ rappresentano rette su un piano a 2 dimensioni. Le soluzioni comuni sono i punti di intersezione delle rette. \\Nel caso specifico dell'esempio 1.1.1 abbiamo che:
\begin{figure}[h!]
    \centering
    \begin{subfigure}{.3\textwidth}
        \centering
        \includegraphics[width=3cm]{images/rette-incidenti.png}
        \caption{1° hanno un punto in comune P=(1,2)}
    \end{subfigure}
    \hfill
    \begin{subfigure}{.3\textwidth}
        \centering
        \includegraphics[width=3cm]{images/rette-coincidenti.png}
        \caption{2° coincidono $\Rightarrow \infty$ punti in comune}
    \end{subfigure}
    \hfill
    \begin{subfigure}{.3\textwidth}
        \centering
        \includegraphics[width=3cm]{images/rette-parallele.png}
        \caption{3° sono parallele  $\Rightarrow \nexists$ punti in comune}
    \end{subfigure}
\end{figure}
\newpage
\subsection{Equazioni a 3 variabili}
Un esempio di equazione a 3 variabili è $x + 2y + 3z = 4$. Ciò crea, invece di una retta, un piano nello spazio 3-dimensionale.
Se adesso consideriamo le equazioni viste sopra $E_1$ ed $E_2$ come equazioni a 3 variabili possiamo vedere che esse corrispondono a 2 piani nello spazio ed i punti in comune formano una retta.\\
\begin{figure}[h!]
    \centering
    \begin{subfigure}{.3\textwidth}
        \centering
        \includegraphics[width=2.5cm]{images/piani-incidenti.png}
        \caption{1° forma una retta}
    \end{subfigure}
    \begin{subfigure}{.3\textwidth}
        \centering
        \includegraphics[width=2.5cm]{images/piani-coincidenti.png}
        \caption{2° i due piani coincidono}
    \end{subfigure}
    \begin{subfigure}{.3\textwidth}
        \centering
        \includegraphics[width=2.5cm]{images/piani-incidenti.png}
        \caption{3° i due piani sono paralleli}
    \end{subfigure}
\end{figure}
Se oltre a $E_1$ ed $E_2$ consideriamo una terza equazione $E_3$ essa corrisponde ad un terzo piano. \\
Possiamo vedere come esso si comporta intersecandolo con l'intersezione fra $E_1$ ed $E_2$, $E_1 \cap E_2$.
\begin{figure}[h!]
    \centering
    \begin{subfigure}{.3\textwidth}
        \centering
        \includegraphics[width=2.5cm]{piano-incotra-retta.png}
        \caption{$E_1 \cap E_2$ è una retta che, intersecata con $E_3$, crea un punto}
    \end{subfigure}
    \begin{subfigure}{.3\textwidth}
        \centering
        \includegraphics[width=2.5cm]{piano-coincide-retta.png}
        \caption{$E_1 \cap E_2$ può essere contenuto in $E_3$ quindi nuova retta}
    \end{subfigure}
    \begin{subfigure}{.3\textwidth}
        \centering
        \includegraphics[width=2.5cm]{piano-non-conindice-retta.png}
        \caption{$E_1 \cap E_2$ e $E_3$ possono non coincidere}
    \end{subfigure}
\end{figure}

\subsection{Caso generale}
Possiamo definire un sistema $(E)$ di $n$ equazioni a $m$ variabili con $n,m > 0$ e con $a_{nm}, b_{n} \in \mathbb{R}$ come:
\begin{flalign}\nonumber
&E_1: a_{11}x_1 + a_{12}x_2 + \ldots + a_{1m}x_m = b_1&&\\\nonumber
&E_2: a_{21}x_1 + a_{22}x_2 + \ldots+ a_{2m}x_m = b_2&&\\\nonumber
&\vdots&&\\
&E_n: a_{n1}x_1 + a_{n2}x_2 + \ldots + a_{nm}x_m = b_n&&\nonumber
\end{flalign}

\begin{definition}[Sistema omogeneo]
Il sistema $(E)$ è \textbf{omogeneo} se $b_1 = \ldots = b_n = 0$. In caso contrario possiamo considerare il sistema omogeneo associato ($E_{om}$) definito come:
\begin{flalign}\nonumber
&E_1: a_{11}x_1 + a_{12}x_2 + \ldots + a_{1m}x_m = 0&&\\\nonumber
&E_2: a_{21}x_1 + a_{22}x_2 + \ldots + a_{2m}x_m = 0&&\\\nonumber
&\vdots&&\\\nonumber
&E_n: a_{n1}x_1 + a_{n2}x_2 + \ldots + a_{nm}x_m = 0&&\nonumber
\end{flalign}
Se $(E)$ è \textbf{omogeneo}, $\exists$ sempre una soluzione comune del tipo $(x_1, \ldots, x_n) = (0_1, \ldots, 0_n)$.
\end{definition}

\begin{proposition}\label{prop-1}
Se $(c_1, ..., c_n)$ e $(d_1, ..., d_n)$ sono soluzioni di $(E)$ $\Longrightarrow$ $c_1 - d_1, ..., c_n - d_n$ è soluzione del sistema omogeneo.
\end{proposition}

\begin{demostration}
Se $(c_1, \ldots, c_m)$ è soluzione vuol dire che :\\
$E_1:a_{i1}c_1 + a_{i2}c_2 + a_{im}c_m = b_i$\\
$E_2:a_{i1}d_1 + a_{i2}d_2 + a_{im}d_m = b_i$\\
Quindi se sottraggo $E1 - E2$ e raccolgo viene:\\
$a_{i1}(c_1 - d_1) + a_{i2}(c_2 - d_m) + a_{im}(c_m - d_m)= 0\:\: \forall \: i,...,n$
\end{demostration}

\begin{theorem}
Se $(c_1,\ldots,c_m)$ è soluzione del sistema $(E)$ tutte le soluzioni $(E)$ sono della forma $(c_1 + e_1, c_2 + e_2, \ldots, c_m + e_m)$ dove $(e_1,\ldots,e_m)$ è soluzione di $E_{om}$.
\end{theorem}
In sinestesi si può semplificare questo teorema scrivendo:
\begin{equation}
    \text{"Soluzione generale" = "Soluzione particolare" + "Soluzione omogenea"}
\end{equation}

\begin{demostration}
La proposizione \ref{prop-1} dice che le soluzioni hanno questa forma. Viceversa se $(e_1,\ldots,e_m)$ sono soluzioni di ($E_{om}$) $\Longrightarrow$ $(c_1 + e_1, c_2 + e_2, \ldots, c_m + e_m)$ sono soluzioni di $(E)$.
\end{demostration}

\begin{example}
Prendiamo n=1 e m=2 e prendiamo come sistema di equazioni $(E): 2x + 3y = 5$ e come equazione omogenea $(E_{om}): 2x + 3y = 0$\\\\
Vediamo che le soluzioni particolari sono $x = y = 1$. Per calcolare le soluzioni omogenee si fa $2x = -3y$ e poi $x = -\frac{3}{2}y$, qui per ogni valore di y trovo un valore di x. \\
La soluzioni omogenea è $(-\frac{3}{2}p, p)$ dove p è un parametro che può essere qualsiasi valore.\\
Sappiamo che "sol. generale" = "sol. particolare" + "sol. omogenea" $\Rightarrow (1,1) + (-\frac{3}{2}t,t) = (1 - \frac{3}{2}t, 1 + t)$.
\end{example}

\begin{observation}
$(0,...,0)$ è sempre soluzione di $(E_{om})$. Quindi se (E) ammette una soluzione questo soluzione è unica $\Longleftrightarrow (0,...,0)$ è l'unica soluzione di $(E_{om})$.
\end{observation}

\subsection{Interpretazione geometrica caso generico}
L'interpretazione geometrica per ($E_{om}$) è un iperpiano attraverso l'origine", e la soluzione è traslazione di questo caso generale per un caso particolare.
\begin{enumerate}
    \item $n=1$, $m=2$ (E) $a_{1n}x_1 + a_{m2}x_2 = b_1$.\\
    Una soluzione $\Longleftrightarrow$ retta ($E_{om}$) $a_1x_1 + a_2x_2 = 0$ una soluzione a (E) $\Rightarrow$ retta attraverso (0,0).
    \item $n=1$, $m=2$, $a_{11}x_1 + a_{12}x_2 + a_{13}x_3 = a$ (E), punto attraverso (0,0,0).
\end{enumerate}

\subsection{Come trovare le soluzioni?}
Per trovare le soluzioni comuni di $(E)$ possiamo usare 3 operazioni per semplificare il sistema:
\begin{enumerate}
    \item Scambiare due equazioni.
    \item Moltiplicare $E_i$ per $\lambda \neq 0$ e fare la somma con $E_j$, $E_j = E_j + \lambda E_i$.
    \item Moltiplicare un'equazione $E_i$ per un costate $\lambda \neq 0$, $E_i \Rightarrow \lambda E_i$.
\end{enumerate}

\begin{observation}
Queste operazioni non cambiano l'insieme delle soluzioni di $(E)$.
\end{observation}

\begin{demostration}
Dimostriamo le 3 proprietà:
\begin{enumerate}
    \item La prima è ovvia quindi non ha bisogno di una dimostrazione.
    \item Se ($c_1, \ldots, c_n$) soluzioni di $E_i$ ed $E_j \Rightarrow$ è anche soluzione di $E_i + \lambda E_j$.\\
    Viceversa se ($c_1, \ldots, c_n$) soluzioni di $E_i$, $E_j + \lambda E_i \Rightarrow$ anche soluzione di ($E_j + \lambda E_i$) - $\lambda = E_j$.
    \item Se ($c_1, \ldots, c_n$) soluzioni di (E) $\Rightarrow$ anche di $\lambda E$ e viceversa.
\end{enumerate}
\end{demostration}


\newpage
\section{Funzioni}
\begin{definition}[Funzione]
- $f: A \longrightarrow B$: \\
Dati due insiemi $A$, $B$, detti dominio e codominio, una funzione è una "legge" o "regola" che associa ad ogni elemento di $A$ uno ed uno solo elemento di $B$.
\end{definition}
\begin{note}
Tipicamente in questo corso le funzioni saranno date come formule del tipo $f(x) = x^2 - 7x - e^x$ specificando dominio e codominio in questo modo $f: \mathbb{R} \longrightarrow \mathbb{R}$. Si noti che la definizione di una funzione \textbf{deve} includere sia la funzione che il suo dominio. Ad esempio $f: \mathbb{R} \longrightarrow \mathbb{R} \: f(x)=x^2$ e $g: \mathbb{R} \geq 0 \longrightarrow \mathbb{R} g(x)=x^2$ sono due funzioni diverse.
\end{note}
\begin{note}
	Se non vengono specificati dominio e codominio allora il dominio è il sottoinsieme più grande di $\mathbb{R}$ in cui la formula ha senso. Per la funzione $f(x)=\frac{1}{x}$ il dominio è ${x \in \mathbb{R} \mid x \neq 0}$.
\end{note}
\begin{example}
    Esempi funzioni:
    \begin{itemize}
        \item $g(x) = x^2 - 7x - e^x$ \hspace{.3cm} $g(0,+\infty) \longrightarrow \mathbb{R}$
        \item $g(x) = x^2$ \hspace{.3cm} $g(0, +\infty) \longrightarrow (0, +\infty)$. \hspace{.3cm}Va bene perché $x^2 > 0$ per qualsiasi valore di x.
        \item $h(x) = x^2$ \hspace{.3cm} $h(0, +\infty) \longrightarrow (-\infty, 0)$. \hspace{.3cm}Questa forma non va bene non definendo una funzione perché la formula non mi da numeri di $(-\infty, 0)$.
        \item $h(x) = x^2$ \hspace{.3cm} $h(0, +\infty) \longrightarrow (-\infty,1)$ \hspace{.3cm}Non va bene perché se prendiamo x=3 $f(3) = 9$ e 9 non fa parte del codominio. 
    \end{itemize}
\end{example}

\subsection{Grafico}
Una funzione $f: A \longrightarrow B$ con $A,B \in \mathbb{R}$ ha un \textbf{grafico} che si indica come:
\begin{equation}
    graph(f) = \{(a,b) \in A \times B\ \mid b = f(a)\}
\end{equation}
\begin{wrapfigure}[8]{l}{7cm}
    \centering
    \includegraphics[width=4.5cm, height=4cm]{Esempio-grafico.png}
    \caption{$f(x) = x^2$ con $f: \mathbb{R} \longrightarrow \mathbb{R}$}
    \label{fig:esempio-grafico}
\end{wrapfigure}
\begin{example}
Esempio punto sulla funzione
\begin{itemize}
    \item Il punto A sta sul grafico si $f(x) = x^2$ esattamente quando $y = x^2$.
    \item Il punto B non sta sul grafico quindi $y \neq x^2$.
\end{itemize}
\end{example}
\begin{note}
    A X B $\subseteq \mathbb{R}$ X $\mathbb{R}$. Dove R X R = $R^2$.
\end{note}
\begin{example}
A e B = $(0, +\infty)$, da qui vediamo che A X B rappresenta il primo quadrante.\\\\
\end{example}

\subsection{Immagine}
\begin{definition}[Immagine]
Prendendo $f: A \longrightarrow B$ e $D \subseteq A$ l'immagine di D tramite f è il sottoinsieme $f(D) \subseteq B$ costituito dagli elementi f(d) dove $d \in D$.
\end{definition}
\begin{example}
    Esempi immagine:
    \begin{itemize}
        \item Immagine di A, $f(A) \subseteq B$ si chiama anche immagine della funzione.
        \item $f(x) = x^2$, $f: \mathbb{R} \longrightarrow \mathbb{R}$ \hspace{.2cm} immagine di g è $[0, +\infty)$ perché $x^2 \geq 0 \: \forall \: x \in \mathbb{R}$.
        \item $g(x) = x?2$, $g:[2, +\infty) \longrightarrow \mathbb{R}$ \hspace{.2cm} l'immagine di g è $[4, +\infty]$ perché se si calcola il punto minore del dominio, cioè 2, torna $g(2) = x^2$ che è uguale a 4, da lì possiamo prendere tutti i punti.
    \end{itemize}
\end{example}

\subsection{Suriettiva}
\begin{definition}[Suriettiva]
Una funzione si dice suriettiva quando ogni elemento del codominio è immagine di almeno un elemento del dominio. Quindi prendendo una f(x), per che sia suriettiva deve l'immagine I essere uguale ad un valore, $I(f) = b$.
\end{definition}
\begin{equation}
	\forall y \in B \exists x \in A
\end{equation}
\begin{note}
	La suriettività si traduce graficamente nel fatto una qualsiasi retta orizzontale intersechi il grafico almeno una volta.
\end{note}
\begin{example}
    Esempi funzioni suriettive:
    \begin{itemize}
        \item $f(x) = x^2$, $f: \mathbb{R} \longrightarrow \mathbb{R}$ non è suriettiva perché tutti i valori del codominio $y < 0$ non hanno un rispettivo nel dominio.
        \item $g(x) = x^2$, $g: \mathbb{R} \longrightarrow (0, +\infty)$ lo è perchè andiamo a restringere il codominio ai punti che hanno un corrispettivo nel dominio.
    \end{itemize}
\end{example}

\subsection{Iniettiva}
\begin{definition}[Iniettiva]
Una funzione iniettiva è una funzione che associa, a elementi distinti del dominio, elementi distinti del codominio. Quindi prendendo una f(x) è iniettiva se prendendo due valori $x_1, x_2$ dove $x_1 \neq x_2 \Longrightarrow f(x_1) \neq f(x_2)$. (Input diversi danno output diversi).
\end{definition}
\begin{equation}
	x_{1}, x_{2} \in A \wedge x_{1} \neq x_{2} \implies f(x_{1}) \neq f(x_{2})
\end{equation}
\begin{note}
	L'iniettività si traduce graficamente nel fatto una qualsiasi retta orizzontale intersechi il grafico al più una volta.
\end{note}
\begin{example}
    Esempi funzioni iniettiva:
    \begin{itemize}
        \item $f(x) = x^2$, $f: \mathbb{R} \longrightarrow \mathbb{R}$ non è iniettiva perché se prendiamo $x_1 = 1$ e $x_2 = -1$ $f(x1) = f(x2).$
        \item $g(x) = x^2$, $g: [0, +\infty) \longrightarrow \mathbb{R}$ è invece iniettiva perché non consideriamo i valori negativi.
    \end{itemize}
\end{example}

\subsection{Biunivoca}
\begin{definition}[Biunivoca]
Una funzione si definisce biunivoca o bigettiva se è sia iniettiva che suriettiva.
\end{definition}

\subsection{Invertibile}
\begin{definition}[Invertibile]
Se una funzione è biunivoca si dice che tale funzione è anche invertibile.
\end{definition}
\begin{wrapfigure}{l}{6cm}
    \centering
    \includegraphics[width=5cm, height=4cm]{Esempio-invertibilita.png}
    \caption{$f(x) = x^2$ e $g(x) = \sqrt{x}$}
    \label{fig:esempio-invertibilità}
\end{wrapfigure}
Se f è una funzione invertibile i grafici di f e di $f^{-1}$ (la funzione inversa) sono simmetrici rispetto alla retta $y=x$ cioè alla bisettrice del primo e del terzo quadrante. \\
\begin{example}
Se vediamo nell'immagine [\ref{fig:esempio-invertibilità}] prendendo l'inverso della funzione $f(x) = x^2$ definita in $[0, +\infty] \longrightarrow \mathbb{R}$ e cioè la funzione $g(x) = \sqrt{x}$ è simmetrica.
\\ \\ \\ \\ \\
\end{example}

\subsection{Funzioni Monotone}
\begin{definition}[Monotone]
Dati $A, B \in \mathbb{R}$ e $f:A \longrightarrow B$. $x_1, x_2 \in A$ con $x_1 < x_2$ se $\forall x_1, x_2$ risulta ciò che è scritto in Tabella \ref{tab:monotone}.
\end{definition}
\begin{table}[h!]
    \centering
    \setlength{\tabcolsep}{6pt}
    \renewcommand{\arraystretch}{1.7}
    \begin{tabular}{|c|c|}
        \hline
        \textbf{[1] Strettamente Crescente} & $f(x_1) < f(x_2) $ \\ \hline
        \textbf{[2]Debolmente Crescente} & $f(x_1) \leq f(x_2) $ \\ \hline
        \textbf{[3]Strettamente Decrescente} & $f(x_1) > f(x_2) $ \\ \hline
        \textbf{[4]Debolmente Decrescente} & $f(x_1) \geq f(x_2) $ \\ \hline
    \end{tabular}
    \caption{Definizioni funzioni crescenti e decrescenti}
    \label{tab:monotone}
\end{table}
Andando a considerare la Tabella \ref{tab:monotone} possiamo dire che:
\begin{itemize}
    \item \textbf{Strettamente monotona} nei casi [1] e [3] della tabella.
    \item \textbf{Debolmente monotona} nei casi [2] e [4] della tabella.
\end{itemize}

\begin{observation}
	Se $f$ è \textbf{strettamente monotona} allora è \textbf{iniettiva} in quanto dati $x_{1} \neq x_{2}$ con $x_{1} < x_{2} \implies f(x_{1}) < f(x_{2})$ e in particolare $f(x_{1}) \neq f(x_{2})$. Non vale però il contrario, infatti una funzione iniettiva non è per forza strettamente monotona (ad esempio data $f(x)=\frac{1}{x}$ con $f:\mathbb{R} \setminus \{0\} \longrightarrow \mathbb{R} \setminus \{0\}$ )
\end{observation}

\begin{observation}
	Se $f$ è \textbf{strettamente crescente} allora è anche \textbf{debolmente crescente}.
\end{observation}
\begin{example}
    Esempi funzioni crescenti e decrescenti:\\
    \begin{figure}[h!]
        \begin{subfigure}{.5\textwidth}
            \centering
            \includegraphics[width=6cm, height=4cm]{funzione-crescente.png}
            \caption{$f(x_1) < f(x-2)$ quindi è crescente}
            \label{fig:funzione-crescente}
        \end{subfigure}
        \begin{subfigure}{.5\textwidth}
            \centering
            \includegraphics[width=6cm, height=4cm]{funzione-decrescente.png}
            \caption{$f(x_1) > f(x-2)$ quindi è decrescente}
            \label{fig:funzione-decrescente}
        \end{subfigure}
    \end{figure}
    \\Possiamo anche federe dalle immagini [\ref{fig:funzione-crescente}] [\ref{fig:funzione-decrescente}] che:
    \begin{itemize}
        \item Se f(x) è \textbf{crescente} l'ordinamene verrà \textbf{mantenuto}.
        \item Se f(x) è \textbf{decrescente} l'ordinamento verrà \textbf{invertito}.\\
    \end{itemize}
\end{example}

\newpage
\begin{observation}
    Osservazione sul rapporto incrementale:\\
\end{observation}
\begin{wrapfigure}[8]{l}{8cm}
    \vspace{-15pt}
    \centering
    \includegraphics[width=6.7cm]{rapporto_incrementale.png}
    \caption{$\frac{\Delta_y}{\Delta_x}$}
    \label{fig:esempio-invertibilità}
\end{wrapfigure}
Definito il \textbf{rapporto incrementale}\footnote{I rapporto incrementale misura quanto il punto della f si sposta in verticale in rapporto a quanto abbiamo l'asciasse in orizzontale.} come:
\begin{equation}
    \frac{\Delta_y}{\Delta_x}=\frac{f(x_1) - f(x_2)}{x_1 - x_2}
\end{equation}

\begin{note}
    Il denominatore ed il numeratori devono essere concordi per fare in modo che il rapporto incrementale sia maggiore di 0 e quindi la funzione crescente. \\ \\\\
\end{note}
Continuando ad analizzare il rapporto incrementale possiamo ricavare anche i casi in cui una funzione e strettamente decrescente o debolmente crescente o debolmente decrescente. Puoi vedere tutte le casistiche nella tabella \ref{tab:analisi-rapporto-incrementale}.
\begin{table}[h!]
    \centering
    \setlength{\tabcolsep}{7pt}
    \renewcommand{\arraystretch}{2}
    \begin{tabular}{|c|c|}
        \hline
        Strettamente Crescente & $\frac{f(x_1) - f(x_2)}{x_1 - x_2} > 0$\\ \hline
        Strettamente Decrescente & $\frac{f(x_1) - f(x_2)}{x_1 - x_2} < 0$ \\ \hline
        Debolmente Crescente & $\frac{f(x_1) - f(x_2)}{x_1 - x_2} \geq 0$ \\ \hline
        Debolmente Decrescente & $\frac{f(x_1) - f(x_2)}{x_1 - x_2} \leq 0$ \\ \hline
    \end{tabular}
    \caption{Analisi rapporto incrementale}
    \label{tab:analisi-rapporto-incrementale}
\end{table}
\begin{observation}
    Se una funzione f(x) è strettamente crescente è a sua volta anche debolmente crescente, mentre una funzione f(x) se è debolmente crescente non è strettamente crescente perché aggiunge una casistica che sarebbe $f(x_1) = f(x_2)$. 
\end{observation}
\begin{example}
    Casistica particolare:\\
    Data $f(x)=\frac{1}{x}$, \hspace{.3cm} $f: \mathbb{R} \: \setminus \: \{0\} \longrightarrow \mathbb{R} \: \setminus \: \{0\}$. Funzione rappresentata nell'immagine [\ref{fig:esempio-particolare}].
    \begin{figure}[h!]
        \centering
        \includegraphics[width=8.7cm]{esempio-particolare.png}
        \caption{$f(x)=\frac{1}{x}$, \hspace{.3cm} $f: \mathbb{R} \: \setminus \: \{0\} \longrightarrow \mathbb{R} \: \setminus \: \{0\}$}
        \label{fig:esempio-particolare}
    \end{figure}
    \\Possiamo vedere che:
    \begin{itemize}
        \item f(x) è strettamente decrescente in $(0, +\infty)$.\\
        Quindi se andiamo a prendere $0 < x_3 < x_4$ abbiamo che $f(x_3) > f(x_4)$.
        \item f(x) è strettamente decrescente in $(-\infty, 0)$.\\
        Quindi se andiamo a prendere $x_1 < x_2 < 0$ abbiamo che $f(x_1) > f(x_2)$.
    \end{itemize}
    Se però andiamo a considerare tutto $\mathbb{R} \: \setminus \: \{0\}$, e quindi prendiamo i punti $x_1 < 0 < x_4$ vediamo che $f(x_1) < f(x_4)$.
    In conclusione si può dire quindi che $f(x)=\frac{1}{x}$) è decrescente in $(-\infty, 0)$ e in $(0, +\infty)$ ma non lo è in tutto $\mathbb{R} \: \setminus \: \{0\}$.
\end{example}

\subsubsection{Composizione con funzioni monotone}
\begin{definition}[Composizione]
	La composizione di funzioni si definisce come $g \circ f:A \longrightarrow C$, $(g \circ f)(a)=g(f(a))$
\end{definition}
Prendendo i considerazioni 3 insiemi A, B, C tali che $A, B, C \subset \mathbb{R}$ e 2 funzioni f(x) e g(x) così definite: \hspace{.2cm} $f: A \longrightarrow B$, $g: B \longrightarrow C $.
\begin{enumerate}
    \item Se f è crescente e g è crescente allora $g \circ f$ è crescente.
    \item Se f è crescente e g è decrescente allora $g \circ f$ è decrescente e viceversa ($ x_{1} < x_{2} \implies f(x_{1}) < f(x_{2}) \implies g(f(x_{1})) < g(f(x_{2})) $).
    \item Se f è decrescente e g è decrescente allora $g \circ f$ è crescente ($ x_{1} < x_{2} \implies f(x_{1}) > f(x_{2}) \implies g(f(x_{1})) < g(f(x_{2})) $).
\end{enumerate}

\begin{example}
    $h(x) = e^{x^3}$\\
    La funzione $h$ si ottiene dalla composizione di:
    \begin{itemize}
        \item $f: \mathbb{R} \longrightarrow \mathbb{R}$ \hspace{.3cm} $f(x) = x^3$. Funzione crescente.
        \item $g: \mathbb{R} \longrightarrow \mathbb{R}$ \hspace{.3cm} $g(t) = e^t$. Funzione decrescente.
    \end{itemize}
    Quindi possiamo scrivere $h(x) = e^{x^3}$ come: \hspace{.3cm} $e^{f(x)} \: \: = \: \: g(f(x)) \: \: = \: \: (g \circ f)(x)$
    Inoltre visto che f è crescente e g è crescente, h è strettamente crescente 
\end{example}
\begin{observation}
    Se prendiamo una funzione f(x) strettamente monotona, allora f(x) è iniettiva. Questa condizione è vera ma NON lo è viceversa: una funzione f(x) iniettiva NON è per forza strettamente monotona. 
\end{observation}
\begin{example}
    Se prendiamo una f(x) tale che: \hspace{.3cm} $f(x) = \frac{1}{x}$ \hspace{.3cm} $\mathbb{R} \setminus \{0\} \longrightarrow \mathbb{R} \setminus \{0\}$
    \\Possiamo vedere rifacendoci all'esempio in figura [\ref{fig:esempio-particolare}] che f è iniettiva ma non monotona.
\end{example}

\subsection{Insieme di definizione}
\begin{definition}[Insieme di definizione]
    Data una funzione f(x) l'insieme di definizione o dominio naturale di una funzione è il più grande sottoinsieme di $\mathbb{R}$ dove ha senso la funzione f(x).
\end{definition}
\begin{example}
    $f(x) = \frac{1}{x}$ \hspace{.3cm} L'insieme di definizione è $\mathbb{R} \setminus \{0\}$
\end{example}

\subsection{Funzioni pari e dispari}
\begin{definition}[Pari]
    La funzione è \textbf{pari} se $f(x) = f(-x) \: \forall x$ nel dominio di $f \longrightarrow f$. Il grafico di una funzione pari è simmetrico rispetto all'asse $y$.
\end{definition}
\begin{definition}[Dispari]
    La funzione è \textbf{dispari} se $f(x) = -f(-x) \: \forall x$ nel dominio di $f \longrightarrow f$. Il grafico di una funzione dispari è simmetrico rispetto all'origine.
\end{definition}
\begin{note}
    Il dominio di $f$ deve essere simmetrico.\\
\end{note}
\begin{example}
Esempio funzioni pari e dispari.\\
\end{example}
$f(-x) = (-x)^2 = x^2 = f(x)$, f(x) è \textbf{pari}: \hfill $f(-x) = (-x)^2 = x^2 = -f(x)$, f(x) è \textbf{dispari}:
\begin{figure}[h!]
    \vspace{-1pt}
    \begin{subfigure}{.5\textwidth}
        \centering
        \includegraphics[width=3cm]{funzione-pari.png}
        \caption{$f(x) = x^2$, \hspace{.2cm} graph(f) con f pari}
    \end{subfigure}
    \begin{subfigure}{.5\textwidth}
        \centering
        \includegraphics[width=2.5cm]{funzione-dispari.png}
        \caption{$f(x) = x^3$, \hspace{.2cm} graph(f) con f dispari}
    \end{subfigure}
\end{figure}
\begin{note}
	Una funzione del tipo $f(x)=x^(2n)$ con $n \in \mathbb{N}$ è sempre pari mentre una funzione del tipo $f(x)=x^(2n+1)$ con $n \in \mathbb{N}$ è sempre dispari.
\end{note}

\subsection{Funzione periodica}
\begin{definition}[Periodicità]
    Una funzione f(x) si dice periodica di periodo $P \in \mathbb{R}$ se $\forall x \: \: f(x + P) = f(x)$. 
\end{definition}
\begin{wrapfigure}{r}{9cm}
    \vspace{-15pt}
    \centering
    \includegraphics[width=7cm]{funzione-periodica.png}
    \caption{$sin(x) = sin(x + 2\pi)$}
    \label{fig:funzione-periodica}
\end{wrapfigure}
Inoltre il dominio di f(x) deve essere tale che $x \in dom(f) \implies x + P \in dom(f)$.
\begin{example}
In figura [\ref{fig:funzione-periodica}] un esempio di funzione periodica.\\\\
\end{example}

\subsection{Funzioni Elementari}
\subsubsection{Lineari}
\textbf{Funzione retta}: $f(x) = ax + b$. \hspace{.3cm} $a,b \in \mathbb{R}$ \\ Dove $a$ (coefficiente angolare) indica la pendenza della retta, mentre $b$ (termine noto) indica il punto di incontro con l'asse $Y$.

\subsubsection{Esponente positivo o negativo}
\textbf{Fun. Esp. positivo:} $f(x) = x^k$, $k \in \mathbb{N}$. \hfill \textbf{Fun. Esp. negativo:} $f(x) = x^k$, $k \in \mathbb{N}$, $k < 0$.
\begin{figure}[h!]
    \begin{subfigure}{.5\textwidth}
        \centering
        \includegraphics[width=4cm, height=3.5cm]{parabole.png}
        \caption{con k pari}
        \label{fig:esponente-positivo-pari}
    \end{subfigure}
    \begin{subfigure}{.5\textwidth}
        \centering
        \includegraphics[width=4cm, height=3.5cm]{esponente-negativo-dispari.png}
        \caption{con k dispari}
        \label{fig:esponente-positivo-dispari}
    \end{subfigure}
\end{figure}
\begin{figure}[h!]
    \vspace{-5pt}
    \begin{subfigure}{.5\textwidth}
        \centering
        \includegraphics[width=4cm, height=3.7cm]{esponente-dispari.png}
        \caption{con k pari}
        \label{fig:esponente-negativo-pari}
    \end{subfigure}
    \begin{subfigure}{.5\textwidth}
        \centering
        \includegraphics[width=4cm, height=3.5cm]{esponsente-negativo-pari.png}
        \caption{con k dispari}
        \label{fig:esponente-negativo-dispari}
    \end{subfigure}
\end{figure}
\begin{observation}
    \textbf{k pari}: Le funzioni con il $k$ pari sono funzioni pari e hanno tutte una forma simile a quella in figura [\ref{fig:esponente-positivo-pari}] per le funzioni con k positive e per le funzioni con k negativo figura [\ref{fig:esponente-negativo-pari}].
\end{observation}
\begin{observation}
    \textbf{k dispari:} Le funzioni con il k positivo e dispari sono funzioni dispari e hanno tutte una forma simile a quella in figura [\ref{fig:esponente-positivo-dispari}] per le funzioni con k positive e per le funzioni con k negativo figura [\ref{fig:esponente-negativo-dispari}].
\end{observation}

\subsubsection{Radici o esponente fratto}
\textbf{Funzionane radici o esponente fratto:} $f(x) = x^{\frac{p}{q}}$ o $f(x) = \sqrt[q]{x^p}$  \: \: con  \: \:  $p, q \in \mathbb{N}$ \: \: e  \: \:  $q \neq 0$. \footnote{In matematica è possibile scrivere una un esponente fratto come radice mettendo il numeratore al radicando della radice e il denominatore all'indice: $x^{\frac{p}{q}} \: = \: \sqrt[q]{x^p}$}
\begin{figure}[h!]
    \begin{subfigure}{.5\textwidth}
        \centering
        \includegraphics[width=6.3cm]{radice-pari.png}
        \caption{con q pari}
        \label{fig:radice-pari}
    \end{subfigure}
    \begin{subfigure}{.5\textwidth}
        \centering
        \includegraphics[width=6cm]{radice-dispari.png}
        \caption{con q dispari}
        \label{fig:radice-dispari}
    \end{subfigure}
\end{figure}
\begin{note}
    $p$ e $q$ non possono essere entrambi pari perché in tal caso sono divisibili fra di loro e quindi portabili ad una forma ridotta.
\end{note}
\begin{observation}
    \textbf{q pari:} Le funzioni con il $q$ pari ha dominio $ x \geq 0$ ed è invertibile sono come funzione $f: [0, +\infty) \longrightarrow [0, +\infty)$. È rappresentata in figura [\ref{fig:radice-pari}].
\end{observation}
\begin{observation}
    \textbf{q dispari:} Le funzioni con il $q$ positivo ha dominio $x \in \mathbb{R}$ ed è ugualmente invertibile su tutto $\mathbb{R}$, è inoltre una funzione dispari. È rappresentata in figura [\ref{fig:radice-dispari}].\\
\end{observation}

\subsubsection{Esponenziale}
\textbf{Funzione esponenziale:} $f(x) = a^x$ con $a \in \mathbb{R}$, \: \: $a > 0$, \: \: $a \neq 1$ \: \: $f: \mathbb{R} \longrightarrow (0, +\infty)$
\begin{figure}[h!]
    \begin{subfigure}{.5\textwidth}
        \centering
        \includegraphics[width=4cm]{esponenziale.png}
        \caption{con $a > 1$}
        \label{fig:esponenziale}
    \end{subfigure}
    \begin{subfigure}{.5\textwidth}
        \centering
        \includegraphics[width=4cm]{esponsenziale-base-minore.png}
        \caption{con $0 < a < 1$}
        \label{fig:esponsenziale-base-minore}
    \end{subfigure}
\end{figure}
\begin{note}
    La funzione esponenziale è sempre positiva.
\end{note}
\begin{observation}
    \textbf{$a > 1$:} La funzione è strettamente crescente, come in nell'immagine [\ref{fig:esponenziale}].
\end{observation}
\begin{observation}
    \textbf{$0 < a < 1$:} La funzione è decrescente, come in nell'immagine [\ref{fig:esponsenziale-base-minore}].
\end{observation}

\subsubsection{Logaritmo}
\textbf{Funzione logaritmo:} $f(x) = \log_a x$, \: \: $f: (0, +\infty) \longrightarrow \mathbb{R}$ \: \: (inversa dell'esponenziale).
\begin{figure}[h!]
    \begin{subfigure}{.5\textwidth}
        \centering
        \includegraphics[width=5cm,height=4cm]{logaritmo.png}
        \caption{con $a > 1$}
        \label{fig:logaritmo}
    \end{subfigure}
    \begin{subfigure}{.5\textwidth}
        \centering
        \includegraphics[width=5cm,height=4cm]{logaritmo-base-minore.png}
        \caption{con $0 < a < 1$}
        \label{fig:logaritmo-base-minore}
    \end{subfigure}
\end{figure}
\begin{observation}
    Casistica particolare - $f(x) = e^x$.\\
    In questa casistica se andiamo a ridurre il codominio la funzione esponenziale è invertibile. $f: \mathbb{R} \longrightarrow (0, +\infty)$.
    Il suo inverso è un caso particolare di logaritmo e di chiama \textbf{logaritmo naturale}. E si può scrive in due modi:
    \begin{itemize}
        \item $\ln{x}$: sarebbe logaritmo in base naturale.
        \item $\log x$: scrivendo il logaritmo senza la base intendiamo il logaritmo in base $e$.
    \end{itemize}
\end{observation}

\subsubsection{Seno e Arcoseno}
\textbf{Seno:} $f(x) = \sin x$, $f: \mathbb{R} \longrightarrow \mathbb{R}$. \hfill
\textbf{Arcoseno:} $f(x) = \arcsin x$, $f: [-1, 1] \longrightarrow [-\frac{\pi}{2}, \frac{\pi}{2}]$
\begin{figure}[h!]
    \begin{subfigure}{.5\textwidth}
        \centering
        \includegraphics[width=6cm]{seno.png}
        \caption{$\sin{x}$}
        \label{fig:seno}
    \end{subfigure}
    \begin{subfigure}{.5\textwidth}
        \centering
        \includegraphics[width=2cm, height=2.3cm]{arcoseno.png}
        \caption{$\arcsin{x}$ o $\sin{x}^{-1}$}
        \label{fig:arcoseno}
    \end{subfigure}
\end{figure}

\begin{observation}
    \textbf{Sin(x):} La funzione $\sin{x}$ (immagine [\ref{fig:seno}]) è periodica per $2\pi$ quindi possiamo scrivere $\sin{(x+2\pi)} = \sin x \: \forall x \in \mathbb{R}$. Inoltre è suriettiva per codominio [-1, 1]. Se invece definiamo $f: [-\frac{\pi}{2}, \frac{\pi}{2}] \longrightarrow [-1, 1]$ la funzione $\sin x$ è strettamente crescente e suriettiva, quindi anche invertibile, e la sua inversa è appunto $\arcsin{x}$.
\end{observation}
\begin{observation}
    \textbf{Arcsin(x):} La funzione $\arcsin{x}$ è l'inverso del seno e può essere scritta anche come $f(x) = \sin{x}^{-1}$, è rappresentata nell'immagine [\ref{fig:arcoseno}].
\end{observation}

\subsubsection{Coseno e Arcocoseno}
\textbf{Coseno:} $f(x) = \cos{x}$, $f: \mathbb{R} \longrightarrow \mathbb{R}$. \hfill
\textbf{Arcocoseno:} $f(x) =\arccos{x}$, $f: [-1, 1] \longrightarrow [0, \pi]$
\begin{figure}[h!]
    \begin{subfigure}{.5\textwidth}
        \centering
        \includegraphics[width=6cm]{coseno.png}
        \caption{$\cos{x}$}
        \label{fig:coseno}
    \end{subfigure}
    \begin{subfigure}{.5\textwidth}
        \centering
        \includegraphics[width=3.5cm, height=2.7cm]{arcocoseno.png}
        \caption{$\arccos{x}$ o $\cos{x}^{-1}$}
        \label{fig:arcocoseno}
    \end{subfigure}
\end{figure}
\vspace{-5pt}
\begin{observation}
    \textbf{Cos(x):} La funzione $\cos{x}$, rappresentata nell'immagine [\ref{fig:coseno}], è periodica per $2\pi$ quindi possiamo scrivere $\cos{(x+2\pi)} = \cos x \: \forall x \in \mathbb{R}$. Inoltre è suriettiva per codominio [-1, 1]. Se invece definiamo $f: [0, \pi] \longrightarrow [-1, 1]$ la funzione $\cos x$ è suriettiva, quindi anche invertibile, e la sua inversa è appunto $\arccos{x}$.
\end{observation}
\begin{observation}
    \textbf{Arccos(x):} La funzione $\arccos{x}$ è l'inverso del seno e può essere scritta anche come $f(x) = \cos{x}^{-1}$ ed è rappresentata nell'immagine [\ref{fig:arcocoseno}].
\end{observation}

\subsubsection{Tangente e Arcotangente}
\textbf{Tangente:} $f(x) = \tan{x}$, $f: \mathbb{R} \longrightarrow \mathbb{R}$ \hfill
\textbf{Arcotangente:} $f(x) = \arctan{x}$, $f: \mathbb{R} \longrightarrow [-\frac{\pi}{2}, \frac{\pi}{2}]$
\begin{figure}[h!]
    \begin{subfigure}{.5\textwidth}
        \vspace{-20pt}
        \centering
        \includegraphics[width=4.2cm]{tangente.png}
        \vspace{-20pt}
        \caption{$\tan{x}$}
        \label{fig:tangente}
    \end{subfigure}
    \begin{subfigure}{.5\textwidth}
        \centering
        \includegraphics[width=5cm]{arcotangente.png}
        \caption{$\arctan{x}$ o $\tan{x}^{-1}$}
        \label{fig:arcotangente}
    \end{subfigure}
\end{figure}
\begin{observation}
    \textbf{Tan(x):} La funzione $\tan{x}$, rappresentata nell'immagine [\ref{fig:tangente}], può essere scritta anche come $\frac{\sin{x}}{\cos{x}}$, ha come dominio $\{x \in \mathbb{R} \: | \:  x \neq \frac{\pi}{2} + k\pi, \: k \in \mathbb{Z}\}$. La funzione tangente è fatta da infiniti intervalli, è quindi periodica per $\pi$; è di base non invertibile, ma se la ristringiamo in $f: [-\frac{\pi}{2}, \frac{\pi}{2}] \longrightarrow \mathbb{R}$ diventa biunivoca ed accetta la funzione inversa che è $\arctan{x}$.
\end{observation}
\begin{observation}
    \textbf{Arctan(x):} La funzione $\arctan{x}$, rappresentate nell'immagine [\ref{fig:arcotangente}], è inversa della funzione $\tan{x}$, può quindi essere scritta anche con la forma $\tan{x}^{-1}$.
\end{observation}
\section{Massimi e minimi}
\subsection{Massimo e minimo intervalli}
\begin{definition}[Massimo]
Dato un insieme A tale che: \\$A \subseteq \mathbb{R}, \: A \neq \O, \: m \in \mathbb{R}$ m si dice \textbf{massimo} di A se $m \geq a \: \forall \: a \in A$ e $m \in A$
\end{definition}
\begin{definition}[Minimo]
Dato un insieme A tale che: \\$A \subseteq \mathbb{R}, \: A \neq \O, \: m \in \mathbb{R}$ m si dice \textbf{minimo} di A se $m \leq a \: \forall \: a \in A$ e $m \in A$
\end{definition}
\begin{example}
    Esempi massini e minimi intervalli:
    \begin{itemize}
        \item Dato $A = [0,1]$ il max(A) = 1 e il suo min(A) = 0
        \item Dato $B = [0, 1)$ il min(B) = 0 mentre B non ha massimo.
    \end{itemize}
\end{example}

\begin{wrapfigure}{r}{8cm}
    \vspace{10pt}
    \centering
    \includegraphics[width=7cm]{dimostrazione-massimo.png}
    \caption{Segmento B}
\end{wrapfigure}
\begin{demostration}
Dimostriamo questo esempio:
\end{demostration}
Supponiamo per assurdo che $m \in \mathbb{R}$ sia il max di B, con ovviamente $m \in B$. Se tale condizione è vera $m < 1$ perché 1 non è incluso nell'insieme B = [0, 1).\\ \\
Poniamo ora $\epsilon = 1 - m$, così facendo $\epsilon$ diventa la lunghezza dell'intervallo fra 1 ed m.\\ \\
Definiamo ora un $m_1 = m + \frac{\epsilon}{2}$. Creando questo valore $m_1$ vediamo che $m_1 \in B$ ma anche che $m < m_1$ che contrasta con la definizione di massimo di B che dovrebbe essere $m \geq b \: \forall \: b \in B$. Così dimostriamo la non esistenza di un valore massimo.

\subsection{Maggiorante e minorante intervalli}
\begin{definition}[Maggiorante]
$A \subseteq \mathbb{R}$, $A \neq \O$, $k \in \mathbb{R}$ si dice \textbf{maggiorante} di A se $k \geq a \: \: \forall \: \: a \in A$. L'insieme di tutti i maggioranti si indica con $M_A$.
\end{definition}
\begin{definition}[Minorante]
$A \subseteq \mathbb{R}$, $A \neq \O$, $k \in \mathbb{R}$ si dice \textbf{minorante} di A se $k \leq a \: \: \forall \: \: a \in A$. L'insieme di tutti i minoranti si indica con $m_A$.
\end{definition}
\begin{example}
A = [0,3] allora 3 è un maggiorante di A, quindi $3 \in M_A$. \\
Mentre $\frac{1}{4}$ non è un maggiorante, quindi $\frac{1}{4} \notin M_A$, perché $1 > A$ e $1 > \frac{1}{2}$.
\end{example}
\begin{observation}
    Se esiste un maggiorante di A allora ne esistono infiniti. Infatti se prendiamo un $k \in M_A$, m è un maggiorante di A $\forall \: \: m \geq k$.
    Questo discorso vale anche per i minoranti, infatti con $k \in m_A$, m è un minorante di A $\forall \: \: m \leq k$.
    \begin{example}
        Esempi per l'osservazione sopra:
        \begin{itemize}
            \item A = $\mathbb{R}$, A non ha maggioranti.
            \item A = [4, $+\infty$] non ha maggioranti ma ha minoranti.
        \end{itemize}
    \end{example}
\end{observation}

\subsection{Intervallo limitato}
\begin{definition}[Limitato superiormente]
    Dato un intervallo A, se $M_A \neq \O$ (insieme dei maggioranti) allora l'intervallo A si dice \textbf{limitato superiormente}
\end{definition}
\begin{definition}[Limitato inferiormente]
    Dato un intervallo A, se $m_A \neq \O$ (insieme dei minoranti) allora l'intervallo A si dice \textbf{limitato inferiormente}
\end{definition}
\begin{definition}[Limitato]
    $A \subset \mathbb{R}$, $A \neq \O$, se A è sia superiormente che inferiormente limitato allora A si dice semplicemente intervallo \textbf{limitato}.
\end{definition}
\begin{observation}
    A è limitato se e solo se $\exists \: \: h,k \in \mathbb{R}$ tale che $k \leq a \leq h \: \: \forall \: \: a \in A$
\end{observation}

\subsubsection{Estremi superiori ed inferiori}
\begin{theorem}[Estremo superiore]
    $A \subset \mathbb{R}$, $A \neq \O$ ed A è superiormente limitato, allora esiste il minimo dell'insieme dei maggioranti. Tale minimo si dice \textbf{estremo superiore} di A e si indica con sup(A).
\end{theorem}
\begin{theorem}[Estremo inferiore]
    $A \subset \mathbb{R}$, $A \neq \O$ ed A è inferiormente limitato, allora esiste il massimo dell'insieme dei minoranti. Tale massimo si dice \textbf{estremo inferiore} di A e si indica con inf(A).
\end{theorem}
\begin{example}
    Esempio estremi superiori ed inferiori:
    \begin{itemize}
        \item A = [0, 1) \hspace{.1cm} $M_A$ = [1, $+\infty$) e $m_A$ = ($-\infty$, 0] \hspace{.1cm} min($M_A$) = sup(A) = 1 \hspace{.2cm} max($m_A$) = inf(A) = 0
        \item B = [0, 1] \hspace{.1cm} $M_B$ = [1, $+\infty$) e $m_A$ = ($-\infty$, 0] \hspace{.1cm} min($M_B$) = sup(B) = 1 \hspace{.2cm} max($m_B$) = inf(B) = 0
    \end{itemize}
    \begin{observation}
        Se esiste max(A) allora max(A) = sup(A) e viceversa se esiste min(A) allora min(A) = inf(A)
    \end{observation}
\end{example}
\begin{note}
    Se A non è superiormente limitato scriviamo sup(A) = $-\infty$ e se non è inferiormente limitato inf(A) = $-\infty$.
\end{note}
\begin{observation}
    $A \neq \O$ e A è superiormente limitata, allora m = sup(A) se e solo se valgono 2 condizioni:
    \begin{enumerate}
        \item $a \leq m \: \: \forall \: \: a \in A$ \hspace{.3cm} Questo dice che m è un maggiorante
        \item $\forall \: \: \epsilon > 0 \: \: \exists \: \: \overline{a}$\footnote{$\overline{a}$ è un semplice metodo di notazione}$\in A \: \: | \: \: \overline{a} > m - \epsilon$ \hspace{.3cm} $m - \epsilon$ mi dice che non ci sono maggioranti più piccoli di m. 
    \end{enumerate}
    Se valgono queste 2 condizioni m è l'estremo sup e viceversa se m è sup(A) allora valgono queste condizioni.
    \begin{note}
        Questa considerazione vale anche per m = inf(A).
    \end{note}
\end{observation}
\begin{observation}
    La scrittura sup(A) $< +\infty$ vuol dire che l'estremo superiore di A è un numero reale, quindi A è superiormente limitato. Viceversa la scrittura inf(A) $> -\infty$ vuol dire che l'estremo inferiore di A è un numero reale, quindi A è inferiormente limitato.
\end{observation}

\subsection{Retta reale estesa}
\begin{definition}[Retta reale estesa]
    La retta reale estesa si indica con $\overline{\mathbb{R}} = \mathbb{R} \cup \{-\infty\} \cup \{+\infty\}$ in modo che valga: $-\infty \leq x \leq +\infty \: \: \forall x \in \overline{\mathbb{R}}$
\end{definition}
\begin{observation}
    Se $x \in \mathbb{R}$ (quindi $x \neq +\infty, x \neq -\infty$) allora $-\infty < x < +\infty$
\end{observation}

\subsubsection{Operazioni in $\overline{\mathbb{R}}$}
\begin{itemize}
    \item Se $x \neq +\infty$ allora $x + (-\infty) = -\infty$.
    \item Se $x \neq -\infty$ allora $x + (+\infty) = +\infty$.
    \item Se $x > 0$ allora $x(+\infty) = +\infty$ e $x(-\infty) = -\infty$.
    \item Se Se $x < 0$ allora $x(+\infty) = -\infty$ e $x(-\infty) = +\infty$.
    \item $(+\infty) + (-\infty)$ e viceversa \hspace{.3cm} $0(+\infty)$ o $0(\infty)$ \hspace{.3cm}\textbf{Sono vietate}
    \item $(+\infty)(+\infty) = +\infty$ \hspace{.2cm} $(+\infty)(-\infty) = -\infty$ \hspace{.2cm} $(-\infty)(-\infty) = +\infty$ \hspace{.2cm} \textbf{Sono consentite}
\end{itemize}
\begin{observation}
    Dato $A \subset \mathbb{Z}$ se A è superiormente limitato, A ha un massimo e se A è inferiormente limitato allora A ha un minimo.
\end{observation}

\subsection{Parte intera di un numero}
\begin{definition}
    Dato $x \in \mathbb{R}$ si dice \textbf{parte intera di x} e si indica con [x] il numero [x] = max$\{m \in \mathbb{Z}: m \leq x\}$
\end{definition}
\begin{wrapfigure}{r}{6cm}
    \vspace{-15pt}
    \centering
    \includegraphics[width=5cm]{parte-intera.png}
    \caption{Parte intera di x}
    \label{fig:my_label}
\end{wrapfigure}
Possiamo spiegarlo in maniera semplice che è il primo numero intero che troviamo alla sinistra di x.
\begin{example}
    $[\frac{25}{10}] = 2$ \hspace{.5cm} $[-\frac{25}{10}] = -2$\\ \\
\end{example}

\vspace{-20pt}
\subsubsection{Grafico di f(x) = [x]}
\begin{wrapfigure}[8]{l}{7cm}
    \vspace{-25pt}
    \centering
    \includegraphics[width=5cm]{funzione-parte-intera.png}
    \caption{Grafico f(x) = [x]}
    \label{fig:funzione-parte-intera}
\end{wrapfigure}
\vspace{20pt}
Possiamo vedere nell'immagine [\ref{fig:funzione-parte-intera}] che tutti numeri vanno a valere in y come il valore del primo intero a sinistra.
\begin{example}
    Esempio per f(x) = [x]:\\
    $f(\frac{1}{2}) = 0$ \hspace{.3cm} $f(\frac{3}{2}) = 1$\\ \\
    $f(\frac{10}{3}) = 3$ \hspace{.3cm} $f(\frac{4}{3}) = 1$\\ \\ \\
\end{example}

\subsection{Limiti, massimi e minimi su funzioni}
Andiamo a fare una serie di definizioni prendendo due insiemi A, B tale che $A \subseteq \mathbb{R}$ e $B \subseteq \mathbb{R}$ ed una funzione $f(A)$ definita come $f: A \longrightarrow B$.
\begin{definition}[Limitata superiormente, inferiormente]
    $f$ si dice limitata superiormente se $f(A)$ è limitata superiormente. Viceversa $f$ si dice limitata inferiormente se $f(A)$ è limitata inferiormente. Se $f$ è sia limitata superiormente che inferiormente si dice che $f$ è limitata.
\end{definition}
\begin{definition}[Massimo e minimo]
    $f$ ha massimo se la sua immagine $f(A)$ ha massimo. Si dice che $M$ è il massimo di $f$ e si scrive $M = max(f)$ se $M = max(f(A))$. Ugualmente $f$ ha minimo se la sua immagine $f(A)$ ha minimo. Si dice che $m$ è il minimo di $f$ e si scrive $m = min(f)$ se $m = min(f(A))$.
\end{definition}
\begin{definition}
    Se f non è limitata superiormente e si scrive $sup(f) = +\infty$. Ugualmente se f non è limitata inferiormente, e si scrive $inf(f) = -\infty$.
\end{definition}
\begin{note}
    Rircoda che sup($f$) corrisponde a scrivere sup($f(A)$) e ugualmente inf($f$) è uguale a inf($f(A)$).
\end{note}
\begin{definition}[Punti di massimo e minimo]
    Se $f$ ha massimo allora ogni $x_0 \in A$ tale che $f(x_0) = max(f)$ si dice punto di massimo per $f$. Similmente se $f$ ha minimo allora ogni $x_0 \in A$ tale che $f(x_0) = min(f)$ si dice punto di minimo per $f$.
\end{definition}
\begin{observation}
    Il massimo di $f$ è unico mentre i punti di massimo possono essere molti.
\end{observation}

\begin{example}
    $f: \mathbb{R} \longrightarrow \mathbb{R}$ \hspace{.3cm} $f(x) = \sin{x}$ [\ref{fig:massimo_sinx}]
\end{example}
max(f) = 1 \hspace{.3cm} $x_0 = \frac{\pi}{2} + 2k\pi, k \in \mathbb{Z}$\\
\begin{wrapfigure}[4]{r}{8cm}
    \vspace{-45pt}
    \centering
    \includegraphics[width=6.5cm]{images/massimo_es1.png}
    \caption{funzione $f(x) = \sin{x}$}
    \label{fig:massimo_sinx}
\end{wrapfigure}
In questo caso essendo la funzione periodica in ogni intervallo di $x_0 = \frac{\pi}{2} + 2k\pi, k \in \mathbb{Z}$ esisterà un punto di massimo mentre il massimo rimarrà sempre 1.
\begin{example}
    $f:(0, +\infty) \longrightarrow \mathbb{R}$ \hspace{.3cm} $f(x) = \frac{1}{x}$ [\ref{fig:massimo-minimo-frazione}]
\end{example}
In questa casistica $f$ non ha ne massimo ne minimo. Questo lo possiamo dimostrare andando ad immaginare una casistica dove esiste un massimo ed un minimo e facendo poi alcune considerazione. \\
\begin{wrapfigure}{r}{6cm}
    \vspace{-15pt}
    \centering
    \includegraphics[width=4.5cm]{images/massimo_es2.png}
    \caption{funzione $f(x) = \frac{1}{x}$}
    \label{fig:massimo-minimo-frazione}
\end{wrapfigure}
Innanzitutto prendiamo per assurdo che $f$ avesse massimo allora $\Longrightarrow \: \: \exists \: \: m$ tale che $f(x) \leq m \: \: \forall \: \: x \in (0, +\infty)$. \\ Se in questa casistica prendessimo un punto x e dicessima che quello è il massimo, $f(\frac{1}{x}) = m$, ma se poi prendiamo un punto che è $\frac{x}{2}$ esso apparitene sempre alla funzione e $f(\frac{x}{2}) = 2m$ e $2m > m$. Quindi vediamo come non è possibile determinare un massimo.\\ \\
Questa funzione non può nemmeno avere un minimo perché $f(x) > 0 \: \: \forall \: \: x$, quindi $inf(f) = 0$. Se $f$ avesse minimo dovrebbe essere $m(f) = inf(f) = 0$ ma questo presuppone che debba esiste un $x_0$ tale che $f(x_0) = 0$ cioè $\frac{1}{x_0} = 0$, ma questo è impossibile.\\
\begin{observation}
    Consideriamo un insieme A $\subset \mathbb{R}$ e una funzione $f: A \longrightarrow \mathbb{R}$, valgono per essi le seguenti osservazioni:
    \begin{itemize}
        \item Se A ha massimo e $f$ è debolmente crescente allora $f$ ha max e max($f$) = $f$(max(A)).
        \item Se A ha minimo e $f$ è debolmente crescente allora $f$ ha min e min($f$) = $f$(min(A)).
        \item Se A ha minimo e $f$ è debolmente crescente allora $f$ ha min e min($f$) = $f$(max(A)).
        \item Se A ha massimo e $f$ è debolmente crescente allora $f$ ha max e max($f$) = $f$(min(A)).
    \end{itemize}
\end{observation}
\begin{figure}[h!]
    \begin{subfigure}{.5\textwidth}
        \centering
        \includegraphics[width=6cm]{images/crescente-max-min.png}
        \caption{Punti max e min $f$ crescente}
        \label{fig:my_label}
    \end{subfigure}
    \begin{subfigure}{.5\textwidth}
        \centering
        \includegraphics[width=4cm]{images/descrescente-max-min.png}
        \caption{Punti max min $f$ decrescente}
        \label{fig:my_label}
    \end{subfigure}
\end{figure}
\begin{observation}
    Se $f: A \longrightarrow \mathbb{R}$ allora m = sup($f$) se e solo se valgono queste due condizioni:
    \begin{enumerate}
        \item $f(x) \leq m \: \: \forall \: \: x \in A$ \hspace{.3cm} Questo vuol dire che m deve essere maggiore o uguale di qualsiasi f(x)
        \item $\forall \: \: \epsilon > 0 \: \: \exists \: \:  \overline{x} \in A \: \: | \: \: f(\overline{x}) > m - \epsilon$ \hspace{.3cm} Questo vuol dire che per qualsiasi valore $\epsilon$ maggiore di 0 deve esistere un $\overline{x}$ appartenendo all'insieme A tale che, se sottraiamo il valore $\epsilon$ a m il risultato deve essere inferiore a $f(\overline{x})$ ciò vuol dire che non ci sono altri valori per il quale la funzione è sempre sotto.
    \end{enumerate}
\end{observation}

\newpage
\section{Valore assoluto}
\begin{definition}[Valore assoluto]
    Dato $x \in \mathbb{R}$ si dice valore assoluto di x il massimo valore fra x e -x e si indica con $|x|$.
    \begin{equation}
        |x| = max(\{x, -(x)\})
    \end{equation}
\end{definition}
\begin{example}
    Esempi valore assoluto:
    \begin{itemize}
        \item $|5| = max(\{5, -5\}) = 5$
        \item $|-3| = max(\{-3, -(-3)\}) = 3$
    \end{itemize}
\end{example}
\subsection{Proprietà valore assoluto}
\begin{table}[h!]
    \setlength{\tabcolsep}{7pt}
    \renewcommand{\arraystretch}{2}
    \centering
    \begin{tabular}{|c|c|}
        \hline
        (1) $x \leq |x| \: \: \forall \: \: x \in \mathbb{R}$ & (2) $|x| = x$ se $x \geq 0, |x| = -x$ se $x \leq 0$ \\
        (3) $|x| \geq 0 \: \: \forall \: \: x \in \mathbb{R}$ & (4) $|x| = 0 \Longleftrightarrow x = 0$ \\
        (5) $|-x| = |x|$ & (6) $-|x| \leq x \leq |x|$ \\
        (7) $|x| \leq M \Longleftrightarrow -M \leq x \leq M$ con $M \geq 0$ & (8) $|x| \geq M \Longrightarrow x \geq M$ oppure $x \leq -M$ \\ \hline
    \end{tabular}
    \caption{Proprietà valore assoluto}
    \label{tab:prop-valore-assoluto}
\end{table}
\subsubsection{Spiegazioni proprietà}
Se stabiliamo un punto M maggiore del valore assoluto la funzione si troverà compreso fra M e -M. Se invece stabiliamo un punto M minore del valore assoluto la funzione sarà maggiore di M e minore di -M. Spiegazione grafica nell'immagine [\ref{fig:prop-6-7}]
\begin{figure}[h!]
    \centering
    \includegraphics[width=10cm]{es-proprieta-valore-assoluto.png}
    \caption{Spiegazione proprietà 7 e 8}
    \label{fig:prop-6-7}
\end{figure}

\subsection{Disuguaglianza triangolare}
\begin{definition}[Disuguaglianza triangolare]
    Dati due valore $a$ e $b$ tali che $a, b \in \mathbb{R}$ risulta che:
    \begin{equation}
        (1)\:\:\:|a + b| \leq |a| + |b| \hspace{.7cm} (2)\:\:\:||a| + |b|| \leq |a - b| 
    \end{equation}
\end{definition}
\begin{demostration}
    Dimostrazione proprietà (1):\\
    Dati due valori $a$ e $b$ calcoliamo il valore assoluto, che per la proprietà (6) in tabella \ref{tab:prop-valore-assoluto}  possiamo scrivere nella seguente forma:
    \begin{equation}
            -|a| \leq a \leq |a| \hspace{.6cm} -|b| \leq b \leq |b|
    \end{equation}
    Ora facciamo una somma di disuguaglianze fra le forme riportate sopra:
    \begin{equation}
        - |a| - |b| \leq a + b \leq |a| + |b|
    \end{equation}
    Possiamo vedere la prima parte $-|a| - |b|$ come un -M, la parte $a + b$ come una $x$ e l'ultima parte $|a| + |b|$ come M. Utilizzando a questo punto la proprietà (7) in tabella \ref{tab:prop-valore-assoluto}, $|x| \leq M$ quindi:
    \begin{equation}
        |a + b| \leq |a| + |b|
    \end{equation}
\end{demostration}
\begin{observation}
Perché una disuguaglianza triangolare a 3 numeri, $|a + b + c| \leq |a| + |b| + |c|$, vale?\\ \\
Perché se $|a + b + c|$ lo dividiamo in $|(a + b) + c|$ possiamo applicare la propria triangolare su 2 valori considerando $(a+b)$ il primo e $c$ il secondo questo fa si che $|(a + b) + c| \leq |a + b| + |c|$ andando poi a riapplicare la disuguaglianza triangolare questa volta solo su $|a + b|$ vediamo che:
\begin{equation}
    |a + b + c| = |(a + b) + c| \leq |a + b| + |c| \leq |a| + |b| + |c|
\end{equation}
Da qui possiamo dedurre che la disuguaglianza triangolare vale indipendentemente dal numero di valori:
\begin{equation}
    |a_1, a_2, a_3, ..., a_n| \leq |a_1| + |a_2| + |a_3| + .... + |a_n|
\end{equation}
\end{observation}
\newpage
\section{Continuità}
\begin{definition}[Funzione continua]
    Dato un insieme A ed una funzione $f(x)$ tale che $A \subset \mathbb{R}$, $f: A \longrightarrow \mathbb{R}$, la funzione $f$ si dice \textbf{continua} in $x_0$ se $\forall \: \: \epsilon > 0 \: \: \exists \: \: \delta > 0$ tale che se data una $x \in A$:
    \begin{equation}\label{funzione-continua}
        |x - x_0| < \delta \Longrightarrow |f(x) - f(x_0)| < \epsilon
    \end{equation}
\end{definition}
La condizione scritta sopra [\ref{funzione-continua}] può essere scritta anche tramite due condizioni:
\begin{itemize}
    \item $|x - x_0| < \delta \Longleftrightarrow x_0 - \delta < x < x_0 + \delta$
    \item $|f(x) - f(x_0)| < \epsilon \Longleftrightarrow f(x_0) - \epsilon < f(x) < f(x_0) + \epsilon$
\end{itemize}
\begin{example}
Ora per capire meglio facciamo un esempio di funzione non continua:
\end{example}
Innanzitutto stabiliamo una $f(x)$ e verifichiamo che $f(x)$ in $x_0 = 0$ non è continua.\\ 
\begin{wrapfigure}{r}{8cm}
\vspace{-20pt}
    \centering
    \includegraphics[width=4.8cm, height=5cm]{images/esempio-non-continuita.png}
    \caption{funzione non continua}
    \label{fig:funzione-non-continua}
\end{wrapfigure}

$
  f(x)=\begin{cases}
    0 \: \: se & x \leq 0\\
    1 \: \: se & x > 0
  \end{cases}
$\\ \\ \\
Come prima cosa stabiliamo un $\epsilon = \frac{1}{2}$. Ora, qualunque sia $\delta > 0$ se andiamo a prendere una $x$ tale che $x \in (0, \delta) \Longrightarrow f(x) = 1$ quindi la disuguaglianza $f(x_0) - \epsilon > f(x) < f(x_0) + \epsilon$, che diventerebbe $0 - \frac{1}{2} < f(x) < 0 + \frac{1}{2}$, è falsa. \\ \\
Deduciamo quindi che in $x_0 = 0$ questa funzione non è continua. \\ \\

\begin{definition}
    Dato un insieme A ed una funzione $f(x)$ tale che $A \subset \mathbb{R}$, $f: A \longrightarrow \mathbb{R}$ ed un insieme $B \subset \mathbb{R}$ si dice che $f$ è continua in B e $f$ è continua in ogni punto $x_0 \in B$.
\end{definition}
Se invece si dice semplicemente che $f$ è continua senza specificare il sotto insieme B vuol dire che $f$ è continua in tutti i punti del suo dominio A.
\begin{example}
    Esempio basato sulla funzione vista sopra:\\\\
    $
      f(x)=\begin{cases}
        0 \: \: se & x \leq 0\\
        1 \: \: se & x > 0
      \end{cases}
    $ \hspace{1cm}
    $f$ è continua in $(-\infty, 0) \cup (0, +\infty)$ 
\end{example}

\subsection{Permanenza del segno}
\begin{theorem}[Permanenza del segno]\label{permanenza-segno}
    Dato un insieme A ed una funzione $f$ tale che $A \subset \mathbb{R}$, $f: A \longrightarrow \mathbb{R}$, $x_0 \in A$. Se $f$ è continua in $x_0$ e $f(x_0) > 0$ allora $\exists \: \: \delta > 0$ t.c. se $x \in A$ è $|x - x_0| < \delta \longrightarrow f(x) > 0$. Analogo risultato se $f(x_0) < 0$.
    \begin{demostration}
    Sappiamo che $f(x_0) > 0$. Ora scegliamo un $\epsilon = \frac{f(x_0)}{2}$ ed utilizziamolo nella definizione di continuità:
    \begin{equation}
        \exists \: \: \delta > 0 \: \: | \: \: x \in A, |x - x_0| < \delta \Longleftarrow |f(x) - f(x_0)| < \epsilon
    \end{equation}
    Ciò che risulta dalle condizioni poste dalla continuità è che:
    \begin{equation}
        f(x_0) - \epsilon < f(x) < f(x_0) + \epsilon
    \end{equation}
    Se prendiamo la prima parte $f(x_0) - \epsilon < f(x)$ e facciamo le dovute sostituzioni risulta che:
    \begin{equation}
        f(x_0) - \frac{f(x_0)}{2} < f(x)
    \end{equation}
    Visto che $f(x_0) - \frac{f(x_0)}{2}$ è sempre maggiore di 0 risulta anche che $f(x)$ è maggiore di 0. $\blacksquare$
    \end{demostration}
    \begin{corollaries}\label{collorartio-permanenza-segno}
        Se $f$ è continua in $x_0$ $f: A \longrightarrow \mathbb{R}$, $x_0 \in A$ e $f(x_0) > M$ con $M \in \mathbb{R}$, $x \in A$, $|x - x_0| < \delta \Longrightarrow f(x) > M$. (Vale anche con $f(x_0) < M \Longrightarrow f(x) < M$)
    \end{corollaries}
    \begin{demostration}[Dimostrazione del corollario \ref{collorartio-permanenza-segno}]
        La dimostrazione di questo corollario è immediata e si fa applicando al teorema precedente \ref{permanenza-segno} la funzione $g(x) = f(x) - M$, perché se la funzione $f(x) - M > 0$ è come dire $f(x) > M$. $\blacksquare$
    \end{demostration}
\end{theorem}

\subsection{Continuità con operazioni fra funzioni}
\begin{theorem}
    Prendendo due funzioni $f$ e $g$ continue in un punto $x_0$ allora le funzioni $f + g$, $f * g$ e $|f|$, se inoltre $f(x_0) \neq 0$ allora anche $\frac{1}{f}$ è continua.
    \begin{corollaries}
        Prendendo due funzioni $f$ e $g$ continue in un punto $x_0$ allora $\frac{f}{g}$ è continua se $g(x_0) \neq 0$
    \end{corollaries}
\end{theorem}

\subsection{Funzioni invertibili e continuità}
\begin{proposition}\label{proposizione-funzione-inversa}
    Prendendo due insiemi I (I deve essere un intervallo) e B tale che $I \subset \mathbb{R}$ e $B \subset \mathbb{R}$ ed una funzione $f: I \longrightarrow B$, se $f$ è continua in $I$ ed è invertibile allora $f^{-1}$ è continua in B.
\end{proposition}
\begin{observation}
    Possiamo osservare che ipotesi della proposizione \ref{proposizione-funzione-inversa} dice che il domino sia un intervallo, questo non può essere omesso. 
\end{observation}
\begin{example}
    Verifichiamo questa osservazione con un' esempio:\\
    Prendiamo una funzione $f(x)$ definita in $f: (-\infty, 1] \cup (2, +\infty) \longrightarrow \mathbb{R}$ \: \: \: $f(x) = 
    \begin{cases}
        x \: \: se & x \leq 1 \\
        x - 1 \: \: se & x > 1
    \end{cases}
    $\\
    Qui di seguito le rappresentazioni della funzione $f(x)$ e della sua inversa $f(x)^{-1}$
\end{example}
\begin{figure}[h!]
    \vspace{10pt}
    \begin{subfigure}{.5\textwidth}
        \centering
        \includegraphics[width=5.5cm]{images/esempio1-osservazione-prop1.png}
        \caption{Osservazione proposizione \ref{proposizione-funzione-inversa}, funzione $f(x)$}
        \label{fig:es1-prop1}
    \end{subfigure}
    \begin{subfigure}{.5\textwidth}
        \centering
        \includegraphics[width=4.2cm, height=5.2cm]{images/esempio2-osservzione-prop1.png}
        \caption{Osservazione proposizione \ref{proposizione-funzione-inversa}, funzione $f(x)^{-1}$}
        \label{fig:es2-prop1}
    \end{subfigure}
\end{figure}
\begin{itemize}
    \item \textbf{Domanda 1°:} $f$ è continua in $x_0 = 1$?\\
    La risposta a questa prima domanda è SI, essendo che noi andiamo a considerare solo i punti all'interno del dominio, quindi la parte compresa fra 1 e 2, dove la funzione presenta una discontinuità, non si considera.
    \item \textbf{Domanda 2°:} $f$ è continua in $x_0 = 2$?\\
La risposta in questo caso è che non ha senso considerare il punto $x_0 = 2$ visto che 2 non fa parte del dominio.
\end{itemize}
Quindi $f$ è continua in tutto il suo dominio. Essendo $f$ continua in tutto il suo dominio allora teoricamente $f^{-1}$ è una funzione invertibile. \\ \\
Possiamo però vedere che la funzione $f^{-1}$, figura [\ref{fig:es2-prop1}] non è continua in $x_0 = 1$ perché essendo la funzione inversa $f^{-1}$ è definita come $f^{-1}: \mathbb{R} \longrightarrow (-\infty, 1] \cup (2, +\infty)$, quindi dobbiamo considerare come dominio tutto $\mathbb{R}$, così facendo ci sono dei punti, in particolare con $x > 0$ che non rientrano nell'intervallo fra $f(x_0) - \epsilon$ e $f(x_0) + \epsilon$ . \\ \\
In conclusione da questo esempio deduciamo che, se $f$ non è definita in un intervallo potrebbe succedere che $f^{-1}$ non è continua anche se $f$ è continua.

\subsection{Continuità delle funzioni elementari}
$f(x) = x$ è una funzione continua. Da questa considerazione segue che tutte le funzioni con polinomi sono continue.
\begin{note}
    Ricorda che anche le funzioni costanti sono sempre continue
\end{note}
Definiamo in maniera generica così una funzione formata da polinomi continua:
\begin{equation}
    P(x) = a_n * x^n + a_{n-1} * x^{n-1} + .... + a_1 * x + a_0 \: \: con \: \: a_0, a_1, ..., a_n \in \mathbb{R}
\end{equation}
Quindi: $x^2 = x * x$ è continua \hspace{.5cm} $x^3 = x^2 * x$ è continua \hspace{.5cm} $x^n$ è continua $\: \: \forall x \in \mathbb{N}$ \\ \\
Le funzioni razionali sono continue nel loro insieme di definizione. Le funzioni razionali sono uguali a quoziente di polinomi:
$f(x) = \frac{p(x)}{q(x)}$ con $p,q$ polinomi, la funzione $f(x)$ è definita se $q(x) \neq 0$.\\ \\
Assumendo che $e^x$, $\sin{x}$, $\cos{x}$ sono funzioni continue quindi anche $\log{x}$, $\arcsin{x}$, $\arccos{x}$, $\tan{x}$, $\arctan{x}$ sono continue.

\subsection{Continuità fra composizione di funzioni}
\begin{theorem}
    Date due funzioni $f: A \longrightarrow \mathbb{R}$ e $g: B \longrightarrow \mathbb{R}$, ed un $x_0 \in A$, $y_0 = f(x_0) \in B$.
    Se $f$ è continua in $x_0$ e $g$ è continua in $y_0$ allora $g \bullet f$ è continua in $x_0$.
\end{theorem}
\begin{example}
    Facciamo un esempio usando la funzione $e^{\cos{x}}$.\\
    $e^{\cos{x}}$ è una funzione continua perché è la composizione di $f(x) = \cos{x}$, funzione continua, e $g(x) = e^y$, pure essa funzione continua.
\end{example}
\begin{observation}
    Data una $f: [a, b] \longrightarrow \mathbb{R}$ continua in $[a, b]$ allora sup($f(x)$) con $x \in (a, b)$ = sup($f(x)$) con $x \in [a, b]$. E ugualmente inf($f(x)$) con $x \in (a, b)$ = inf($f(x)$) con $x \in [a, b]$.
\end{observation}
\begin{example}
    $f(x) = x^2$ con $f: [0,1] \longrightarrow \mathbb{R}$\\
    sup($f(x)$) = $f(1) = 1$ con $x \in [0, 1]$ \hspace{.5cm} sup($f(x)$) = $f(1) = 1$ con $x \in (0, 1)$ \footnote{Ricorda che sup(imm($f(x)$)) = sup(0, 1) = 1}
\end{example}

\subsection{Teorema degli zeri}
\begin{theorem}[Teorema degli zeri]
    Data una $f: [a, b] \longrightarrow \mathbb{R}$ continua. Se $f(a) \cdot f(b) < 0$ allora $\exists c \in (a, b)$ tale che $f(c) = 0$
\end{theorem}
Questo teorema dice che prendendo una funzione, che deve essere obbligatoriamente continua, se i valori di $f(x)$ nei due estremi moltiplicati fra di loro risultano minori di 0 la funzione passa per 0 in un ponto $c$ e questo accade perché se il prodotto fra i due estremi torni inferiore a 0 vuol dire che hanno segno discorde.\\
\begin{example}
    Facciamo un esempio di un caso in cui la funzione NON è continua:
\end{example}
\begin{wrapfigure}{r}{5cm}
    \vspace{-10pt}
    \centering
    \includegraphics[width=4.5cm, height=2cm]{images/esempio-teorema-zeri.png}
    \caption{$f(x) = [x] + \frac{1}{2}$}
    \label{fig:esempio-teorema-zeri}
\end{wrapfigure}
Prendiamo $f(x) = [x] + \frac{1}{2}$ \: $f: [1, -1] \longrightarrow \mathbb{R}$\\ \\
Se ora prendiamo la f(x) nei due estremi e facciamo il prodotto torna che: \\
$f(1) \cdot f(-1) < 0$ ma $\nexists x \in [-1.1]$ t.c. $f(x) = 0$ come possiamo vedere nell'immagine \ref{fig:esempio-teorema-zeri}.\\ \\

\subsection{Teorema valori intermedi}
\begin{theorem}[Teorema dei valori intermedi]
    Prendendo un intervallo $I \subset R$, ed una funzione $f: I \longrightarrow \mathbb{R}$ continua, allora $f(I)$ è un intervallo.
\end{theorem}
Questo teorema dice che se il nostro dominio è un intervallo e la $f$ è continua all'ora anche il codominio o immagine di $f$ sarà un intervallo.
\begin{corollaries}
    Prendendo sempre un $I \subset R$, una funzione $f: I \longrightarrow \mathbb{R}$ continua, se $f$ assume $y_1$ e $y_2$ allora assume anche tutti i valori compresi fra $y_1$ e $y_2$.
\end{corollaries}

\subsection{Teorema di Weirstrass}
\begin{theorem}[Teorema di Weirstrass]
    Data una funzione $f: [a, b] \longrightarrow \mathbb{R}$ continua. Allora f ha massimo e minimo.
\end{theorem}
\begin{note}
    Notare che $a,b \in \mathbb{R}$ e non in $\overline{\mathbb{R}}$ perché $a, b \neq \pm \infty$ e gli estremi devono essere compresi.
\end{note}
\begin{example}
    Facciamo ora un esempio per confermare come il teorema di Weirstrass possa valore solo con un intervallo chiuso:\\
    Dato $f(x) = \frac{1}{x}$ con $f(x): (0, 1] \longrightarrow \mathbb{R}$ \\ \\in questo caso $f$ ha come dominio un intervallo non chiuso a sinistra\\
    $f$ è continua ma non ha max perché sup($f$) = $+\infty$
\end{example}
\begin{example}
    Facciamo ora un esempio per confermare come il teorema di Weirstrass possa valore solo con un intervallo limitato:\\
    Dato $f(x) = \arctan{x}$ con $f(x): \mathbb{R} \longrightarrow \mathbb{R}$ \\ \\in questo caso $f$ è una funzione continua definita come $-\frac{\pi}{1} < f(x) < \frac{\pi}{2}$\\
    Possiamo notare però che f non toccherà mai ne $-\frac{\pi}{2}$ ne $\frac{\pi}{2}$ e quindi non ha ne massimo ne minimo.
\end{example}
\newpage
\section{Limiti}
\subsection{Intorni}
\begin{definition}[Intorno]
    Dato $x_0 \in \mathbb{R}$ si dice \textbf{intorno} di $x_0$ un insieme del tipo $(x_0 - \epsilon, x_0 + \epsilon)$ dove $\epsilon \in \mathbb{R}$, e $\epsilon > 0$. Inoltre $\epsilon$ si dice raggio dell'intorno
\end{definition}
\begin{itemize}
    \item Un insieme del tipo $[x_0, x_0 + \epsilon]$ si dice \textbf{intorno destro} di $x_0$.
    \item Un insieme del tipo $[x_0 - \epsilon, x_0]$ si dice \textbf{intorno sinistro} di $x_0$.
\end{itemize}
\begin{definition}
    Se $x_0 = +\infty$ un intorno di $x_0$ è un insieme del tipo $(a, +\infty)$\footnote{$(a, +\infty)$ è una semiretta} dove $a \in \mathbb{R}$
\end{definition}
\begin{definition}[Punto di accumulazione]
    Dato $A \subset \mathbb{R}$ e $x_0 \in \overline{R}$ $x_0$ si dice \textbf{punto di accumulazione} per A se $\forall \: U$ intorno di $x_0$ risulta che $U \cap A \setminus \{x_0\} \neq 0$
\end{definition}
Questa definizione vuol dice che "vicino" a $x_0$ ci sono altri punti di A oltre a $x_0$ ($x_0$ potrebbe anche non appartenere ad A).
\begin{example}
    Prendiamo un intervallo A = (2, 3).
\end{example}
Se prendiamo un punto $x_0$ che appartiene a A, quindi $x_0 \in (a,b)$, allora ogni intorno di $x_0$ interseca A in infiniti punti, quindi $x_0$ è un punto di accumulazione di A.\\
\begin{definition}[Intorno bucato]
    Se invece non andiamo a considerare $x_0$ nel suo intorno si dice \textbf{Intorno bucato} e si scrive come $\{x_0 - \epsilon, x_0 + \epsilon\} \setminus \{x_0\}$
\end{definition}
\begin{wrapfigure}{r}{8cm}
    \centering
    \includegraphics[width=7cm]{images/esempio-intorno-su-estremi.png}
    \caption{Punto di acc $x_0 = 2$ dell'intervallo A}
\end{wrapfigure}

Ora andiamo a dimostrare come tutti i punti [2,3] $\in$ acc(A).
Se poniamo per esempio $x_0 = 2$. Se andiamo a prendere un intorno di $x_0$ nonostante il $\epsilon$ possa essere piccolissimo esisteranno sempre infiniti punti nell'intersezione fra $U$ intorno e $A$ ($U \cap A \setminus \{x_0\}$) perché qualsiasi sia l'epsilon $2 + \epsilon$ rientrerà sempre in A.\\
Questo anche con $x_0 = 3$. 
\begin{note}
Nota che oltre a tutto [2,3] $\in$ A non esisto altri punti di accumulazione di un intervallo A.
\end{note}

\begin{definition}[Punto isolato]
    Dato un insieme A, $x_0 \in A$ si dice \textbf{punto isolato} di A se esiste un $U$ intorno di $x_0$ tale che $U \cap A = \{x_0\}$
\end{definition}
\begin{example}
    Facciamo un osservazione con un intorno spezzato per vedere un caso di punto isolato.
\end{example}
\begin{wrapfigure}{l}{8cm}
    \centering
    \includegraphics[width=7.3cm, height=2.7cm]{images/esempio-intorno-sepezzato.png}
    \caption{Punto di acc $x_0 = 5$ dell'intervallo C}
\end{wrapfigure}

Se prendiamo un punto C = $(2,3) \cup \{5\}$ non possiamo dire che tutti i punti dell'intervallo C siano punti di accumulazione perché se prendiamo $x_0 = 5$ possono esistere dei casi in cui il suo intorno non interseca C (con U intorno di $x_0 = 5$, $U \cap C \setminus {5} = \O$).\\
Diciamo quindi che in questo caso acc(c) = [2,3]\\\\
\begin{example}
    Esempio in cui verifichiamo come, dato un insieme D = $(3, +\infty)$, sia $+\infty \in$ acc(D).\\
    Come prima cosa prendiamo un $U$ intorno di $x_0 = +\infty$. Quindi $U = (a, +\infty)$.\\
    Definiamo ora il punto maggiore fra 3 ed $a$, $b =$ max($3,a$), questo punto sarà l'estremo sinistro dei punti di accumulazione.
    Facciamo ora l'intersezione:
    \begin{center}
        $U \cap D \setminus \{x_0\} = (2, +\infty) \cap (a, +\infty) \setminus {+\infty} = (b, +\infty) \neq \O$.
    \end{center}
    Vediamo dunque che $+\infty$ è un punto i accumulazione di D, quindi acc(D) = $[b, +\infty]$.
\end{example}
\newpage
\begin{example}
    Esempio prendendo come insieme $E = \mathbb{N}$.
\end{example}
\begin{wrapfigure}{r}{7cm}
    \vspace{-15pt}
    \centering
    \includegraphics[width=6.2cm]{images/insieme-N.png}
    \caption{Insieme $\mathbb{N}$}
    \label{fig:insieme-N}
\end{wrapfigure}

Se osserviamo l'immagine \ref{fig:insieme-N} vediamo chiaramente come tutti gli elemento di $\mathbb{N}$ sia punti isolare e quindi non siano punti di accumulazione. Ma, per l'esempio visto sopra, $+\infty$ è l'unico punto di accumulazione di $\mathbb{R}$. Acc($\mathbb{N}$) = $+\infty$. \\
\begin{note}
    Allo stesso modo prendendo in considerazione l'insieme $\mathbb{Z}$ i suoi punti di accumulazione sono acc($\mathbb{Z}$) = $\{-\infty, +\infty\}$
\end{note}
\begin{definition}
    Dato un insieme $A \subset \mathbb{R}$, ed un $x_0 \in A$, si dice $x_0$ punto interno ad A se esiste un $U$ intorno di $x_0$ tale che $U \subset A$. L'insieme dei punti interni si indica con int(A).
\end{definition}
\begin{example}
    Dato un A = [3, 5] i punti intesi sono (3,5) e non [3,5] perché se prendiamo $x_0 = 3$ o $x_0 = 5$ essendo che l'intorno di $x_0$ è [$x_0 - \epsilon$, $x_0 + \epsilon$] rimarrà sempre una parte fuori, in particolare quella di sinistra per $x_0 = 3$, e quella di destra per $x_0 = 5$.
\end{example}

\subsubsection{Minimi e massimi locali}
\begin{definition}[Minimi e massimi locali e locali stretti]
    Dato un insieme $A \subset \mathbb{R}$, una funzione $f: A \longrightarrow \mathbb{R}$ ed un punto $x_0 \in A$ si dice che $x_0$ è:
    \begin{itemize}
        \item \textbf{Minimo locale} (o relativo) se esiste un $U$ intorno di $x_0$ tale che $f(x) \geq f(x_0) \: \forall \: x \in U \cap A$
        \item \textbf{Minimo locale stretto} se esiste un $U$ intorno di $x_0$ tale che $f(x) > f(x_0) \: \forall \: x \in U \cap A \setminus \{x_0\}$
        \item \textbf{Massimo locale} (o relativo) se esiste un $U$ intorno di $x_0$ tale che $f(x) \leq f(x_0) \: \forall \: x \in U \cap A$
        \item \textbf{Massimo locale stretto} se esiste un $U$ intorno di $x_0$ tale che $f(x) < f(x_0) \: \forall \: x \in U \cap A \setminus \{x_0\}$
    \end{itemize}
\end{definition}
Questa definizione vuol dire che se andiamo a prendere un intorno di $x_0$, il punto $x_0$ può essere definito minimo o massimo di quel determinato intorno se è il punto più "in basso" o più "in alto" rispetto a tutti gli altri punti dell'intorno.
\begin{figure}[h!]
    \begin{subfigure}{.5\textwidth}
        \centering
        \includegraphics[width=7.7cm]{images/min-max-locale.png}
        \caption{Minimo e massimo locale}
        \label{fig:min-max-locale}
    \end{subfigure}
    \begin{subfigure}{.5\textwidth}
        \centering
        \includegraphics[width=5.5cm]{images/min-max-locale-stretto.png}
        \caption{Minimo e massimo locale stretto}
        \label{fig:min-max-locale-stretto}
    \end{subfigure}
\end{figure}

Come si può vedere dalle immagini [\ref{fig:min-max-locale}] [\ref{fig:min-max-locale-stretto}] noi andiamo a considerare solo i punti all'interno dell'intorno di $x_0$, infatti esisterebbero altri punti esterni a $U$ intorno maggiori o minori, ma non li consideriamo.
\begin{note}
    Nota che se $x_0$ è punto di minimo allora è anche punto di minimo locale, qualsiasi sia l'intorno che prendiamo in considerazione.
\end{note}

\subsection{I limiti}
\begin{definition}[Limite]
    Dato un $A \subset \mathbb{R}$, una $f: A \longrightarrow \mathbb{R}$, ed un $x_0$ punto di accumulazione per A, si dice che $l \in \overline{\mathbb{R}}$ è il limite per $x$ che tende a $x_0$ di $f(x)$ se $\forall$ V intorno di $l$, $\exists \: U$ intorno di $x_0$ t.c. $x \in U \cap A \setminus \{x_0\} \Longrightarrow f(x) \in V$
\end{definition}
Questa definizione dice che un valore $l$ per essere definito come limite di una funzione con $x$ che tende a $x_0$ bisogna che per qualsiasi intorno che andiamo a prendere di $l$ deve esistere una intorno di $x_0$ chiamato U tale che, se una $x$ appartiene ad U allora la $f(x)$ apparterrà all'intorno di $l$. \\
Se ci rifacciamo alle definizioni di intorno vediamo che $x \in U \cap A \setminus \{x_0\}$ vuol dire che $|x-x_0| < \delta$ e che $f(x) \in V $ vuol dire che $l - \epsilon < f(x_0) < l + \epsilon$.\\
Questa definizione può essere scritta in altre parole dicendo che:
\begin{center}
    $\lim\limits_{x\to x_0}f(x) = l$\footnote{La notazione $\lim\limits_{x\to x_0}f(x)$ è quella con cui andiamo a scrivere i limiti e vuol dire limite di $f(x)$ con $x$ che tende a $x_0$ è uguale a $l$ valore del limite} $ \Longleftrightarrow \forall \epsilon > 0 \: \: \exists \delta > 0 $ tale che $x \in A, |x - x_0| < \delta \land x \neq x_0 \Longrightarrow |f(x) - f(x_0)| < \epsilon$
\end{center}
\begin{example}
Alcuni esempi di limiti:
\begin{itemize}
    \item $\lim\limits_{x\to x_0}f(x) = \pm \infty$ \hspace{.5cm} $V = (a, \pm \infty)$ \hspace{.5cm} $f(x) \in V$ se e solo se $f(x) > a$\\
    Il risultato di questo limite è $\pm \infty$ se $\forall a \in \mathbb{R} \: \exists \: \delta > 0$ t.c. $|x-x_0|<\delta, x \in A, x\neq x_0 \Longrightarrow f(x) > a$
    \item $\lim\limits_{x\to \pm \infty}f(x) = l$ \hspace{.5cm} se $l \in \mathbb{R}$ se e solo se $x \to \infty$\\
    Il risultato del limite è un valore appartenete a $\mathbb{R}$ se $\forall \epsilon > a \: \exists a \in \mathbb{R}$ t.c. $x > a \Longrightarrow |f(x) - l| < \epsilon$
    \item $\lim\limits_{x\to \pm \infty}f(x) = \pm \infty$ \:
    se e solo se $\forall a \in \mathbb{R} \exists b \in \mathbb{R}$ t.c. $x > b \Longrightarrow f(x) > a$
\end{itemize}
\end{example}
\begin{theorem}[Unicità dei limiti]
Se esiste un limite di $f$ con $x \to x_0$, questo limite è unico.
\end{theorem}

\subsection{Continuità con i limiti}
Rivediamo le definizioni di limiti (con il limite che sia un numero finito)  e continuità accanto:
\begin{enumerate}
    \item $\lim\limits_{x\to x_0}f(x) = l$ con $x_0 \in A$, $l \in \mathbb{R}$ è vera se e solo se $\forall\epsilon > 0 \: \: \exists \delta >0$ t.c. $x \in A, x \neq x_0$ è $|x - x_0| < \delta \Longrightarrow |f(x) - l| < \epsilon$
    \item $f$ è continua in $x_0$ se e solo se $\forall \epsilon > 0 \: \: \exists \delta > 0$ t.c. $|x - x_0| < \delta$ con $x \in A \Longrightarrow |f(x) - f(x_0)| < \epsilon$
\end{enumerate}
Notiamo subito che fra la definizione (1) e la (2) c'è come unica differenza che nella prima c'è $l$ mentre nella seconda c'è $f(x)$. Possiamo dunque trarre una serie di osservazioni.
\begin{observation}
Data una funzione $f(x)$ essa è continua in $x_0 \Longrightarrow \lim\limits_{x\to x_0}f(x) = l$
\end{observation}
\begin{observation}
Una funzione è sempre continua nei punti isolati.
\end{observation}
\begin{observation}
Nella definizione di limite non serve che $x_0$ sia nel dominio di una funzione, basta che sia un punto di accumulazione per il dominio.
\end{observation}
\begin{example}
Esempio di continuità con i limiti:
\end{example}
\begin{wrapfigure}{r}{6cm}
    \vspace{-50pt}
    \centering
    \includegraphics[width=5cm, height=4.7cm]{images/es-continuita-limiti.png}
    \caption{$\lim\limits_{x\to 0}f(x) = 3$}
\end{wrapfigure}

$f(x) = 
    \begin{cases}
        3 \: \: se & x \neq 0 \\
        2 \: \: se & x = 0
    \end{cases}
    $\\ \\
$\lim\limits_{x\to 0}f(x) = 3$, senza considerare f in $x = 0$.\\
Secondo la definizione di continuità di una funzione vista sopra (dove andiamo a guardare il valore del limite in $x_0$):
\begin{center}
    $|x - x_0| < \delta$, $x \in A$, $x \neq x_0$ allora $|f(x) - l| < \epsilon$.
\end{center}
Se andiamo però a vedere $\lim\limits_{x\to 0}f(x) = 3$ mentre $f(0) = 2$ e ovviamente $2 \neq 3$ quindi f non è continua in $x_0$.



\subsection{Limite destro e sinistro}
\begin{definition}[Limite destro e sinistro]
    Se dato un $A \subset \mathbb{R}$, un $x_0 \in Acc(A)$, un $x_0 \in \mathbb{R}$ ($x_0$ deve essere un numero finito), ed  $f: A \to \mathbb{R}$, allora si dice che $l \in \overline{\mathbb{R}}$ è il limite di $f(x)$ per x che tende a $x_0$ da \textbf{destra} (si scrive come $\lim\limits_{x\to x_0^+}f(x) = l$) se:
    \begin{center}
        $\forall \: V$ intorno di $l \: \exists \: \delta > 0$ t.c. $x_0 < x < x_0 + \delta$, $x \in A \Longrightarrow f(x) \in V$
    \end{center}
    Si dice limite \textbf{sinistro} (si scrive come $\lim\limits_{x\to x_0^-}f(x) = l$) se:
    \begin{center}
        $\forall \: V$ intorno di $l \: \exists \: \delta > 0$ t.c. $x_0 - \delta < x < x_0$, $x \in A \Longrightarrow f(x) \in V$
    \end{center}
\end{definition}
\begin{example}
Se prendiamo una $f: (-\infty, 0) \cup (0, +\infty) \to \mathbb{R}$, 
$f(x) = 
    \begin{cases}
        -1 \: \: se & x < 0 \\
        1 \: \: se & x > 0
    \end{cases}
    $ \\ 
    Il $\lim\limits_{x \to 0^+} f(x) = 1$ mentre $\lim\limits_{x \to 0^-} f(x) = -1$. Ciò perché andiamo nel caso del limite destro a guardare il valore "alla destra" di 0 e nel limite sinistro il valore "alla sinistra".
\end{example}
\begin{observation}
$\lim\limits_{x \to x_0^+} = l$ se e solo se $\lim\limits_{x \to x_0^+} = l_1$, $\lim\limits_{x \to x_0^-} = l_2$ e $l_1 = l_2$. Cioè per far in modo che il limite di una funzione che tende ad un valore $x_0$ sia unico bisogna che il limite destre e quello sinistro siano uguali. Nell'esempio precedente infatti possiamo notare che non esiste un unico limite perché i valori del destro e del sinistro sono diversi.
\end{observation}

\subsection{Limite da sopra e da sotto}
Dato un $A \subset \mathbb{R}$, una $f: A \to \mathbb{R}$, ed un $x_0 \in Acc(A)$
\begin{definition}
 Si dice che $\lim\limits_{x \to x_0}f(x) = l^+$ (con $l \in \mathbb{R}$) se $\lim\limits_{x\to x_0}f(x) = l$ ed esiste un $U$ intorno di $x_0$ t.c. $x \in U \cap A \setminus \{x_0\} \Longrightarrow f(x) > l$
\end{definition}
\begin{definition}
 Mentre analogamente si dice che $\lim\limits_{x \to x_0}f(x) = l^-$ (con $l \in \mathbb{R}$) se $\lim\limits_{x\to x_0}f(x) = l$ ed esiste un $U$ intorno di $x_0$ t.c. $x \in U \cap A \setminus \{x_0\} \Longrightarrow f(x) < l$
\end{definition}
Queste due definizione vogliono dire che la funzione può tendere ad un valore "da sopra" nel caso del + e "da sotto" nel caso del -.
\begin{figure}[h!]
    \begin{subfigure}{.5\textwidth}
        \centering
        \includegraphics[width=5.5cm]{images/limite-tende-sopra.png}
        \caption{Limite che tende da sopra}
    \end{subfigure}
    \begin{subfigure}{.5\textwidth}
        \centering
        \includegraphics[width=5.5cm]{images/lim-tende-sotto.png}
        \caption{Limite che tende sa sotto}
    \end{subfigure}
\end{figure}
\begin{example}
Un esempio è con $f(x) = \frac{1}{x}$ dove $\lim\limits_{x\to x_0}f(x) = 0^+$
\end{example}

\subsection{Permanenza del segno}
\begin{theorem}[Permanenza del segno]
Dato un $A \subset \mathbb{R}$, un $x_0 \in Acc(A)$ se esiste $\lim\limits_{x\to x_0}f(x) = l$, dove $l \in \overline{\mathbb{R}}$ e $l \neq 0$ allora esiste un intorno $U$ di $x_0$ t.c se $x \in U \cap A \setminus \{x_0\}$ allora $f(x)$ ha lo stesso segno di $l$.
\end{theorem}
\begin{example}
$f: (0, +\infty) \to \mathbb{R}$ \hspace{.5cm} $f(x) = \frac{1}{x}$ \hspace{.5cm} $\lim\limits_{x\to 0^+}f(x) = +\infty$ \\
Quindi visto che $+\infty > 0$ se prendiamo un intorno di $x_0$ qualsiasi $f(x)$ con $x$ appartenente all'intersezione fra il dominio e l'intorno (escluso $x_0$) tornerà che $f(x) > 0$. 
\end{example}

\subsection{Non esistenza di un limite}
Ci sono casistiche di funzioni nel quale un limite non esiste, e quindi no può essere calcolato. Per verificare ciò vediamo alcuni esempi.
\begin{example}
$\lim\limits_{x\to x_0} \sin(x)$ Non esiste. Vediamo perché.
\end{example}
\begin{wrapfigure}{l}{8cm}
    \vspace{-10pt}
    \centering
    \includegraphics[width=7cm, height=3cm]{images/es-limite-non-esiste.png}
    \caption{Limite che non esiste}
    \label{fig:limite-non-esiste}
\end{wrapfigure}

Supponiamo per assurdo che: \\$\lim\limits_{x\to x_0} \sin(x) = l$\\ \footnote{Ricorda che per la definizione di limite la f(x) deve essere compresa fra $f(x_0) + \epsilon$ e $f(x_0) - \epsilon$ qualsiasi sia il valore di $\epsilon$}Prendiamo ora un valore $\epsilon < \frac{1}{2}$. \\ \\
Se esistesse il limite $l \in \mathbb{R}$ allora dovrebbe esistere $a > 0$ t.c. $x > a \Longrightarrow l - \epsilon < \sin{x} < l + \epsilon$ ma questo assurdo perché vorrebbe dire che $\sin{x}$ oscilla con ampiezza minore di $2\epsilon$ mentre $\sin{x}$ oscilla con ampiezza 2. 
\begin{note}
Nota che nell'immagine \ref{fig:limite-non-esiste} le parti rosse escono dall''intervallo $[l-\epsilon, l+\epsilon]$.
\end{note}

\subsection{Continuità destra e sinistra}
\begin{definition}[Continuità destra e sinistra]\label{continuità-destra-sinistra}
Dato un $A \subset \mathbb{R}$, un $x_0 \in Acc(A)$:
\begin{itemize}
    \item se $\lim\limits_{x \to x_0^+}f(x) = f(x_0)$ allora si dice che $f$ è \textbf{continua a destra} in $x_0$.
    \item se $\lim\limits_{x \to x_0^-}f(x) = f(x_0)$ allora si dice che $f$ è \textbf{continua a sinistra} in $x_0$.
\end{itemize}
\end{definition}

\begin{example}
Data una $
f(x) = \begin{cases}
    1 \: \: se & x \geq 0 \\
    -1 \: \: se & x < 0
\end{cases}$\\
Il $\lim\limits_{0^+}f(x) = 1$ mentre $\lim\limits_{0^-}f(x) = -1$\\
Questo esempio ci dice, come spiegato nella definizione sopra (\ref{continuità-destra-sinistra}), che la funzione è continua a destra nel caso di $0^+$ mentre con $0^-$ la funzione non è continua a sinistra perché il risultato del limite $l \neq f(x_0)$. 
\end{example}
\begin{observation}
    Nel esempio sopra possiamo vedere che la funzione è continua in $x_0^+$ ma non in $x_0^-$. Sin può osservare infatti come una funzione $f$ è continua in un punto $x_0$ se e solo se è continua sia a destra che ha sinistra, perché ciò vorrebbe dire che entrambi i limiti, quello da $x_0^-$ e $x_0^+$, avrebbero uno stesso risultato:
    \begin{center}
        $\lim\limits_{x\to x_0^+}f(x) = l_1$ \:\:\: $\lim\limits_{x\to x_0^-}f(x) = l_2$ \:\:\: $l_1 = l_2 = f(x_0)$
    \end{center}
\end{observation}

\subsection{Teorema di confronto}
\begin{theorem}[Teorema di confronto]
    Dato un $A \subset \mathbb{R}$, un $x_0 \in Acc(x)$, e due funzioni $f,g: A \to \mathbb{R}$. Se esiste un $\lim\limits_{x\to x_0}f(x) = l_1$ e $\lim\limits_{x\to x_0}g(x) = l_2$ e se esiste un $U$ intorno di $x_0$ t.c. $x \in U \cap A \setminus \{x_0\}$ e $f(x) \leq g(x)$ allora $l_1 \leq l_2$.
\end{theorem}
Questo teorema in maniera sintetica dice che se una funzione "sta sotto" l'altra a sua volta anche il limite della prima starà sotto il secondo, detto in altre parole la disuguaglianza passa ai limiti:
\begin{center}
    Se $f(x) \leq g(x)$ allora $\lim\limits_{x\to x_0}f(x) \leq \lim\limits_{x\to x_0}g(x)$
\end{center}

\begin{observation}
    Se però esiste $f(x) < g(x)$ non potrei dire che $\lim\limits_{x\to x_0}f(x) < \lim\limits_{x\to x_0}g(x)$. Perché:
\end{observation}
Se prendiamo come esempio due funzioni una $f(x) = -\frac{1}{x}$ e una $g(x) = \frac{1}{x}$ vediamo che $f(x) < g(x)$ ma se calcoliamo i limiti $\lim\limits_{x\to x_0}f(x) = 0$ e $\lim\limits_{x\to x_0}g(x) = 0$ e quindi i limiti sono uguali. Possiamo dunque dire che le disuguaglianze passano al limite ma diventano sempre deboli:
\begin{center}
        Se $f(x) < g(x)$ allora $\lim\limits_{x\to x_0}f(x) \leq \lim\limits_{x\to x_0}g(x)$
\end{center}

\subsection{Teorema somma e prodotto}
\begin{theorem}[Teorema somma e prodotto]
    Dato un $A \subset \mathbb{R}$, un $x_0 \in Acc(A)$, e due funzioni $f,g: A \to \mathbb{R}$. Supponiamo che esistano i limiti $\lim\limits_{x\to x_0}f(x) = l_1$ e $\lim\limits_{x\to x_0}g(x) = l_2$ con $l_1, l_2 \in \overline{\mathbb{R}}$.
    \begin{itemize}
        \item Se ha senso $l_1 + l_2$ allora esiste $\lim\limits_{x\to x_0}(f + g)(x) = l_1 + l_2$.
        \item Se ha senso $l_1 \cdot l_2$ allora esiste $\lim\limits_{x\to x_0}(f + g)(x) = l_1 \cdot l_2$.
    \end{itemize}
\end{theorem}
\begin{note}
Sono esclusi i casi $l_1 = +\infty$ e $l_2 = -\infty$ (o viceversa) per il prodotto. Sono invece esclusi i casi $l_1 = 0$ e $l_2 = \pm\infty$ (o viceversa) per la somma. Questi casistiche sono dette indeterminate e non possono essere calcolate in maniera diretta.
\end{note}

\subsection{Teorema dei carabinieri}
\begin{theorem}[Teorema dei carabinieri]
    Dato un $A \subset \mathbb{R}$, un $x_0 \in Acc(A)$, e due funzioni $f,g,h: A \to \mathbb{R}$. Se esiste $\lim\limits_{x\to x_0}f(x) = l$ e $\lim\limits_{x\to x_0}h(x) = l$ (i due limiti hanno lo stesso risultato) e se esiste un intorno $U$ di $x_0$ t.c. $x \in A \cup U \setminus \{x_0\}$, se $f(x) \leq g(x) \leq h(x)$ allora esiste $\lim\limits_{x\to x_0}g(x) = l$.
\end{theorem}
Il teorema dei carabinieri dice in maniera sintetica che se due funzioni hanno lo stesso limite ed una è inferiore all'altra se esiste una $g(x)$ in mezzo a queste due funzioni avrà lo stesso limite per uno stesso $x_0$, quindi dall'esistenza dei limiti di $f$ e $h$ (uguali) deduco l'esistenza del limite di $g$
\begin{example}
Facciamo un esempio prendendo $\lim\limits_{x\to +\infty}\frac{2 + \sin{(x)}}{x}$.
Prendendo due funzioni $f(x) = \frac{1}{x}$ e $h(x) = \frac{3}{x}$ sapiamo che $\frac{1}{x} \leq \frac{2 + \sin{(x)}}{x} \leq \frac{3}{x}$. \\
Se poi andiamo a calcolare i limiti per $x \to +\infty$ di $f(x)$ e di $h(x)$ vediamo che $\lim\limits_{x\to +\infty}f(x) = 0$ e $\lim\limits_{x\to +\infty}h(x) = 0$.
Allora per il teorema dei carabinieri $\lim\limits_{x\to +\infty}\frac{2 + \sin{(x)}}{x} = 0$
\end{example}

Alcune conseguenze del teorema dei carabinieri visto sopra:
\begin{proposition}
Dato un $A \subset \mathbb{R}$, un $x_0 \in Acc(A)$, e due funzioni $f,g: A \to \mathbb{R}$:
\begin{itemize}
    \item Se $f$è lim. inferiormente in intorno di $x_0$ e $\lim\limits_{x\to x_0}g(x) = +\infty \Longrightarrow \lim\limits_{x\to x_0}(f + g)(x) = +\infty$.
    \item Se $f$è lim. superiormente in intorno di $x_0$ e $\lim\limits_{x\to x_0}g(x) = -\infty \Longrightarrow \lim\limits_{x\to x_0}(f + g)(x) = -\infty$.
    \item Se $f$è limitata in un intorno di $x_0$ e $\lim\limits_{x\to x_0}g(x) = 0 \Longrightarrow \lim\limits_{x\to x_0}(f \cdot g)(x) = 0$.
\end{itemize}
\end{proposition}

\begin{example}
Prendiamo il $\lim\limits_{x\to +\infty}x + \sin(x)$\\
$\lim\limits_{x\to +\infty}x = +\infty$ \hspace{.5cm} $\lim\limits_{x\to +\infty}\sin(x)$ non esiste.\\
Data l'inesistenza del secondo limite non posso applicare il teorema sul limite della somma ma $\sin(x)$ è limitata inferiormente quindi:
Per il teorema dei carabinieri $x - 1 \leq x + \sin(x) \leq x + 2$, e visto che $\lim\limits_{x\to +\infty}x - 1 = +\infty$ e $\lim\limits_{x\to +\infty}x + 2 = +\infty$ possiamo dire che $\lim\limits_{x\to +\infty}\sin(x) = +\infty$
\end{example}

\subsection{Limitatezza funzione con i limiti}
\begin{theorem}
    Dato un $A \subset \mathbb{R}$, un $x_0 \in Acc(A)$, e $f: A \to \mathbb{R}$. Se esiste $\lim\limits_{x\to x_0}f(x) = l$ e $l \in \mathbb{R}$ (quindi $l$ non è $\pm\infty$) allora $f$ è limitata in un intorno di $x_0$ cioè $\exists \: U$ intorno di $x_0$ e $\exists M \in \mathbb{R}$ con $M > 0$ t.c. $x \in U \cap A \Longrightarrow |f(x)| \leq M$.
\end{theorem}
Questo teorema dice che se prendiamo una funzione che ha un limite per $x\to x_0$ che è un valore diverso da $\pm\infty$ e prendiamo un intorno di $x_0$ esisterà un valore M dove per qualsiasi $x \in U \cap A$ il $|f(x)| \leq M$ che corrisponderebbe a $-M \leq f(x) \leq M$ quindi la funzione sarà limitata nell'intorno selezionato.
\begin{example}
Se prendiamo $f(x) = \frac{1}{x}$ è limitata in un intorno di $+\infty$ perché $\lim\limits_{x\to x_0}f(x) = 0$.
\end{example}

\begin{definition}
Dato un $A \subset \mathbb{R}$, un $x_0 \in Acc(A)$, e $f: A \to \mathbb{R}$ possiamo dire che:
\begin{itemize}
    \item Se $\lim\limits_{x\to x_0}f(x) = 0$ allora si dice che $f$ è \textbf{infinitesima} per $x$ che tende a $x_0$.
    \item Se $\lim\limits_{x\to x_0}f(x) = +\infty$ allora si dice che $f$ è \textbf{diverge positivamente} per $x$ che tende a $x_0$.
    \item Se $\lim\limits_{x\to x_0}f(x) = -\infty$ allora si dice che $f$ è \textbf{diverge negativamente} per $x$ che tende a $x_0$.
    \item Se $\lim\limits_{x\to x_0}f(x) = l$ ed $l \in \mathbb{R}$ ($l$ è finito) allora si dice che $f$ è \textbf{converge} in $l$ per $x$ che tende a $x_0$.
\end{itemize}
\end{definition}

\subsection{Forme indeterminate}
\begin{table}[h!]
    \setlength{\tabcolsep}{7pt}
    \renewcommand{\arraystretch}{1.5}
    \centering
    \begin{tabular}{|c c c|}
        \hline
        $[1]$ $(+\infty) + (-\infty)$ & $[2]$ $(-\infty) + (+\infty)$ & $[3]$ $0 \cdot (\pm \infty)$ \\
        $[4]$ $(\pm \infty)^0$ & $[5]$ $(0^+)^0$ & $[6]$ $(1)^{\pm \infty}$\\ 
        \hline
    \end{tabular}
    \caption{Forme indeterminate}
\end{table}
\begin{demostration}
Dimostriamo come la forma [1] e la [2] siano indeterminate (facciamo un esempio considerandone una, ma sono equivalente).\\
Prendiamo un $f(x) = 2x$ e $g(x) = -x$ e facciamo i limiti di entrambi, ed il limite della somma.\\\\
$\lim\limits_{x\to +\infty}f(x) = +\infty$ e $\lim\limits_{x\to +\infty}g(x) = -\infty$, la somma $\lim\limits_{x\to +\infty}(f + g)(x) = 2x - x = x = +\infty$\\
In questo cosa il limite di $(+\infty) + (-\infty)$ torna $+\infty$.\\ \\
Ora prendiamo invece altre due funzioni $f(x) = \frac{x}{2}$ e $g(x) = -x$ e calcoliamo come prima i limiti di entrambi ed il limite della loro somma.\\\\
$\lim\limits_{x\to +\infty}f(x) = +\infty$ e $\lim\limits_{x\to +\infty}g(x) = -\infty$, la somma $\lim\limits_{x\to +\infty}(f + g)(x) = (\frac{x}{2} - x) = -\frac{x}{2} = -\infty$\\
In questo caso invece il limite di $(+\infty) + (-\infty)$ torna $-\infty$.\\\\
Alla domanda, quale scegliamo? La risposta è nessuna delle due, infatti non potendo avere un risultato fisso diciamo che questa è una forma indeterminata.\\
Nota che questa dimostrazione è valida anche per la forma $0 \cdot (\pm \infty)$.
\end{demostration}
Per le forme [4], [5] e [6] possiamo tramite dei calcoli algebrici spiegarle riconducendoci alle prime 3 forme.\\\\
Possiamo infatti vederle come $f(x)^{g(x)} = e^{\log(f(x)^{g(x)}}) = e^{g(x) \cdot \log(f(x))}$ e quindi possiamo analizzare i casi in cui $\lim\limits_{x\to x_0}g(x) \cdot \lim(f(x))$ è indeterminato:
\begin{enumerate}
    \setcounter{enumi}{3}
    \item Con $g\to 0$ e $f\to +\infty \Longrightarrow \log(f(x)) \to +\infty = 0 \cdot +\infty$ (quindi $(+\infty)^0$ è indeterminata).
    \item Con $g\to 0$ e $f\to +0^+ \Longrightarrow \log(f(x)) \to -\infty = 0 \cdot -\infty$ (quindi $(0^+)^0$ è indeterminata).
    \item Con $g\to \pm\infty$ e $f\to 1 \Longrightarrow \log(f(x)) \to 0 = 0 \cdot \pm\infty$ (quindi $(1)^{\pm\infty}$ è indeterminata).
\end{enumerate}

\subsection{Calcolo dei limiti}
\begin{proposition}
Dato un $A \subset \mathbb{R}$, un $x_0 \in Acc(A)$, e $f: A \to \mathbb{R}$ possiamo vedere che nel calcolare alcuni limiti si verificano delle situazioni ricorrenti:
\begin{itemize}
    \item Se $\lim\limits_{x\to x_0}f(x) = 0^+ \Longrightarrow \lim\limits_{x\to x_0}\frac{1}{f(x)} = +\infty$.
    \item Se $\lim\limits_{x\to x_0}f(x) = 0^- \Longrightarrow \lim\limits_{x\to x_0}\frac{1}{f(x)} = -\infty$.
    \item Se $\lim\limits_{x\to x_0}f(x) = +\infty \Longrightarrow \lim\limits_{x\to x_0}\frac{1}{f(x)} = 0^+$.
    \item Se $\lim\limits_{x\to x_0}f(x) = -\infty \Longrightarrow \lim\limits_{x\to x_0}\frac{1}{f(x)} = 0^-$.
    \item Se $\lim\limits_{x\to x_0}f(x) = l$ con $l \neq 0, \pm\infty \Longrightarrow \lim\limits_{x\to x_0}\frac{1}{f(x)} = \frac{1}{l}$.
\end{itemize}
\end{proposition}
\begin{note}
Nota che se abbiamo $\lim\limits_{x\to x_0}f(x) = 0$ (non $0^+$ o $0^-$) non si conclude nulla su $\lim\limits_{x\to x_0}\frac{1}{f(x)}$
\end{note}

\begin{proposition}
Dati due valori $a,b \in \overline{\mathbb{R}}$, una $f:(a,b) \to \mathbb{R}$ con $f$ debolmente crescente. Allora esistono $\lim\limits_{x\to a^+}f(x) = inf(f(x))$ quando $x \in (a,b)$ e $\lim\limits_{x\to b^-}f(x) = sup(f(x))$ con $x \in (a,b)$. (Analogamente con $f$ debolmente crescente)
\end{proposition}

\begin{example}
$f:(9, -\infty)\to \mathbb{R}$ con $f(x) = -\frac{1}{x}$\\
Se calcoliamo i limiti viene che $\lim\limits_{x\to 0^+}-\frac{1}{x} = +\infty = sup(f)$ \hspace{.1cm} mentre $\lim\limits_{x\to 0^-}-\frac{1}{x} = 0 = inf(f)$
\end{example}

\subsubsection{Limiti fondamentali}
\begin{table}[h!]
    \setlength{\tabcolsep}{7pt}
    \renewcommand{\arraystretch}{1.5}
    \centering
    \begin{tabular}{|c c|c|}
        \hline
        $\lim\limits_{x\to +\infty}x^n = +\infty$ & $\lim\limits_{x\to +\infty}\frac{1}{x^n} = \frac{1}{+\infty} = 0$ & $\lim\limits_{x\to +\infty}a^x = +\infty$ e $\lim\limits_{x\to -\infty}a^x = 0^+$ se $a \geq 1$ \\\hline
        $\lim\limits_{x\to +\infty}e^x = +\infty$ & $\lim\limits_{x\to -\infty}e^x = 0^+$ & $\lim\limits_{x\to +\infty}a^x = 1$ e $\lim\limits_{x\to -\infty}a^x = 1$ se $a = 1$  \\\hline
        $\lim\limits_{x\to 0^+}\log(x) = -\infty$ & $\lim\limits_{x\to +\infty}\log(x) = +\infty$ & $\lim\limits_{x\to +\infty}a^x = 0^+$ e $\lim\limits_{x\to -\infty}a^x = +\infty$ se $0 < a < 1$ \\
        \hline
    \end{tabular}
    \vspace{-5pt}
    \caption{Limiti fondamentali}
\end{table}
Questi limiti scritti sopra sono alcuni dei limiti fondamentali (considera quando c'è $n$ come $n\in \mathbb{N}$)

\subsubsection{Limiti di polinomi}
Se prendiamo una funzione generale così definitiva:
\begin{center}
    $p(x) = a_nx^n + a_{n-1}x^{n-1} + ... + a_1x + a_0$ con $a_0, a_1, ..., a_n \in \mathbb{R}$, $n$ è il grado del polinomio $n \in N$
\end{center}
è possibile trovare una standardizzazione per la risoluzione di $\lim\limits_{x\to +\infty}p(x)$
\begin{example}
Prendiamo in $\lim\limits_{x\to +\infty}3x^2 -7x + 1$.\\
Questa è una forma indeterminata $\lim\limits_{x\to +\infty}3x^2 -7x + 1 = +\infty -\infty + 1$, per risolvere si raccogliere:
\begin{center}
    $\lim\limits_{x\to +\infty}3x^2(1 - \frac{7x}{3x^2} + \frac{1}{3x^2}) = \lim\limits_{x\to +\infty}+\infty \cdot (1 - \frac{7x}{+\infty} + \frac{1}{+\infty}) = \lim\limits_{x\to +\infty}+\infty \cdot (1 - 0 + 0) = +\infty$
\end{center}
\end{example}
\hspace{-15pt}Come regola generale presa la funzione $p(x)$ scritta sopra risolviamo il limite tendente a $\pm\infty$ raccogliendo:
\begin{center}
    \vspace{-10pt}
    $\lim\limits_{x\to \pm\infty}p(x) = \lim\limits_{x\to \pm\infty}a_nx^n (1 + \frac{a_{n-1}}{a_n} \cdot \frac{x^{n-1}}{x^n} + ... + \frac{a_{1}}{a_n} \cdot \frac{x}{x^n} + \frac{a_{0}}{a_n} \cdot \frac{1}{x^n})$
\end{center}
Poi visto che i vari $\frac{x^{n-1}}{x^n}$, $\frac{x}{x^n}$ ecc. si annullano e quindi:
\begin{center}
    $\lim\limits_{x\to \pm\infty} a_nx^4 + a_{n-1}x^{n-1} + ... + a_1x + a_0 = \lim\limits_{x\to \pm\infty}a_nx^n$
\end{center}

\subsubsection{Funzioni razionali}
Se prendiamo una situazione $\frac{p(x)}{q(x)}$ con $p,q$ due polinomi quindi
\begin{center}
    $p(x) = a_nx^n + ... + a_1x + a_0$ \hspace{1cm} $q(x) = b_mx^m + ... + b_1x + b_0$
\end{center}
Possiamo sviluppare il limite seguendo la logica vista nei singoli limiti di polinomi:
\begin{center}
    $\lim\limits_{x\to \pm\infty} = \lim\limits_{x\to \pm\infty} \frac{a_nx^n (1 + \frac{a_{n-1}}{a_n}\cdot\frac{x^{n-1}}{x^n} + ... + \frac{a_0}{a_n}\cdot\frac{1}{x^n})}{b_nx^n (1 + \frac{b_{n-1}}{b_n}\cdot\frac{x^{n-1}}{x^n} + ... + \frac{b_0}{b_n}\cdot\frac{1}{x^n})} = \lim\limits_{x\to \pm\infty}\frac{a_nx^n}{b_mx^n}$
\end{center}

\begin{example}
$\lim\limits_{x\to +\infty}\frac{7x^4 + 5x^2}{-2x^3 + x} = \lim\limits_{x\to +\infty}\frac{7x^4}{-2x^3} = \lim\limits_{x\to +\infty}\frac{7x}{-2} = -\infty$
\end{example}

\subsubsection{Limiti notevoli}
In tabella \ref{tab:limiti-notevoli} alcuni limiti notevoli, cioè limiti che all'apparenza possono sembrare il risultato ma che in realtà tornano un risultato finito.
\begin{table}[h!]
    \centering
    \setlength{\tabcolsep}{10pt}
    \renewcommand{\arraystretch}{2.5}
    \begin{tabular}{|c|c|}
        \hline
        $\lim\limits_{x\to 0}\frac{\sin(x)}{x} = 1$ & $\lim\limits_{x\to 0} \frac{1-\cos(x)}{x^2} = \frac{1}{2}$ \\\hline
        $\lim\limits_{x\to 0}\frac{e^x-1}{x} = 1$ & $\lim\limits_{x\to 0}\frac{\log(1+x)}{x} = 1$\\
        \hline
    \end{tabular}
    \caption{Limiti notevoli}
    \label{tab:limiti-notevoli}
\end{table}
\begin{demostration}
Dimostriamo $\lim\limits_{x\to 0} \frac{1-\cos(x)}{x^2} = \frac{1}{2}$.
\begin{enumerate}
    \item $\lim\limits_{x\to 0} \frac{1-\cos(x)}{x^2} = \frac{1}{2} = \frac{(1-\cos(x)) \cdot (1+\cos(x))}{x^2 \cdot (1+\cos(x))}$ \hspace{.7cm} Moltiplico e divido per $(1+\cos(x))$.
    \item $\lim\limits_{x\to 0} \frac{(1-\cos(x)) \cdot (1+\cos(x))}{x^2 \cdot (1+\cos(x))} = \frac{1-\cos^2(x)}{x^2 \cdot (1 + \cos(x))} = \frac{\sin^2(x)}{x^2 \cdot (1 + \cos(x))}$ \hspace{.7cm} Utilizzo le formule goniometriche.
    \item $\lim\limits_{x\to 0}\frac{\sin^2(x)}{x^2 \cdot (1 + \cos(x))} = \lim\limits_{x\to 0}\frac{\sin(x)}{x} \cdot \frac{\sin(x)}{x} \cdot \frac{1}{1 + \cos(x)}$ \hspace{.7cm} Spezziamo la divisioni in 3 parti.
    \item $\lim\limits_{x\to 0}\frac{\sin(x)}{x} = 1$ \: \: $\lim\limits_{x\to 0}\frac{\sin(x)}{x} = 1$ \: \: $\lim\limits_{x\to 0}\frac{1}{1 + \cos(x)} = \frac{1}{1 + 1}$ \hspace{.7cm} Facciamo il limite dei singoli pezzi.
    \item $\lim\limits_{x\to 0}\frac{1-\cos(x)}{x^2} = 1 \cdot 1 \cdot \frac{1}{2} = \frac{1}{2}$ \hspace{.7cm} Dimostrazione finita. $\blacksquare$
\end{enumerate}
\end{demostration}

\subsubsection{Logaritmi e potenze}
Vediamo una serie di casi di calcolo di limiti con logaritmi e potenze.
\begin{itemize}
    \item $\lim\limits_{x\to +\infty}\frac{\log(x)}{x} = \frac{+\infty}{+\infty}$ forma indeterminata.\\
    Eseguiamo un cambio di variabili con $y = \log(x)$ e $x = e^y$. Se $x\to +\infty \Longrightarrow y = \log(x) \to +\infty$\\\\
    Torna che $\lim\limits_{x \to +\infty}\frac{\log(x)}{x} = \lim\limits_{y\to +\infty}\frac{y}{e^y} = 0$
    \item $\lim\limits_{x\to +\infty}\frac{(\log(x))^\beta}{x^\alpha}$ con $\alpha, \beta \in \mathbb{R}$ e $\alpha, \beta > 0$\\
    Possiamo risolvere con un cambio di variabile $y = \log(x)$, $x = e^y$ e se $x \to +\infty \Longrightarrow y\to +\infty$\\\\
    Quindi $\lim\limits_{x\to +\infty}\frac{(\log(x))^\beta}{x^\alpha} = \lim\limits_{y \to +\infty}\frac{y^\beta}{(e^y)^\alpha} = \lim\limits_{y \to +\infty}\frac{y^\beta}{e^{y\cdot\alpha}} = 0$  (l'esponenziale cresce più velocemente).
    \item $\lim\limits_{x\to 0^+}x\log(x) = 0 \cdot (-\infty)$ forma indeterminata.\\
    Facciamo il cambio di variabile $y = \log(x)$, e $x = e^y$ con $x\to 0^+ \Longrightarrow y\to -\infty$.\\\\
    $\lim\limits_{x\to 0^+}x\log(x) = \lim\limits_{y\to -\infty}e^y \cdot y = 0^+ \cdot (-\infty)$ ancora indeterminata.\\
    Possiamo fare un altro cambio di varibile con $z = -y$, e $y = -z$ e se $y \to -\infty \Longrightarrow z \to +\infty$\\\\
    $\lim\limits_{y\to -\infty}e^y \cdot y = \lim\limits_{z\to +\infty}e^{-z} \cdot (-z) = \frac{-z}{e^z} = 0$
    \item $\lim\limits_{x\to 0^+}x^\alpha \cdot \log(x)$ con $\alpha > 0$.\\
    Cambio di variabile con $y = x^\alpha$, e $x = y^{\frac{1}{\alpha}}$ e con $x\to 0^+ \Longrightarrow y\to^+$\\\\
    $\lim\limits_{x\to 0^+}x^\alpha \cdot \log(x) = \lim\limits_{y\to 0^+}y \cdot \log(y^{\frac{1}{\alpha}}) = \lim\limits_{y\to 0^+}\frac{y}{\alpha} \cdot \log(y) = \frac{1}{\alpha}\lim\limits_{y\to 0^+} y \cdot \log(y) = 0$ per l'esempio sopra.
\end{itemize}

\subsection{Limite della composizione di funzioni}
\begin{theorem}[Limite della composizione di funzioni]
    Dati $A,B \subset \mathbb{R}$, una $f: A \to B$, ed una $g: B \to \mathbb{R}$, un punto $x_0 \in Acc(A)$. Se esiste $\lim\limits_{x\to x_0}f(x) = y_0$ e $y_0 \in Acc(B)$ e $\exists \lim\limits_{x\to x_0}g(y) = l \in \overline{\mathbb{R}}$ e se verifichiamo almeno delle seguenti ipotesi:
    \begin{enumerate}
        \item $y_0 \in B$ e g è continua in $y_0$.
        \item Esiste $U$ intorno di $x_0$ t.c. se $x \in U \cap A \setminus \{x_0\} \Longrightarrow f(x) \neq y_0$
    \end{enumerate}
    Allora $\lim\limits_{x\to x_0}(g \circ f)(x) = l$. Cioè:
    \begin{center}
        \vspace{-5pt}
        $\lim\limits_{x\to x_0}(g \circ f)(x) = \lim\limits_{y\to y_0}g(y)$
    \end{center}
\end{theorem}
\begin{example}
Facciamo un esempio andando a calcolare il $\lim\limits_{x\to -\infty}\arctan(x^2)$.\\
Questo limite è una composizione fra $f(x) = x^2$ e $g(y) = \arctan(y)$, che può essere scritto come $(g \circ f)(x) = g(f(x)) = g(x^2) = \arctan(x^2)$.\\
Noi abbiamo che $x_0 = -\infty$ mentre $t_0 = \lim\limits_{x\to x_0}f(x) = \lim\limits_{x\to -\infty}x^2 = +\infty$.\\
Vediamo dunque che l'ipotesi (1) non è verificata perché $y_0 = +\infty$ e non appartiene al dominio di $g$.\\
Mentre possiamo vedere che l'ipotesi (2) è ovviamente verificata perché chiedo che $f(x) \neq y_0$ cioè $f(x) \neq +\infty$ che è ovviamente sempre vero. Possiamo dunque applicare il teorema:\\
$\lim\limits_{y\to y_0}g(y) = \lim\limits_{y\to +\infty}\arctan(y) = \frac{\pi}{2} \Longrightarrow \lim\limits_{x\to -\infty}\arctan(x^2) = \frac{\pi}{2}$
\end{example}
\begin{observation}
    Quello che osserviamo nel teorema del limite della composizione di funzioni + un teorema di cambiamento di variabili. Infatti andando a prendere l'esempio di prima vediamo che:
    \begin{center}
        Da $\lim\limits_{x\to +\infty}\arctan(x^2)$ cambiamo variabile e ponto $y = x^2$, $\lim\limits_{y\to +\infty}\arctan(y) = \frac{\pi}{2}$
    \end{center}
    Nel caso $x\to -\infty$ dobbiamo vedere a quanto tende $y$, quindi $\lim\limits_{x\to -\infty} = \lim\limits_{x\to -\infty}x^2 = +\infty$
\end{observation}
\begin{observation}
    Un altra osservazione è del perché è inserita l'ipotesi (2) nel teorema. Facciamo un esempio per capire il suo scopo.\\
    Prendiamo $f: \mathbb{R} \to \mathbb{R}$, definita come $f(x) = 1 \forall x \in \mathbb{R}$.\\
    Poi prendiamo anche una $g: \mathbb{R} \to \mathbb{R}$ definita come $g(x) = \begin{cases}
        3 se & y = 1\\
        5 se & y \neq 1\\
    \end{cases}$. Facciamo la composizioni di queste due funzioni e valutiamo il limite con $x\to 0$.\\
    $(g \circ f)(x) = g(f(x)) = g(1) = 3 \forall x \in \mathbb{R} \Longrightarrow \lim\limits_{x\to 0}(g \circ f)(x) = 3$.
    Ma  $\lim\limits_{y \to y_0}g(y) = \lim\limits_{y\to 1}g(y) = 5$.\\
    $y_0 = \lim\limits_{x \to x_0}f(x) = \lim\limits_{x\to 0}f(x) = 1$.\\
    Vediamo dunque che $\lim\limits_{x\to x_0} \neq \lim\limits_{y \to y_0}g(y)$.\\
    Ma infatti in questo esempio non abbiamo considerato che non vale l'ipotesi (2) e nemmeno la (1).
\end{observation}

\subsection{Teorema di Weirstrass generalizzato}
\begin{theorem}[Teorema di Weirstrass generalizzato]
Siano $a,b \in \overline{\mathbb{R}}$ e $f: (a,b) \to \mathbb{R}$ continua t.c. $\exists \: \lim\limits_{x \to a}f(x) = l_1$ e $\exists \: \lim\limits_{x \to b}f(x) = l_2$, valgono i seguenti risultati:
\begin{enumerate}
    \item $f$ è limitata inferiormente $\Longleftrightarrow$ $l_1 \neq -\infty$ e $l_2 \neq -\infty$.
    \item $f$ è limitata superiormente $\Longleftrightarrow$ $l_1 \neq +\infty$ e $l_2 \neq +\infty$.
    \item $f$ è limitata  $\Longleftrightarrow$ $l_1 \in \mathbb{R}$ e $l_2 \in \mathbb{R}$.
    \item $f$ ha minimo $\Longleftrightarrow \: \exists x_0 \in (a,b)$ t.c. $f(x_0) \leq min\{l_1, l_2\}$.
    \item $f$ ha massimo $\Longleftrightarrow \: \exists x_0 \in (a,b)$ t.c. $f(x_0) \geq max\{l_1, l_2\}$.
 \end{enumerate}
\end{theorem}
\begin{observation}
I risultati precedenti valgono anche nel caso $a \in \mathbb{R}$ e $f: [a,b) \to \mathbb{R}$ oppure $b\in \mathbb{R}$ e $f: (a,b] \to \mathbb{R}$ (f sempre continua).
\end{observation}
\begin{wrapfigure}[9]{l}{9cm}
    \vspace{-10pt}
    \centering
    \includegraphics[width=8cm]{images/es-weirstrass-generalizzato.png}
    \vspace{-7pt}
    \caption{Massimi e minimi con Weirstrass}
    \label{fig:werstrass-generalizzato}
\end{wrapfigure}

Come possiamo vedere nella figura \ref{fig:werstrass-generalizzato} se la funzione sale sopra il limite maggiore dovrà necessariamente scendere e quindi si andrà a creare un massimo.\\\\
Ugualmente se la funzione scende sotto il limite minore vuol dire che poi risalirà creando dunque un minimo.\\\\\\

\begin{example}
Prendiamo $f(x) = \frac{1}{x - x^2}$ definita in $f:(0,1) \to \mathbb{R}$ e calcoliamo il limite agli estremi:\\ \\
$\lim\limits_{x\to 0^+}\frac{1}{x \cdot (1 - x)} = \frac{1}{0^+ \cdot 1} = \frac{1}{0^+} = +\infty$ \hspace{.7cm}
$\lim\limits_{x\to 1^-}\frac{1}{x \cdot (1 - x)} = \frac{1}{1 \cdot (1 - 1^-)} = \frac{1}{1 \cdot 0^+)} = \frac{1}{0^+} = +\infty$\\ \\
In questo caso per il teorema visto la funzione $f(x)$ ha minimo.
\end{example}
\begin{example}
Con $f(x) = \frac{x^2 + x|x| + x}{1 + x^2}$ che va da $f: \mathbb{R}\to \mathbb{R}$ verifichiamo se c'è massimo e o minimo.\\
$f(x) = \begin{cases}
    \frac{2x^2 + x}{1 + x^2}& se \: \: x \geq 0\\
    \frac{x}{1 + x^2}& se \: \: x < 0\\
\end{cases}$ \hspace{.7cm} $\lim\limits_{x\to +\infty}\frac{x^2 + x|x| + x}{1 + x^2} = 2 $ \hspace{.3cm} $\lim\limits_{x\to -\infty}\frac{x^2 + x|x| + x}{1 + x^2} = 0 $\\\\
Quello che ci dobbiamo domandare è se $\exists x_0$ t.c. $f(x) \leq 0$ e o $f(x) \geq 2$.\\\\
Se $x<0 \Longrightarrow f(x) = \frac{2x^2 + x}{1 + x^2} < 0 \forall x < 0$ quindi $f$ ha minimo.\\
Mentre se $x \geq 0 \Longrightarrow f(x) = \frac{x}{1 + x^2} \geq 0 \Longrightarrow 2x^2 + x \geq 2 + 2x^2 \Longrightarrow x \geq 2$ quindi $f$ ha anche massimo.
\end{example}
\newpage
\section{Infinitesimi}
\subsection{O-piccolo}
\begin{definition}[O-piccolo]
Prendiamo $A \subset \mathbb{R}, x_0 \in Acc(A)$, $f,g: A \to \mathbb{R}$ ($x_0 \in \overline{\mathbb{R}}$). Si dice che $f$ è \textbf{o-piccolo} di $g$ per x che tende a $x_0$, e si scrive $f(x) = o(g(x))$ per $x \to x_0$ se esiste una funzione $\omega(x)$ t.c. $\lim\limits_{x \to x_0} \omega(x) = 0$ e $f(x) = g(x) \cdot \omega(x)$.
\end{definition}
\begin{observation}
Se esiste un intorno $U$ di $x_0$ t.c. $g(x) \neq 0 \forall x \in U \setminus \{x_0\}$ allora $f(x) = o(g(x)) \Longleftrightarrow \lim\limits_{x\to x_0}\frac{f(x)}{g(x)}=0$ (vuol dire che $f(x) = \omega(x) \cdot g(x) = \frac{f(x)}{g(x)} = \omega(x) \to 0$), possiamo infatti scrivere:
\begin{center}
    \vspace{-8pt}
    $\lim\limits_{x\to 0}\frac{f(x)}{g(x)} = 0$ allora $f(x) = o(g(x))$
\end{center}
\end{observation}
Intuitivamente possiamo dire anche che se $f(x) = o(g(x))$ vuol dire che $f(x)$ è infinitesimamente più piccola di $g(x)$ per $x\to x_0$.
\begin{example}
Se prendiamo una $f(x) = x^3$ e $g(x) = x^2$, $f(x) = o(g(x))$ per $x\to 0$.\\
Infatti $\frac{f(x)}{g(x)} = \frac{x^3}{x^2} = x \to 0$ per $x\to 0$.\\
Possiamo vedere l'applicazione della definizione con $f(x) = g(x) \cdot \omega(x)$ con $\omega(x) = x$ e visto $\omega(x) \to 0$.
\end{example}

\subsection{Proprietà o-piccolo}
Dato un $A \subset \mathbb{R}$, un $x_0 \in Acc(A)$, e due funzioni $f,g: A \to \mathbb{R}$ e con tutti gli o-piccoli che si intendono per $x\to x_0$, valgono le seguenti proprietà.
\begin{enumerate}
    \item $f(x) \cdot o(g(x)) = o(f(x) \cdot g(x))$.
    \item Se $k \in \mathbb{R}$, e $k \neq 0 \Longrightarrow o(k \cdot g(x)) = o(g(x))$.
    \item $o(g) + o(g) = o(g)$. \footnote{Scrivere $o(g(x))$ oppure $o(g)$ è equivalente}
    \item Se $\lim\limits_{x\to x_0}f(x) = 0 \Longrightarrow f(x) \cdot g(x) = o(g(x))$.
    \item Se $\lim\limits_{x\to x_0}f(x) = 0 \Longrightarrow o(g) + o(f \cdot g) = o(g)$.
    \item $o(o(g)) = o(g)$.
    \item $o(f + g) = o(f) + o(g)$.
    \item $o(g) \cdot o(f) = o(f \cdot f)$.
\end{enumerate}

\begin{observation}
Facciamo un osservazione relativa alla proprietà (3) e di essa valga anche nel caso $o(g) - o(g)$.\\
$o(g) - o(g) = o(g) + (-1)\cdot o(g) = o(g) + o(-1 \cdot g) = o(g) + o(g) = o(g)$. \\
Vediamo dunque che la proprietà (2) comprende anche i casi con il meno.
\end{observation}

\begin{example}
Facciamo un esempio per capire meglio l'osservazione sopra. \\
Prendiamo $f(x) = x^3$, $g(x) = x^2$ e $h(x) = x^4$ , vediamo che $x^3 = o(x^2)$ e $x^3 = o(x^2)$ ma che $x^3 - x^4 \neq 0$.
\end{example}

\begin{observation}
Una casistica molto frequente e quella con $g = $ potenza di $x$ (o di $x - x_0$).\\\\
Infatti se prendiamo $\alpha, \beta \in \mathbb{R}$ con $\alpha > \beta \Longrightarrow x^\alpha = o(x^\beta)$ perché $x^\alpha = x^\beta \cdot x^{\alpha - \beta}$.\\
Quindi quando $\omega(x) = x^{\alpha-\beta} \to 0$ perché $\alpha > \beta$.
Mentre quando $\omega(x) = \frac{x^\alpha}{x^\beta} \to 0$ sempre perché $\alpha > \beta$.
\end{observation}

\begin{example}
Prendiamo $f(x) = \tan(x) \cdot \sin(x)$ e dico che $f(x) = o(x)$ per $x\to 0$.
Infatti $\lim\limits_{x\to 0}\frac{f(x)}{x} = \lim\limits_{x\to 0}\frac{\tan(x) \cdot \sin(x)}{x} = \lim\limits_{x\to 0}\tan(x) \cdot \lim\limits_{x\to 0}\frac{\sin(x)}{x} = 0 \cdot 1 = 0$ (ricorda il limite notevole $\lim\limits_{x\to0}\frac{\sin(x)}{x} = 1$)
\end{example}

\newpage
\subsection{Sviluppi al primo ordine}
\begin{itemize}
    \item Dai limiti notevoli sappiamo che $\lim\limits_{x\to 0}\frac{\sin(x)}{x} = 1 \Longrightarrow \lim\limits_{x\to 0}\frac{\sin(x)}{x} - 1 = 0$.\\
    Possiamo dunque dire che $\lim\limits_{x\to 0}\frac{\sin(x) - x}{x} = 0$ quindi per definizione:
    \begin{center}
        \vspace{-5pt}
        $\sin(x) - x = o(x)$ \:\:\: e che \:\:\: $\sin(x) = x - o(x)$ per $x\to 0$
    \end{center} 
    \item Dal limite notevole $\lim\limits_{x\to 0}\frac{1 - \cos(x)}{x^2} = \frac{1}{2}$ ottengo, come prima, che:
    \begin{center}
        \vspace{-5pt}
        $1 - \cos(x) - \frac{1}{2} = o(x^2)$ \:\:\: e che \:\:\: $\cos(x) = 1 - \frac{x^2}{2} + o(x^2)$
    \end{center}
    \item $\lim\limits_{x\to 0}\frac{\tan(x)}{x} = \lim\limits_{x\to 0}\frac{\sin(x)}{\cos(x)} \cdot \frac{1}{x} = \lim\limits_{x\to 0}\frac{\sin(x)}{x} \cdot \frac{1}{\cos(x)} = 1 \cdot \frac{1}{1} = 1 \Longrightarrow \tan(x) = x + o(x)$
    \item $\lim\limits_{x\to x_0}\frac{e^x - 1}{x} = 1 \Longrightarrow e^x = 1 + x + o(x)$
    \item $\lim\limits_{x\to 0} \frac{\log(1 + x)}{x} = 1 \Longrightarrow \log(1 + x) = x + o(x)$
\end{itemize}

\begin{example}
    Esempio risolvendo $(\tan(x))^2$ in termini di o-piccoli. Sappiamo che $\tan(x) = x + o(x)$.\\\\
    $\tan(x)^2 \:\: = \:\: (x + o(x))^2 = x^2 + 2x \cdot o(x) + (o(x))^2 \:\: = \:\: x^2 + o(2x^2) + o(x^2) \:\: = \:\: x^2 + o(x^2) + o(x^2) \:\: = \:\: x^2 + o(x^2)$\\\\
    Quindi il risultato è che $\tan(x)^2 = x^2 + o(x^2) $
\end{example}

\begin{example}
    Proviamo a risolvere $\lim\limits_{x\to 0}\frac{\cos(\sin^2(x)) - 1}{x^4}$. Ricorda che $\sin(x) = x + o(x)$, quindi \\\\
    Ricorda che $\sin(x) = x + o(x)$, quindi $\sin^2(x) = (x + o(x))^2 = x^2 + o(x^2)$ \\\\
    $\cos(\sin^2(x)) - 1 = \cos(x^2 + o(x^2)) - 1$ poniamo $t = x^2 + o(x^2)$\\\\
    Abbiamo quindi che in termini di o-piccolo $\cos(t) = 1 + \frac{t^2}{2} + o(t^2)$ con $t\to 0$\\\\
    Possiamo fare questa sostituzione perché $\cos(t) = 1 + \frac{t^2}{2} + o(t^2)$ vale con $t\to 0$, se $t = x^2 + o(x^2)$ ottengo che se $x\to 0$ allora $x^2 + o(x^2) \to 0$ quindi $t\to 0$.\\\\
    Ri-sostituendo la $t$ abbiamo che $\cos(t) = 1 - \frac{t^2}{2} + o(t^2) = 1 - \frac{(x^2 + o(x^2))^2}{2} + o((x^2 + o(x^))^2) =$\\\\
    $= 1 - \frac{x^4 + 2x^2 \cdot o(x^2) + (o(x^2))^2}{2} + o(x^4 + 2x^2 \cdot o(x^2) + o(x^2)^2) = 1 - \frac{x^4 + o(x^4) + o(x^4)}{2} + o(x^4 + o(x^4) + o(x^4)) =$\\\\
    $= 1 - \frac{x^4}{2} + o(x^4) + o(x^4) = 1 - \frac{x^4}{2} + o(x^4)$ quindi abbiamo che:\\\\
    $\frac{\cos(\sin^2(x)) - 1}{x^4} = \frac{1 - \frac{x^4}{2} + o(x^4) -1}{x^4} = \frac{-\frac{x^4}{2} + o(x^4)}{x^4} = -\frac{1}{2} + \frac{o(x^4)}{x^4}$\\\\
    Visto che $\frac{o(x^4)}{x^4}$ tende a 0 abbiamo che $\lim\limits_{x\to 0}\frac{\cos(\sin^2(x)) - 1}{x^4} = -\frac{1}{2}$
\end{example}

\subsection{O-grande}
\begin{definition}[O-grande]
    Dato $A \subset \mathbb{R}$, $x_0 \in Acc(A)$, e $f,g: A \to \mathbb{R}$. Se $\exists M \in \mathbb{R}$ t.c. $|f(x)| \geq M \cdot |g(x)|    \forall x \in U \cap A \setminus \{x_0\}$ dove $U$ è un intorno di $x_0$, allora si dice che $f$ è O-grande di $g$ per $x$ che tende a $x_0$ e si scrive $f(x) = O(g(x))$ per $x\to x_0$.
\end{definition}

\begin{observation}
Se $g$ non si annulla in un intorno di $x_0$ allora possiamo scrivere che:
    \begin{center}
        $f(x) = O(g) \Longleftrightarrow |\frac{f(x)}{g(x)}| \geq M$ in un intorno di $x_0$
    \end{center}
\end{observation}

\begin{example}
    Facciamo un esempio prendendo $f(x) = x\sin(x)$ e $g(x) = x$.\\
    Vediamo che $|\frac{f(x)}{g(x)}| = |\frac{x\sin(x)}{x}| = |\sin(x)| \geq 1$ quindi $f(x) = O(g(x))$ per $x\to x_0$ per qualunque $x_0 \to \overline{\mathbb{R}}$
\end{example}

\begin{definition}
    Dato $A \subset \mathbb{R}$, $x_0 \in Acc(A)$, e $f,g: A \to \mathbb{R}$ infinitesime per $x\to x_0$ (cioè $\lim\limits_{x\to x_0}f(x) = 0$ e $\lim\limits_{x\to x_0}g(x) = 0$). Se esistono $L, \alpha \in \mathbb{R}$ con $L \neq 0$ t.c. $f(x) = L \cdot (g(x))^\alpha + o((g(x))^\alpha)$ per $x\to x_0$ si dice che $f$ è infinitesima di ordine $\alpha$ rispetto a $g$ con parte principali $L(g(x))^\alpha$ per x che tende a $x_0$.\\
    Stessa definizioni del caso in. cui $f$ e $g$ siano divergenti (cioè $\lim\limits_{x\to x_0}f(x) = \pm\infty$ e $\lim\limits_{x\to x_0}g(x) = \pm\infty$)
\end{definition}

\begin{example}
    Prendiamo $f(x) = 3\sin(x) + x^2$ e $g(x) = x$ con $x_0 = 0$.\\
    $f$ è di ordine 1 rispetto a $g$ per $x\to 0$ con parte principale $3x$. Infatti $3\sin(x) + x^2 = 3x + o(x)$.\\
    (Perché $\sin(x) = x + o(x) \Longrightarrow 3\sin(x) + x^3 = 3x + o(x) + x^2 = 3x + o(x)$)
\end{example}

\begin{example}
    Prendiamo il caso con $f(x) = 5x^4 + (2\sin(x)) \cdot x^2 + 3x$ e $g(x) = x$.\\
    $f$ è di ordine 4 rispetto a $x$ per $x\to +\infty$ con parte principale $5x^4$\\
    Questo perché $(2\sin(x)) \cdot x^2 + 3x = o(x^4)$ quindi, $f(x) = 5x^4 + o(x^4)$ infatti $\frac{(2\sin(x)) \cdot x^2 + 3x}{x^4} \to 0$
\end{example}

\begin{example}
    Guardiamo un esempio con $f(x) = \log(e^{3x} + x^2)$ per $x\to +\infty$\\\\
    $\log(e^{3x} + x^2) = \log(e^{3x} \cdot (1 + \frac{x^2}{e^{3x}}) = \log(e^{3x}) + \log(1 + \frac{x^2}{e^{3x}}) = 3x + \log(1 + \frac{x^2}{e^{3x}})$\\
    Abbiamo che $\frac{x^2}{e^{3x}}\to 0$ per $x\to +\infty$. Possiamo dunque dire che $f(x)$ è di ordine 1 rispetto a $x$ con parte principale $3x$ per $x\to +\infty$. Quindi $f(x) = 3x + o(x)$
\end{example}
\newpage
\section{Asintoti}

\subsection{Asintoto orizzontale}
\begin{definition}[Asintoto orizzontale]
Data una $f: A \to \mathbb{R}$, un $a \in \mathbb{R}$. Se esiste $\lim\limits_{x\to x_0}f(x) = l \in \mathbb{R}$ (finito) si dice che $f$ ha un \textbf{asintoto orizzontale} di equazione $y = l$ per x che tende a $\pm\infty$.
\end{definition}
\begin{example}
Prendiamo $f(x) = e^x$ con $f:\mathbb{R}\to \mathbb{R}$.
\end{example}
\begin{wrapfigure}[7]{l}{7cm}
    \vspace{-5pt}
    \centering
    \includegraphics[width=5.5cm]{images/asintoto-esponenziale.png}
    \caption{Asintoto orizzontale di $e^x$}
\end{wrapfigure}

Andiamo come prima cosa a calcolare il limite: $\lim\limits_{x\to -\infty}f(x) = 0$.
Possiamo così vedere che $f$ ha un asintoto orizzontale di equazione $y=0$ per $x\to -\infty$. \\\\
Come possiamo notare nell'immagine a fianco (asintoto segnato dalla linea blu in basso).\\\\\\\\\\

\begin{example}
Facciamo un altro esempio prendendo questa volta $f(x) = \arctan(x)$ con $f: \mathbb{R}\to \mathbb{R}$
\end{example}
\begin{wrapfigure}[9]{r}{7.5cm}
    \vspace{-5pt}
    \centering
    \includegraphics[width=5.5cm]{images/asintoto-arctan.png}
    \caption{Asintoto orizzontale di $\arctan(x)$}
\end{wrapfigure}

Anche qui compre prima cosa calcoliamo il limite sia vero $+\infty$ che verso $-\infty$ della funzione:\\ $\lim\limits_{x \to +\infty}\arctan(x) = \frac{\pi}{2}$
$\lim\limits_{x \to -\infty}\arctan(x) = -\frac{\pi}{2}$\\\\
Vediamo dunque due asintoti con equazioni $y=\frac{\pi}{2}$ e $y=-\frac{\pi}{2}$ rispettivamente con $x\to +\infty$ e $x\to -\infty$.
Possiamo vedere i due asintoti nell'immagine a fianco (rette in blu).\\

\subsection{Asintoto verticale}
\begin{definition}[Asintoto verticale]
Dato un $A \subset \mathbb{R}$, $x_0\in Acc(A)$, $x_0 \in \mathbb{R}$, una $f:A\to \mathbb{R}$. Se $f$ diverge per $x$ che tende a $x_0$ da destra o da sinistra (o da entrambe le parti) si dice che f ha un \textbf{asintoto verticale} di equazione $x=x_0$.
\end{definition}

\begin{example}
Prendiamo la funzione $f(x) = \frac{1}{x}$ definita come $f: \mathbb{R} \setminus \{0\} \to \mathbb{R}$.
\end{example}
\begin{wrapfigure}[9]{r}{7.5cm}
    \vspace{-10pt}
    \centering
    \includegraphics[width=3cm]{images/asintoto-verticale-1.png}
    \caption{Asintoto verticale di $\frac{1}{x}$}
\end{wrapfigure}

Andiamo a calcolare nel punto di discontinuità, che è lo 0, il limite sia da destra che da sinistra:\\\\
$\lim\limits_{x\to 0^+}\frac{1}{x} = +\infty$, $\lim\limits_{x\to 0^-}\frac{1}{x} = -\infty$.\\\\
Vediamo dunque la $f$ ha un asintoto verticale di equazione $x=0$. Possiamo vedere l'asintoto nell'immagine a fianco (asintoto verticale segnato in blu).\\

\begin{observation}
Una funzione al massimo ha 2 asintoti orizzontali (uno a $+\infty$ ed uno a $-\infty$) ma può anche avere $\infty$ asintoti verticali, come nel caso di $f(x) = \tan(x)$ che ha $\infty$ asintoti verticali.
\end{observation}

\subsection{Asintoto obliquo}
\begin{definition}[Asintoto obliquo]
    Data una $f:(a, +\infty) \to \mathbb{R}$. Se esiste $\lim\limits_{x\to +\infty}\frac{f(x)}{x} = m$ con $m \in \mathbb{R}$ e $m\neq 0$, e se esiste anche $\lim\limits_{x\to +\infty}f(x) - mx = q$ con $q \in \mathbb{R}$ allora si dice che $f$ ha un \textbf{asintoto obliquo} di equazione $y = mx + q$ per $x\to +\infty$. Lo stesso vale con $x \to -\infty$.
\end{definition}

\begin{example}
Facciamo un esempio di calcolo dei asintoto obliquo con $f(x) = \frac{2x^2 + 3x +2}{x-5}$.\\
$\lim\limits_{x\to +\infty}\frac{f(x)}{x} = \frac{2x^2 + 3x +2}{x^2-5x} = 2$, quindi $m=2$\\
$\lim\limits_{x\to +\infty}f(x) - mx = \frac{2x^2 + 3x +2}{x-5} - 2x = \lim\limits_{x\to +\infty}\frac{2x^2 + 3x +2 - 2x(x-5)}{x-5} = \lim\limits_{x\to +\infty} \frac{3x + 2 + 10x}{x-5} = \lim\limits_{x\to +\infty}\frac{13x + 2}{x-5} = 13$\\\\
Abbiamo dunque che esiste un asintoto obliquo di equazione $y=2x +13$ per $x\to +\infty$
\end{example}

\begin{observation}
Una funzione può avere al massimo 2 asintoti obliqui (uno a $+\infty$ ed uno a $-\infty$). Inoltre non può avere contemporaneamente un asintoto orizzontale ed uno obliquo "dalla stessa parte".
\end{observation}

\begin{example}
Prendiamo $f(x) = 3x + 5\log(x)$ definita come $f: (0,+\infty) \to \mathbb{R}$. Proviamo ora a calcolare l'asintoto obliquo.\\\\
$\lim\limits_{x\to +\infty}\frac{f(x)}{x} = \lim\limits_{x\to +\infty} \frac{3x + 5\log(x)}{x} = 3 + \lim\limits_{x\to +\infty}\frac{5\log(x)}{x} = 3 + 0 = 3$ quindi $m=3$.\\
$\lim\limits_{x\to +\infty}f(x) - mx = \lim\limits_{x\to +\infty} 3x + 5\log(x) -3x = 5\log(x) = +\infty$.\\\\
Visto che la $q$ non torna un numero finito vediamo che questa funzione non ha asintoto obliquo.
\end{example}
\input{derivate}
\newpage
\section{Sviluppi di Taylor}

\subsection{Fattoriale}
\begin{definition}[Fattoriale]
Dato un $n \in \mathbb{N}$ con $n \geq 1$ definiamo un fattoriale come il prodotto dei primi n numeri naturali:
\begin{center}
    $n! = 1 \cdot 2 \cdot 3 \cdot 4 \cdot ... \cdot n$
\end{center}
\end{definition}
\begin{note}
Nota che $0! = 1$ per definizione.
\end{note}
\begin{example}
$1! = 1$ \hspace{.5cm} $2! = 1 \cdot 2 = 2$ \hspace{.5cm} $3! = 1 \cdot 2 \cdot 3 = 6$ \hspace{.5cm} $4! = (1 \cdot 2 \cdot 3) \cdot 4 = 24$
\end{example}
\hspace{-15pt}Possiamo definire un uguaglianza per definire il fattoriale:
\begin{center}
    $(n+1)! = n! \cdot (n+1)$ dove $(n+1)! = [1 \cdot 2 \cdot ... \cdot n] \cdot (n+1) = n! \cdot (n+1)$
\end{center}

\subsection{Sommatorie}
Supponiamo di avere dei numeri naturali indicizzati con un numero naturale.
\begin{center}
    $a_1, a_2, ..., a_n$ e $a_j \in \mathbb{R}$ con $j \in \mathbb{N}$
\end{center}
Per esempio si potrebbe prendere $a_j = \frac{1}{j}$ quindi: $a_1 = \frac{1}{1}$, $a_2 = \frac{1}{2}$, $a_3 = \frac{1}{3}$, ecc.
Oppure possiamo $a_j = \sqrt{j}$ quindi: $a_1 = \sqrt{1}$, $a_2 = \sqrt{2}$, ecc.

\begin{definition}[Sommatoria]
Definisco sommatoria degli $a_j$ per $j$ che va da $m$ ad $n$ dove $m,n \in \mathbb{N}$ e $m \leq n$, e si scrivere\footnote{Usiamo $j$ per convenzione ma è possibile utilizzare qualsiasi variabile}:
\begin{center}\vspace{-5pt}
    $\sum\limits_{j = m}^n a_j = a_m + a_{m+1} + a_{m+2} + ... + a_n$
\end{center}
\end{definition}
\begin{example}
$\sum\limits_{j = 1}^5 \frac{1}{j} = \frac{1}{1} + \frac{1}{2} + ... + \frac{1}{5}$
\end{example}
\begin{example}
$\sum\limits_{j = 0}^3 j^2 = 0^2 + 1^2 + 2^2 + 3^2 = 1 + 4 + 9 = 14$
\end{example}

\subsection{Formula di Taylor}
\subsubsection{Taylor con resto di Peano}
Supponiamo di avere una funzione $f$ derivabile nel punto $x_0 \in (a,b)$, allora abbiamo visto che posso scrivere $f(x) = f(x_0) + f'(x_0) \cdot (x-x_0) + o(x - x_0)$ per $x\to x_0$. Abbiamo dunque un polinomi di grado 1 ungule a $f(x_0) + f'(x_0) \cdot (x-x_0)$ ed un resto $o(x - x_0)$, $f$ quindi differisce dal polinomio per un resto che è infinitesimo rispetto a $x- x_0$ cioè $\lim\limits_{x\to x_0} = \frac{o(x - x_0)}{x - x_0} = 0$.\\\\
Posso precisare meglio la quantità di $o(x - x_0)$ ma $f$ deve essere derivabile più volte nel punto $x_0$.

\begin{definition}[Formula di Taylor con resto di Peano]
Dato una funzione $f:(a,b) \to \mathbb{R}$ e $x_0 \in (a,b)$. Se $f$ è derivabile n volte in $x_0$ ed almeno $n-1$ volte nel resto dell'intervallo (a,b) (cioè in $(a,b) \setminus \{x_0\}$) allora esiste un unico polinomio $P_n(x)$ di grado $\leq n$ ed una funzione $R_n(x)$ tale che:
\begin{center}
    $f(x) = P_n(x) + R_n(x)$ e $R_n(x) = o(x - x_0)^n$ per $x\to x_0$
\end{center}
Il polinomio $P_n(x)$ ha la seguente forma:
\begin{center}\vspace{-5pt}
    $P_n(x) = \sum\limits_{j=0}^n \frac{f^{(j)}(x_0)}{j!} \cdot (x - x_0)^j$
\end{center}
\end{definition}

Scritto in maniera esplicita:\\
$P_n = f(x_0) + f'(x_0) \cdot (x - x_0) + f''(x_0)\frac{f''(x_0)}{2} \cdot (x-x_0)^2 + ... + \frac{f^{(n)}(x_0)}{n!} \cdot (x - x_0)^n$

\begin{observation}
Il grado massimo del polinomio è correlato all'ordine di infinitesimo del resto. Cioè $P_n$ è di grado n e $R_n = o(x - x_0)^n$. Questo vuol dire che: $f(x) - P_n(x) = o((x-x_0)^n)$, $o((x-x_0)^n)$ è la differenza fra la funzione ed il polinomio che l'approssima.
\end{observation}

\subsubsection{Taylor con resto di Lagrange}
\begin{definition}[Formula di Taylor con resto di Lagrange]
Dato una funzione $f:(a,b) \to \mathbb{R}$ e $x_0 \in (a,b)$ e $f$ derivabili in $n+1$ volte in $(a,b) \setminus \{x_0\}$ e n volte in $x_0$. Allora $f(x) = P_n + R_n(x)$ ed esiste $z$ compreso tra $x$ e $x_0$ tale che:
\begin{center}
    $R_n(x) = \frac{f^{n+1}(z) \cdot (x - x_0)^{n+1}}{(n+1)!}$
\end{center}
\end{definition}
\hspace{-15pt}Dico un punto compreso fra $x$ e $x_0$ perché a priori non so quali dei due valori sta a destra e quale sta a sinistra, quindi parlo semplicemente di punto compreso.

\subsubsection{Esempi di formula di Taylor}
\begin{example}
$f(x) = e^x$ e $f'(x) = e^x$, $f''(x) = e^x$, ... $f^{(j)}(x) = e^x \: \forall j \in \mathbb{N}$. La calcolo in $x_0 = 0$ \footnote{Si dice che in questo caso si fa centrato in 0}.\\
$f(0) = 1$, $f'(0) = 1$, ..., $f^{j}(0) = 1$. Quindi $e^x = (\sum\limits_{j=0}^n\frac{x^j}{j!}) + o(x^n) = (\sum\limits_{j=0}^n\frac{f^{(j)}(0)}{j!}\cdot (x-0)^j) + o(x^n)$\\
$e^x = 1 + x + \frac{x^2}{2} + \frac{x^3}{3!} + \frac{x^4}{4!} + ... + \frac{x^n}{n!} + o(x^n)$.\\
Per esempio in ordine 2: $e^x = 1 + x + \frac{x^2}{2} + o(x^2)$, se lo confrontiamo con il limite notevole $e^x = 1 + x + o(x)$ vediamo che $o(x)$ (che è $R_1(x)$)in realtà è $\frac{x^2}{2} + o(x^2)$ (che è $R_2(x)$).
\end{example}
\begin{observation}
$R_2(x)$ in particolare è un $o(x)$ perché se faccio $\frac{R_2(x)}{x} = \frac{\frac{x^2}{2} + o(x^2)}{x} = \frac{x}{2} + o(x) \to 0$ se $x\to 0$.
Quella con il grado 2 è più precisa di quella con il grado 1.
\end{observation}

\begin{example}
$f(x) = \sin{x}$, $f'(x) = \cos{x}$, $f''(x) = -\sin{x}$, $f'''(x)=-\cos{x}$.\\
$f(0) = 0$, $f'(0) = 0$, $f''(0) = 0$, $f'''(0) = -1$. $\sin{x} = \sum\limits_{i=0}^n\frac{f^{(i)}(0)}{j!} \cdot x^j + R_n(x)$.\\
$\sin{x} = 0 + \frac{x}{1} + 0 \cdot \frac{x^2}{2} - \frac{x^3}{3! + o(x^3)} = x - \frac{x^3}{6} + o(x^3)$. Ordine $n=3$.\\
In questo caso $P_3(x) = x - \frac{x^3}{6}$ e $R_3(x) = o(x^3)$.\\
Proviamo con ordine 4: $\sin{x} = 0 + 1 \cdot x + 0 \cdot \frac{x^2}{2} - 1 \frac{x^3}{3!} + o \cdot \frac{x^4}{4!} + o(x^4) = x - \frac{x^3}{6} + o(x^4)$. \\
In questo caso invece $P_4(x) = x - \frac{x^3}{6}$ e $R_4(x) = o(x^4)$, vediamo che in questo caso $P_3(x) = P_4(x)$.\\\\
Ora confrontiamo:\\
$\sin{x} = x - \frac{x^3}{6} + o(x^3)$ ordine 3 \hspace{.5cm} $\sin{x} = x - \frac{x^3}{6} + o(x^4)$ ordine 4.\\
Possiamo vedere che sono vere entrambi ma la seconda è più precisa perché ha un resto più piccolo.\\
Allo stesso modo $\sin{x} = x + o(x)$ ma visto che sappiamo che la derivata seconda del seno calcolato in 0 è 0 possiamo scrivere in maniera più precisa $\sin{x} = x + o(x^2)$.
\end{example}

\subsection{Taylor per le funzioni elementari}
Possiamo dunque ora scrivere le varie formule di Taylo per delle funzioni ricorrenti.\\

\hspace{-15pt}\textbf{Formula seno:} $\sin{x} = (\sum\limits_{j = 0}^n \frac{(-1)^j \cdot x^{2j +1}}{(2j + 1)!}) + o(x^{2n+2})$

\begin{example}
Proviamo questa formula con $n=2$.\\\\
$\frac{(-1)^0 \cdot x^{2 \cdot 0 + 1}}{(2 \cdot 0 + 1} + \frac{(-1)^1 \cdot x^{2 \cdot 1 + 1}}{(2 \cdot 1 + 1)!} + \frac{(-1)^2 \cdot x^{2 \cdot 2 \cdot 1}}{(2 \cdot 2 + 1)!} + o(x^{2 \cdot 2 + 2}) = x - \frac{x^3}{3!} + \frac{x^5}{5!} + o(x^6)$.\\
\end{example}

\hspace{-15pt}\textbf{Formula coseno:} $\cos{x} = (\sum\limits_{j = 0}^n \frac{(-1)^j \cdot x^{2j +1}}{(2j)!}) + o(x^{2n+1})$
\begin{example}
Formula di Taylor di grado 7 per il coseno:\\
$\cos{x} = 1 - \frac{x^2}{2} + \frac{x^4}{4!} - \frac{x^6}{6!} + o(x^7)$.\\
\end{example}

\hspace{-15pt}\textbf{Formula logaritmo:} $\log(1+x) = (\sum\limits_{j=1}^n(-1)^{j+1}\frac{x^j}{j}) + o(x^n)$
\begin{example}
Facciamo un esempio con $n=4$ della formula del logaritmo:\\
$\log(1+x) = x - \frac{x^2}{2} + \frac{x^3}{3} - \frac{x^4}{4} + o(x^4)$
\end{example}

\begin{note}
Nota che il coseno è una funzione peri ed il polinomio dalla funzione di Taylor contiene sempre potenze pari mentre il seno essendo dispari contiene solo dispari.\\
\end{note}

\hspace{-15pt}\textbf{Formula tangente:} per la tangente la formula è molto complicata quindi scriviamo semplicemente:\\
$\tan(x) = x + o(x^2)$ e $\tan(x) = x + \frac{x^3}{3} + \frac{2x^5}{15} + o(x^6)$.\\

\hspace{-15pt}\textbf{Formula Arcotangente:} $\arctan(x) = (\sum\limits_{j = 0}^n (-1)^j \frac{x^{2j +1}}{2j +1}) + o(x^{2n+2})$
Quindi sviluppata al settimo grado:\\
$\arctan(x) = x - \frac{x^3}{3} + \frac{x^5}{5} - \frac{x^7}{7} + o(x^8)$
\begin{note}
Nota che anche nell'arcotangente come nel logaritmo non c'è il fattoriale.\\
\end{note}

\hspace{-15pt}\textbf{Formula Binomiale:} dato $\alpha \in \mathbb{R}$ possiamo scrivere:\\\\
$(1+ \alpha) = 1 + \alpha x + \frac{\alpha(\alpha -1)}{2} \cdot x^2 + \frac{\alpha(\alpha -1)(\alpha -2)}{3!}\cdot x^3 + ... + \frac{\alpha(\alpha -1)(\alpha -2)...(\alpha - n+1)}{n!}\cdot x^n + o(x^n)$.

\begin{example}
Con $\alpha = \frac{1}{2}$ quindi $\sqrt{1 + x} = (1 + x)^{\frac{1}{2}}$.\\
$(1 + x)^{\frac{1}{2}} = 1 + \frac{1}{2}x + \frac{\frac{1}{2}(\frac{1}{2}-1)}{2} \cdot x^2 + o(x^2) = 1 + \frac{x}{2} - \frac{1}{8}x^2 + o(x^2)$. 
\end{example}

\begin{example}
Con invece $\alpha = -1 $ quindi con $(1 + x)^{-1} = \frac{1}{1+x}$.\\
$\frac{1}{1+x} = 1 - x + \frac{(-1)(-2)}{2!} \cdot x^2 + \frac{(-1)(-2)(-3)}{3!} \cdot x^3 + o(x^3) = 1 - x + \frac{2}{2}x^2 - \frac{3!}{3!}\cdot x^3 + o(x^3) = 1 - x + x^2 + x^3 + o(x^3)$.\\\\
Quindi se sostituiamo $x = -t$ abbiamo che:\\
$\frac{1}{1-t} = 1 - (-t) + (-t)^2 - (-t^3) + o(t^3) = 1 + t + t^2 + t^3 + o(t^3)$, generalizzando possiamo scrivere:\\
$\frac{1}{1-t} = 1 - (-t) + (-t)^2 - (-t^3) + ... + t^n + o(t^n)$
\end{example}

\begin{table}[h!]
    \setlength{\tabcolsep}{5pt}
    \renewcommand{\arraystretch}{2.2}
    \centering
    \begin{tabular}{|c|c|}
        \hline
        $e^x$ & $1 + x + \frac{x^2}{2!} + \frac{x^3}{3!} + \frac{x^4}{4!} + ... + \frac{x^n}{n!} + o(x^n)$  \\
        $\log(1+x)$ & $x - \frac{x^2}{2} + \frac{x^3}{3} - \frac{x^4}{4} + \frac{x^5}{5} + ... + (-1)^{n-1}\frac{x^n}{n} + o(x^n)$ \\
        $\sin(x)$ & $x - \frac{x^3}{3!} + \frac{x^5}{5!} - \frac{x^7}{7!} + ... + (-1)^n \frac{x^{2x+1}}{(2n+1)!} + o(x^{2n+2})$ \\
        $\cos(x)$ & $1 - \frac{x^2}{2!} + \frac{x^4}{4!} - \frac{x^6}{6!} + ... + (-1)^n\frac{x^2n}{(2n)!} + o(x^{2n+1})$ \\
        $\tan(x)$ & $x + \frac{x^3}{3} + \frac{2}{15}x^5 + o(x^6)$\\
        $\arctan(x)$ & $x - \frac{x^3}{3} + \frac{x^5}{5} - \frac{x^7}{7} + ... + (-1)^n\frac{x^{2x+1}}{(2n + 1)} + o(x^{2n+2})$\\
        $\arcsin{x}$ & $x + \frac{x^3}{6} + \frac{3}{40}x^5 + o(x^6)$\\
        $\sqrt{1+x}$ & $1 + \frac{1}{2}x - \frac{1}{8}x^2 + \frac{1}{16}x^3 + o(x^3)$\\
        $(1+x)^{\alpha}$ & $1 + \alpha x + \frac{\alpha(\alpha - 1)}{2}x^2 + \frac{\alpha(\alpha - 1)(\alpha - 2)}{6}x^3 + o(x^3)$\\
        \hline
    \end{tabular}
    \caption{Formule di taylor}
\end{table}


\subsection{Utilizzo di Taylor nei limiti}
\begin{example}
Calcolare $\lim\limits_{x\to 0}\frac{\sin{x} - x}{e^x - \log(1 + x) - 1}$. Si può utilizzare gli o-piccoli:\\
$\sin{x} = x + o(x^2)$ \hspace{.5cm} $e^x = 1 + x + o(x)$ \hspace{.5cm} $\log(1 + x) = x + o(x)$\\\\
$\frac{\sin{x} - x}{e^x - \log(1 + x) - 1} = \frac{x + o(x^2) - x}{1 + x + o(x) - (x + o(x)) - 1} = \frac{o(x^2)}{o(x)}$ ma anche questo è indeterminato.\\\\
Dobbiamo quindi andare un po' avanti negli sviluppi del numeratore e del denominatore.\\\\
$\sin{x} = x - \frac{x^3}{6} + o(x^4)$ \hspace{.5cm} $e^x = 1 + x + \frac{x^2}{2} + o(x^2)$ \hspace{.5cm} $\log(1 + x) = x -\frac{x^2}{2} o(x^2)$\\\\
$\frac{\sin{x} - x}{e^x - \log(1 + x) - 1} = \frac{x - \frac{x^3}{6} + o(x^4) - x}{1 + x + \frac{x^2}{2} + o(x^2) - (x - \frac{x^2}{2} + (x^2)) - 1} = \frac{-\frac{x^3}{6} + o(x^4)}{\frac{x^2}{2} + \frac{x^2}{2} + o(x^2)} = \frac{-\frac{x^3}{6} + o(x^4)}{x^2 + o(x^2)} = \frac{-\frac{x}{6} + o(x^4)}{1 + o(x^2)} = \frac{0}{1} = 0$
\end{example}

\begin{example}
$\lim\limits_{x\to 0}\frac{(\sin{x})^2 - \sin{x^2}}{x^4}$\\
$\sin{t} = t + o(t^2)$ \hspace{.5cm} $t= x^2$\\
$\sin{x}^2 = (x + o(x^2))^2 = x^2 + 2x \cdot o(x^2) + (o(x^2))^2 = x^2 + o(x^3) + o(x^4) = x^2 + o(x^3)$ \hspace{.3cm}$\sin{x^2} = x^2 + o(x^4)$\\\\
$\frac{(\sin{x})^2 - \sin{x^2}}{x^4} = \frac{x^2 + o(x^3) - x^2 + o(x^4)}{x^4} = \frac{o(x^2)}{x^4} = \frac{o(x^2)}{x^3} \cdot \frac{1}{x} = 0 \cdot \infty$ \\
Questa è una forma indeterminata perché $\frac{o(x^2)}{x^3} \to 0$ e $\frac{1}{x} \to \infty$. Quindi aumentiamo il grado dell'approssimazione andando a migliorare $(\sin{x})^2$. $\sin{x} = x - \frac{x^3}{6} + o(x^4)$\\
$(\sin{x})^2 = (x - \frac{x^3}{6} + o(x^4))^2 = x^2 + \frac{x^6}{36} + (o(x^4))^2 - 2x \cdot \frac{x^2}{6} + 2x \cdot o(x^4) - 2 \cdot \frac{x^3}{6} \cdot o(x^4) = x^2 \frac{x^6}{36} + o(x^8) -  \frac{x^4}{3} + o(x^5) + o(x^7) = x^2 - \frac{x^4}{3} + o(x^5)$\\
$\frac{(\sin{x})^2 - \sin{x^2}}{x^4} = \frac{x^2 - \frac{x^4}{3} + o(x^5) - x^2 + o(x^4)}{x^4} = \frac{x^2 - \frac{x^4}{3} - x^2 + o(x^4)}{x^4} = \frac{-\frac{x^4}{3} + o(x^4)}{x^4} = \frac{-\frac{1}{3} + o(1)}{1} \to -\frac{1}{3} + 0 = -\frac{1}{3}$. (Divido sopra e sotto per $x^4$)
\end{example}


\newpage
\section{Convessità}
\subsection{Funzione convessa}
\begin{definition}[Convessa]
Dato un $I \subset \mathbb{R}$ intervallo\footnote{Si parla sempre di intervalli quando si parla di convessità perché la convessità non ha senso sennò} ed una $f: I \to \mathbb{R}$. $f$ si dice \textbf{convessa} in I se, presi due punti qualsiasi sul grafico di $f$ il segmento che li unisce è sopra il grafico di $f$.
\end{definition}
\begin{wrapfigure}[6]{r}{6cm}
    \vspace{-10pt}
    \centering
    \includegraphics[width=3.8cm]{images/convessa.png}
    \caption{Funzione convessa}
\end{wrapfigure}
\hspace{-15pt}In formule si esprime dicendo che: $f$ si dice convessa in $I$ se $\forall x_1, x_2 \in I$ con $x_1 < x_2$ e $\forall t \in (0,1)$ risulta che:
\begin{center}
    $f(x_1 + t(x_2 - x_1)) \leq f(x_1) + t(f(x_2) - f(x_1))$
\end{center}
Se la stessa disuguaglianza vale con il $<$ (minore stretto) allora $f$ si dice strettamente convessa.

\subsection{Funzione concava}
\begin{definition}[Concava]
$f$ si dice concava se $-f$ è convessa. Strettamente concava se $-f$ è strettamente convessa.
\end{definition}
\begin{wrapfigure}[6]{l}{6cm}
    \vspace{-10pt}
    \centering
    \includegraphics[width=3.8cm]{images/concava.png}
    \caption{Funzione concava}
\end{wrapfigure}
Se andiamo a scrivere in formule una funzione concava è uguale a:
\begin{center}
    $f(x_1 + t(x_2 - x_1)) \geq f(x_1) + t(f(x_2) - f(x_1))$
\end{center}
\begin{note}
Nota che, come per la concavità, se andiamo scrivere $>$ (maggiore stretto) allora $f$ si dice strettamente concava.
\end{note}

\vspace{15pt}
\subsection{Calcolo della convessità}
\begin{proposition}
Dato $I \subset \mathbb{R}$ intervallo, $f: I \to \mathbb{R}$ derivabile 2 volte. Sono equivalenti:
\begin{enumerate}
    \item $f$ è convessa (strettamente convessa).
    \item $f'$ è debolmente crescente (strettamente crescente).
    \item $f'' \geq 0$ ($f'' > 0$).
\end{enumerate}
\end{proposition}

\begin{note}
La proposizione è uguale per la concavità ma con il segno scambiato.
\end{note}

\begin{example}
$f(x) = x^2$ da $f:\mathbb{R} \to \mathbb{R}$.\\
$f'8x) = 2x$, $f''(x) = 2 > 0 \: \forall x \in \mathbb{R} \Longrightarrow f$ è convessa (anche strettamente) in tutto $\mathbb{R}$.
\end{example}

\begin{example}
$f(x) = e^x$ e $f'(x) = e^x$, $f''(x) = e^x > 0$ sempre $\Longrightarrow f:\mathbb{R} \to \mathbb{R}$ è strettamente convessa.
\end{example}

\begin{example}
$f(x) = \log(x)$ con $f:(0,+\infty) \to \mathbb{R}$.\\
$f'(x) = \frac{1}{x}$, $f''(x) = \frac{1}{x^2} < 0 \: \forall x > 0 \Longrightarrow f$ è strettamente concava. 
\end{example}

\subsection{Interpretazione geometrica}
\begin{wrapfigure}[7]{l}{6cm}
    \vspace{-13pt}
    \centering
    \includegraphics[width=5.5cm]{images/interpretazione-geometrica-convessita.png}
\end{wrapfigure}
Dire che $f'$ è crescente vuol dire che diciamo che il coefficiente angolare sulla tangente cresce, e questo vuol dire che se noi pensiamo alla retta tangente come un punto che tocca il grafico e mano a mano si sposta sul grafico e così facendo va a cambiare inclinazione ruotando, quindi possiamo dire che "la tangente ruota in senso antiorario".\\
\newpage
\begin{example}
Esempio di funzione concava e convessa solo in sotto intervalli del dominio.
\end{example}
\begin{wrapfigure}[4]{r}{8cm}
    \vspace{-10pt}
    \centering
    \includegraphics[width=7cm]{images/seno-concavo-convesso.png}
\end{wrapfigure}

$f(x) = \sin{x}$, $f: [0, 2\pi]$. $f'(x) = \cos{x}$ e $f''(x) = -\sin{x}$.\\
$-\sin{x} \geq 0. \Longleftrightarrow \sin{x} \leq 0 \Longleftrightarrow x \in [\pi, 2\pi]$.\\
$f''(x) \geq 0 \Longleftrightarrow x \in [\pi, 2\pi]$ \hspace{.5cm} $f''(x) \leq 0 \Longleftrightarrow x \in [0, \pi]$\\\\

\begin{proposition}
Prendiamo un $I \subset \mathbb{R}$ intervallo, una $f: I \to \mathbb{R}$ derivabile. Allora $f$ è convessa in $I$ se e solo se $\forall \: x_0 \in I$ il grafico di $f$ è sopra la retta tangente nel punto $(x_0, f(x_0))$ cioè, $\forall \: x_0, x\in I$:
\[f(x) \geq f(x_0) + f'(x_0)(x-x_0)\]
Concava se vale il $\leq$. Stret. convessa se vale $>$ con $x \neq x_0$ e stret. concava se vale $<$ con $x\neq x_0$.
\end{proposition}

\begin{note}
Il grafico  di $f(x_0) + f'(x_0)(x-x_0)$ è la retta tangente.
\end{note}

\begin{example}
$f(x) = e^{-|x|}$, questa è una funzione pari e $f(x) = e^{-x}$ se $x \geq 0$.\\
Questa funzione non è ne concava ne convessa in tutto $\mathbb{R}$, perché ci sono dei tratti dove $f$ sta sotto altri dove sta sopra.\\\\
$f(x) = e^{-|x|} = \begin{cases}e^{-x} & \text{ se } x\geq 0\\e^{x} & \text{ se } x < 0\end{cases}$ \\\\
Quindi se $x>0$ $f'(x) = -e^{-x}$ e $f''(x) = e^{-x} > 0 \Longrightarrow f$ è convessa sull'insieme $\{x>0\}.$\\
Mentre se $x<0$ $f'(x) = e^{-x}$ e $f''(x) = e^{x} > 0 \Longrightarrow f$ è convessa sull'insieme $\{x\leq0\}.$
\end{example}
\hspace{-15pt}Da questo esempio vediamo che se prendiamo $f$ in due intervalli separati, in entrambi questi intervalli è convessa ma nell'unione dei due intervalli $f$ smette di essere convessa. Il motivo è che abbiamo un punto in $x=0$ di non derivabilità.

\begin{example}
Se invece prendiamo $f(x) = e^{|x|}$ quindi $f(x) = \begin{cases}e^x & \text{ se } x\geq 0 \\e^{-x} & \text{ se } x< 0 \end{cases}$.\\\\
In questo caso $f$ è convessa in $(-\infty, 0]$ ed è convessa anche in $[0, +\infty)$ e in questo caso $f$ è convessa anche in tutto $\mathbb{R}$.
\end{example}

\hspace{-15pt}Possiamo notare che nel secondo esempio se calcoliamo $f'_-(0) = -1$ e $f'_+(0) = 1$ mentre se vediamo l'esempio prima $f'_-(0) = 1$ e $f'_+(0) = -1$.

\begin{proposition}
Prendiamo un $I \subset \mathbb{R}$ intervallo, $x_0$ punto interno di $I$, $f:\mathbb{R} \to \mathbb{R}$ derivabile in $I \setminus \{x_0\}$. Siano $I_1 = \{x \in I \: |\: x<x_0\}$ e $I_2 = \{x\in I \: |\: x > x_0\}$ abbiamo che se $f$ è convessa in $I_1$ e $I_2$ e $x_0$ è un punto angoloso per $f$ allora $f$ è convessa in $I$ se e solo se $f'_-(x_0) \leq f'_+(x_0)$.
\end{proposition}

\hspace{-15pt}Questa cosa perché, se noi prendiamo una funzione che presenta un angolo e tracciamo la tangente, data dalla derivata, a sinistra notiamo che mano a mano che ci spostiamo verso destra questa tangente "ruoterà" sul grafico, nel punto $x_0$ avremo due tangenti una dalla derivata destra ed una dalla sinistra, possiamo notare che se la funzione rimane concava o convessa questa tangente continuerà a "ruotare" nello stesso verso senza fare "uno scatto" nel suo andamento, in caso contrario allora non manterrà la concavità o la convessità.

\subsection{Flessi}
\begin{definition}[Flesso]
Dato un $I\subset \mathbb{R}$ intervallo, $f: I\to \mathbb{R}$, $x_0$ punto interno ad $I$ si dice punto di flesso se $f$ è derivabile in $x_0$ ed esiste un intorno $U \subset I$ di $x_0$ t.c. la quantità
\[\frac{f(x) - (f(x_0) + f'(x_0)(x-x_0))}{x-x_0} \text{ non cambia segno in }U \setminus \{x_0\}\]
\end{definition}

\hspace{-15pt}Dire che $\frac{f(x) - (f(x_0) + f'(x_0)(x-x_0))}{x-x_0}$ non cambia segno vuol dire che il grafico della funzione passa da sopra a sotto la tangente (o viceversa).

\begin{definition}[Flesso a tangente verticale]
Se invece $f'(x) = \pm\infty$ ($f$ non è derivabile), $f$ è continua in $x_0$, e se $f$ è convessa in un intorno destro di $x_0$ e concava in un intorno sinistro di $x_0$ (o viceversa) allora $x_0$ si dice punto di flesso a tangente verticale. 
\end{definition}

\hspace{-15pt}Un flesso verticale è un cambiamento di convessità con un flesso verticale.

\begin{observation}
Se avete una funzione $f: I \to \mathbb{R}$, $I$ intervallo ed $f$ derivabile due volte in $I$. Allora se $f''(x_0)=0$ e $f$ cambia segno in $x_0$ allora $x_0$ è punto di flesso.
\end{observation}

\hspace{-15pt}Cambia segno vuol dire che $f''(x) \leq 0$ se $x\leq x_0$ e $f''(x) \geq 0$ se $x\geq x_0$ (o viceversa), con $x \in U$ intorno di $x_0$.

\begin{example}
Calcoliamo il flesso di $f(x) = x^3$, $f'(x) = 3x^2$, $f''(x) =6x$.\\
Vediamo dall'immagine che esiste un flesso in $x=0$, infatti:\\
$f''(x) = 0$, $f''(x) \leq 0$ se $x \leq 0$ e $f''(x) \geq 0$ se $x \geq 0$.
\end{example}

\begin{observation}
$f''(x_0) = 0$ non è sufficiente per aver un flesso
\end{observation}

\begin{example}
Prendiamo per verificare l'osservazione $f(x) = x^4$, $f'(x) = 4x^3$, $f''(x) = 12x^2$.\\
Anche se $f(0)=0$ abbiamo che $f''(x) \geq 0 \forall x \in \mathbb{R} \Longrightarrow $ f è convessa in $\mathbb{R}$.
\end{example}

\begin{observation}
Ci possono essere punti di flesso dove non esiste la derivata seconda.
\end{observation}

\begin{example}
$f(x) = x \cdot |x|$\hspace{.3cm} $f(x) = \begin{cases}x^2 & \text{ se } x \geq 0\\ -x^2 & \text{ se } x<0\end{cases}$\hspace{.3cm} $f'(x) = \begin{cases}2x & \text{ se } x>0 \\ -2x & \text{ se } x<0\end{cases}$\\\\
Possiamo vedere che $x_0 = 0$ è punto di flesso, infatti $f$ è derivabile in $x_0 = 0$ infatti $f'(0) = \lim\limits_{x\to 0}\frac{f(x) - f(0)}{x-0} = \lim\limits_{x\to 0}|x| = 0$.
La retta tangente in $x=0$ è $y=0$. $f$ passa da sopra la tangente in $x_0 = 0$, quindi $x_0$ è un punto di flesso. Però non esiste la derivata seconda in $x_0 = 0$ perché in questo punto c'è un punto angoloso.
\end{example}

\begin{observation}
Se abbiamo una funzione $f: I \to \mathbb{R}$, con $I \subset \mathbb{R}$, $f$ convessa nei punti interni di $I$, ed $f$ continua in tutto $I \Longrightarrow f$ è convessa in $I$.\\
Quindi se abbiamo $f:[a,b] \to \mathbb{R}$ convessa in $(a,b)$ ed $f$ continua in $[a,b] \Longrightarrow f$ è convessa in $[a,b]$.
\end{observation}

\newpage
\section{Studio di funzione}
\subsection{Punti da seguire}
Data una funzione $f(x)$ bisogna andare ad eseguire una serie di passi. $f(x)$ viene di solito assegnata senza specificare il dominio.
\begin{enumerate}
    \item Determinare l'insieme di definizione di $f$.
    \item Determinare l'insieme di continuità di $f$.
    \item Determinare l'insieme di derivabilità di $f$.
    \item Vedere eventuali asintoti orizzontali, verticali o obliqui.
    \item Studiare la monotonia della funzione.
    \item Trovare punti di massimo o di minimo locali.
    \item Determinare massimo e minimo d $f$ oppure estremo sup. ed inf.
    \item Studiare la convessità di $f$ (con eventuali punti di flesso).
\end{enumerate}

\subsection{Esempio studio di funzione}
\begin{example}
Studiamo la funzione $f(x) = \log|x| - \frac{x^2 - 1}{4x}$.
\begin{enumerate}
    \item $|x| > 0 \Longleftrightarrow x\neq 0$ e $4x \neq 0 \longleftarrow x \neq 0$. \textbf{Insieme di definizione} è $\mathbb{R} \setminus \{0\}$.
    \item La $f$ è \textbf{continua} in tutto $\mathbb{R} \setminus \{0\}$ (composizione funzioni continue e prodotto e sottrazioni funzioni continue).
    \item $f$ \textbf{derivabile} in tutto $\mathbb{R} \setminus \{0\}$ (sempre perché tutte queste funzioni sono derivabili in tutto il loro insieme di definizione, il valore assoluto non è derivabile in 0 ma non lo si considera).
    \item Per vedere gli \textbf{asintoti} dobbiamo fare i limiti ai bordo e sui punti non interi al dominio:\\
    $\lim\limits_{x\to -\infty}f(x) = \log|x| - \frac{x^2 -1}{4x} = \log|x| - \frac{x}{4} + \frac{1}{4x}$ \hspace{.5cm} $\lim\limits_{x\to -\infty}\log|-\infty| - \frac{-\infty}{4} + \frac{1}{4(-\infty)} = + \infty$.\\
    $\lim\limits_{x\to 0^-}\log|0^-| - \frac{0^-}{4} + \frac{1}{4(0^-)} = -\infty - 0 - \infty$. \hspace{.5cm} 
    $\lim\limits_{x\to 0^+}\log|0^+| - \frac{0^+}{4} + \frac{1}{4(0^+)} = -\infty - 0 + \infty$\\
    $\lim\limits_{x\to 0^+}f(x) = \lim\limits_{x\to 0^+}(-\frac{x}{4}) + \lim\limits_{x\to 0^+}\log|x| + \frac{1}{4x} = 0 + \lim\limits_{x\to 0^+}\frac{4x\log|x| + 1}{4x} = 0 + \frac{0+1}{4 \cdot 0^+} = +\infty$\\
    $\lim\limits_{x\to +\infty} \log|+\infty| - \frac{\infty}{4} + \frac{1}{4 \cdot \infty} = \infty - \infty + 0$\\
    $\lim\limits_{x\to +\infty} x(\frac{\log|x|}{4} - \frac{1}{4}) + \lim\limits_{x\to +\infty} \frac{1}{4x} = \infty(0 - \frac{1}{4}) + 0 = -\infty.$
    Abbiamo quindi un asintoto verticale di equazione $x=0$ e non ci sono asintoti orizzontali. Vediamo se ci sono asintoti obliqui:
    $\lim\limits_{x\to +\infty}\frac{f(x)}{x} = \lim\limits_{x\to +\infty}(\log|x| -\frac{x}{4} + \frac{1}{4x}) \cdot \frac{1}{x} = 0 - \frac{1}{4} + 0 = -\frac{1}{4}$, quindi $m= -\frac{1}{4}$\\
    $\lim\limits_{x\to +\infty}f(x) -mx = \lim\limits_{x\to +\infty} \log|x| - \frac{x}{4} + \frac{1}{4x} + \frac{1}{4}\cdot x = \infty + 0 = \infty$. \\
    Non c'è asintoto obliquo per $x\to +\infty$ e neanche a $x\to -\infty$ perché i conti sono uguali.
    \item Studiamo ora la \textbf{monotonia} di $f$.
    
    $\log|x| = \begin{cases}\log(x) & \text{ se } x>0 \\ \log(-x) & \text{ se } x<0\end{cases}$\hfill
    $D(\log|x|) = \begin{cases}D(\log(x)) = \frac{1}{x} & \text{ se } x>0 \\D(\log(-x)) = \frac{1}{-x} \cdot (-1) = \frac{1}{x} & \text{ se } x<0\end{cases}$
    
    Quindi possiamo notare che $D(\log |x|) = \frac{1}{x}$.\\
    $f(x) = \log|x| - \frac{x}{4} + \frac{1}{4x}$ \hspace{.3cm} $f'(x) = \frac{1}{x} - \frac{1}{4} -\frac{1}{4x^2} = \frac{-x^2 + 4x -1}{4x^2}$ conferma che è derivabile ovunque tranne che in $x = 0$.\\\\
    Il denominatore è $> 0$ in tutto il dominio, allora il segno di $f'$ è lo sesso del numeratore. Per trovare il segno bisogna trovare dove si annulla il numeratore.\\
    $-x^2 + 4x - 1 = 0 \Longleftrightarrow x^2 - 4x + 1 = 0$ \hspace{.3cm} $x = 2 \pm \sqrt{4-1} = 2 \pm \sqrt{3}$.\\
    $f$ è decrescente in $(-\infty,0)$, decrescente in $(0,2-\sqrt{3}]$, crescente in $[2-\sqrt{3}, 2+\sqrt{3}]$, decrescente in $[2+\sqrt{3},+\infty)$. Questa separazione va fatta perché il teorema di Lagrange prevede intervalli e lo 0 interrompeva l'intervallo.
    \item Vedendo la monotonia possiamo anche dire i punti di \textbf{massimo e minimo locali}. \\
    $x = 2 - \sqrt{3}$ è punti di minimo locale. \hspace{.3cm} $x = 2 + \sqrt{3}$ è punti di massimo locale.\\
    Per calcolare esattamente questi punti dove si collocano nel grafico basta sostituirli in $f(x)$.
    \item Dal fatto che $\lim\limits_{x\to +\infty} = +\infty$ otteniamo che $sup(f) = +\infty \Longrightarrow f$ non ha \textbf{massimo}. Dal fatto che $\lim\limits_{x\to 0^-}=-\infty$ otteniamo che $inf(f) = -\infty \Longrightarrow f$ non ah \textbf{minimo}.
    \item Come ultima calcoliamo al derivata seconda e troviamo la \textbf{convessità}.\\
    $f'(x) = \frac{1}{x} - \frac{1}{4} - \frac{1}{4x^2}$ \hspace{.5cm} $f''(x) = -\frac{1}{x^2} +\frac{1}{2x^3} = \frac{-2x + 1}{2x^2}$\\
    Segno del numeratore $-2x + 1 > 0 \Longleftrightarrow 1 > 2x \Longleftrightarrow x < \frac{1}{2}$.\\
    Segno del denominatore $2x^3 > 0 \Longleftrightarrow x > 0$.\\
    $f$ è concava in $(-\infty, 0)$ \hspace{.3cm} convessa in $(0,\frac{1}{2}]$ \hspace{.3cm} concava in $[\frac{1}{2},
    +\infty)$.\\
    Il punti di ascissa $x=\frac{1}{2}$ è punto di flesso visto che c'è un cambio di convessità.
\end{enumerate}
\end{example}
% !TeX spellcheck = it_IT
\newpage
\section{Integrali}
\begin{wrapfigure}[5]{r}{5.5cm}
    \vspace{-25pt}
    \centering
    \includegraphics[width=4cm]{images/area-sottografico.png}
\end{wrapfigure}
In questo corso tratteremo gli integrali detti \textbf{di Riemain}.
Sia $f: [a,b] \to \mathbb{R}$, limitata. (ad esempio una funzione continua). L'idea della definizione è che l'integrale (definito) di $f(x)$ in $[a,b]$ rappresenta l'area del sotto grafo di $f$  (questo è vero se $f \geq 0$ su $[a,b]$).

\begin{definition}[Suddivisione di un intervallo]
Una \textbf{suddivisione} di [a,b] è un insieme di $A = \{x_0, x_1, ..., x_n\}$ con $a = x_0 < x_1 < x:2 < ... < x_n = b$.
\end{definition}
\begin{observation}
Le lunghezze degli intervalli $[x_{i-1}, x_u]$ non sono necessariamente uguali.\\
Inoltre $\sum\limits_{i=1}^n(x_i - x_{i-1}) = b - a = $ lunghezza di $[a,b]$.
\end{observation}

\begin{definition}[Somma inferiore]
Dato una suddivisione di un intervallo A, si dice somma inferiore di $f$ relativa alla suddivisione di A 
\vspace{-5pt}
\[S'(f,A) = \sum^n\limits_{i=1} \big( \inf(f(x))_{x \in [x_{i-1}, x_i]} \big) \cdot (x_i - x_{i - 1})\]
\end{definition}
E la somma delle aree dei rettangoli rossi. Approssima l'area del sotto grafico di $f(x)$ per difetto.
\begin{definition}[Somma superiore]
Dato una suddivisione di un intervallo A, si dice somma superiore di $f$ relativa alla suddivisione di A 
\vspace{-5pt}
\[S'(f,A) = \sum^n\limits_{i=1} \big( \sup(f(x))_{x \in [x_{i-1}, x_i]} \big) \cdot (x_{i-1} - x_i)\]
\end{definition}
Somma delle aree dei rettangoli rossi. Questa volta è un'approssimazione per eccesso dell'area del sotto grafico.
\begin{observation}
Non server che $f$ sia continua per dare tutte queste definizione, ma soltanto che sia limitata.
\end{observation}
\begin{definition}[Somme indipendente dalle suddivisioni]
Le somme inferiori e superiori indipendenti dalle suddivisioni si definiscono come:
\begin{itemize}
    \item $S'(f) = sup\{S'(f,A) \:\: |$ A suddivisione di $[a,b]\}$ si dice somma inferiore di $f$.
    \item $S''(f) = inf\{S'(f,A) \:\: |$ A suddivisione di $[a,b]\}$ si dice somma superiore di $f$.
\end{itemize}
\end{definition}
Aggiungendo punti le somme inferiori crescono (e le somme superiori calano).

\begin{figure}[h!]
    \begin{subfigure}{.3\textwidth}
        \centering
        \includegraphics[width=4cm]{somma-superiore.png}
        \caption{Somma superiore}
    \end{subfigure}
    \begin{subfigure}{.3\textwidth}
        \centering
        \includegraphics[width=4.5cm]{somma-inferiore.png}
        \caption{Somma inferiore}
    \end{subfigure}
    \begin{subfigure}{.3\textwidth}
        \centering
        \includegraphics[width=4.5cm]{somma-indipendente.png}
        \caption{Somma indipendente}
    \end{subfigure}
    \caption{Somme delle sezioni}
\end{figure}

\begin{definition}[Integrabile secondo Rieman]
Se $S'(f) = S''(f)$ si dice che $f$ è \textbf{integrabile secondo Rieman} su $[a,b]$ e il valore comune si dice integrale di $f$ su $[a,b]$ e si indica come:
\[\int_{a}^b f(x)\:dx \: \: = S'(f) = S''(f)\]
\end{definition}
\newpage
\begin{observation}
Questa definizione ha senso anche quando $f$ può prendere anche valori negativi.
\end{observation}
\begin{wrapfigure}[4]{r}{6cm}
    \vspace{-15pt}
    \centering
    \includegraphics[width=5cm]{images/area-positiva-negativa.png}
\end{wrapfigure}
Se $f \leq 0 \Longrightarrow \int_a^b f(x)\:dx \leq 0$ ed è l'opposto dell'area in figura.
In generale $\int_a^b f(x)\:dx$ è la somma algebrica delle aree in figura (si sommano le aree dove l'integrale è positivo e si sottraggono quelle dove è negativo). \\

\begin{theorem}
Se $f: [a,b] \to \mathbb{R}$ è continua, allora è integrabile.
\end{theorem}

\begin{observation}
Ci sono anche funzioni non continue che sono integrabili, ad esempio una funzione con un punto in cui c'è un salto.
\end{observation}

\begin{definition}
Una $f: [a,b] \to \mathbb{R}$ è generalmente continua se è limitata e ha eventualmente un numero finito di punti di discontinuità.
\end{definition}

\begin{example}
Funzione non generalmente continua. $f(x) = \begin{cases}\frac{1}{x} & x\neq 0 \\ 0 & x=0\end{cases}$ con $f: [-1,1] \to \mathbb{R}$
\\C'è un solo punto di discontinuità, ma $f$ non è limitata $\Longrightarrow$ non è generalmente continua.
\end{example}

\begin{theorem}
Se $f:[a,b] \to \mathbb{R}$ è generalmente continua, allora f è integrabile.
\end{theorem}

\begin{example}
$f(x) = \begin{cases}\sin\frac{1}{x} & x\neq 0 \\ 0 & x=0\end{cases}$ con $f: [0,1] \to \mathbb{R}$.\\
$f(x)$ non è continua ma è generalmente continua $\Longrightarrow$ integrabile
\end{example}

\begin{example}
Esempio di una funzione non integrabile. (Esempio con la funzione di Dirichlet).
\end{example}
\begin{wrapfigure}[7]{r}{5cm}
    \vspace{-25pt}
    \centering
    \includegraphics[width=5cm]{images/funzione-dirichlet.png}
\end{wrapfigure}
$f(x) = \begin{cases}1 & x\in\mathbb{Q} \\ 0 & x \notin \mathbb{Q} \end{cases}$ con $f: [0,1] \to \mathbb{R}$.\\
Per qualsiasi intervallo $[x_{i-1}, x_i] \subseteq [0,1]$ si ha che:\\\\
$sup(f(x))_{x \in [x_{i-1}, x_i]} = 1$ e $inf(f(x))_{x \in [x_{i-1}, x_i]} = 0$. \\
Segue che $S'(f,A) = 0 \:\: \forall A$ suddivisione di $[0,1] \Longrightarrow S'(f) = 0$ e $S''(f,A) = 1 \:\: \forall A $ suddivisione di $[0,1] \Longrightarrow S''(f) = 1$. Quindi $S'(f) \neq S''(f) \Longrightarrow f$ non è integrabile.\\\\

\begin{wrapfigure}[3]{l}{5cm}
    \vspace{-30pt}
    \centering
    \includegraphics[width=4cm]{images/differenze-aree-integrale.png}
\end{wrapfigure}

Se $f$ è integrabile, $S''(f,A) - S'(f,A)$ (la differenza, l'area della regione verde nell'immagine) "tende a 0" al raffinarsi delle suddivisioni.
\vspace{15pt}
\subsection{Calcolo degli integrali}
\begin{theorem}
Siano $f,g$ integrabili su $[a,b]$ e un numero $k \in \mathbb{R}$, allora: $f+g, k\cdot, |f|$ sono integrabili, e si ha che:
\begin{enumerate}
    \item $\int_a^b (f+g)\:dx = \int_a^b f(x)\:dx + \int_a^b g(x)\:dx$.
    \item $\int_a^b (k\cdot f)\:dx = k \cdot \int_a^b f(x) \:dx$.
    \item Se $f(x) \leq g(x) \forall x \in [a,b]$ allora $\int_a^b f(x) \:dx \leq \int_a^b g(x)\:dx$.
    \item $\big|\int_a^b f(x) \:dx \big| \leq \int_a^b |f(x)|\:dx$.
    \item Se $a < c < b$ allora $\int_a^b f(x) \:dx = \int_a^c f(x) \:dx + \int_c^b f(x)\:dx$.
\end{enumerate}
\end{theorem}

\newpage
\begin{figure}[h!]
    \centering
    \begin{subfigure}{.3\textwidth}
        \centering
        \includegraphics[width=4.5cm]{images/teorema-calcolo-integrali-1.png}
        \caption{Caso 1°}
    \end{subfigure}
    \begin{subfigure}{.3\textwidth}
        \centering
        \includegraphics[width=4.5cm]{images/teorema-calcolo-integrali-2.png} 
        \caption{Caso 2°}
    \end{subfigure}
    \begin{subfigure}{.3\textwidth}
        \centering
        \includegraphics[width=4.5cm]{images/teorema-calcolo-integrali-3.png}
        \caption{Caso 3°}
    \end{subfigure}
\end{figure}

\begin{observation}
Osserviamo anche che se $f: [a,b] \to \mathbb{R}$ è constante, cioè $f(x) = k \:\: \forall x \in [a,b]$ allora $\int_a^b f(x) \:dx = k \cdot (b-a)$
\end{observation}

\subsection{Media Integrabile}
\begin{definition}[Media Integrabile]
Se $f: [a,b] \to \mathbb{R}$ integrabile, si dice \textbf{media integrabile} di $f$ su $[a,b]$.
\vspace{-10pt}
\[m = \frac{1}{b-a} \cdot \int_a^b f(x) \:dx\]\\
\end{definition}
\begin{wrapfigure}[2]{l}{5cm}
\vspace{-45pt}
    \centering
    \includegraphics[width=4cm]{images/media-integrabile.png}
\end{wrapfigure}
\vspace{-10pt}
Graficamente, m è l'altezza di un rettangolo di base $b-a$, con la stessa area del sotto grafico di $f$.
\vspace{15pt}
\begin{theorem}[Teorema della media integrale]
Sia $f:[a,b] \to \mathbb{R}$ integrabile, allora:
\vspace{-5pt}
\[inf(f(x))_{[a,b]} \leq \frac{1}{b-a} \cdot \int_a^b f(x) \:dx \leq sup(f(x))_{[a,b]}\]
Se $f$ è continua, allora $\exists x \in [a,b]$ tale che:
$f(z) = \frac{1}{b-a} \cdot \int_a^b f(x) \:dx$
\end{theorem}

\begin{demostration}
$\forall x \in [a,b]$ abbiamo $inf(f(x))_{[a,b]} \leq f(x) \leq sup(f(x))_{[a,b]}$. Integriamo questa disuguaglianza usando la proprietà (3) del teorema, e otteniamo:\\
$\int_a^b inf(f(x))_{[a,b]}\:dx \leq \int_a^b f(x)\:dx \leq \int_a^b sup(f(x))_{[a,b]}\:dx$. Sia $\int_a^b inf(f(x))_{[a,b]}\:dx$ che $\int_a^b sup(f(x))_{[a,b]}\:dx$ sono costanti $\Longrightarrow \big( inf(f(x))_{[a,b]} \big)(b-a) \leq \int_a^b f(x)\:dx \leq \big( sup(f(x))_{[a,b]} \big)(b-a)$.\\
Dividendo per $(b-a)$ ottengo proprio: $inf(f(x))_{[a,b]} \leq \frac{1}{b-a} \cdot \int_a^b f(x) \:dx \leq sup(f(x))_{[a,b]}$.\\\\
Se $f$ è continua, allora per il teorema di Weirstrass $inf(f) = min(f)$ e $sup(f) = max(f)$. Inoltre per il teorema dei valor intermedi $f$ prende tutti i valori compresi tra il $min(x)$ e $max(f)$. La media integrale è un tale valore per quanto visto, quindi $\exists z \in [a,b]$ tale che $f(z) = \frac{1}{b-a} \cdot \int_a^b f(x)\:dx$. $\blacksquare$
\end{demostration}

\begin{observation}
Se $b<a$, definiamo $\int_a^b f(x)\:dx = - \int_b^a f(x)\:dx$, e definiamo anche $\int_a^a f(x) = 0$.
\end{observation}

\begin{example}
$\int_2^1 x^3\:dx = -\int_1^2 x^3\:dx$
\end{example}

\hspace{-15pt}Le proprietà viste precedentemente valgono anche con i valori scambiati come nell'esempio sopra.
\begin{observation}
La media integrale ha senso anche quando gli estremi sono scambiati. Se $b < a$, allora $\frac{1}{b-a}\int_a^b f(x)\:dx = (\frac{1}{b-a})\big(-\int_a^b f(x)\:dx \big) = \frac{1}{a-b}\int_a^bf(x)\:dx$
\end{observation}

\begin{definition}[Primitiva]
Prendiamo un $I\subseteq \mathbb{R}$ intervallo, $f: I \to \mathbb{R}$, una funzione $F: I \to \mathbb{R}$ si dice primitiva di $f$ se $F$ è derivabile in $I$ e vale che $F'(x) = f(x) \:\: \forall x \in I$.
\end{definition}

\begin{example}
$f(x) = 2x$. Una primitiva è $F(x) = x^2$. Non è l'unica primitiva, $G(x) = x^2 + k$, $k\in \mathbb{R}$ ho comunque $G'(x) = 2x + 0 = f(x)$ quindi queste funzioni sono tutte primitive di $f(x) = 2x$. 
\end{example}
\hspace{-15pt}In generale, se $F$ è primitiva di $f$, tutte le funzioni $G(x) = F(x) + k$ con $k \in \mathbb{R}$ sono pure primitive di $f(x)$.

\begin{observation}
In effetti due primitive di $f(x)$ differiscono sempre per una costante.
\end{observation}

\begin{demostration}
Siano F e G due primitive di $f$. Allora ho che $F' = f$, $G' = f$. Quindi $(F - G)' = F' - G' = f - f = 0$. Visto che siamo su un intervallo, concludo che $F - G$ è costante $K \in \mathbb{R} \Longrightarrow F(x) = G(x) + k \:\: \forall x \in I$.
\end{demostration}

\begin{definition}[Integrale indefinito]
\textbf{L'integrale indefinito} di $f(x)$ è l'insieme di tutte le primitive di $f(x)$ e si indica con $\int f(x)\:dx$ (senza gli estremi).
\end{definition}

\begin{observation}
$\int f(x)\:dx$ non indica una singola funzione, ma un insieme di funzioni.
\[\int f(x)\:dx = \{F: I \to \mathbb{R} \:\: | \:\: F \: derivabile \: e \: F'=f\}\]
\end{observation}

\begin{example}
Se prendiamo per esempio $\int 2x\:dx = \{x^2 + k \:\:|\:\: k \in \mathbb{R}\}$ di solito si abbrevia scrivendo $\int 2x\:dx = x^2 + k$. 
\end{example}

\hspace{-15pt}L'integrale di Riemainn $\int_a^b f(x)\:dx$ invece è un numero reale, e rappresenta l'area del sotto grafico di $f$, e si dice \textbf{integrale definito} e $a,b$ sono gli \textbf{estremi di integrazione} ("a" è inferiore e "b" superiore). 

\subsection{Formule per integrali indefiniti}
Dalle formule per le derivate seguono formule per le primitive di una funzione f(x). Vedere la tabella di seguito.
\begin{table}[h!]
    \centering
    \setlength{\tabcolsep}{6pt}
    \renewcommand{\arraystretch}{1.5}
    \begin{tabular}{|c||c|}
        \hline
        $\int e^x=e^x + k$ & $\int \frac{1}{x}\:dx=\log|x| + k$\\
        
        $\int \cos(x)\:dx=\sin{x} + k$ & $\int \sin{x}\:dx=-\cos(x) + k$ \\
        
        $\int \frac{1}{1+x^2}\:dx=\arctan{x} + k$ & $\int \frac{1}{\sqrt{1 - x^2}}\:dx=\arcsin{x}$\\
        
        $\int \frac{1}{(\sin{x})^2}\:dx=-\cot{x}$ & $\int \frac{1}{(\cos{x})^2}\:dx=\tan{x}$ \\
        
        $\int x^n \:dx=\frac{1}{n+1}x^{n+1} + k$ & $\int -\frac{1}{x^2}\:dx=\frac{1}{x}$\\
        \hline
    \end{tabular}
    \caption{Formule primitive}
\end{table}
\vspace{-10pt}
\subsection{Teorema fondamentale del calcolo integrale}
\begin{theorem}[Teorema fondamentale del calcolo integrale]
Sia $I \subseteq \mathbb{R}$ un intervallo, $a \in I$, $f: I \to \mathbb{R}$ continua. Allora la funzione $F(x) = \inf_a^x f(t) \:dt$ (chiamata anche funzione integrale) è una primitiva di $f$, cioè $F(x)$ è derivabile e $F'(x) = f(x)$.
\end{theorem}

\begin{demostration}
Mostriamo che $F$ è derivabile calcolandone il rapporto incrementale in $x_0 \in I$ arbitrario, e poi facendo il limite.\\\\
$\frac{F(x) - F(x_0)}{x - x_0} = \frac{1}{x - x_0}\big( \int_a^x f(y) \:dt - \int_a^{x_0} f(y) \:dt \big) = \frac{1}{x - x_0} \int_{x_0}^x f(t) \:dt$. In risultato è la media integrale di $f$ sull'intervallo di estremi $x$ e $x_0$.\\\\
Visto che $f$ è continua, per il teorema della media integrale $\exists \: z(x)$ compreso tra $x_0$ e $x$ tale che $f(z(x)) = \frac{1}{x - x_0} \int_{x_0}^x f(t) \:dt$.\\
Quindi $F'(x_0) = \lim\limits_{x\to x_0}\frac{F(x) - F(x_0)}{x - x_0} = \lim\limits_{x\to x_0}f(z(x))$. Cambio variabile e prendo $y = z(x)$. Devo capire a cosa tende $y$ quando $x\to x_0$. So che $z(x)$ è compreso tra $x_0$ e $x$ (ad esempio se $x \leq x_0$, so che $x \leq z(x) \leq x_0$) quindi per il teorema dei carabinieri ho che $\lim\limits_{cx \to x_0}y = x_0$.\\\\
Segue che $\lim\limits_{x \to x_0}f(z(x)) = \lim\limits_{y \to x_0} f(y) = f(x_0)$ (questo per la continuità di f). Questo dimostra che $F'(x_0) = f(x_0)$m quindi $F'(x) = f(x) \:\: \forall x \in I$. $\blacksquare$
\end{demostration}

\newpage
\subsection{Teorema di Torricelli}
\begin{theorem}[Teorema di Torricelli]
$I \subseteq \mathbb{R}$ intervallo, $f: I \to \mathbb{R}$ funzione continua, $a \in I$. Se G è una primitiva di $f$ in I, allora $\exists k \in \mathbb{R}$ tale che $G(x) = \int_a^x f(t) \:dt + k$ e $\forall \alpha, \beta \in I$ abbiamo che $\int_{\alpha}^{\beta}f(t) \:dt = G(\beta) - G(\alpha)$.
\end{theorem}

\hspace{-15pt}A livello di notazioni si va a scrivere: $[G(x)]_{\alpha}^{\beta} = G(\beta) - G(\alpha)$

\begin{example}
Prendiamo $\int_1^3 x \:dx$. Una primitiva di $f(x) = x$ è $G(x) = \frac{x^2}{2}$. \\
Quindi $\int_1^3 x \:dx = [\frac{x^2}{2}]_1^3 = \frac{9}{2} - \frac{1}{2} = \frac{8}{2} = 4$. (Se prendiamo un'altra primitiva ad esempio $F(x) = \frac{x^2}{2} + 1$, trovato $\int_1^3 x \:dx = [\frac{x^2}{2} + 1]_1^3 = \frac{8}{2} = 4$)
\end{example}

\subsection{Integrali con estremi variabili}
\begin{theorem}
Dato un $I \subseteq \mathbb{R}$ intervallo, $f: I \to \mathbb{R}$ continua. Abbiamo poi $A\subseteq \mathbb{R}$, e $\alpha,\beta:A \to I$ derivabili. Sia $G(x) = \int_{\alpha(x)}^{\beta(x)} f(t) \:dt$. Allora $G(x)$ è derivabile e si ha:
\[G'(x) = f(\beta(x)) \cdot \beta'(x) - f(\alpha(x)) \cdot \alpha'(x)\]
In particolare se $\alpha(x) = a$ constante e $\beta(x) = x$, si ha $G(x) = \int_a^x f(t)\:dt$, e la formula scritta sopra è uguale a $f(x) \cdot \ - f(a) \cdot 0 = f(x)$. (Come della conclusione del teorema fondamentale)
\end{theorem}

\begin{example}
$G(x) = \int_{x^2}^{\sin{x}}e^t \cdot \arctan(t) \:dt$ \hfill $f(t) = e^t \arctan(t)$, $\alpha(x) = x^2$, $\beta(x) = \sin(x)$.\\
Abbiamo $G'(x) = f(\beta(x)) \cdot \beta'(x) - f(\alpha(x)) \cdot \alpha'(x) = e^{\sin{x}} \cdot \arctan(\sin(x)) \cdot \cos(x) - e^{x^2} \cdot \arctan(x^2) \cdot 2$.\\
Applicazione: $\lim\limits_{x\to 0}\frac{\int_0^{x^2} e^t \cdot \arctan(x) \:dt}{\sin(x^4)} = \frac{\int_0^0 (...)}{\sin(0)} = \frac{0}{0}$. Usiamo de l'hopital.\\
$\lim\limits_{x\to 0} \frac{e^{x^2}\cdot \arctan(x^2) \cdot 2x}{\cos(x^4) \cdot 4x^3} = \lim\limits_{x\to 0}\frac{e^{x^2}}{\cos(x^4)}\cdot\frac{\arctan(x^2) \cdot x}{2x^3} = \lim\limits_{x\to 0} \frac{e^{x^2}}{\cos(x^4)}\cdot \frac{\arctan(x^2)}{2x^2} = \lim\limits_{x\to 0}\frac{e^{x^2}}{\cos(x^4)}\cdot\frac{x^2 + o(x^2)}{2x^2} = 1 \cdot \frac{1}{2}$
\end{example}

\subsection{Metodi di calcolo per integrali indefiniti}
\subsubsection{Integrazione per parti}
Prendiamo $f,g: I \to \mathbb{R}$ con $I\subseteq \mathbb{R}$ intervallo, $f$ continua e $g$ di classe $C^1$ ($g$ e derivabile e la derivata è continua). Se $F$ è una primitiva di $f$ allora:
\vspace{-5pt}
\[\int f\cdot g\:dx = F \cdot g - \int F \cdot g' \:dx\]

\begin{demostration}
Se faccio la derivata del prodotto $(F \cdot g)' = F'\cdot g + F\cdot g' = fg + F'g$. \\
(Se due funzioni sono uguali anche gli integrali indefiniti delle due funzioni sono uguali)Integrando ambo i membri ottengo che $\int (Fg)'\:dx = \int (fg)\:dx + \int F\cdot g' \:dx = \int F\cdot g\:dx = \int (fg)\:dx + \int F\cdot g' \:dx$. Abbiamo così dimostrato la formula. $\blacksquare$
\end{demostration}

\hspace{-15pt}Esempi ed esercizi guarda i lucidi delle lezioni (gli appunti del professore).

\begin{observation}
Se il ho $\log(f(x))' = \frac{f'(x)}{f(x)}$ (sto supponendo che $f(x) > 0$), quindi segue che $\int \frac{f'(x)}{f(x)} \:dx = \log(f(x)) + k$.
\end{observation}

\subsubsection{Integrazione per sostituzione}
Supponiamo di avere $I,J \subseteq \mathbb{R}$ intervalli, $f: I \to \mathbb{R}$ continua. Prendiamo poi $\phi: J \to I$ di classe $C^1$. Se $F$ è una primitiva di $f$, allora $\int (f \circ \phi) \cdot \phi' \:dx = (F \circ \phi) + k$.

\begin{demostration}
$(F \circ \phi)' = (F'(\phi)) \cdot \phi' = (f \circ \phi) \cdot \phi'$ per la regola di derivazione di funzioni composte, Integrando trovo che: $\int (f \circ \phi) \cdot \phi' \: dx = \int (F \circ \phi)' \:dx = (F \circ \phi) + k$. $\blacksquare$
\end{demostration}

\begin{example}
Prendiamo $\int xe^{x^2}\:dx$. Pongo $t=x^2$ (funzione di x), $\frac{dt}{dx} = dx$ quindi $dt = 2xdx$, $\frac{dt}{2} = xdx$.\\
$\int e^t \cdot \frac{dt}{2} = \frac{1}{2}\int e^t \:dt = \frac{1}{2} e^t +c$ e poi si torna in $x$ sostituendo $t=x^2$ quindi torna $\frac{1}{2}e^{x^2} + k$.\\\\
Per gli integrali definiti possiamo fare in due modi. Prendiamo $\int_0^2 xe^{x^2}\:dx$:
\begin{enumerate}
    \item Calcolare $\int xe^{x^2}\:dx$. Abbiamo che che è $\frac{1}{2}e^{x^2} + k$. Poi $\int_0^2 xe^{x^2}\:dx = [\frac{1}{2}e^{x^2} + k]_0^2 = \frac{1}{2}(e^4-1)$.
    \item Possiamo usare la sostituzione, ricordandosi di cambiare gli estremi: $\int_0^2 xe^{x^2}\:dx$ pongo come prima  $dt = 2xdx$, $\frac{dt}{2} = xdx$.\\
    Quindi $\int_0^2 xe^{x^2}\:dx = \int \frac{e^t}{2}\:dt$ e bisogna calcolare gli estremi vedendo quanto vale t negli estremi.\\
    $x= 0$ quindi $t = 0^2 = 0$ e $x = 2$ quindi $t = 2^2 = 4$. Alla fine avremo  $\int_0^4 xe^{x^2}\:dx$, da qui poi si va avanti come prima.
\end{enumerate}
\end{example}

\begin{example}
$\int \sqrt{1-x^2}\:dx$. $x = \sin(t)$, $t = \arcsin(x)$ e $\frac{dx}{dt} = \cos(t)$ quidi $dx = \cos(t) \:dt$.\\
$= \int \sqrt{1-\sin(t)^2} \cdot \cos(t) \:dt = \int \sqrt{\cos(t)^2} \cdot \cos(t) \:dt = \int |\cos(t)| \cdot \cos(t) \:dt$ (il valore assoluto si toglie visto che $\cos(t) \geq 0$ nell'intervallo in cui stiamo integrando che è fra $-\frac{\pi}{2}$ e $\frac{\pi}{2}$).\\
$\int \cos(t)^2 \:dt = \frac{t + \sin(t)\cdot\cos(t)}{2} + c = \frac{\arcsin(x) + x \cdot \sqrt{1 - x^2}}{2} + c$.
\end{example}

\hspace{-15pt}Se andiamo a fare l'integrale di $f(x) = \sqrt{1-x^2}$ si va a calcolare l'area del cerchio unitario.\\
Infatti $4 \int_0^1 \sqrt{1-x^2}\:dx = 4\big[\frac{\arcsin(x) + x\sqrt{1-x^2}}{2}\big]_0^1 = 4 \cdot \frac{\arcsin(1)}{2} = 2 \frac{\pi}{2} = \pi$.\\\\
Se volessimo calcolare $\int \frac{1}{\sqrt{1 - x^2}}\:dx = \arcsin(x) + k$ visto che $(\arcsin(x))' = \frac{1}{\sqrt{1 - x^2}}$.
\begin{observation}
Ho anche $(\arccos(x))' = -\frac{1}{\sqrt{1-x^2}}$, quindi $\int \frac{1}{\sqrt{1-x^2}}\:dx = -\int -\frac{1}{\sqrt{1-x^2}}\:dx = -\arccos(x) + k'$. Segue che $\arcsin(x) - (-\arccos(x))$ è costante. Per vedere quanto vale basta calcolare in $x=0$, e trovo $\arcsin(0) + \arccos(0) = 0 + \frac{\pi}{2}$. Quindi  $\arcsin(x) + \arccos(x) = \frac{\pi}{2} \:\: \forall x \in [-1,1]$.
\end{observation}

\subsection{Integrali di funzioni razionali}
Consideriamo integrali nella forma $\int \frac{p(x)}{q(x)} \:dx$ dove $p(x)$ e $q(x)$ sono polinomi in x ed il grado di $q(x) \leq 2$, $\deg(q(x)) \leq 2$.
\begin{itemize}
    \item Caso con denominatore ha grado 1, $\deg(q(x)) = 1$.\\
    Esempio caso particolare con numeratore costante con $\int \frac{1}{ax + b}\:dx$. In questo caso usiamo la sostituzione $y = ax+b$ e $dy = a \cdot dx$.\\
    $= \int \frac{1}{y} \cdot \frac{dy}{a} = \frac{1}{a} \int \frac{1}{y}\:dy = \frac{1}{a} \log|ax+b| + c$.\\\\
    Caso con $\deg(p(x)) > 0$. Usiamo il caso precedente ma facendo prima la divisione di polinomi di $p(x)$ per $q(x) = ax + b$. Cioè scriviamo:\\
    $p(x) = (ax + b) \cdot Q(x) + R(x)$ dove $Q(x)$ e $R(x)$ sono polinomi e $\deg R(x) < \deg(ax + b) = 1$, (allora R(x) è una costante ed è uguale a $p(-\frac{b}{a})$).\\\\
    C'è un algoritmo per fare la divisione in maniera veloce:
    \begin{example}
    Prendiamo $\int \frac{2x^2 + 1}{x+1}\:dx$. $\frac{p(x)}{ax + b} = \frac{2x^2 +1}{x+1}$, quindi $a=1$ e $b=1$.\\
    Divido $2x^2 + 1$ per $x+1$. (Fare la divisione, vedere gli appunti delle lezioni per il modo preciso).\\
    Il risultato è: $\int \frac{(x+1)(2x-2)+3}{x+1}\:dx = \int \frac{(x+1)(2x-2)}{(x+1)}dx + \int \frac{3}{x+1} = \int (2x-2)dx + 3\log|x+1| + c = x^2 -2x + 3\log|x+1| + c$
    \end{example}
    
    \item Caso con grado denominatore uguale a 2, $\deg(q(x)) = 2$.\\
    Il primo passaggio è sempre quello di fare la divisione scrivendo $p(x) = (ax^2 + bx + c) \cdot Q(x) + R(x)$ dove $\deg R(x) <2$ cioè $R(x) = cx + d$.\\
    Quindi $\int \frac{p(x)}{q(x)}dx = \int \frac{(ax^2 + bx + c) \cdot Q(x) + R(x)}{ax^2 + bx + c} dx = \int Q(x)dx + \int \frac{R(x)}{ax^2 + bx + c}dx$, dove $R(x) = cx+d$.\\
    Per calcolare gli integrali di questa forma rimane da vedere come calcare $\int \frac{cx + d}{ax^2 + bx + c}dx$.\\
    Ci sono usa serie di casi particolare da analizzare, a seconda del numero di radici reali del denominatore:
    \begin{enumerate}
        \item Due radici coincidenti e numeratore costante. $\int \frac{dx}{(x-a)^2}$, si fa una sostituzione del tipo $y= x-a$ e $dy= dx$.
        $\int \frac{dy}{y^2} = \int y^{-2}\:dy = \frac{1}{-1} \cdot y^{-1}+c = -\frac{1}{y} + c = -\frac{1}{x-a} + c$.
        \item Due radici reali distinte e numeratore costante. $\int \frac{dx}{(x-a)(x-b)}$ con $a\neq b$. Si cercano due numeri reali A e B tali che valga:\\
        $\frac{1}{(x-a)(x-b)} = \frac{A}{(x-a)} + \frac{B}{(x-b)} = \frac{A(x-b) + B(x-a)}{(x-a)(x-b)} = \frac{x(A+B) - Ab - Ba}{(x-a)(x-b)}$. Se voglio che valga questa uguaglianza, per il principio di identità dei polinomi deve essere che:\\\\
        $\begin{cases}A+B=0\\-Ab-Ba = 1\end{cases}=$\hspace{.3cm}$\begin{cases}B = -A\\-Ab + Aa = 1\end{cases}=$\hspace{.3cm}$\begin{cases}A+B=0\\A=\frac{1}{a-b}\end{cases}=$\hspace{.3cm} $\begin{cases}B = -\frac{1}{a-b}\\A=\frac{1}{a-b}\end{cases}$\\\\
        A questo punto posso sostituire con l'espressione trovata sopra:\\
        $\int \frac{dx}{(x-a)(x-b)} = \int (\frac{1}{a-b} \cdot \frac{1}{(x-a)} - \frac{1}{a-b}\cdot\frac{1}{(x-b)})dx = \frac{1}{a-b} (\log|x-a) - \log|x-b|) + c$
    \end{enumerate}
    
    \item Denominatore senza radici reali e numeratore costante.\\
    $\int \frac{dx}{1+x^2}dx = \arctan(x) + c$. Generalizzando $\int \frac{dx}{k^2 + x^2}$ con $k \in \mathbb{R}$ e $k\neq 0$.\\
    $\int \frac{dx}{k^2 + x^2} = \frac{1}{k^2}\cdot \int \frac{dx}{1 + (\frac{x}{k})^2}$ facciamo poi una sostituzione con $y= \frac{x}{k}$ e $dy = \frac{dx}{k}$.\\
    $\frac{1}{k^2} \int \frac{1}{a+y^2} \cdot k \: dy = \frac{1}{k} \cdot \arctan(y) + c = \frac{1}{k} \cdot \arctan(\frac{x}{k}) + c$.\\
    Il casi generale con il denominatore come $ax?2 +bx + c$ senza radici reali, cioè $\Delta < 0$. In realtà posso supporre che $a = 1$:
    $\int \frac{1}{ax^2 + bx + c}dx = \frac{1}{a}\cdot \int \frac{1}{x^2 + \frac{b}{a}x + \frac{c}{a}}dx$. Quindi guardo polinomi della forma $x^2 + bx + c$ con $\Delta < 0$. Io posso fare $x^2 + bx + c = (x^2 + bx + \frac{b^2}{4}) - \frac{b^2}{4} + c = (x + \frac{b}{2})^2 + \frac{1}{4}(-b^2 + 4c)$.\\
    $\int \frac{dx}{x^2 + x + c} = \int \frac{dx}{(x + \frac{b}{2})^2 + k^2}$, con $k^2 = \frac{1}{4}(-b^2 + 4c) > 0$. Se poi andiamo a sostituire con $y = x + \frac{b}{2}$ e $dy = dx$ abbiamo $\int \frac{1}{y^2 + k^2}dx$ e questo lo sappiamo fare perché visto sopra ed è $\frac{1}{k}\arctan(\frac{x + \frac{b}{2}}{k})+c$.
    
    \begin{example}
    $\int \frac{dx}{x^2 + 2x + 10} = \int \frac{dx}{x^2 + 2x + 1 - 1 + 10} = \int \frac{dx}{(x+1)^2 + 9} = \int \frac{dy}{y^2 + 9} = \frac{1}{3}\arctan(\frac{x+1}{3})+c$.\\
    (se si fosse scelto $k=-3$ invece che $k=3$ sarebbe venuto lo stesso risultato perché $-\frac{1}{3}\arctan(-\frac{y}{3}) + c = \frac{1}{3}\arctan(\frac{y}{3})$)
    \end{example}
    
    \item Caso nel quale il denominatore non è costante, cioè ha grado 1, bisogna vedere come comportarsi con il numeratore.\\
    $\int \frac{ax + b}{x^2 + cx + d}dx = \frac{a}{2} \int \frac{2x + \frac{2b}{a}}{x^2 + cx + d}dx = \frac{a}{2}\int \frac{2x + c - c + \frac{2b}{a}}{x^2 + cx + d} = \frac{a}{2}\int \frac{2x + c}{x^2 + cx + d}dx + \frac{a}{2}\int \frac{-c \frac{2b}{a}}{x^2 + cx + d}dx$ ora per il primo integrale il numeratore è la derivata del denominatore, mentre nel secondo essendoci una costante al numeratore lo sappiamo fare.\\
    $\frac{a}{2}\log|x^2 + cx + d| + \frac{a}{2}\int \frac{-c + \frac{2b}{a}}{x^2 + cx + d}dx$.
    \begin{example}
    $\int \frac{4x + 5}{x^2 + 2x - 1}dx = 2\int \frac{2x + \frac{5}{2} + 2 - 2}{x^2 + 2x - 1}dx = 2 \int \frac{2x + 2}{x^2 + 2x - 1}dx + 2 \int \frac{\frac{1}{2}}{x^2 + 2x - 1}dx =$\\ $=2\log|x^2 + 2x - 1| + \int \frac{1}{x^2 + 2x - 1}$
    \end{example}
\end{itemize}

\subsection{Integrali impropri}
Gli Integrali impropri o generalizzati estendono la definizione di integrale definito al caso in cui l'integrale non è limitato, oppure l'intervallo di integrazione non è limitato.
\begin{example}
Dobbiamo dare un senso per esempio a $\int_0^{+\infty}e^{-x}\:dx$.
\end{example}
\begin{wrapfigure}[6]{r}{5cm}
    \vspace{-25pt}
    \centering
    \includegraphics[width=4.5cm]{images/esempio-integrale-improprio-1.png}
\end{wrapfigure}
Intuitivamente rappresenta l'area di tutto il sotto grafico sopra $(0,+\infty)$.
Formalmente definiremo un limite: \\\\
$\lim\limits_{M\to +\infty}\int_0^Me^{-x}\:dx = \lim\limits_{M\to + \infty}[-e^{-x}]^M_0=\lim\limits_{M\to +\infty}-e^{-M}+1 = 1$.\\
In questo caso il sotto grafico $[0,+\infty)$ ha area finita uguale a 1.

\begin{example}
Se invece prendiamo $\int_0^{+\infty}\frac{1}{1+x}dx = \lim\limits_{M\to +\infty}\int_0^{M}\frac{1}{1+x}dx = \lim\limits_{M\to +\infty}[\log(1+x)]_0^M = \lim\limits_{M\to +\infty}(\log(1+M)-0) = +\infty$. In questo caso l'area del sotto grafico è infinito.
\end{example}
\newpage
\begin{example}
Facciamo un' altro esempio di integrale improprio con $\int_0^1 \frac{1}{\sqrt{x}}dx$.
\end{example}
\begin{wrapfigure}[5]{l}{4.5cm}
    \vspace{-25pt}
    \centering
    \includegraphics[width=3.7cm]{images/esempio-intergale-improprio-2.png}
\end{wrapfigure}
$\int_0^1 \frac{1}{\sqrt{x}}dx$ notiamo che la funzione $\frac{1}{\sqrt{x}}$ non è limitata sull'intervallo compreso fra $[0,1)$.\\\\
$\lim\limits_{M\to 0^+}\int_M^1 \frac{1}{\sqrt{x}}dx = \lim\limits_{M\to 0^+}[2\sqrt{x}]_M^1 = \lim\limits_{M\to 0^+}(2-2\sqrt{M}) = 2$, l'area del sotto grafico di $\frac{1}{\sqrt{x}}$ sopra a $[0,1]$.
\vspace{10pt}
\begin{example}
$\int_0^1 \frac{1}{x}dx = \lim\limits_{M\to 0^+}\int_0^1 \frac{1}{x}dx = \lim\limits_{M\to 0^+}[\log(x)]_M^1 = \lim\limits_{M\to 0^+} (0-\log(M)) = + \infty$.\\
Quindi in questo caso il sotto grafico ha area infinita.
\end{example}
\begin{definition}[Integrali impropri o generalizzati]
Dati due punti $a\in \mathbb{R}$ e $b \in \mathbb{\overline{R}}$, $a<b$ e $f:[a,b)\to \mathbb{R}$ che sia integrabile in tutti gli intervalli $[a,M]$ con $a<M<b$. Se esiste $\lim\limits_{M\to b^-}\int_a^M f(x)\:dx = L$, definiamo $\int_a^b f(x)\:dx = L$. 
\begin{itemize}
    \item Se L è reale finito si dice che l'integrale di $f(x)$ su $[a,b)$ converge (oppure che $f(x)$ è integrabile "in senso generalizzato su $[a,b)$").
    \item Se L è uguale a $+\infty$ si dice che l'integrale diverge positivamente (o "a $+\infty$").
    \item Se L è uguale a $-\infty$ si dice che l'integrale diverge negativamente (o "a $-\infty$").
\end{itemize}
\end{definition}
\hspace{-15pt}Vedendo gli esempi visti sopra possiamo dire che:\\
$\int_0^{+\infty}e^{-x}\:dx$ converge \hfill $\int_0^{+\infty}\frac{1}{1+x}\:dx$ diverge pos. \hfill $\int_0^1 \frac{1}{\sqrt{x}}\:dx$ converge \hfill $\int_0^1\frac{1}{x}\:dx$ diverge pos.

\begin{example}
Esempio in cui il limite non esiste: 
\end{example}
\begin{wrapfigure}[2]{l}{4.5cm}
    \vspace{-25pt}
    \centering
    \includegraphics[width=4cm]{images/esempio-integrale-improprio-3.png}
\end{wrapfigure}
$\int_0^{+\infty}\cos(x)\:dx = \lim\limits_{M\to +\infty}\int_0^M\cos(x)\:dx = \lim\limits_{M\to +\infty}[\sin(x)]_0^M = \lim\limits_{M\to +\infty}(\sin(M)-0)$ e questo non esiste.\\\\

\hspace{-15pt}Analogamente si definisce $\int_a^b f(x)\:dx$ quando $f:(a,b] \to \mathbb{R}$ con $a \in \mathbb{\overline{R}}$, $b\in \mathbb{R}$ e $f$ integrabile su $[M,B] \forall a < M < b$ come $\lim\limits_{M\to a^+}\in_M^b f(x) \:dx$ (se esiste).\\\\
Se però abbiamo $f:(a,b)\to \mathbb{R}$ questa funzione "ha un problema" in entrambi a e b, ad esempio $\int_{-\infty}^{+\infty}\frac{1}{1+x^2}\:dx$ oppure $\int_{-1}^1 \frac{1}{1-x^2}\:dx$.

\begin{definition}
Sia $f: (a,b)\to \mathbb{R}$ con $a,b \in \mathbb{\overline{R}}$ che sia integrabile su $[M_1,M_2]$ con $a<M_1<M_2<b$. Scegliamo arbitrariamente $c\in (a,b)$, se esistono entrambi $\int_a^c f(x)\:dx$ e $\int_c^b f(x)\:dx$ allora si definisce:
\[\int_a^b f(x)\:dx = \int_a^c f(x)\:dx + \int_c^b f(x)\:dx \text{ Se la somma non è indeterminata (cioè non è} +\infty-\infty)\]
E in questo caso si diche che $f$ è integrabile in senso improprio su (a,b)
\end{definition}

\begin{observation}
L'esistenza e il valore di $\int_a^b f(x) \:dx$ non dipende dalla scelta di $c\in (a,b)$.\\
Se scelgo $d\in (a,b)$ ho $\int_a^b f(x)\:dx = \int_a^c f(x)\:dx + \int_c^b f(x)\:dx$ e $\int_d^b f(x)\:dx = \int_d^c f(x)\:dx + \int_c^b f(x)\:dx$.\\
Sommando queste due equazioni ottengo $\int_a^b f(x)\:dx + \int_d^b f(x)\:dx = \int_a^c f(x)\:dx + \int_c^b f(x)\:dx + \int_c^d f(x)\:dx + \int_d^c f(x)\:dx$ la seconda somma $\int_c^d f(x)\:dx + \int_d^c f(x)\:dx = 0$ quindi vediamo che il risultato non cambia.
\end{observation}

\begin{example}
$\int_{-\infty}^{+\infty} \frac{1}{1+x^2}\:dx$. Scelgo $c=0$.\\
$\int_{-\infty}^0 \frac{1}{1+x^2}\:dx = \lim\limits_{M\to -\infty} \int_M^0\frac{1}{1+x^2}\:dx = \lim\limits_{M\to -\infty} [\arctan(x)]_{-M}^0 = \lim\limits_{M\to -\infty}[0 -\arctan(M)] = +\frac{\pi}{2}$\\\\
$\int_0^{+\infty}\frac{1}{1+x^2}=$ (stessi conti di prima) $= \frac{\pi}{2}$.
Quindi $\int_{-\infty}^{+\infty}\frac{1}{1+x^2}\:dx = \frac{\pi}{2}+\frac{\pi}{2} = 2$ (converge)
\end{example}

\begin{example}
$\int_0^{+\infty}\frac{1}{x^2}dx$, scegliamo $c=1$.\\
$\int_0^1 \frac{1}{x^2}\:dx = \lim\limits_{M\to 0^+} \int_M^1\frac{1}{x^2}\:dx = \lim\limits_{M\to 0^+}[-x^{-1}]_M^1 = \lim\limits_{M\to 0^+}[-1+\frac{1}{M}]_M^1 = +\infty$ (diverge positivamente).\\
$\int_1^{+\infty}\frac{1}{x^2}\:dx = \lim\limits_{M\to +\infty}\int_1^M \frac{1}{x^2}\:dx = \lim\limits_{M\to +\infty}[-x^{-1}]_1^M = \lim\limits_{M\to +\infty}[-\frac{1}{M} + 1] = 1$.\\
Quindi $\int_0^+\infty \frac{1}{x^2}\:dx = +\infty + 1 = +\infty$ quindi il grafico diverge positivamente.
\end{example}

\begin{example}
Prendiamo $\int_{-1}^1 \frac{1}{x}\:dx$ in questo caso spezziamo in $c=0$.\\
$\int_{-1}^1 \frac{1}{x}dx = \int_{-1}^0 \frac{1}{x}\:dx + \int_0^1\frac{1}{x}\:dx$.\\
$\int_{-1}^0 \frac{1}{x}\:dx = \lim\limits_{M\to 0^-}\int_{-1}^M =  \lim\limits_{M\to 0^-} [\log(-x)]_{-1}^M =  \lim\limits_{M\to 0^-} \log(-M) = -\infty$.\\
$\int_0^1\frac{1}{x}\:dx = \lim\limits_{M\to 0^+}\int_M^1\frac{1}{x}\:dx = \lim\limits_{M\to 0^+} [\log(x)]_M^1 = \lim\limits_{M\to 0^+} -\log(M) = +\infty$.\\
Se vado a fare la soma ho che la somma è indeterminata $\int_{-1}^1 \frac{1}{x}\:dx = +\infty - \infty$ e dunque non esiste.\\\\
Attenzione a non fare $\int_{-1}^1 \frac{1}{x}\:dx = [\log|x|]_{-1}^1 = \log(1) - \log(1) = 0$ perché è sbagliato, il teorema di Torricelli non si applica perché $f$ non è integrabile su $[-1,1]$. Bisogna trattarlo come integrale improprio. 
\end{example}

\begin{observation}
I potrebbe pensare che ha senso dire che $\int_{-1}^1 \frac{1}{x}\:dx = 0$ visto che $\frac{1}{x}$ è dispari, e le aree sopra e sotto si sovrappongono perfettamente. Si preferisce dire comunque che l'integrale non esiste.\\\\
Si potrebbe sommare $\int_{-1}^a\frac{1}{x}\:dx + \int_b^1\frac{1}{x}\:dx$ e far tendere $a\to 0^-$ e $b\to 0^+$.\\
Il problema è che il risultante del limite dipende da come viene fatto questo limite.
\begin{example}
$\lim\limits_{b\to 0^+}(\int_{-1}^{-b}\frac{1}{x}\:dx + \int_b^1\frac{1}{x}\:dx) = \lim\limits_{b\to 0^+}(\log(b)-\log(b))=0$. \\
Ma per esempio se prendiamo $-2b$ invece che b abbiamo $\lim\limits_{b\to 0^+}(\int_{-1}^{-2b}\frac{1}{x}\:dx + \int_b^1\frac{1}{x}\:dx) = \lim\limits_{b\to 0^+}(\log(2b)-\log(b)) = \lim\limits_{b\to 0^+} \log(\frac{2b}{b}) = \log(2)$ ed il risultato è diverso.
\end{example}
\end{observation}

\hspace{-15pt} Se ci sono "più problemi" sull'intervallo di integrazione si spezza in tanti intervalli quanto basta per ricondursi a integrali impropri in cui c'è solo un problema.
\begin{example}
$\int_{-\infty}^{+\infty}\frac{1}{x^4-1}dx$ ci sono problemi sia agli estremi, più ha due asintoti a -1 e 1.\\
Quindi si spezza come $\int_{-\infty}^{+\infty} = \int_{-\infty}^{-2} + \int_{-2}^{-1} + \int_{-1}^{0} + \int_{0}^{1} + \int_{1}^{2} + \int_{2}^{+\infty}$ e la somma ha senso se hanno senso (cioè i limiti esistono) e non è indeterminata.
\end{example}

\begin{observation}
In questi casi si scrive comunque $\int_{-1}^1 \frac{1}{x}\:dx$ e non $\int_{[-1,0)\cup(0,1]}\frac{1}{x}\:dx$.
\end{observation}

\begin{proposition}
Data una $f:[a,b)\to \mathbb{R}$ integrabile su $[a,M] \forall \: a<M<b$ e supponiamo che $f$ abbia segno costante. Allora esiste (finito o infinito) $\int_a^b f(x)\:dx$. Ed esiste un enunciato analogo per il caso simmetrico $f:(a,b]\to \mathbb{R}$.
\end{proposition}

\begin{demostration}
Supponiamo ad esempio che $f\geq 0$ su $[a,b)$. Mostriamo che $F(x) = \int_a^x f(t)\:dt$ è debolmente crescente. Seguirà che $\exists \lim\limits_{x\to b^-}F(x)$ che è proprio $\int_a^b f(t)\:dt$. Infatti se $x_1 < x_2$, allora $F(x_2) = \int_a^{x_2}f(t)\:dt = \int_a^{x_1} + \int_{x_1}^{x^2}f(t)\:dt \geq \int_a^{x_1}F(x_1)$.
Il pezzo $\int_{x_1}^{x^2} \geq 0$ perché $f(t) \geq 0$ e $x_2 > x_1$. $\blacksquare$
\end{demostration}

\subsubsection{Integrali impropri notevoli}
Con la forma: $\int_1^{+\infty}\frac{1}{x^{\alpha}}$ con $\alpha \in \mathbb{R}$.
\begin{itemize}
    \item Se $\alpha = 1$, $\int \frac{1}{x}\:dx = \log|x| \longrightarrow \int_1^{+\infty}\frac{1}{x}\:dx = \lim\limits_{M\to +\infty}[\log(x)]_1^M = \lim\limits_{M\to +\infty}(\log(M)) = +\infty$, diverge.
    \item Se $\alpha \neq 1$, $\int \frac{1}{x} = \int x^{-\alpha}\:dx = \frac{1}{1-\alpha}x^{1-\alpha} + c$.\\
    Quindi $\int_1^{+\infty}\frac{1}{x^{\alpha}} = \lim\limits_{M\to +\infty} [\frac{1}{1-\alpha}x^{1-\alpha}]_1^M = \lim\limits_{M\to +\infty} (\frac{1}{1-\alpha}M^{1-\alpha} - \frac{1}{1-\alpha})$.
    \item Se $1-\alpha > 0$, cioè $\alpha < 1$, il limite è $+\infty$.
    \item se $1-\alpha < 0$, cioè $\alpha > 1$, il limite è finito e vale $-\frac{1}{1-\alpha} = \frac{1}{\alpha-1}>0$.
\end{itemize}

\begin{example}
$\int_1^{+\infty}\frac{1}{x^2}\:dx$ converge, e $\int_1^{+\infty}\frac{1}{\sqrt{x}}\:dx$ diverge a $+\infty$.
\end{example}

\hspace{-15pt}Con la forma: $\int_0^1\frac{1}{x^{\alpha}}$ con $\alpha \in \mathbb{R}$.
\begin{itemize}
    \item Se $\alpha = 1$, $\int_0^{1}\frac{1}{x}\:dx = \lim\limits_{M\to 0^+}[\log(x)]_M^1 = \lim\limits_{M\to 0^+}(\log(M)) = +\infty$ diverge.
    \item Se $\alpha \neq 1$, $\int \frac{1}{x} = \int x^{-\alpha}\:dx = \frac{1}{1-\alpha}x^{1-\alpha} + c$.\\
    Quindi $\int_0^1\frac{1}{x^{\alpha}} = \lim\limits_{M\to 0^+} [\frac{1}{1-\alpha}x^{1-\alpha}]_M^1 = \lim\limits_{M\to 0^+} (\frac{1}{1-\alpha} - \frac{1}{1-\alpha}M^{1-\alpha})$.
    \item Se $1-\alpha > 0$, cioè $\alpha < 1$, il limite è finito e vale $-\frac{1}{1-\alpha} >0$.
    \item se $1-\alpha < 0$, cioè $\alpha > 1$, il limite è finito e vale $+\infty$.
\end{itemize}

\begin{observation}
Quindi questo implica che $\int_0^{+\infty}\frac{1}{x^{\alpha}}\:dx = +\infty \:\: \forall \alpha \in \mathbb{R}$.
\end{observation}

\subsection{Criteri per la convergenza di integrali impropri}
\subsubsection{Criterio del confronto}
Prendiamo un $a\in \mathbb{R}$, $b\in \mathbb{\overline{R}}$ (deve essere $+\infty$), e $f,g: [a,+\infty)\to \mathbb{R}$ integrabile in ogni $[a,M] \: \forall \: a<M<b$. Se $\exists \: U$ intorno sinistro di $b$ tale che $0 \leq f(x) \leq g(x) \:\: \forall \: x\in U \cap [a,b)$.
\begin{enumerate}
    \item Se $\int_a^b g(x)\:dx$ converge, allora anche $\int_a^b f(x)\:dx$ converge.
    \item Se $\int_a^b f(x)\:dx$ diverge $(a,+\infty)$, allora anche $\int_a^b g(x)\:dx$ diverge $(a,+\infty)$.
\end{enumerate}
C'è un enunciato analogo se $f,g: (a,b]$.

\begin{example}
$\int_1^{+\infty}\frac{dx}{x^4+3x^3+x+1}$, chiamiamo $f(x) = \frac{dx}{x^4+3x^3+x+1}$ è continua in $[1,+\infty)$ perché $x^4+3x^3+x+1 > 0 \:\: \forall x \geq 1$.\\
Inoltre $0\leq f(x) \leq \frac{1}{x^4} \forall x\in [1,+\infty)$. Visto che $\int_1^{+\infty}\frac{1}{x^4}\:dx$ converge, per confronto concludiamo che $\int_1^{+\infty}f(x)\:dx$ converge.
\end{example}

\subsubsection{Criterio del confronto asintotico o C.A.}
Prendiamo $a\in \mathbb{R}$, $b\in \mathbb{\overline{R}}$, e $f,g: [a,b)\to \mathbb{R}$ integrabile in ogni $[a,M] \: \forall a<M<b$. Se $\exists U$ intorno sinistro di $b$ tale che $f(x) \geq 0$, $g(x) \geq 0 \forall x \in U \cap [a,b)$ e $\lim\limits_{x\to b^-}\frac{f(x)}{g(x)}=l$. Allora:
\begin{itemize}
    \item Se $l\neq 0,+\infty$, $\int_a^b f(x)\:dx$ converge $\Longleftrightarrow \int_a^b g(x)\:dx$ converge.
    \item Se $l = 0$ e $\int_a^b g(x)\:dx$ converge $\Longrightarrow \int_a^b f(x)\:dx$ converge.
    \item Se $l = +\infty$ e $\int_a^b f(x)\:dx$ converge $\Longrightarrow \int_a^b g(x)\:dx$ converge.
\end{itemize}
C'è un enunciato analogo se $f,g: (a,b]$.\\
Esempio: nel secondo caso $\lim\limits_{x\to b^-}\frac{f(x)}{g(x)}=0 \Longrightarrow$ per $x$ vicine a b vale $\frac{f(x)}{g(x)}\leq 1 \Longrightarrow f(x) \leq g(x)$ vicino $b$.

\begin{observation}
Le implicazioni di questi criteri non si invertono.
\end{observation}

\begin{example}
$\frac{1}{x^2} \leq \frac{1}{x}$ ($f(x) \leq g(x)$) per $x\geq 0$ e $\int_1^{+\infty}\frac{1}{x}\:dx$ diverge non si può concludere che $\int_1^{+\infty}\frac{1}{x^2}\:dx$ diverge. Il criterio del confronto non vale in maniera inversa.
\end{example}

\begin{example}
$\int_0^1 \frac{1}{x-\sin(x)}\:dx$, prendiamo $f(x) = \frac{1}{x-\sin{x}}$ è continua in $(0,1]$ e $f(x) > 0$ in $(0,1]$.\\
Il metodo è usare Taylor per confrontare la $f(x)$ con una certa forma $\frac{1}{x^{\alpha}}$. Sviluppiamo il denominatore in 0 (il punto "problematico")\\
$x-\sin{x} = x - (x-\frac{x^3}{6} + o(x^3)) = \frac{x^3}{6} + o(x^3)$. $f(x) = \frac{1}{x-\sin{x}} = \frac{1}{\frac{x^3}{6}}$ attorno a 0.\\
Uso il criterio del confronto asintotico con $g(x) = \frac{1}{x^3}$. $\lim\limits_{x\to 0^+}\frac{f(x)}{\frac{1}{x^3}} = \lim\limits_{x\to 0^+}\frac{x^3}{x-\sin(x)} = \lim\limits_{x\to 0^+} \frac{x^3}{x-\sin{x}} = \lim\limits_{x\to 0^+}\frac{x^3}{\frac{x^3}{6} + o(x^3)} = \frac{1}{\frac{1}{6}} = 6$. Per il C.A. concludo che $\int_0^1 f(x)\:dx$ si comporta come $\int_0^1 \frac{1}{x^3}$ che sappiamo diverge. Quindi $\int_0^1 \frac{1}{x-\sin(x)}\:dx$ diverge.
\end{example}

\begin{observation}
I criteri del confronto e del confronto asintotico si possono usare anche per funzioni negative, cambiando opportunamente le conclusioni.\\
Ad esempio: se $g(x) \leq f(x) \leq 0$ per $x\in [a,b)$ allora:
\begin{itemize}
    \item Se $\int_a^b g(x)\:dx$ converge allora anche $\int_a^b f(x)\:dx$ converge.
    \item Se $\int_a^b f(x)\:dx$ diverge (a $-\infty$ per forza) allora anche $\int_a^b g(x)\:dx$ diverge (a $-\infty$).
\end{itemize}
\end{observation}

\subsubsection{Criterio dell'assoluta convergenza}
Questo criterio si applica a funziono a segno variabile.
\begin{definition}
$f$, integrabile su ogni intervallo chiuso $[a,b]\subseteq I$, si dice assolutamente integrabile su I se $|f|$ è integrabile (eventualmente in senso generalizzato) su I, cioè $\int_I |f(x)|\:dx$ converge.
\end{definition}

\begin{definition}[Parte positiva e negativa]
Prendiamo un $x\in \mathbb{R}$. Definiamo:
\begin{itemize}
    \item La \textbf{parte positiva} di $x$ è $x^+ = max{x,0}$ cioè è x se $x \geq 0$ ed è 0 se $x < 0$. 
    \item Mentre la \textbf{parte negativa} di $x$ è $x^- = -min{x,0}$ che è $-x$ quando $x\leq 0$ e 0 se $x>0$.
\end{itemize}
\end{definition}

\begin{example}
$4^+ = 4$, $4^- = 0$, $(-3)^+ = 0$, $(-3)^- = 3$
\end{example}

\begin{observation}
Ogni $x = x^+ - x^-$ mentre $|x| = x^+ + x^-$. Segue che $x^+ = \frac{|x| + x}{2}$ e $x^- = \frac{|x|-x}{2}$.\\
Analogamente, se $f(x)$ è una funzione ho $f(x) = (f(x))^+ - (f(x))^-$, e $|f(x)| = (f(x))^+ + (f(x))^-$.
\end{observation}

\begin{proposition}[Criterio dell'assoluta integrabilità]
Se $f$ è assolutamente integrabile su $I$ allora $f$ è integrabile (in senso generalizzato) su I. 
\end{proposition}

\hspace{-15pt}Per questa proposizione non vale il viceversa.

\begin{demostration}
$|f(x)| = (f(x))^+ + (f(x))^-$ quindi:\\
$0 \leq (f(x))^+ \leq |f(x)|$ e $0 \leq (f(x))^- \leq |f(x)|$\\\\
Per confronto, supponendo che $\int_I |f|\:x$ converga, concludo che convergono $\int_I f(x)^+\:x$ e $\int_I f(x)^-\:x$.\\
Visto che $f(x) = (f(x))^+ - (f(x))^-$, concludo che: \\
$\int_I f(x)\:dx = \int_I (f(x)^+ - f(x)^-)\:dx = \int_I f(x)^+\:dx - \int_I f(x)^- \:dx$.\\\\
Ad esempio se $I = [a,b)$, abbiamo che:\\
$\int_a^M f(x)\:dx = \int_a^M (f(x)^+ - f(x)^-)\:dx = \int_a^M f(x)^+\:dx - \int_a^M f(x)^- \:dx$, passando al limite per $M\to b^-$ so che i limiti di $\int_a^M f(x)^+\:dx - \int_a^M f(x)^- \:dx$ esistono, quindi esiste anche $\lim\limits_{M\to b^-}\int_a^M f(x)\:dx$. $\blacksquare$
\end{demostration}

\begin{corollaries}
$f,g: [a,b)\to \mathbb{R}$ con $a \in \mathbb{R}$ e $b \in \mathbb{\overline{R}}$ integrabili in $[a,M] \forall a < M < b$. Se $\exists U$ intorno sinistro di $b$ tale che $|f(x)| \leq g(x) \forall x \in U \cap [a,b)$ e se $\int_a^b g(x) \:dx$ converge $\Longrightarrow \int_a^b f(x)\:dx$ converge. (Confronto + assoluta integrabilità)
\end{corollaries}

\begin{example}
$\int_1^{+\infty}\frac{\sin{x}}{x^2}$. $f(x) = \frac{\sin{x}}{x^2}$ a segno variabile su $[1,+\infty)$.\\
$|f(x)| = \frac{|\sin{x}|}{x^2} \leq \frac{1}{x^2}$, prendo $g(x) = \frac{1}{x^2}$ nel corollario di sopra.\\
Visto che $\int_1^{+\infty}\frac{1}{x^2}\:dx$ converge, concludo che $\int_1^{+\infty}\frac{\sin{x}}{x^2}$ converge.
\end{example}

\begin{example}
$\int_1^{+\infty}\frac{\sin{x}}{x}$. Procedendo alla stesso modo di sopra $f(x) = \frac{\sin{x}}{x}$ a segno variabile.\\
$|f(x)| = \frac{|\sin{x}|}{x^2} \leq \frac{1}{x}$ prendo $g(x) = \frac{1}{x^2}$. Questa volta però $\int_1^{+\infty}\frac{1}{x}\:dx$ diverge.\\
Quindi non posso concludere niente su $\int_1^{+\infty}\frac{\sin{x}}{x}$. In questo caso possiamo:
$\int_1^{+\infty}\frac{\sin{x}}{x} = \lim\limits_{M\to +\infty}\int_1^{M}\frac{\sin{x}}{x} = $ (integro per parti) $= \int_1^{M}\sin{x}\frac{1}{x}\:dx = [-\frac{\cos{x}}{x}]_1^M - \int_1^M \frac{\cos{x}}{x^2}\:dx =  \lim\limits_{M\to +\infty}(-\frac{\cos{M}}{M} + \frac{\cos{1}}{1} - \int_1^M \frac{\cos{x}}{x^2})\:dx =  \lim\limits_{M\to +\infty}(-\frac{\cos{M}}{M} + \cos{1}) - \lim\limits_{M\to +\infty}\int_1^M \frac{\cos{x}}{x^2}\:dx$.\\
Il risultato finale è uguale a $\int_1^M \frac{\cos{x}}{x^2}\:dx = \int_1^{+\infty}\frac{\cos{x}}{x^2}$ che converge come il caso con seno (visto nell'esempio prima). Mentre la parte $-\frac{\cos{M}}{M}$ tende a 0, quindi $\int_1^{+\infty}\frac{\sin{x}}{x}$ converge.
\end{example}

\begin{observation}
Stesso discorso per $\int_1^{+\infty}\frac{\cos{x}}{x}$ che converge.
\end{observation}

\begin{example}
Vediamo come $\int_1^{+\infty}\frac{|\sin{x}|}{x}$ diverge(questo da un esempio di $f(x)$ tale che $\int_1^{+\infty}f(x)\:dx$ converge, ma $\int_1^{+\infty}|f(x)|\:dx$ diverge).\\\\
Osserviamo che $|\sin{x}| \geq (\sin{x}^2)$ (perché $-1 \leq \sin{x} \leq 1$). Quindi $\int_1^{M}\frac{|\sin{x}|}{x} \geq \int_1^{M}\frac{\sin{x}^2}{x}\:dx = \int_1^{M}\frac{(1-\cos{2x})}{2x} = \int_1^{M}\frac{1}{2x} - \int_1^{M}\frac{\cos{sx}}{2x} = \int_1^{M}\frac{1}{2x} - \frac{1}{2}\int_2^{2M}\frac{\cos{t}}{t}\:dt$ con $t=2x$ e $dt=2dx$. Il primo integrale diverge ed il secondo converge perché si ritorna ad un caso visto prima ($\int_2^{+\infty}\frac{\cos(t)}{t}\:dt$). \\
Quindi in conclusione la somma diverge a $+\infty$ quindi $\int_1^{+\infty}\frac{|\sin{x}|}{x}$ diverge a $+\infty$.
\end{example}

\subsubsection{Integrali impropri ricorrenti}
\underline{\textbf{TIPO 1°}}. Vediamo ora gli integrali del tipo $\int_2^{+\infty}\frac{1}{x^{\alpha}\log(x)^{\beta}}\:dx$ con $\alpha, \beta \in \mathbb{R}$.
\begin{itemize}
    \item Caso con $\alpha > 1$: possiamo prendere un $\gamma \in \mathbb{R}$ tale che $\alpha > \gamma > 1$.\\
    $f(x) = \frac{1}{x^{\alpha}(\log(x)^{\beta})}$ e $g(x) = \frac{1}{x^{\gamma}}$. $f(x), g(x) \geq 0$ e $\lim\limits_{x\to +\infty}\frac{f(x)}{g(x)} = \lim\limits_{x\to +\infty} \frac{x^{\gamma}}{x^{\alpha}(\log(x)^{\beta})} = \lim\limits_{x\to +\infty} \frac{1}{x^{\alpha-\gamma}(\log(x))^{\beta}}$ questo limite è 0.\\
    Quindi visto che $\gamma > 1$, quindi $\int_2^{+\infty}\frac{1}{x^{\gamma}}\:dx$ converge e per C.A. concludiamo che $\int_2^{+\infty}\frac{1}{x^{\alpha}(\log(x)^{\beta})}\:dx$ converge.
    \item Caso con $\alpha < 1$: possiamo prendere un $\gamma \in \mathbb{R}$ tale che $\alpha < \gamma < 1$.\\
    $f(x) = \frac{1}{x^{\alpha}(\log(x)^{\beta})}$ e $g(x) = \frac{1}{x^{\gamma}}$. $f(x), g(x) \geq 0$.\\
    Questa volta $\lim\limits_{x\to +\infty}\frac{f(x)}{g(x)} = \lim\limits_{x\to +\infty}\frac{x^{\gamma-\alpha}}{(\log(x)^{\beta})} = +\infty$.\\
    Visto che $\gamma < 1$, $\int_2^{+\infty}\frac{1}{x^{\gamma}}\:dx$ diverge per C.A. Possiamo quindi concludere che $\int_2^{+\infty}\frac{1}{x^{\alpha}(\log(x)^{\beta})}\:dx$ diverge $\forall \beta \in \mathbb{R}$.
    \item Caso con $\alpha = 1$: $\int_2^{+\infty}\frac{1}{x^{\alpha}(\log(x)^{\beta})}\:dx$.\\
    $\int_2^{M}\frac{1}{x^{\alpha}(\log(x)^{\beta})}\:dx$ = con $t = \log(x)$ e $dt = \frac{1}{x}\:dx = \int_{\log(2)}^{\log(M)}\frac{1}{t^{\beta}}\:dt = \lim\limits_{M\to +\infty}\frac{1}{x^{\alpha}(\log(x)^{\beta})} = \int_{\log(2)}^{+\infty}\frac{1}{t^{\beta}}\:dt$ che converge se $\beta > 1$ e diverge a $+\infty$ se $\beta \leq 1$.
\end{itemize}
\underline{\textbf{TIPO 2°}}. Analogamente studiamo $\int_0^{\frac{1}{2}}\frac{1}{x^{\alpha}|\log(x)|^{\beta}}\:dx$ con $\alpha, \beta \in \mathbb{R}$.\\
$\int_M^{\frac{1}{2}}\frac{1}{x^{\alpha}|\log(x)|^{\beta}}\:dx =$ con $t=\frac{1}{x}$ quindi $x = \frac{1}{t}$, e $dx = -\frac{1}{t^2}\:dt \int_{\frac{1}{M}}^2 \frac{-dt}{t^2 \cdot t^{-\alpha} |l-\log(t)|^{\beta}} =$ (se $M\to 0^+$ allora $\frac{1}{M}\to +\infty$) $=\int_2^{\frac{1}{M}} \frac{dt}{t^{2-\alpha}\cdot|\log(t)|^{\beta}} = \lim\limits_{M\to 0^+}\frac{dt}{t^{2-\alpha}\cdot|\log(t)|^{\beta}} = \int_2^{\frac{1}{M}} \frac{dt}{t^{2-\alpha}\cdot|\log(t)|^{\beta}}$ e questo l'abbiamo appena studiato.
Segue che $\int_0^{\frac{1}{2}}\frac{1}{x^{\alpha}|\log(x)|^{\beta}}\:dx$ abbiamo che:
\begin{itemize}
    \item $2-\alpha > 1$ ($\alpha < 1$) converge $\forall \beta \in \mathbb{R}$.
    \item $2-\alpha < 1$ ($\alpha > 1$) diverge a $+\infty$ $\forall \beta \in \mathbb{R}$.
    \item $2-\alpha = 1$ ($\alpha = 1$) con $\beta > 1$ converge.
    \item $2-\alpha = 1$ ($\alpha = 1$), $\beta \leq 1$ diverge a $+\infty$.
\end{itemize}

\hspace{-15pt}\underline{\textbf{TIPO 3°}}. Vediamo come ultimo gli integrali della forma $\int_a^{+\infty} \frac{1}{x^{\alpha}}$.
\begin{itemize}
    \item Questo integrale converge se $a > 0$ e $\alpha > 1$.
    \item Invece l'integrale diverge a $+\infty$ se $a > 0$ e $\alpha \leq 1$-
\end{itemize}

\subsubsection{Esempi riassuntivi}
\begin{example}
Primo esempio riassuntivo: $\int_{x_0}^{x_0+1}\frac{dx}{x-x_0}$
\begin{itemize}
    \item Converge se $\alpha < 1$.
    \item Diverge a $+\infty$ se $\alpha \geq 1$.
\end{itemize}
Dato M tale che $x_0 < M < x_0 + 1$. $\int_M^{x_0+1}\frac{dx}{(x-x_0)^{\alpha}}$ = con $t = x - x_0$ e $dt = dx = \int_{M-x_0}^1 \frac{dt}{t^{\alpha}}$.\\
$\lim\limits_{M\to x_0^+}\int_M^{x_0+1}\frac{dx}{(x-x_0)^{\alpha}} = \lim\limits_{M\to x_0^+}\int_{M-x_0}^1 \frac{dt}{t^{\alpha}} = \int_0^1 \frac{dt}{t^{\alpha}}$ e sappiamo che questo converge se $\alpha < 1$ e diverge a $+\infty$ se $\alpha \geq 1$.
\end{example}

\begin{example}
Secondo esempio riassuntivo: $\int_0^2 \frac{x^3 + 1}{x^2 -4}\:dx$. $f(x) = \frac{x^3 + 1}{x^2 -4}$ è definita e continua in $[0,2)$ quindi integrale va trattato come integrale improprio. Bisogna notare anche che $f(x) < 0$ (sempre positiva) in tutto $[0,2)$ perché $x^3+1 > 0$ per $x>0$ e $x^2 -4 < 0$ per $0 \leq x < 2$.\\
Avendo segno costante si possono usare i criteri del confronto e del confronto asintotico.\\
$f(x) = \frac{x^3+1}{x^2-4} = \frac{x^3 +1}{(x-2)(x+2)}$ il pezzo problematico è $g(x) = \frac{1}{x-2}$.\\
Usiamo C.A. con $g(x) = \frac{1}{x-2}$ (notare $g(x) < 0$ in $[0,2)$). Poi facciamo $\lim\limits_{x\to 2}\frac{f(x)}{g(x)}$:\\
$\lim\limits_{x\to 2} \frac{x^3 +1}{(x-2)(x+2)}\cdot(x-2) = \frac{9}{4} \neq 0, +\infty$. Per C.A. $\int_0^2 f(x)\:dx$ ha lo stesso comportamento di $\int_0^2 \frac{1}{x-2}\:dx$ che sappiamo diverge negativamente (sostituzione $t=2-x$ per ricondursi a $\int \frac{dt}{t}$).\\
Quindi $\int_0^2 \frac{x^3 + 1}{x^2 -4}\:dx = -\infty$ (si scrive sono $-\infty$ che vuol dire che diverge negativamente)
\end{example}

\begin{example}
Terzo esempio riassuntivo: $\int_0^{+\infty}\frac{\log(1+x^2)}{\sqrt{1+x^2}}\:dx$. $f(x) = \frac{\log(1+x^2)}{\sqrt{1+x^2}}$ è definita e continua in $\mathbb{R}$ ($1 + x^2 > 0 \forall x \in \mathbb{R}$). Infine $f(x) \geq 0 \forall x \in \mathbb{R}$. Quindi l'unico problema è a $+\infty$.\\
Per $x$ grandi $\frac{\log(1+x^2)}{\sqrt{1+x^2}}$ sarà circa $\frac{\log(x^2)}{\sqrt{x^2}} = \frac{2\log(x)}{x} \geq \frac{1}{x}$ e sappiamo che $\int_1^{+\infty}\frac{1}{x}\:dx$ diverge, quindi probabilmente anche il nostro divergerà.\\
Facciamo C.A. con $g(x) = \frac{1}{x}$. $\lim\limits_{x\to +\infty}\frac{f(x)}{g(x)} = \lim\limits_{x\to +\infty} \frac{\log(1+x^2)}{\sqrt{1+x^2}}\cdot x = \lim\limits_{x\to +\infty} \frac{\log(1+x^2)}{\sqrt{1+\frac{1}{x^2}}} = \frac{+\infty}{1} = +\infty$.\\
Quindi visto che $\int_1^{+\infty}\frac{1}{x}\:dx$ diverge, concludo che $\int_1^{+\infty}f(x)\:dx$ diverge positivamente. Segue che $\int_0^{+\infty}f(x)\:dx = \int_0^{1}f(x)\:dx + \int_1^{+\infty}f(x)\:dx = +\infty$ cioè il nostro integrale diverge positivamente.
\end{example}

\begin{example}
Quarto esempio riassuntivo: $\int_0^{+\infty}\frac{x^2}{(2+3x^4)\cdot\arctan(x^{\frac{5}{2}})}\:dx$. $f(x) = \frac{x^2}{(2+3x^4)\cdot\arctan(x^{\frac{5}{2}})}$ definita e continua su $(0,+\infty)$ e $f(x) > 0$ su $(0,+\infty)$, ci sono 2 problemi, in 0 e a $+\infty$ quindi spezziamo in $\int_0^{+\infty}f(x)\:dx = \int_0^1 f(x)\:dx + \int_1^{+\infty}f(x)\:dx$ e studiamo i due pezzi.
\begin{itemize}
    \item Caso $\int_0^1$: per $x\to 0^+$ si ha $\arctan(x^{5/2}) = x^{5/2} + o(x^{5/2})$ quindi $f(x) = \frac{x^2}{(2+3x^4)(x^{\frac{5}{2}} + o(x^{5/2}))} = \frac{x^2}{2x^{\frac{5}{2}} + o(x^{\frac{5}{2}})} = \frac{1}{2x^{\frac{1}{2}} + o(x^{\frac{1}{2}})}$ prendo $g(x) = \frac{1}{x^{\frac{1}{2}}} = \frac{1}{\sqrt{x}}$.\\
    Ho $\lim\limits_{x\to 0^+}\frac{f(x)}{g(x)} = \lim\limits_{x\to 0^+}\frac{1}{2x^{\frac{1}{2}} + o(x^{\frac{1}{2}})} = \frac{1}{2} \neq 0, +\infty$. Visto che $\int_0^1 \frac{1}{\sqrt{x}}\:dx$ converge per C.A. concludo che converge anche $\int_0^1 f(x)\:dx$.
    \item Caso $\int_1^{+\infty}$: per $x\to +\infty$, $f(x) = \frac{x^2}{(2+3x^4)\cdot\arctan(x^{\frac{5}{2}})} = \frac{x^2}{x^4(\frac{2}{x^4}+3)\cdot\arctan(x^{\frac{5}{2}})} = \frac{1}{x^2} \cdot \frac{1}{(\frac{2}{x^4} + 3)\cdot \arctan(x^{\frac{5}{2}})}$ la seconda parte per $x\to +\infty$ fa $\frac{1}{3\cdot \frac{\pi}{2}} = \frac{2}{3\pi}$, prendo quindi $g(x) = \frac{1}{x^2}$.\\
    $\lim\limits_{x\to +\infty}\frac{f(x)}{g(x)} = \lim\limits_{x\to +\infty}\frac{x^2}{x^2\cdot(\frac{2}{x^4} + 3)\cdot \arctan(x^{\frac{5}{2}})} = \frac{2}{3\pi} \neq 0, +\infty$. Vito che $\int_1^{+\infty}\frac{1}{x^2}\:dx$ converge, per C.A. converge anche $\int_1^{+\infty}f(x)\:dx$.
\end{itemize}
In conclusione, anche $\int_0^{+\infty}f(x)\:dx = \int_0^1 f(x)\:dx + \int_1^{+\infty}f(x)\:dx$ converge.
\end{example}

\begin{example}
Quinto esempio (con segno variabile): $\int_0^{+\infty}\frac{\sin{x}}{x^{3/2}(x^2+1)}\:dx$. $f(x) = \frac{\sin{x}}{x^{3/2}(x^2+1)}$ definita e continua in $(0,+\infty)$, problemi in $x=0$ e $x=+\infty$, $f(x)$ a segno variabile.\\
Spezziamo $\int_0^{+\infty}f(x)\:dx = \int_0^1f(x)\:dx + \int_1^{+\infty}f(x)\:dx$
\begin{itemize}
    \item Caso $\int_0^1$: osserviamo che $f(x) \geq 0$ per $0 \leq x \leq 1$ perché ($\sin(x) \geq 0$ per $0 \leq x \leq 1 < \frac{\pi}{2}$). Quindi posso usare confronto e C.A. per $x\to 0^+$ $f(x) =\frac{\sin{x}}{x^{3/2}(x^2+1)} = \frac{x+o(x)}{x^{3/2} + o(x^{3/2})} = \frac{1 + o(1)}{x^{1/2}+o(x^{1/2})}$, prendo quindi $g(x) = \frac{1}{x^{1/2}}$.\\
    $\lim\limits_{x\to 0^+}\frac{f(x)}{g(x)} = \lim\limits_{x\to 0^+} \frac{1 + o(1)}{x^{1/2}+o(x^{1/2})} \cdot x^{1/2} = 1 \neq 0,+\infty$. Visto che $\int_0^1 \frac{1}{\sqrt{x}}\:dx$ converge, per C.A. concludiamo che $\int_0^1 f(x)\:dx$ converge.
    \item Caso $\int_1^{+\infty}$ qui $f(x)$ non è costante ma oscilla tra valori positivi e negativi. Proviamo ad usare assoluta convergenza:\\
    $|f(x)| = \frac{|\sin(x)|}{x^{3/2}(x^2+1)} \leq \frac{1}{x^{3/2}(x^2 + 1)} \leq \frac{1}{x^{3/2}}\cdot \frac{1}{x^2} = \frac{1}{x^{7/2}}$. Visto che $\int_1^{+\infty}\frac{1}{x^{7/2}}\:dx$ converge, per confronto ho che $\int_1^{+\infty}|f(x)|\:dx$ converge e per il criterio dell'assoluta integrabilità segue che $\int_1^{+\infty}f(x)$ converge.
\end{itemize}
In conclusione, anche $\int_0^{+\infty}f(x)\:dx = \int_0^1 f(x)\:dx + \int_1^{+\infty}f(x)\:dx$ converge.
\end{example}
\newpage
\section{Successioni}
\begin{definition}[Successione]
Una successione\footnote{Nelle successioni si è soliti scrivere n al posto di x come simbolo per la variabile ess. $f(n)$} è una funzione $f: S\to \mathbb{R}$ dove S è una semiretta di $\mathbb{N}$, cioè $S = \{n \in \mathbb{R}\:|\: x\geq n_0\}$ per qualche $n_0$.
\end{definition}

\begin{example}
Consideriamo $f(n) = n^2$ con $S = \mathbb{N}$.
\end{example}
\begin{wrapfigure}[6]{l}{6cm}
    \vspace{-20pt}
    \centering
    \includegraphics[width=5cm]{images/esempio-successione-1.png}
\end{wrapfigure}
Da questa funzione posso calcolare tutti i valori: $f(0) = 0^2 = 0$, $f(1) = 1^2 = 1$, $f(2) = 2^2 = 4$\\\\
E possibile disegnare un grafico di una successione che è composto da una serie di punti sparsi.\\

\begin{example}
$f(n) = \frac{1}{n}$, come S non posso prendere tutti i naturali perché con 0 non ha senso quindi $S = \{n \in \mathbb{N} \: |\: n \geq 1\}$.  $f(1) = \frac{1}{1}=1$, $f(2) = \frac{1}{2}$, $f(3) = \frac{1}{3}$.
\end{example}

\subsection{Notazione}
Nelle successioni invece di scrivere $f(n)$ di solito una successione si denota con $a_n$. Negli esempio di prima si sarebbe: $a_n = n^2$, $a_n = \frac{1}{n}$.\\
L'intera successione si denota con $\{a_n\}$ oppure $\{a_n\}_{n\in \mathbb{N}}$, $\{a_n\}_{n\in S}$.
\begin{example}
$a_n = \frac{1}{n-5}$. La formula ha senso per $n\neq 5$, quindi si può prendere $S = \{n \in \mathbb{N} \:|\: n\geq 6\}$ (avrei anche potuto prendere $n \geq 7$ o $n \geq 8$).
\end{example}

\begin{example}
$a_n = \sqrt{5 - n}$. La formula ha senso se $5-n \geq 0$ cioè $n \leq 5$. Nessuna semiretta va bene perché in una successione n diventa sicuramente più grande ad un certo punto quindi non definisce una successione.
\end{example}

\subsection{Limiti di Successioni}
Come per le funzioni bisogna guardare come si comporta la successioni all'avvicinarsi ad un limite. L'unico limite che ha senso è il limite per $n\to +\infty$, perché $+\infty$ è l'unico punto di accumulazione di tutto il dominio (perché $S \subseteq \mathbb{N}$).
\begin{definition}[Limite di successione]
Si ha che $\lim\limits_{n\to +\infty}a_n = l$ se $\forall \:\: U$ intorno di l si ha che $\exists \: \overline{n}\in \mathbb{N}$ tale che $a_n \in U \:\: \forall \: n\geq \overline{n}$.\\
Si dice che $a_n$ converge a $l$ se $\lim\limits_{n\to +\infty}a_n = l$ e $l\in \mathbb{R}$ e che diverge a $\pm \infty$ se $\lim\limits_{n\to +\infty}a_n = \pm \infty$.
\end{definition}

\begin{figure}[h!]
\centering
\begin{subfigure}{.45\textwidth}
    \vspace{-15pt}
    \centering
    \includegraphics[width=5cm]{images/limite-successione-1.png}
    \caption{Graficamente se il limite è in $\mathbb{R}$ quindi $l \in \mathbb{R}$}
\end{subfigure}
\begin{subfigure}{.45\textwidth}
    \centering
    \includegraphics[width=4cm]{images/limite-successione-2.png}
    \caption{E con $l = +\infty$}
\end{subfigure}
\end{figure}

Esiste una \textbf{Terminologia} quando si parla di queste cose: se $P(n)$ è un predicato la cui verità dipende da $n\in \mathbb{N}$ (esempio: $P(n) =$ "n è pari") si dice che $P(n)$ è vero definitivamente se $\exists \: \overline{n}\in \mathbb{N}$ tale che $P(n)$ è vero $\forall n \geq \overline{n}$.\\
Quindi $\lim\limits_{n\to +\infty} a_n = l$ se $\forall \: U$ introno di l si ha che $a_n \in U$ definitivamente.

\subsection{Sottosuccessioni (estratte)}
\begin{definition}[Sottosuccessione]
Dato $a_n: S \to \mathbb{R}$ una successione, consideriamo $k_n: \mathbb{N} \to S$ strettamente crescente (cioè $k_n > k_m$ quando $n>m$), possiamo considerare la composizione $a_{k_n}$. Questa è una nuova successione detta sottosuccessione di $\{a_n\}$ (In pratica scegliamo solo un certo sottoinsieme di indici, in modo crescente).
\end{definition}

\begin{example}
Prendiamo la successione $a_n = \frac{1}{n}$.
\end{example}
\begin{wrapfigure}[4]{r}{6cm}
    \vspace{-40pt}
    \centering
    \includegraphics[width=5cm]{images/esempio-sottosuccessioni.png}
\end{wrapfigure}
Per avere una sottosuccessione prendo $k_n: \mathbb{N}\to S$,e prendo $n \mapsto 2n+1$. Abbiamo $a_{k_n} = \frac{1}{k_n} = \frac{1}{2n+1}$.\\
Quindi graficamente: \\
$a_{k_0} = \frac{1}{0+1} = 1$, $a_{k_1} = \frac{1}{2\cdot 1 +1} = \frac{1}{3} = a_3$, $a_{k_2} = \frac{1}{2 \cdot 2+1} = \frac{1}{5} = a_5$.\\\\
\begin{theorem}
Data una successione $\lim\limits_{n\to +\infty}a_n = l$ se e solo se vale $\lim\limits_{n\to +\infty}a_{k_n} = l$ per ogni sottosuccessione di $\{a_n\}$.
\end{theorem}
\hspace{-15pt} A volta si può usare per dimostrare che una successione non ha limite.
\begin{example}
$a_n= (-1)^h = \begin{cases}-1 & \text{se n è pari} \\ -1 & \text{se n è dispari} \end{cases}$ \\\\
Questo successione non ha limite e si dimostra con il teorema visto sopra. Infatti, consideriamo le sottosuccessioni $\{a_{2n}\}$ e $\{a_{2n+1}\}$ date da indici pari e dispari. \\
Abbiamo che $a_{2n} = (-1)^{2n} = (1)^n = 1$ che converge a 1 mentre, $a_{2n+1} = (-1)^{2n+1} = -1$ e quindi converge a -1. Visto che questi limiti esistono e sono diversi, segue dal teorema che $\{a_n\}$ non può avere limite.
\end{example}

\begin{observation}
Per i limiti di successioni valogono molti dei teoremi visti per le funzioni, ad esepio:
\begin{itemize}
    \item Formule per limiti di somme, prodotti, quozienti, esponenziali etc.
    \item Teorema di permanenza del segno.
    \item Teorema dei carabinieri.
    \item Teorema del confronto, ed altri...
\end{itemize}
\end{observation}

\begin{example}
Per esempio il teorema della permanenza del segno per le successioni dice: se abbiamo una successione che $\lim\limits_{n\to +\infty} a_n = l > 0$, allora $a_n > 0$ definitivamente.
\end{example}

\subsection{Monotonia}
\begin{definition}[Monotonia]
Una successione $\{a_n\}$ essa si dice:
\begin{itemize}
    \item \textbf{Debolmente crescente} se $n>m \Longrightarrow a_n \geq a_n$.
    \item \textbf{Strettamente crescente} se $n > m \Longrightarrow a > a_m$.
    \item \textbf{Debolmente decrescente} se $n > m \Longrightarrow a_n \leq a_m$.
    \item \textbf{Strettamente decrescente} se $n > m \Longrightarrow a_n < a_m$.
\end{itemize}
Successione è monotona quando vale una di queste 4 proprietà.
\end{definition}

\begin{observation}
$\{a_n\}$ è debolmente crescente se e solo se vale $a_{n+1} \geq a_n \forall \: n \in S$ (basta guardare termini successivi).\\
Infatti, se so che $a_{n+1} \geq a_n \forall \: n \in \mathbb{N}$, poi se $n > m$ allora $a_n \geq ... \geq a_{m+2} \geq a_{m+1} \geq a_{m}$.
\end{observation}

\begin{example}
Prendiamo $a_n=n^2$ e controlliamo che è strettamente crescente: vediamo che $a_{n+1} > a_n$. Infatti $a_{n+1} = (n+1)^2 = n^2 + 2n + 1$ e $a_n = n^2$ e quindi $n^2 + 2n + 1 > n^2 \Longleftrightarrow 2n+1 > 0$ che è vero $\forall \:n \in \mathbb{N}$.
\end{example}

\begin{theorem}
Se $\{a_n\}$ è monotona (cioè debolmente crescente o decrescente) allora ammette limite.
Se è debolmente crescente, il limite non può essere $-\infty$ e se Se è debolmente decrescente, il limite non può essere $+\infty$
\end{theorem}

\subsection{Limitatezza}
\begin{definition}[Limitatezza]
Una successione $\{a_n\}$ è \textbf{limitata superiormente} se $\exists\: M \in \mathbb{R}$ tale che $a_n \subseteq M \:\forall\: \in S$ e \textbf{limitata inferiormente} se $\exists \:m \in \mathbb{R}$ tale che $a_n \geq m \forall \: n \in S$ e \textbf{limitata} se è limitata sia inferiormente e superiormente. (immagine \ref{limitatezza-successioni})
\end{definition}

\begin{observation}
Una successione convergente (che ha limite finito) è limitata. Questo non è vero per funzioni di variabile reale.
\end{observation}

\begin{example}
$f(x) = \frac{1}{x}$, $f: (0,+\infty)\to \mathbb{R}$ abbiamo $\lim\limits_{x\to +\infty}f(x) = 0$ ma f non è limitata, perché $\lim\limits_{x\to 0^+}f(x) = +\infty$ però $a_n = \frac{1}{n}$ invece è limitata.
\end{example}

\begin{theorem}
Se $\lim\limits_{n\to +\infty}a_n = +\infty$, allora $\{a_n\}$ ha minimo (cioè $\exists \:n_{min} \in \mathbb{N}$ tale che $a_n \geq a_{n_{min}} \: \forall \:n \in S$). Se invece $\lim\limits_{n\to +\infty}a_n = -\infty$ allora $a_n$ ha massimo.  (immagine \ref{teorema-minimo})
\end{theorem}
\begin{figure}[h!]
\centering
\begin{subfigure}{.45\textwidth}
    \vspace{-25pt}
    \centering
    \includegraphics[width=5cm]{images/limitatezza-successioni.png}
    \caption{Graficamente definizione di limiti inf, sup}
    \label{limitatezza-successioni}
\end{subfigure}
\begin{subfigure}{.45\textwidth}
    \vspace{-15pt}
    \centering
    \includegraphics[width=4cm]{images/teorema-minimo.png}
    \caption{Graficamente teorema minimo massimo}
    \label{teorema-minimo}
\end{subfigure}
\caption{Raffigurazione di definizione limitatezza e teorema minimo massimo}
\end{figure}
\hspace{-15pt}Ci si può chiedere come domanda se una successione $\{a_n\}$ è limitata, necessariamente massimo e minimo? La risposte è no.
\begin{example}
Se prendiamo $a_n = \frac{1}{n}$ è limitata: $1 \geq \frac{1}{n} > 0$ ma non ha minimo. $max\{a_n\} = 1$ e $inf\{a_n\} = 0$ (uguale a $\lim\limits_{n\to +\infty} a_n$). Non ha minimo perché non esiste $n \in \mathbb{N}$ tale che $\frac{1}{n} = 0$
\end{example}
\hspace{-15pt}Inoltre è possibile chiedersi se $\{a_n\}$ è limitata, esiste almeno uno tra massimo minimo? E la risposta anche in questo caso è no.
\begin{example}
Prendiamo $a_n = (1-\frac{1}{n})(-1)^n = \begin{cases}1-\frac{1}{n} & \text{per n pari} \\ -(1 - \frac{1}{n}) & \text{per n dispari}\end{cases}$
\end{example}
\begin{figure}[h!]
\centering
\begin{subfigure}{.3\textwidth}
    \vspace{-25pt}
    \centering
    \includegraphics[width=4.5cm]{images/esempio-mas-min-succesioni-1.png}
    \caption{n pari}
\end{subfigure}
\begin{subfigure}{.3\textwidth}
    \centering
    \includegraphics[width=4cm]{images/esempio-mas-min-succesioni-2.png}
    \caption{n dispari}
\end{subfigure}
\begin{subfigure}{.3\textwidth}
    \centering
    \includegraphics[width=4cm]{images/esempio-mas-min-succesioni-3.png}
    \caption{Complessivamente}
\end{subfigure}
\end{figure}
\hspace{-15pt}Complessivamente possiamo vedere la la successione oscilla avvicinandosi con $sun\{a_n\} = 1$ e $inf\{a_n\} = -1$, e non esistono massimo e minimo, anche se $a_n$ è limitata, visto che $-1 < a_n < 1$.

\newpage
\begin{example}
Prendiamo $a_n = \frac{(-1)^n}{n}$ e ci chiediamo se ha limite e sa ha massimo e o minimo.
\end{example}
\begin{wrapfigure}[4]{l}{6cm}
    \vspace{-15pt}
    \centering
    \includegraphics[width=5cm]{images/esempio-mas-min-successioni-4.png}
\end{wrapfigure}
Abbiamo che $\lim\limits_{n\to +\infty} = 0$. Infatti abbiamo che $-\frac{1}{n} \leq a_n \leq \frac{1}{n}$ e visto che $\lim\limits_{n\to +\infty}-\frac{1}{n}=\lim\limits_{n\to +\infty}\frac{1}{n} = 0$ per il teorema dei carabinieri abbiamo che $\lim\limits_{n\to +\infty}a_n = 0$. Quindi ha massimo e minimo il massimo è in $n=2$ ed il minimo in $n=1$.\\\\

\begin{theorem}
Se ho usa successione che converge $\lim\limits_{n\to +\infty}a_n = l$ finito allora:
\begin{itemize}
    \item $\exists \: \overline{n}\in \mathbb{N}$ tale che $a_{\overline{n}} \geq l \Longrightarrow \{a_n\}$ ha massimo.
    \item $\exists \: \overline{n}\in \mathbb{N}$ tale che $a_{\overline{n}} \leq l \longrightarrow \{a_n\}$ ha minimo.
\end{itemize}
\end{theorem}

\subsection{Legame tra limiti di funzione e successioni}
\begin{theorem}
Prendiamo una funzione definita in $A \subseteq \mathbb{R}$ sottoinsieme $f:A \to \mathbb{R}$, e $x_0 \in acc(A)$. Allora abbiamo che $\lim\limits_{x\to x_0}f(x) = l$ se e solo se $\lim\limits_{n\to +\infty}f(a_n) = l$ per ogni successione $\{a_n\}\subseteq A$ tale che $\lim\limits_{n\to +\infty}a_n = x_0$ e $a_n \neq x_0$ definitivamente.
\end{theorem}
\begin{wrapfigure}[2]{r}{6cm}
    \vspace{-35pt}
    \centering
    \includegraphics[width=4.7cm]{images/legame-lim-successioni-funzioni.png}
\end{wrapfigure}
Questo teorema a volte si può utilizzare per dimostrare che non esiste $\lim\limits_{x\to x_0}f(x)$.\\\\
\begin{example}
Dimostriamo che non esiste $\lim\limits_{x\to +\infty}\sin(x)$.\\
Esibiamo due successioni $a_n, b_n$ che tendono a $+\infty$,  tali che $\lim\limits_{n\to +\infty}\sin(a_n)$ e $\lim\limits_{n\to +\infty}\sin(b_n)$ esistono, ma sono diversi.\\
Prendo $a_n = n\pi$. Abbiamo $\lim\limits_{a\to +\infty}a_n = n\pi = +\infty$. Inoltre $\lim\limits_{n\to +\infty}\sin(a_n)= \lim\limits_{n\to +\infty}\sin(n\pi) = 0$ e $b_n = \frac{\pi}{2} + 2n\pi$. Di nuovo, $\lim\limits_{n\to +\infty}b_n = +\infty$ ma questa volta $\lim\limits_{n\to +\infty}\sin(b_n) = \sin(\frac{\pi}{2} + 2n\pi) = 1$.
\end{example}
\begin{wrapfigure}[4]{l}{6cm}
    \vspace{-15pt}
    \centering
    \includegraphics[width=5cm]{images/esempio-dim-con-legame-succ-fun.png}
\end{wrapfigure}

Per il teorema concludo che non esiste il $\lim\limits_{x \to +\infty}\sin(x)$
In particolare il teorema implica che se $\lim\limits_{x\to +\infty}f(x) = l$, allora $\lim\limits_{n\to +\infty}f(n) = l$. Attenzione che non è vero il viceversa.\\\\
\begin{example}
$f(x) = \sin(x\pi)$. Abbiamo $f(n) = \sin(n\pi) = 0$. Quindi $\lim\limits_{n\to +\infty}f(n) = 0$, ma non esiste $\lim\limits_{x\to +\infty}\sin(x\pi)$.
\end{example}

\subsection{Calcolo dei limiti di successioni}
\begin{theorem}
Se abbiamo due successioni $a_n \to l$ e $b_n \to l'$ allora $a_n + b_n \to l + l'$, $a_n \cdot b_n \to l \cdot l'$, $\frac{a_n}{b_n} \to \frac{l}{l'}$ (se $l' \neq 0$ e $b_n \neq 0$ definitivamente), $a_n^{n^n} \to l^{l'}$ (se $l>0$ e $a_n > 0$ definitivamente), se $a_n = c \forall\: n \in \mathbb{N}$ allora $\lim\limits_{n\to +\infty}a_n = c$.
\end{theorem}
\hspace{-15pt}Questo teorema vale solo se supponiamo che non vengono forme indeterminate che sono le stesse viste con le funzioni.
\begin{theorem}
Se $f: A \to \mathbb{R}$ e $x_0 \in acc(A)$ e $\lim\limits_{x\to x_0}f(x) = l$ e $a_n: S \to A$ tale che $a_n \to x_0$ e $a_n \neq x_0$ definitivamente allora $\lim\limits_{n\to +\infty}f(a_n) = l$. In particolare se $\lim\limits_{x\to +\infty}f(x) = l$, allora $\lim\limits_{n\to +\infty}f(n) = l$.
\end{theorem}

\begin{example}
Alcuni esempi di calcolo dei limiti con successioni:
\begin{itemize}
    \item $\lim\limits_{n\to +\infty}(n^2 + 2n)$. Partendo da $\lim\limits_{n\to +\infty}n = +\infty$ troviamo $(\lim\limits_{n\to +\infty}n^2) + \lim\limits_{n\to +\infty}(2n) = (\lim\limits_{n\to +\infty}n)(\lim\limits_{n\to +\infty}n) + 2 \lim\limits_{n\to +\infty}n = (+\infty) \cdot (+\infty) + 2(+\infty) = +\infty$.
    \item $\lim\limits_{n\to +\infty}(n^2 - 2n) = +\infty - \infty$ possiamo fare $n^2 - 2n = n(n-2) \to +\infty \cdot (+\infty) = +\infty$. Si poteva anche dire $f(x) = x^2 - 2x$ visto che $\lim\limits_{x\to +\infty} (x^2 - 2x)=+\infty$ allora $\lim\limits_{n\to +\infty}f(n) = +\infty$.
    \item $\lim\limits_{n\to +\infty}\frac{n^2 - 2n}{n} = +\frac{+\infty}{+\infty}$ possiamo però fare $\frac{n^2 - 2n}{n} = n-2 \to +\infty$.
    \item $\lim\limits_{n\to +\infty}e^n$, consideriamo $f(x) = e^x$, so che $\lim\limits_{x\to +\infty}e^x = +\infty$ quindi $\lim\limits_{n\to +\infty}f(n) = +\infty$.
    \item $\lim\limits_{n\to +\infty}n\cdot \sin\frac{1}{n} = +\infty \cdot 0$, pongo $f(x) = x \cdot \sin\frac{1}{x}$ e calcoliamo $\lim\limits_{x\to +\infty}x\cdot \sin\frac{1}{x} = \frac{\sin\frac{1}{x}}{\frac{1}{x}}$ e $\frac{1}{x} \to 0$ quando $x\to +\infty$, poniamo $t = \frac{1}{x}$ e viene $\lim\limits_{x\to +\infty}\frac{\sin{t}}{t} = 1$.\\\\
    Altro modo utilizzando taylor: poniamo $\sin{t} = t + o(t)$ per $t\to 0$ sostituisco $t=\frac{1}{n}$ (infatti $\frac{1}{n}\to 0$ quando $n\to +\infty$), po $\sin{\frac{1}{n}} = \frac{1}{n} + o(\frac{1}{n})$ quindi $n\cdot \sin{\frac{1}{n}} = n \cdot (\frac{1}{n} + o(\frac{1}{n})) = 1 + o(1) \to 1$ per $x\to +\infty$.
\end{itemize}
\end{example}

\begin{observation}
$f(n)$ può avere limite anche se $f(x)$ non c'è l'ha infatti per esempio:
$f(x) = \sin{\pi x}$ non ha limite per $x\to +\infty$ ma $f(n) = \sin{\pi n}$ ha limite. \\
Quindi il metodo di utilizzare la funzione può non sempre funzionare.
\end{observation}

\begin{example}
Ci chiediamo se esiste $\lim\limits_{n\to +\infty}\sin(n)$. Vediamo che il limite non esiste:\\
Chiediamo quando $\sin(x) \geq \frac{1}{2}$ in $[0,\pi]$ succede esattamente per $x \in [\frac{\pi}{6},\frac{5}{6}\pi]$. L'intervallo ha lunghezza $\frac{5}{6}\pi - \frac{1}{6}\pi = \frac{4}{6}\pi = \frac{2}{8}\pi > 2$. Quindi l'intervallo contiene almeno due numeri interi (in $\mathbb{N}$) e lo stesso vale per tutti gli traslati di multipli di $2\pi$.\\
Questo ci permette di costruire una successione crescete $h_n$ di numeri naturali tale che $\sin(h_n) \geq \frac{1}{2} \: \forall \: n \in \mathbb{N}$.
Questo mi dice che se esiste $\lim\limits_{n\to +\infty}\sin(n) = l$, allora sicuramente $l \geq \frac{1}{2}$ (conseguenza della permanenza del segno). \\
Posso fare lo stesso discorso partendo da $\sin(x) \leq -\frac{1}{2}$, e trovo che $l \leq -\frac{1}{2}$. Questo è assurdo, e mi dimostra che non esiste $\lim\limits_{n\to +\infty}\sin(n)$.
\end{example}

\begin{example}
$\lim\limits_{n\to +\infty}n^2 \cdot \sin{n}$, ci chiediamo se esiste il limite.\\
Considerando la successione dell'esempio precedente $h_n$, troviamo una sottosuccessione $h_n^2 \cdot \sin(h_n)$, $\sin(h_n) \geq \frac{1}{2}$ quindi $h_n^2 \cdot \sin(h_n) \geq \frac{1}{2}\cdot h_n^2 \to +\infty$. Se $k_n$ è una successione di naturali tale che $\sin(k_n) \leq -\frac{1}{2} \forall \: n$, abbiamo una sottosuccessione $k_n^2 \cdot \sin(k_n) \leq -\frac{1}{2}kn^2 \to +\infty$. Quindi ho due sottosuccessioni di $n^2\sin{n}$ che hanno limiti diversi. Segue che non esiste $\lim\limits_{n\to +\infty}n^2 \cdot \sin{n}$.
\end{example}

\begin{theorem}
Sia $\{a_n\}_{n \in \mathbb{N}}$ (nota \footnote{In questa scrittura ci sono tutti i numeri naturali}) una successione,e $\{a_{h_n}\}$ e $\{a_{k_n}\}$ due sottosuccessioni tale che $\{h_n \: | \: n \in \mathbb{N}\} \cup \{k_n \: | \: n \in \mathbb{N}\} = \mathbb{N}$. (si dice che le due sottosuccessioni "saturano tutti gli indici").\\
Se $\exists \: \lim\limits_{n\to +\infty}\_{h_n}$ e $\exists \lim\limits_{n\to +\infty}a_{k_n}$ e sono uguali, allora esiste anche $\lim\limits_{n\to +\infty}a_n$ ed è uguale agli altri due.
\end{theorem}
\hspace{-15pt}Un caso tipico in cui si utilizza questo teorema è quando si prendono gli indici pari e dispari.
\begin{example}
$\Large{\lim\limits_{n\to +\infty}\frac{(\log{n+1})^{(-1)^h}}{n^3}}$. Guardiamo gli indici pari $k_n = 2n$ con il quale ho $\frac{(\log(2n+1))^1}{(2n)=3} \to 0$, e poi guardiamo gli indici dispari $h_n = 2n+1$ dove viene $\frac{(\log(2n+1))^{-1}}{(2n+1)^3} = \frac{1}{(2n+1)^3\log(2n+1)} = \frac{1}{+\infty} = 0$ quindi le sottosuccessioni saturano tutti gli indici.\\
Usando il teorema concludiamo che $\Large{\lim\limits_{n\to +\infty}\frac{(\log{n+1})^{(-1)^h}}{n^3}} = 0$.
\end{example}

\subsubsection{Criterio del rapporto}
\begin{theorem}[Criterio del rapporto]
Sia $\{a_n\}$ una successione.  Se $a_n > 0$ definitivamente, e se esiste $\lim\limits_{n\to +\infty}\frac{a_{n+1}}{a_n} = l$ allora:
\begin{enumerate}
    \item Se $0 \leq l \leq 1$, allora $\lim\limits_{n\to +\infty} a_n = 0$.
    \item Se $l > 1$, allora $\lim\limits_{n\to +\infty}a_n = +\infty$.
\end{enumerate}
\end{theorem}

\begin{observation}
Se $l = 1$, non si può dire niente sul comportamento di $a_n$.
\end{observation}

\begin{example}
Esempi del criterio del rapporto con $l=1$:
\begin{itemize}
    \item Prendo $a_n = 1 \forall \: n \in \mathbb{N}$. Allora $\frac{a_{n+1}}{a_n} = \frac{1}{1} = 1 \to 1$ quindi $l=1$ e $a_n$ converge a 1.
    \item Con $a_n = n$. Allora $\frac{a_{n+1}}{a_n} = \frac{n+1}{n} \to 1$ quindi $l=1$ e $a_n \to +\infty$.
    \item Con $a_n = \frac{1}{n}$, di nuovo $\frac{a_{n+1}}{a_n} = \frac{n}{n+1} \to 1$ sempre $l=1$ e $a_n \to 0$.
\end{itemize}
\end{example}

\begin{example}
Esempi di applicazioni del criterio del rapporto:
\begin{itemize}
    \item $a_n = (\frac{1}{2})^n$. Usando il criterio $\frac{a_{n+1}}{a_n} = \frac{(\frac{1}{2})^{n+1}}{(\frac{1}{2})^n} = \frac{2^n}{2^{n+1}} = \frac{1}{2} = l$ e ho $0 \leq l \leq 1$. Quindi $a_n \to 0$.
    \item $a_n = 2^n$ (si può usare $f(x) = 2^x$ e il fatto che $\lim\limits_{x\to +\infty}2^x = +\infty$). Usiamo il criterio del rapporto quindi $\frac{a_n+1}{a_n} = \frac{2^{n+1}}{2^n} = 2 = l$ e con $l>2$ concludo che $a_n \to +\infty$.
    \item $a_n = n!$. Criterio del rapporto che $\frac{a_{n+1}}{a_n} = \frac{(n+1)!}{n!} = \frac{(n+1)n!}{n!} = n+1 \to +\infty = l$ quindi $l>1$ e quindi $a_n \to +\infty$. In questo caso si poteva anche osservare che $n! > n$ e $n \to +\infty$ quindi per confronto segue che $n! \to +\infty$.
\end{itemize}
\end{example}

\hspace{-15pt}Confronto di $n!$ con $n^k$, $b^n$, $n^n$:
\begin{itemize}
    \item \textbf{Potenza} ($n^k$) con ($k\geq 1$). Vogliamo guardare che $\lim\limits_{n\to +\infty}\frac{n!}{n^k} = \frac{+\infty}{+\infty}$ forma indeterminata.\\
    Usiamo quindi il criterio del rapporto per $a_n = \frac{n!}{n^k}$. Ho $\frac{a_{n+1}}{a_n} = \frac{(n+1)!}{(n+1)^k} \cdot \frac{n^k}{n!} = \frac{(n+1)!}{n!} \cdot \frac{n^k}{(n+1)^k} = (n+1) \cdot (\frac{n}{n+1})^k \to +\infty \cdot (1)^k = +\infty = l$ quindi $l > 1$.\\
    Segue che $\frac{n!}{n^k} \to +\infty$, quindi $n!$ "tende a $+\infty$ più velocemente di $n^k$".
    \item \textbf{Esponenziale} ($b^n$) con $b > 1$. $\lim\limits_{n\to +\infty} \frac{n!}{b^n} = \frac{+\infty}{+\infty}$ forma indeterminata. Guardiamo quindi il rapporto, per $a_n = \frac{n!}{b^n}$. Quindi abbiamo $\frac{a_{n+1}}{a_n} = \frac{(n+1)!}{b^{n+1}} \cdot \frac{b^n}{n!} = \frac{(n+1)!}{n!} \cdot \frac{b^n}{b^{n+1}} = (n+1)\cdot \frac{1}{b} \to +\infty = l > 1$. Segue che $\frac{n!}{b^n} \to +\infty$, quindi $n!$ tende a $+\infty$ più velocemente di $b^n$.
    \item \textbf{Esponenziale potentissimo} ($n^n$). Notare che $n^n \to +\infty$ ad esempio perché $n^n \geq n$ e $n\to +\infty$. Facciamo $\lim\limits_{n\to +\infty}\frac{n^n}{n!} = \frac{+\infty}{+\infty}$ forma indeterminata. Usiamo il criterio del rapporto per $a_n = \frac{n^n}{n!}$. $\frac{a_{n+1}}{a_n} = \frac{(n+1)^{n+1}}{(n+1)!}\cdot \frac{n!}{n^n} = \frac{n!}{(n+1)!} \cdot \frac{(n+1)^{n+1}}{n^n} = \frac{1}{n+1} \cdot \frac{(n+1)^{n+1}}{n^n} = (\frac{n+1}{n})^n = (1 + \frac{1}{n})^n$ che è un limite notevole che $\to e > 1$. (per vederlo ad esempio si può scrivere $(1 + \frac{1}{n})^n = e^{\log(1+\frac{1}{n})^n} = e^{n\cdot \log(1+\frac{1}{n})} = e^{n\cdot(\frac{1}{n} + o(\frac{1}{n}))} = e^{1 + o(1)} \to e^1 = e$). Quindi segue che $\frac{n^n}{n!} \to +\infty$ quindi $n^n$ tende a $+\infty$ più velocemente di $n!$.
\end{itemize}


\subsubsection{Criterio della radice}
\begin{theorem}
Se $a_n > 0$ definitivamente, e $\exists \lim\limits_{n\to +\infty}\sqrt[n]{a_n} = l$, allora:
\begin{enumerate}
    \item Se $0 \leq l < 1$, allora $\lim\limits_{n\to +\infty} a_n = 0$.
    \item Se $l > 1$, allora $\lim\limits_{n\to +\infty}a_n = +\infty$.
\end{enumerate}
\end{theorem}

\begin{observation}
Se $l=1$ non si può dire niente riguardo al comportamento di $a_n$ come sul criterio del rapporto.
\end{observation}

\begin{demostration}
Dimostrazione dei due casi del criterio della radice.
\begin{enumerate}
    \item Suppongo che $0 \leq l \leq 1$ e fisso un $m\in \mathbb{R}$ tale che $l < m < 1$. Visto che $\sqrt[n]{a_n}\to l$ definitivamente avrò $\sqrt[n]{a_n} < m$, quindi $a_n < m^n$. Ora visto che $m < 1$ abbiamo visto che $m^n \to 0$, quindi visto che $0 < a_n < m^n$ per il teorema dei carabinieri segue che $a_n \to 0$.
    \item Questo punto si fa analogo, se invece $l > 1$ scelto $m \in \mathbb{R}$ tale che $1 < m < l$. Visto che $\sqrt[n]{a_n} \to l$ avrò $\sqrt[n]{a_n} > m$ definitivamente segue che definitivamente ho $a_n > m^n$ e visto che $n > 1$ ho $m^n \to +\infty$. Per confronto segue che $a_n \to +\infty$. $\blacksquare$
\end{enumerate}
\end{demostration}

\subsubsection{Relazione fra criteri del rapporto e della radice}
\begin{theorem}[Relazione fra rapporto e radice]
Se $a_n > 0$ definitivamente e se $\exists \lim\limits_{n\to +\infty}\frac{a_{n+1}}{a_n} = l$, allora $\exists \lim\limits_{n\to +\infty}\sqrt[n]{a_n}$ ed è uguale a l.
\end{theorem}

\begin{observation}
Questo teorema è vero anche con $l=1$.
\end{observation}

\begin{observation}
Potrebbe esiste $\lim\limits_{n\to +\infty}\sqrt[n]{a_n}$ e non esistere il $\lim\limits_{n\to +\infty} \frac{a_{n+1}}{a_n}$ (quindi questo teorema vale solo per un verso e non il viceversa).
\end{observation}

\begin{example}
Alcuni esempi utilizzando quest'ultimo teorema.
\begin{itemize}
    \item Fissiamo un $a > 0$. Proviamo a calcolare $\lim\limits_{n\to +\infty} \sqrt[n]{a}$. (Si può fare in diversi modi come $\sqrt[n]{a} = a^{\frac{1}{n}} \to a^0 = 1$).
    Usiamo l'ultimo teorema $a_n = a$ successione costante. Abbiamo quindi $\frac{a_{n+1}}{n} = \frac{a}{a} = 1$. Per il teorema segue che $\sqrt[n]{a_n} = \sqrt[n]{a} = 1$.
    \item Proviamo a fare $\lim\limits_{n\to +\infty}\sqrt[n]{n}$. usiamo il teorema con $a_n = n$. Abbiamo quindi $\frac{a_{n+1}}{a_n} = \frac{n+1}{n} \to 1$. Quindi segue che $\sqrt[n]{a_n} = \sqrt[n]{n} \to 1$.
    \item Nello stesso modo dell'esempio sopra si vede che $\sqrt[n]{p(n)}\to 1$ dove $p(n)$ è un polinomio in n.
\end{itemize}
\end{example}

\begin{example}
Esiste $\lim\limits_{n\to +\infty}\sqrt[n]{a_n}$ ma non esiste $\lim\limits_{n\to +\infty}\frac{a_{n+1}}{a_n}$.
Prendiamo $a_n = \begin{cases}1 & \text{ se n è pari} \\ 2 & \text{ se n è dispari }\end{cases}$\\\\
Abbiamo $\sqrt[n]{1} \leq \sqrt[n]{a_n} \leq \sqrt[n]{2} \: \forall \: n \in \mathbb{N}$. Abbiamo appena visto che $\sqrt[n]{1} \to 1$ e $\sqrt[n]{2} \to 1$, per il teorema dei carabinieri segue che $\sqrt[n]{2} \to 1$.\\\\
Ora $\frac{a_{n+1}}{a_n} = \begin{cases}\frac{2}{1}=2 & \text{ se n è pari} \\ \frac{1}{2} & \text{ se n è dispari }\end{cases}$ \hspace{.5cm} e questa successione non ha limite.
\end{example}

\begin{example}
Calcoliamo $\lim\limits_{n\to +\infty}\sqrt[n]{2 + \sin{n}}$. Usare il rapporto non sembra promettente perché se $a_n = 2 + \sin{n}$, sarebbe $\frac{a_{n+1}}{a_n} = \frac{2 + \sin{n+1}}{2 + \sin{n}}$. Visto che $-1 \leq \sin{n} \leq 1$ abbiamo che $\sqrt[n]{1} \leq \sqrt[n]{2 + \sin{n}} \leq \sqrt[n]{3}$, in questo caso sia $\sqrt[n]{1} \to 1$ che $\sqrt[n]{3} \to 1$ quindi per il teorema dei carabinieri, il limite è 1.
\end{example}

\hspace{-15pt}In riferimento all'esempio di prima possiamo dire più in generale, che se $a_n$ è limitata $m \leq a_n \leq M$ definitivamente (definitivamente limitata), con $m > 0$ allora ho $\sqrt[n]{m} \leq \sqrt[n]{a_n} \leq \sqrt[n]{M}$ e come sopra visto che $\sqrt[n]{1} \to 1$ e $\sqrt[n]{3} \to 1$ concludo che $\sqrt[n]{a_n} \to 1$.

\begin{example}
$\lim\limits_{n\to +\infty}\sqrt[n]{n!}$, pongo $a_n = n!$ e $\frac{a_{n+1}}{a_n} = \frac{(n+1)!}{n!} = n+1 \to +\infty$. Dall'ultimo teorema visto segue che $\sqrt[n]{a_n} = \sqrt[n]{n!} \to +\infty$.
\end{example}
\end{document}
