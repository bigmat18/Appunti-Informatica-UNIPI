\newpage
\section{Successioni}
\begin{definition}[Successione]
Una successione\footnote{Nelle successioni si è soliti scrivere n al posto di x come simbolo per la variabile ess. $f(n)$} è una funzione $f: S\to \mathbb{R}$ dove S è una semiretta di $\mathbb{N}$, cioè $S = \{n \in \mathbb{R}\:|\: x\geq n_0\}$ per qualche $n_0$.
\end{definition}

\begin{example}
Consideriamo $f(n) = n^2$ con $S = \mathbb{N}$ da questa funzione posso calcolare tutti i valori: $f(0) = 0^2 = 0$, $f(1) = 1^2 = 1$, $f(2) = 2^2 = 4$
\end{example}

\hspace{-15pt}E possibile disegnare un grafico di una successione che è composto da una serie di punti sparsi.

\begin{example}
$f(n) = \frac{1}{n}$, come S non posso prendere tutti i naturali perché con 0 non ha senso quindi $S = \{n \in \mathbb{N} \: |\: n \geq 1\}$.  $f(1) = \frac{1}{1}=1$, $f(2) = \frac{1}{2}$, $f(3) = \frac{1}{3}$.
\end{example}

\subsection{Notazione}
Nelle successioni invece di scrivere $f(n)$ di solito una successione si denota con $a_n$. Negli esempio di prima si sarebbe: $a_n = n^2$, $a_n = \frac{1}{n}$.\\
L'intera successione si denota con $\{a_n\}$ oppure $\{a_n\}_{n\in \mathbb{N}}$, $\{a_n\}_{n\in S}$.
\begin{example}
$a_n = \frac{1}{n-5}$. La formula ha senso per $n\neq 5$, quindi si può prendere $S = \{n \in \mathbb{N} \:|\: n\geq 6\}$ (avrei anche potuto prendere $n \geq 7$ o $n \geq 8$).
\end{example}

\begin{example}
$a_n = \sqrt{5 - n}$. La formula ha senso se $5-n \geq 0$ cioè $n \leq 5$. Nessuna semiretta va bene perché in una successione n diventa sicuramente più grande ad un certo punto quindi non definisce una successione.
\end{example}

\subsection{Limiti di Successioni}
Come per le funzioni bisogna guardare come si comporta la successioni all'avvicinarsi ad un limite. L'unico limite che ha senso è il limite per $n\to +\infty$, perché $+\infty$ è l'unico punto di accumulazione di tutto il dominio (perché $S \subseteq \mathbb{N}$).
\begin{definition}[Limite di successione]
Si ha che $\lim\limits_{n\to +\infty}a_n = l$ se $\forall \:\: U$ intorno di l si ha che $\exists \: \overline{n}\in \mathbb{N}$ tale che $a_n \in U \:\: \forall \: n\geq \overline{n}$.\\
Si dice che $a_n$ converge a $l$ se $\lim\limits_{n\to +\infty}a_n = l$ e $l\in \mathbb{R}$ e che diverge a $\pm \infty$ se $\lim\limits_{n\to +\infty} = \pm \infty$.
\end{definition}

\hspace{-15pt}Graficamente se il limite è in $\mathbb{R}$ quindi $l \in \mathbb{R}$, e con $l = +\infty$.\\
Esiste una \textbf{Terminologia} quando si parla di queste cose: se $P(n)$ è un predicato la cui verità dipende da $n\in \mathbb{N}$ (esempio: $P(n) =$ "n è pari") si dice che $P(n)$ è vero definitivamente se $\exists \: \overline{n}\in \mathbb{N}$ tale che $P(n)$ è vero $\forall n \geq \overline{n}$.\\
Quindi $\lim\limits_{n\to +\infty} a_n = l$ se $\forall \: U$ introno di l si ha che $a_n \in U$ definitivamente.

\subsection{Sottosuccessioni (estratte)}
\begin{definition}[Sottosuccessione]
Dato $a_n: S \to \mathbb{R}$ una successione, consideriamo $k_n: \mathbb{N} \to S$ strettamente crescente (cioè $k_n > k_m$ quando $n>m$), possiamo considerare la composizione $a_{k_n}$. Questa è una nuova successione detta sottosuccessione di $\{a_n\}$ (In pratica scegliamo solo un certo sottoinsieme di indici, in modo crescente).
\end{definition}

\begin{example}
$a_n = \frac{1}{n}$. Per avere una sottosuccessione prendo $k_n: \mathbb{N}\to S$,e prendo $n \mapsto 2n+1$. Abbiamo $a_{k_n} = \frac{1}{k_n} = \frac{1}{2n+1}$. Quindi graficamente:
\end{example}

\begin{theorem}
Data una successione $\lim\limits_{n\to +\infty}a_n = l$ se e solo se vale $\lim\limits_{n\to +\infty}a_{k_n} = l    $ per ogni sottosuccessione di $\{a_n\}$.
\end{theorem}
\hspace{-15pt} A volta si può usare per dimostrare che una successione non ha limite.
\begin{example}
$a_n= (-1)^h = \begin{cases}-1 & \text{se n è pari} \\ -1 & \text{se n è dispari} \end{cases}$ \\\\
Questo successione non ha limite e si dimostra con il teorema visto sopra. Infatti, consideriamo le sottosuccessioni $\{a_{2n}\}$ e $\{a_{2n+1}\}$ date da indici pari e dispari. \\
Abbiamo che $a_{2n} = (-1)^{2n} = (1)^n = 1$ che converge a 1 mentre, $a_{2n+1} = (-1)^{2n+1} = -1$ e quindi converge a -1. Visto che questi limiti esistono e sono diversi, segue dal teorema che $\{a_n\}$ non può avere limite.
\end{example}

\begin{observation}
Per i limiti di successioni valogono molti dei teoremi visti per le funzioni, ad esepio:
\begin{itemize}
    \item Formule per limiti di somme, prodotti, quozienti, esponenziali etc.
    \item Teorema di permanenza del segno.
    \item Teorema dei carabinieri.
    \item Teorema del confronto, ed altri...
\end{itemize}
\end{observation}

\begin{example}
Per esempio il teorema della permanenza del segno per le successioni dice: se abbiamo una successione che $\lim\limits_{n\to +\infty} a_n = l > 0$, allora $a_n > 0$ definitivamente.
\end{example}

\subsection{Monotonia}
\begin{definition}[Monotonia]
Una successione $\{a_n\}$ essa si dice:
\begin{itemize}
    \item \textbf{Debolmente crescente} se $n>m \Longrightarrow a_n \geq a_n$.
    \item \textbf{Strettamente crescente} se $n > m \Longrightarrow a > a_m$.
    \item \textbf{Debolmente decrescente} se $n > m \Longrightarrow a_n \leq a_m$.
    \item \textbf{Strettamente decrescente} se $n > m \Longrightarrow a_n < a_m$.
\end{itemize}
Successione è monotona quando vale una di queste 4 proprietà.
\end{definition}

\begin{observation}
$\{a_n\}$ è debolmente crescente se e solo se vale $a_{n+1} \geq a_n \forall \: n \in S$ (basta guardare termini successivi).\\
Infatti, se so che $a_{n+1} \geq a_n \forall \: n \in \mathbb{N}$, poi se $n > m$ allora $a_n \geq ... \geq a_{m+2} \geq a_{m+1} \geq a_{m}$.
\end{observation}

\begin{example}
Prendiamo $a_n=n^2$ e controlliamo che è strettamente crescente: vediamo che $a_{n+1} > a_n$. Infatti $a_{n+1} = (n+1)^2 = n^2 + 2n + 1$ e $a_n = n^2$ e quindi $n^2 + 2n + 1 > n^2 \Longleftrightarrow 2n+1 > 0$ che è vero $\forall \:n \in \mathbb{N}$.
\end{example}

\begin{theorem}
Se $\{a_n\}$ è monotona (cioè debolmente crescente o decrescente) allora ammette limite.
Se è debolmente crescente, il limite non può essere $-\infty$ e se Se è debolmente decrescente, il limite non può essere $+\infty$
\end{theorem}

\subsection{Limitatezza}
\begin{definition}[Limitatezza]
Una successione $\{a_n\}$ è \textbf{limitata superiormente} se $\exists\: M \in \mathbb{R}$ tale che $a_n \subseteq M \:\forall\: \in S$ e \textbf{limitata inferiormente} se $\exists \:m \in \mathbb{R}$ tale che $a_n \geq m \forall \: n \in S$ e \textbf{limitata} se è limitata sia inferiormente e superiormente.
\end{definition}

\begin{observation}
Una successione convergente (che ha limite finito) è limitata. Questo non è vero per funzioni di variabile reale.
\end{observation}

\begin{example}
$f(x) = \frac{1}{x}$, $f: (0,+\infty)\to \mathbb{R}$ abbiamo $\lim\limits_{x\to +\infty}f(x) = 0$ ma f non è limitata, perché $\lim\limits_{x\to 0^+}f(x) = +\infty$ però $a_n = \frac{1}{n}$ invece è limitata.
\end{example}

\begin{theorem}
Se $\lim\limits_{n\to +\infty}a_n = +\infty$, allora $\{a_n\}$ ha minimo (cioè $\exists \:n_{min} \in \mathbb{N}$ tale che $a_n \geq a_{n_{min}} \: \forall \:n \in S$). Se invece $\lim\limits_{n\to +\infty}a_n = -\infty$ allora $a_n$ ha massimo.
\end{theorem}
\hspace{-15pt}Ci si può chiedere come domanda se una successione $\{a_n\}$ è limitata, necessariamente massimo e minimo? La risposte è no.
\begin{example}
Se prendiamo $a_n = \frac{1}{n}$ è limitata: $1 \geq \frac{1}{n} > 0$ ma non ha minimo. $max\{a_n\} = 1$ e $inf\{a_n\} = 0$ (uguale a $\lim\limits_{n\to +\infty} a_n$). Non ha minimo perché non esiste $n \in \mathbb{N}$ tale che $\frac{1}{n} = 0$
\end{example}
\hspace{-15pt}Inoltre è possibile chiedersi se $\{a_n\}$ è limitata, esiste almeno uno tra massimo minimo? E la risposta anche in questo caso è no.
\begin{example}
Prendiamo $a_n = (1-\frac{1}{n})(-1)^n = \begin{cases}1-\frac{1}{n} & \text{per n pari} \\ -(1 - \frac{1}{n}) & \text{per n dispari}\end{cases}$\\\\
Complessivamente possiamo vedere la la successione oscilla avvicinandosi con $sun\{a_n\} = 1$ e $inf\{a_n\} = -1$, e non esistono massimo e minimo, anche se $a_n$ è limitata, visto che $-1 < a_n < 1$.
\end{example}

\begin{example}
Prendiamo $a_n = \frac{(-1)^n}{n}$ e ci chiediamo se ha limite e sa ha massimo e o minimo.\\
Abbiamo ceh $\lim\limits_{n\to +\infty} = 0$. Infatti abbiamo che $-\frac{1}{n} \leq a_n \leq \frac{1}{n}$ e visto che $\lim\limits_{n\to +\infty}-\frac{1}{n}=\lim\limits_{n\to +\infty}\frac{1}{n} = 0$ per il teorema dei carabinieri abbiamo che $\lim\limits_{n\to +\infty}a_n = 0$. Quindi ha massimo e minimo il massimo è in $n=2$ ed il minimo in $n=1$.
\end{example}

\begin{theorem}
Se ho usa successione che converge $\lim\limits_{n\to +\infty}a_n = l$ finito allora:
\begin{itemize}
    \item $\exists \: \overline{n}\in \mathbb{N}$ tale che $a_{\overline{n}} \geq l \Longrightarrow \{a_n\}$ ha massimo.
    \item $\exists \: \overline{n}\in \mathbb{N}$ tale che $a_{\overline{n}} \leq l \longrightarrow \{a_n\}$ ha minimo.
\end{itemize}
\end{theorem}

\subsection{Legame tra limiti di funzione e successioni}
\begin{theorem}
Prendiamo una funzione definita in $A \subseteq \mathbb{R}$ sottoinsieme $f:A \to \mathbb{R}$, e $x_0 \in acc(A)$. Allora abbiamo che $\lim\limits_{x\to x_0}f(x) = l$ se e solo se $\lim\limits_{n\to +\infty}f(a_n) = l$ per ogni successione $\{a_n\}\subseteq A$ tale che $\lim\limits_{n\to +\infty}a_n = x_0$ e $a_n \neq x_0$ definitivamente.
\end{theorem}

\hspace{-15pt}Questo teorema a volte si può utilizzare per dimostrare che non esiste $\lim\limits_{x\to x_0}f(x)$.
\begin{example}
Dimostriamo che non esiste $\lim\limits_{x\to +\infty}\sin(x)$.\\
Esibiamo due successioni $a_n, b_n$ che tendono a $+\infty$,  tali che $\lim\limits_{n\to +\infty}\sin(a_n)$ e $\lim\limits_{n\to +\infty}\sin(b_n)$ esistono, ma sono diversi. \\
Prendo $a_n = n\pi$. Abbiamo $\lim\limits_{a\to +\infty}a_n = n\pi = +\infty$. Inoltre $\lim\limits_{n\to +\infty}\sin(a_n)= \lim\limits_{n\to +\infty}\sin(n\pi) = 0$ e $b_n = \frac{\pi}{2} + 2n\pi$. Di nuovo, $\lim\limits_{n\to +\infty}b_n = +\infty$ ma questa volta $\lim\limits_{n\to +\infty}\sin(b_n) = \sin(\frac{\pi}{2} + 2n\pi) = 1$.\\
Per il teorema concludo che non esiste il $\lim\limits_{x \to +\infty}\sin(x)$
\end{example}

\hspace{-15pt}In particolare il teorema implica che se $\lim\limits_{x\to +\infty}f(x) = l$, allora $\lim\limits_{n\to +\infty}f(n) = l$. Attenzione che non è vero il viceversa.
\begin{example}
$f(x) = \sin(x\pi)$. Abbiamo $f(n) = \sin(n\pi) = 0$. Quindi $\lim\limits_{n\to +\infty}f(n) = 0$, ma non esiste $\lim\limits_{x\to +\infty}\sin(x\pi)$.
\end{example}