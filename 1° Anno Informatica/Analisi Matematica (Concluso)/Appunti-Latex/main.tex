\documentclass[a4paper,10pt]{article}
\usepackage[utf8]{inputenc}

% ----  Useful packages % ---- 
\usepackage{amsmath}
\usepackage{graphicx}
\usepackage{amsfonts}
\usepackage{amsthm}
\usepackage{amssymb}
\usepackage{makecell}
% ----  Useful packages % ---- 

\usepackage{wrapfig}
\usepackage{caption}
\usepackage{subcaption}
\usepackage{xcolor}
\usepackage{hyperref}
\hypersetup{
    colorlinks,
    citecolor=black,
    filecolor=black,
    linkcolor=black,
    urlcolor=black
}

\graphicspath{ {./images/} }

% ---- Set page size and margins replace ------
\usepackage[letterpaper,top=2cm,bottom=2cm,left=3cm,right=3cm,marginparwidth=1.75cm]{geometry}
% ---- Set page size and margins replace ------

% ------- NOTA ------
\theoremstyle{remark}
\newtheorem{note}{Note}[subsection]
% ------- NOTA ------

% ------- OSSERVAZIONE ------
\theoremstyle{definition}
\newtheorem{observation}{Osservazione}[subsection]
% ------- OSSERVAZIONE ------

% ------- DEFINIZIONE ------
\theoremstyle{plain}
\newtheorem{definition}{Definizione}[subsection]
% ------- DEFINIZIONE ------

% ------- ESEMPIO ------
\theoremstyle{definition}
\newtheorem{example}{Esempio}[subsection]
% ------- ESEMPIO ------

% ------- DIMOSTRAZIONE ------
\theoremstyle{definition}
\newtheorem{demostration}{Dimotrazione}[subsection]
% ------- DIMOSTRAZIONE ------

% ------- TEOREMA ------
\theoremstyle{definition}
\newtheorem{theorem}{Teorema}[subsection]
% ------- TEOREMA ------

% ------- COROLLARIO ------
\theoremstyle{plain}
\newtheorem{corollaries}{Corollario}[theorem]
% ------- COROLLARIO ------

% ------- PROPOSIZIONE ------
\theoremstyle{plain}
\newtheorem{proposition}{Proposizione}[subsection]
% ------- PROPOSIZIONE ------

% ---- Footer and header ---- 
\usepackage{fancyhdr}
\pagestyle{fancy}
\fancyhf{}
\fancyhead[LE,RO]{A.A 2022-2023}
\fancyhead[RE,LO]{Analisi Matematica}
\fancyfoot[RE,LO]{\rightmark}
\fancyfoot[LE,RO]{\thepage}

\renewcommand{\headrulewidth}{.5pt}
\renewcommand{\footrulewidth}{.5pt}
\newcolumntype{P}[1]{>{\centering\arraybackslash}p{#1}}
% ---- Footer and header ---- 

% ----  Language setting ---- 
\usepackage[italian, english]{babel}
% ----  Language setting ---- 

\title{\textbf{Analisi Matematica}}
\author{Realizzato da: Giuntoni Matteo e Ghirardini FIlippo}
\date{A.A. 2022-2023}

\begin{document}
\begin{titlepage} %crea l'enviroment
	\begin{figure}[t] %inserisce le figure
		\centering\includegraphics[width=0.98\textwidth]{marchio_unipi_pant541.png}
	\end{figure}
	\vspace{20mm}
	
	\begin{Large}
		\begin{center}
			\textbf{Dipartimento di Informatica\\ Corso di Laurea Triennale in Informatica\\}
			\vspace{20mm}
			{\LARGE{Corso 3° anno - 6 CFU}}\\
			\vspace{10mm}
			{\huge{\bf Ingegneria del Software}}\\
		\end{center}
	\end{Large}
	
	
	\vspace{36mm}
	%minipage divide la pagina in due sezioni settabili
	\begin{minipage}[t]{0.47\textwidth}
		{\large{\bf Professore:}\\ \large{Prof. Jacopo Soldani}}
	\end{minipage}
	\hfill
	\begin{minipage}[t]{0.47\textwidth}\raggedleft
		{\large{\bf Autore:}\\ \large{Filippo Ghirardini}}
	\end{minipage}
	
	\vspace{25mm}
	
	\hrulefill
	
	\vspace{5mm}
	
	\centering{\large{\bf Anno Accademico 2024/2025}}
	
\end{titlepage}

\tableofcontents
\newpage
\maketitle
\begin{center}
    \vspace{-20pt}
    \rule{11cm}{.1pt} 
\end{center}

% !TeX spellcheck = it_IT
\section{Introduzione}
\subsection{Sistemi di equazioni}
L'algebra lineare è lo studio delle soluzioni di sistemi di equazioni lineari utilizzando spazi vettoriali.
\begin{example}
Un esempio di sistemi di equazioni:
\begin{enumerate}
    \item
    $\begin{rcases*}
    	E_1: x + y = 5 \\ E_2: x + 2y = 6
    \end{rcases*}
	\Rightarrow E_2 - E_1$ (sostituzione):
	$\begin{cases}
		y = 5 - 3 = 2 \\ x = 3 - 2 = 1
	\end{cases}$ 
	Un unica soluzione.
    \item 
    $\begin{rcases*}
    	E_1: x + y = 3 \\ E_2: 2x + 2y = 6
    \end{rcases*}
	\Rightarrow E_2 - 2E_1$: 0 = 0.\\\\
    Infatti $E_2 = 2E_1 \Rightarrow$ hanno le stesse soluzioni $\Rightarrow$ $\exists \infty$ soluzioni.
    \item 
    $\begin{rcases*}
    	E_1: x + y = 3 \\ E_2: 2x + 2y = 5
    \end{rcases*}
	\Rightarrow E_2 - 2E_1: 0 = -1$ è impossibile infatti $\nexists$ soluzioni comuni.
\end{enumerate}
Possiamo vedere da questi esempi che abbiamo tre possibili risultati: 1 soluzione, $\infty$ e 0.
\end{example}

\subsection{Interpretazioni geometrica}
In ogni caso le equazioni $E_1$ ed $E_2$ rappresentano rette su un piano a 2 dimensioni. Le soluzioni comuni sono i punti di intersezione delle rette. \\Nel caso specifico dell'esempio 1.1.1 abbiamo che:
\begin{figure}[h!]
    \centering
    \begin{subfigure}{.3\textwidth}
        \centering
        \includegraphics[width=3cm]{images/rette-incidenti.png}
        \caption{1° hanno un punto in comune P=(1,2)}
    \end{subfigure}
    \hfill
    \begin{subfigure}{.3\textwidth}
        \centering
        \includegraphics[width=3cm]{images/rette-coincidenti.png}
        \caption{2° coincidono $\Rightarrow \infty$ punti in comune}
    \end{subfigure}
    \hfill
    \begin{subfigure}{.3\textwidth}
        \centering
        \includegraphics[width=3cm]{images/rette-parallele.png}
        \caption{3° sono parallele  $\Rightarrow \nexists$ punti in comune}
    \end{subfigure}
\end{figure}
\newpage
\subsection{Equazioni a 3 variabili}
Un esempio di equazione a 3 variabili è $x + 2y + 3z = 4$. Ciò crea, invece di una retta, un piano nello spazio 3-dimensionale.
Se adesso consideriamo le equazioni viste sopra $E_1$ ed $E_2$ come equazioni a 3 variabili possiamo vedere che esse corrispondono a 2 piani nello spazio ed i punti in comune formano una retta.\\
\begin{figure}[h!]
    \centering
    \begin{subfigure}{.3\textwidth}
        \centering
        \includegraphics[width=2.5cm]{images/piani-incidenti.png}
        \caption{1° forma una retta}
    \end{subfigure}
    \begin{subfigure}{.3\textwidth}
        \centering
        \includegraphics[width=2.5cm]{images/piani-coincidenti.png}
        \caption{2° i due piani coincidono}
    \end{subfigure}
    \begin{subfigure}{.3\textwidth}
        \centering
        \includegraphics[width=2.5cm]{images/piani-incidenti.png}
        \caption{3° i due piani sono paralleli}
    \end{subfigure}
\end{figure}
Se oltre a $E_1$ ed $E_2$ consideriamo una terza equazione $E_3$ essa corrisponde ad un terzo piano. \\
Possiamo vedere come esso si comporta intersecandolo con l'intersezione fra $E_1$ ed $E_2$, $E_1 \cap E_2$.
\begin{figure}[h!]
    \centering
    \begin{subfigure}{.3\textwidth}
        \centering
        \includegraphics[width=2.5cm]{piano-incotra-retta.png}
        \caption{$E_1 \cap E_2$ è una retta che, intersecata con $E_3$, crea un punto}
    \end{subfigure}
    \begin{subfigure}{.3\textwidth}
        \centering
        \includegraphics[width=2.5cm]{piano-coincide-retta.png}
        \caption{$E_1 \cap E_2$ può essere contenuto in $E_3$ quindi nuova retta}
    \end{subfigure}
    \begin{subfigure}{.3\textwidth}
        \centering
        \includegraphics[width=2.5cm]{piano-non-conindice-retta.png}
        \caption{$E_1 \cap E_2$ e $E_3$ possono non coincidere}
    \end{subfigure}
\end{figure}

\subsection{Caso generale}
Possiamo definire un sistema $(E)$ di $n$ equazioni a $m$ variabili con $n,m > 0$ e con $a_{nm}, b_{n} \in \mathbb{R}$ come:
\begin{flalign}\nonumber
&E_1: a_{11}x_1 + a_{12}x_2 + \ldots + a_{1m}x_m = b_1&&\\\nonumber
&E_2: a_{21}x_1 + a_{22}x_2 + \ldots+ a_{2m}x_m = b_2&&\\\nonumber
&\vdots&&\\
&E_n: a_{n1}x_1 + a_{n2}x_2 + \ldots + a_{nm}x_m = b_n&&\nonumber
\end{flalign}

\begin{definition}[Sistema omogeneo]
Il sistema $(E)$ è \textbf{omogeneo} se $b_1 = \ldots = b_n = 0$. In caso contrario possiamo considerare il sistema omogeneo associato ($E_{om}$) definito come:
\begin{flalign}\nonumber
&E_1: a_{11}x_1 + a_{12}x_2 + \ldots + a_{1m}x_m = 0&&\\\nonumber
&E_2: a_{21}x_1 + a_{22}x_2 + \ldots + a_{2m}x_m = 0&&\\\nonumber
&\vdots&&\\\nonumber
&E_n: a_{n1}x_1 + a_{n2}x_2 + \ldots + a_{nm}x_m = 0&&\nonumber
\end{flalign}
Se $(E)$ è \textbf{omogeneo}, $\exists$ sempre una soluzione comune del tipo $(x_1, \ldots, x_n) = (0_1, \ldots, 0_n)$.
\end{definition}

\begin{proposition}\label{prop-1}
Se $(c_1, ..., c_n)$ e $(d_1, ..., d_n)$ sono soluzioni di $(E)$ $\Longrightarrow$ $c_1 - d_1, ..., c_n - d_n$ è soluzione del sistema omogeneo.
\end{proposition}

\begin{demostration}
Se $(c_1, \ldots, c_m)$ è soluzione vuol dire che :\\
$E_1:a_{i1}c_1 + a_{i2}c_2 + a_{im}c_m = b_i$\\
$E_2:a_{i1}d_1 + a_{i2}d_2 + a_{im}d_m = b_i$\\
Quindi se sottraggo $E1 - E2$ e raccolgo viene:\\
$a_{i1}(c_1 - d_1) + a_{i2}(c_2 - d_m) + a_{im}(c_m - d_m)= 0\:\: \forall \: i,...,n$
\end{demostration}

\begin{theorem}
Se $(c_1,\ldots,c_m)$ è soluzione del sistema $(E)$ tutte le soluzioni $(E)$ sono della forma $(c_1 + e_1, c_2 + e_2, \ldots, c_m + e_m)$ dove $(e_1,\ldots,e_m)$ è soluzione di $E_{om}$.
\end{theorem}
In sinestesi si può semplificare questo teorema scrivendo:
\begin{equation}
    \text{"Soluzione generale" = "Soluzione particolare" + "Soluzione omogenea"}
\end{equation}

\begin{demostration}
La proposizione \ref{prop-1} dice che le soluzioni hanno questa forma. Viceversa se $(e_1,\ldots,e_m)$ sono soluzioni di ($E_{om}$) $\Longrightarrow$ $(c_1 + e_1, c_2 + e_2, \ldots, c_m + e_m)$ sono soluzioni di $(E)$.
\end{demostration}

\begin{example}
Prendiamo n=1 e m=2 e prendiamo come sistema di equazioni $(E): 2x + 3y = 5$ e come equazione omogenea $(E_{om}): 2x + 3y = 0$\\\\
Vediamo che le soluzioni particolari sono $x = y = 1$. Per calcolare le soluzioni omogenee si fa $2x = -3y$ e poi $x = -\frac{3}{2}y$, qui per ogni valore di y trovo un valore di x. \\
La soluzioni omogenea è $(-\frac{3}{2}p, p)$ dove p è un parametro che può essere qualsiasi valore.\\
Sappiamo che "sol. generale" = "sol. particolare" + "sol. omogenea" $\Rightarrow (1,1) + (-\frac{3}{2}t,t) = (1 - \frac{3}{2}t, 1 + t)$.
\end{example}

\begin{observation}
$(0,...,0)$ è sempre soluzione di $(E_{om})$. Quindi se (E) ammette una soluzione questo soluzione è unica $\Longleftrightarrow (0,...,0)$ è l'unica soluzione di $(E_{om})$.
\end{observation}

\subsection{Interpretazione geometrica caso generico}
L'interpretazione geometrica per ($E_{om}$) è un iperpiano attraverso l'origine", e la soluzione è traslazione di questo caso generale per un caso particolare.
\begin{enumerate}
    \item $n=1$, $m=2$ (E) $a_{1n}x_1 + a_{m2}x_2 = b_1$.\\
    Una soluzione $\Longleftrightarrow$ retta ($E_{om}$) $a_1x_1 + a_2x_2 = 0$ una soluzione a (E) $\Rightarrow$ retta attraverso (0,0).
    \item $n=1$, $m=2$, $a_{11}x_1 + a_{12}x_2 + a_{13}x_3 = a$ (E), punto attraverso (0,0,0).
\end{enumerate}

\subsection{Come trovare le soluzioni?}
Per trovare le soluzioni comuni di $(E)$ possiamo usare 3 operazioni per semplificare il sistema:
\begin{enumerate}
    \item Scambiare due equazioni.
    \item Moltiplicare $E_i$ per $\lambda \neq 0$ e fare la somma con $E_j$, $E_j = E_j + \lambda E_i$.
    \item Moltiplicare un'equazione $E_i$ per un costate $\lambda \neq 0$, $E_i \Rightarrow \lambda E_i$.
\end{enumerate}

\begin{observation}
Queste operazioni non cambiano l'insieme delle soluzioni di $(E)$.
\end{observation}

\begin{demostration}
Dimostriamo le 3 proprietà:
\begin{enumerate}
    \item La prima è ovvia quindi non ha bisogno di una dimostrazione.
    \item Se ($c_1, \ldots, c_n$) soluzioni di $E_i$ ed $E_j \Rightarrow$ è anche soluzione di $E_i + \lambda E_j$.\\
    Viceversa se ($c_1, \ldots, c_n$) soluzioni di $E_i$, $E_j + \lambda E_i \Rightarrow$ anche soluzione di ($E_j + \lambda E_i$) - $\lambda = E_j$.
    \item Se ($c_1, \ldots, c_n$) soluzioni di (E) $\Rightarrow$ anche di $\lambda E$ e viceversa.
\end{enumerate}
\end{demostration}


\newpage
\section{Funzioni}
\begin{definition}[Funzione]
- $f: A \longrightarrow B$: \\
Dati due insiemi $A$, $B$, detti dominio e codominio, una funzione è una "legge" o "regola" che associa ad ogni elemento di $A$ uno ed uno solo elemento di $B$.
\end{definition}
\begin{note}
Tipicamente in questo corso le funzioni saranno date come formule del tipo $f(x) = x^2 - 7x - e^x$ specificando dominio e codominio in questo modo $f: \mathbb{R} \longrightarrow \mathbb{R}$. Si noti che la definizione di una funzione \textbf{deve} includere sia la funzione che il suo dominio. Ad esempio $f: \mathbb{R} \longrightarrow \mathbb{R} \: f(x)=x^2$ e $g: \mathbb{R} \geq 0 \longrightarrow \mathbb{R} g(x)=x^2$ sono due funzioni diverse.
\end{note}
\begin{note}
	Se non vengono specificati dominio e codominio allora il dominio è il sottoinsieme più grande di $\mathbb{R}$ in cui la formula ha senso. Per la funzione $f(x)=\frac{1}{x}$ il dominio è ${x \in \mathbb{R} \mid x \neq 0}$.
\end{note}
\begin{example}
    Esempi funzioni:
    \begin{itemize}
        \item $g(x) = x^2 - 7x - e^x$ \hspace{.3cm} $g(0,+\infty) \longrightarrow \mathbb{R}$
        \item $g(x) = x^2$ \hspace{.3cm} $g(0, +\infty) \longrightarrow (0, +\infty)$. \hspace{.3cm}Va bene perché $x^2 > 0$ per qualsiasi valore di x.
        \item $h(x) = x^2$ \hspace{.3cm} $h(0, +\infty) \longrightarrow (-\infty, 0)$. \hspace{.3cm}Questa forma non va bene non definendo una funzione perché la formula non mi da numeri di $(-\infty, 0)$.
        \item $h(x) = x^2$ \hspace{.3cm} $h(0, +\infty) \longrightarrow (-\infty,1)$ \hspace{.3cm}Non va bene perché se prendiamo x=3 $f(3) = 9$ e 9 non fa parte del codominio. 
    \end{itemize}
\end{example}

\subsection{Grafico}
Una funzione $f: A \longrightarrow B$ con $A,B \in \mathbb{R}$ ha un \textbf{grafico} che si indica come:
\begin{equation}
    graph(f) = \{(a,b) \in A \times B\ \mid b = f(a)\}
\end{equation}
\begin{wrapfigure}[8]{l}{7cm}
    \centering
    \includegraphics[width=4.5cm, height=4cm]{Esempio-grafico.png}
    \caption{$f(x) = x^2$ con $f: \mathbb{R} \longrightarrow \mathbb{R}$}
    \label{fig:esempio-grafico}
\end{wrapfigure}
\begin{example}
Esempio punto sulla funzione
\begin{itemize}
    \item Il punto A sta sul grafico si $f(x) = x^2$ esattamente quando $y = x^2$.
    \item Il punto B non sta sul grafico quindi $y \neq x^2$.
\end{itemize}
\end{example}
\begin{note}
    A X B $\subseteq \mathbb{R}$ X $\mathbb{R}$. Dove R X R = $R^2$.
\end{note}
\begin{example}
A e B = $(0, +\infty)$, da qui vediamo che A X B rappresenta il primo quadrante.\\\\
\end{example}

\subsection{Immagine}
\begin{definition}[Immagine]
Prendendo $f: A \longrightarrow B$ e $D \subseteq A$ l'immagine di D tramite f è il sottoinsieme $f(D) \subseteq B$ costituito dagli elementi f(d) dove $d \in D$.
\end{definition}
\begin{example}
    Esempi immagine:
    \begin{itemize}
        \item Immagine di A, $f(A) \subseteq B$ si chiama anche immagine della funzione.
        \item $f(x) = x^2$, $f: \mathbb{R} \longrightarrow \mathbb{R}$ \hspace{.2cm} immagine di g è $[0, +\infty)$ perché $x^2 \geq 0 \: \forall \: x \in \mathbb{R}$.
        \item $g(x) = x?2$, $g:[2, +\infty) \longrightarrow \mathbb{R}$ \hspace{.2cm} l'immagine di g è $[4, +\infty]$ perché se si calcola il punto minore del dominio, cioè 2, torna $g(2) = x^2$ che è uguale a 4, da lì possiamo prendere tutti i punti.
    \end{itemize}
\end{example}

\subsection{Suriettiva}
\begin{definition}[Suriettiva]
Una funzione si dice suriettiva quando ogni elemento del codominio è immagine di almeno un elemento del dominio. Quindi prendendo una f(x), per che sia suriettiva deve l'immagine I essere uguale ad un valore, $I(f) = b$.
\end{definition}
\begin{equation}
	\forall y \in B \exists x \in A
\end{equation}
\begin{note}
	La suriettività si traduce graficamente nel fatto una qualsiasi retta orizzontale intersechi il grafico almeno una volta.
\end{note}
\begin{example}
    Esempi funzioni suriettive:
    \begin{itemize}
        \item $f(x) = x^2$, $f: \mathbb{R} \longrightarrow \mathbb{R}$ non è suriettiva perché tutti i valori del codominio $y < 0$ non hanno un rispettivo nel dominio.
        \item $g(x) = x^2$, $g: \mathbb{R} \longrightarrow (0, +\infty)$ lo è perchè andiamo a restringere il codominio ai punti che hanno un corrispettivo nel dominio.
    \end{itemize}
\end{example}

\subsection{Iniettiva}
\begin{definition}[Iniettiva]
Una funzione iniettiva è una funzione che associa, a elementi distinti del dominio, elementi distinti del codominio. Quindi prendendo una f(x) è iniettiva se prendendo due valori $x_1, x_2$ dove $x_1 \neq x_2 \Longrightarrow f(x_1) \neq f(x_2)$. (Input diversi danno output diversi).
\end{definition}
\begin{equation}
	x_{1}, x_{2} \in A \wedge x_{1} \neq x_{2} \implies f(x_{1}) \neq f(x_{2})
\end{equation}
\begin{note}
	L'iniettività si traduce graficamente nel fatto una qualsiasi retta orizzontale intersechi il grafico al più una volta.
\end{note}
\begin{example}
    Esempi funzioni iniettiva:
    \begin{itemize}
        \item $f(x) = x^2$, $f: \mathbb{R} \longrightarrow \mathbb{R}$ non è iniettiva perché se prendiamo $x_1 = 1$ e $x_2 = -1$ $f(x1) = f(x2).$
        \item $g(x) = x^2$, $g: [0, +\infty) \longrightarrow \mathbb{R}$ è invece iniettiva perché non consideriamo i valori negativi.
    \end{itemize}
\end{example}

\subsection{Biunivoca}
\begin{definition}[Biunivoca]
Una funzione si definisce biunivoca o bigettiva se è sia iniettiva che suriettiva.
\end{definition}

\subsection{Invertibile}
\begin{definition}[Invertibile]
Se una funzione è biunivoca si dice che tale funzione è anche invertibile.
\end{definition}
\begin{wrapfigure}{l}{6cm}
    \centering
    \includegraphics[width=5cm, height=4cm]{Esempio-invertibilita.png}
    \caption{$f(x) = x^2$ e $g(x) = \sqrt{x}$}
    \label{fig:esempio-invertibilità}
\end{wrapfigure}
Se f è una funzione invertibile i grafici di f e di $f^{-1}$ (la funzione inversa) sono simmetrici rispetto alla retta $y=x$ cioè alla bisettrice del primo e del terzo quadrante. \\
\begin{example}
Se vediamo nell'immagine [\ref{fig:esempio-invertibilità}] prendendo l'inverso della funzione $f(x) = x^2$ definita in $[0, +\infty] \longrightarrow \mathbb{R}$ e cioè la funzione $g(x) = \sqrt{x}$ è simmetrica.
\\ \\ \\ \\ \\
\end{example}

\subsection{Funzioni Monotone}
\begin{definition}[Monotone]
Dati $A, B \in \mathbb{R}$ e $f:A \longrightarrow B$. $x_1, x_2 \in A$ con $x_1 < x_2$ se $\forall x_1, x_2$ risulta ciò che è scritto in Tabella \ref{tab:monotone}.
\end{definition}
\begin{table}[h!]
    \centering
    \setlength{\tabcolsep}{6pt}
    \renewcommand{\arraystretch}{1.7}
    \begin{tabular}{|c|c|}
        \hline
        \textbf{[1] Strettamente Crescente} & $f(x_1) < f(x_2) $ \\ \hline
        \textbf{[2]Debolmente Crescente} & $f(x_1) \leq f(x_2) $ \\ \hline
        \textbf{[3]Strettamente Decrescente} & $f(x_1) > f(x_2) $ \\ \hline
        \textbf{[4]Debolmente Decrescente} & $f(x_1) \geq f(x_2) $ \\ \hline
    \end{tabular}
    \caption{Definizioni funzioni crescenti e decrescenti}
    \label{tab:monotone}
\end{table}
Andando a considerare la Tabella \ref{tab:monotone} possiamo dire che:
\begin{itemize}
    \item \textbf{Strettamente monotona} nei casi [1] e [3] della tabella.
    \item \textbf{Debolmente monotona} nei casi [2] e [4] della tabella.
\end{itemize}

\begin{observation}
	Se $f$ è \textbf{strettamente monotona} allora è \textbf{iniettiva} in quanto dati $x_{1} \neq x_{2}$ con $x_{1} < x_{2} \implies f(x_{1}) < f(x_{2})$ e in particolare $f(x_{1}) \neq f(x_{2})$. Non vale però il contrario, infatti una funzione iniettiva non è per forza strettamente monotona (ad esempio data $f(x)=\frac{1}{x}$ con $f:\mathbb{R} \setminus \{0\} \longrightarrow \mathbb{R} \setminus \{0\}$ )
\end{observation}

\begin{observation}
	Se $f$ è \textbf{strettamente crescente} allora è anche \textbf{debolmente crescente}.
\end{observation}
\begin{example}
    Esempi funzioni crescenti e decrescenti:\\
    \begin{figure}[h!]
        \begin{subfigure}{.5\textwidth}
            \centering
            \includegraphics[width=6cm, height=4cm]{funzione-crescente.png}
            \caption{$f(x_1) < f(x-2)$ quindi è crescente}
            \label{fig:funzione-crescente}
        \end{subfigure}
        \begin{subfigure}{.5\textwidth}
            \centering
            \includegraphics[width=6cm, height=4cm]{funzione-decrescente.png}
            \caption{$f(x_1) > f(x-2)$ quindi è decrescente}
            \label{fig:funzione-decrescente}
        \end{subfigure}
    \end{figure}
    \\Possiamo anche federe dalle immagini [\ref{fig:funzione-crescente}] [\ref{fig:funzione-decrescente}] che:
    \begin{itemize}
        \item Se f(x) è \textbf{crescente} l'ordinamene verrà \textbf{mantenuto}.
        \item Se f(x) è \textbf{decrescente} l'ordinamento verrà \textbf{invertito}.\\
    \end{itemize}
\end{example}

\newpage
\begin{observation}
    Osservazione sul rapporto incrementale:\\
\end{observation}
\begin{wrapfigure}[8]{l}{8cm}
    \vspace{-15pt}
    \centering
    \includegraphics[width=6.7cm]{rapporto_incrementale.png}
    \caption{$\frac{\Delta_y}{\Delta_x}$}
    \label{fig:esempio-invertibilità}
\end{wrapfigure}
Definito il \textbf{rapporto incrementale}\footnote{I rapporto incrementale misura quanto il punto della f si sposta in verticale in rapporto a quanto abbiamo l'asciasse in orizzontale.} come:
\begin{equation}
    \frac{\Delta_y}{\Delta_x}=\frac{f(x_1) - f(x_2)}{x_1 - x_2}
\end{equation}

\begin{note}
    Il denominatore ed il numeratori devono essere concordi per fare in modo che il rapporto incrementale sia maggiore di 0 e quindi la funzione crescente. \\ \\\\
\end{note}
Continuando ad analizzare il rapporto incrementale possiamo ricavare anche i casi in cui una funzione e strettamente decrescente o debolmente crescente o debolmente decrescente. Puoi vedere tutte le casistiche nella tabella \ref{tab:analisi-rapporto-incrementale}.
\begin{table}[h!]
    \centering
    \setlength{\tabcolsep}{7pt}
    \renewcommand{\arraystretch}{2}
    \begin{tabular}{|c|c|}
        \hline
        Strettamente Crescente & $\frac{f(x_1) - f(x_2)}{x_1 - x_2} > 0$\\ \hline
        Strettamente Decrescente & $\frac{f(x_1) - f(x_2)}{x_1 - x_2} < 0$ \\ \hline
        Debolmente Crescente & $\frac{f(x_1) - f(x_2)}{x_1 - x_2} \geq 0$ \\ \hline
        Debolmente Decrescente & $\frac{f(x_1) - f(x_2)}{x_1 - x_2} \leq 0$ \\ \hline
    \end{tabular}
    \caption{Analisi rapporto incrementale}
    \label{tab:analisi-rapporto-incrementale}
\end{table}
\begin{observation}
    Se una funzione f(x) è strettamente crescente è a sua volta anche debolmente crescente, mentre una funzione f(x) se è debolmente crescente non è strettamente crescente perché aggiunge una casistica che sarebbe $f(x_1) = f(x_2)$. 
\end{observation}
\begin{example}
    Casistica particolare:\\
    Data $f(x)=\frac{1}{x}$, \hspace{.3cm} $f: \mathbb{R} \: \setminus \: \{0\} \longrightarrow \mathbb{R} \: \setminus \: \{0\}$. Funzione rappresentata nell'immagine [\ref{fig:esempio-particolare}].
    \begin{figure}[h!]
        \centering
        \includegraphics[width=8.7cm]{esempio-particolare.png}
        \caption{$f(x)=\frac{1}{x}$, \hspace{.3cm} $f: \mathbb{R} \: \setminus \: \{0\} \longrightarrow \mathbb{R} \: \setminus \: \{0\}$}
        \label{fig:esempio-particolare}
    \end{figure}
    \\Possiamo vedere che:
    \begin{itemize}
        \item f(x) è strettamente decrescente in $(0, +\infty)$.\\
        Quindi se andiamo a prendere $0 < x_3 < x_4$ abbiamo che $f(x_3) > f(x_4)$.
        \item f(x) è strettamente decrescente in $(-\infty, 0)$.\\
        Quindi se andiamo a prendere $x_1 < x_2 < 0$ abbiamo che $f(x_1) > f(x_2)$.
    \end{itemize}
    Se però andiamo a considerare tutto $\mathbb{R} \: \setminus \: \{0\}$, e quindi prendiamo i punti $x_1 < 0 < x_4$ vediamo che $f(x_1) < f(x_4)$.
    In conclusione si può dire quindi che $f(x)=\frac{1}{x}$) è decrescente in $(-\infty, 0)$ e in $(0, +\infty)$ ma non lo è in tutto $\mathbb{R} \: \setminus \: \{0\}$.
\end{example}

\subsubsection{Composizione con funzioni monotone}
\begin{definition}[Composizione]
	La composizione di funzioni si definisce come $g \circ f:A \longrightarrow C$, $(g \circ f)(a)=g(f(a))$
\end{definition}
Prendendo i considerazioni 3 insiemi A, B, C tali che $A, B, C \subset \mathbb{R}$ e 2 funzioni f(x) e g(x) così definite: \hspace{.2cm} $f: A \longrightarrow B$, $g: B \longrightarrow C $.
\begin{enumerate}
    \item Se f è crescente e g è crescente allora $g \circ f$ è crescente.
    \item Se f è crescente e g è decrescente allora $g \circ f$ è decrescente e viceversa ($ x_{1} < x_{2} \implies f(x_{1}) < f(x_{2}) \implies g(f(x_{1})) < g(f(x_{2})) $).
    \item Se f è decrescente e g è decrescente allora $g \circ f$ è crescente ($ x_{1} < x_{2} \implies f(x_{1}) > f(x_{2}) \implies g(f(x_{1})) < g(f(x_{2})) $).
\end{enumerate}

\begin{example}
    $h(x) = e^{x^3}$\\
    La funzione $h$ si ottiene dalla composizione di:
    \begin{itemize}
        \item $f: \mathbb{R} \longrightarrow \mathbb{R}$ \hspace{.3cm} $f(x) = x^3$. Funzione crescente.
        \item $g: \mathbb{R} \longrightarrow \mathbb{R}$ \hspace{.3cm} $g(t) = e^t$. Funzione decrescente.
    \end{itemize}
    Quindi possiamo scrivere $h(x) = e^{x^3}$ come: \hspace{.3cm} $e^{f(x)} \: \: = \: \: g(f(x)) \: \: = \: \: (g \circ f)(x)$
    Inoltre visto che f è crescente e g è crescente, h è strettamente crescente 
\end{example}
\begin{observation}
    Se prendiamo una funzione f(x) strettamente monotona, allora f(x) è iniettiva. Questa condizione è vera ma NON lo è viceversa: una funzione f(x) iniettiva NON è per forza strettamente monotona. 
\end{observation}
\begin{example}
    Se prendiamo una f(x) tale che: \hspace{.3cm} $f(x) = \frac{1}{x}$ \hspace{.3cm} $\mathbb{R} \setminus \{0\} \longrightarrow \mathbb{R} \setminus \{0\}$
    \\Possiamo vedere rifacendoci all'esempio in figura [\ref{fig:esempio-particolare}] che f è iniettiva ma non monotona.
\end{example}

\subsection{Insieme di definizione}
\begin{definition}[Insieme di definizione]
    Data una funzione f(x) l'insieme di definizione o dominio naturale di una funzione è il più grande sottoinsieme di $\mathbb{R}$ dove ha senso la funzione f(x).
\end{definition}
\begin{example}
    $f(x) = \frac{1}{x}$ \hspace{.3cm} L'insieme di definizione è $\mathbb{R} \setminus \{0\}$
\end{example}

\subsection{Funzioni pari e dispari}
\begin{definition}[Pari]
    La funzione è \textbf{pari} se $f(x) = f(-x) \: \forall x$ nel dominio di $f \longrightarrow f$. Il grafico di una funzione pari è simmetrico rispetto all'asse $y$.
\end{definition}
\begin{definition}[Dispari]
    La funzione è \textbf{dispari} se $f(x) = -f(-x) \: \forall x$ nel dominio di $f \longrightarrow f$. Il grafico di una funzione dispari è simmetrico rispetto all'origine.
\end{definition}
\begin{note}
    Il dominio di $f$ deve essere simmetrico.\\
\end{note}
\begin{example}
Esempio funzioni pari e dispari.\\
\end{example}
$f(-x) = (-x)^2 = x^2 = f(x)$, f(x) è \textbf{pari}: \hfill $f(-x) = (-x)^2 = x^2 = -f(x)$, f(x) è \textbf{dispari}:
\begin{figure}[h!]
    \vspace{-1pt}
    \begin{subfigure}{.5\textwidth}
        \centering
        \includegraphics[width=3cm]{funzione-pari.png}
        \caption{$f(x) = x^2$, \hspace{.2cm} graph(f) con f pari}
    \end{subfigure}
    \begin{subfigure}{.5\textwidth}
        \centering
        \includegraphics[width=2.5cm]{funzione-dispari.png}
        \caption{$f(x) = x^3$, \hspace{.2cm} graph(f) con f dispari}
    \end{subfigure}
\end{figure}
\begin{note}
	Una funzione del tipo $f(x)=x^(2n)$ con $n \in \mathbb{N}$ è sempre pari mentre una funzione del tipo $f(x)=x^(2n+1)$ con $n \in \mathbb{N}$ è sempre dispari.
\end{note}

\subsection{Funzione periodica}
\begin{definition}[Periodicità]
    Una funzione f(x) si dice periodica di periodo $P \in \mathbb{R}$ se $\forall x \: \: f(x + P) = f(x)$. 
\end{definition}
\begin{wrapfigure}{r}{9cm}
    \vspace{-15pt}
    \centering
    \includegraphics[width=7cm]{funzione-periodica.png}
    \caption{$sin(x) = sin(x + 2\pi)$}
    \label{fig:funzione-periodica}
\end{wrapfigure}
Inoltre il dominio di f(x) deve essere tale che $x \in dom(f) \implies x + P \in dom(f)$.
\begin{example}
In figura [\ref{fig:funzione-periodica}] un esempio di funzione periodica.\\\\
\end{example}

\subsection{Funzioni Elementari}
\subsubsection{Lineari}
\textbf{Funzione retta}: $f(x) = ax + b$. \hspace{.3cm} $a,b \in \mathbb{R}$ \\ Dove $a$ (coefficiente angolare) indica la pendenza della retta, mentre $b$ (termine noto) indica il punto di incontro con l'asse $Y$.

\subsubsection{Esponente positivo o negativo}
\textbf{Fun. Esp. positivo:} $f(x) = x^k$, $k \in \mathbb{N}$. \hfill \textbf{Fun. Esp. negativo:} $f(x) = x^k$, $k \in \mathbb{N}$, $k < 0$.
\begin{figure}[h!]
    \begin{subfigure}{.5\textwidth}
        \centering
        \includegraphics[width=4cm, height=3.5cm]{parabole.png}
        \caption{con k pari}
        \label{fig:esponente-positivo-pari}
    \end{subfigure}
    \begin{subfigure}{.5\textwidth}
        \centering
        \includegraphics[width=4cm, height=3.5cm]{esponente-negativo-dispari.png}
        \caption{con k dispari}
        \label{fig:esponente-positivo-dispari}
    \end{subfigure}
\end{figure}
\begin{figure}[h!]
    \vspace{-5pt}
    \begin{subfigure}{.5\textwidth}
        \centering
        \includegraphics[width=4cm, height=3.7cm]{esponente-dispari.png}
        \caption{con k pari}
        \label{fig:esponente-negativo-pari}
    \end{subfigure}
    \begin{subfigure}{.5\textwidth}
        \centering
        \includegraphics[width=4cm, height=3.5cm]{esponsente-negativo-pari.png}
        \caption{con k dispari}
        \label{fig:esponente-negativo-dispari}
    \end{subfigure}
\end{figure}
\begin{observation}
    \textbf{k pari}: Le funzioni con il $k$ pari sono funzioni pari e hanno tutte una forma simile a quella in figura [\ref{fig:esponente-positivo-pari}] per le funzioni con k positive e per le funzioni con k negativo figura [\ref{fig:esponente-negativo-pari}].
\end{observation}
\begin{observation}
    \textbf{k dispari:} Le funzioni con il k positivo e dispari sono funzioni dispari e hanno tutte una forma simile a quella in figura [\ref{fig:esponente-positivo-dispari}] per le funzioni con k positive e per le funzioni con k negativo figura [\ref{fig:esponente-negativo-dispari}].
\end{observation}

\subsubsection{Radici o esponente fratto}
\textbf{Funzionane radici o esponente fratto:} $f(x) = x^{\frac{p}{q}}$ o $f(x) = \sqrt[q]{x^p}$  \: \: con  \: \:  $p, q \in \mathbb{N}$ \: \: e  \: \:  $q \neq 0$. \footnote{In matematica è possibile scrivere una un esponente fratto come radice mettendo il numeratore al radicando della radice e il denominatore all'indice: $x^{\frac{p}{q}} \: = \: \sqrt[q]{x^p}$}
\begin{figure}[h!]
    \begin{subfigure}{.5\textwidth}
        \centering
        \includegraphics[width=6.3cm]{radice-pari.png}
        \caption{con q pari}
        \label{fig:radice-pari}
    \end{subfigure}
    \begin{subfigure}{.5\textwidth}
        \centering
        \includegraphics[width=6cm]{radice-dispari.png}
        \caption{con q dispari}
        \label{fig:radice-dispari}
    \end{subfigure}
\end{figure}
\begin{note}
    $p$ e $q$ non possono essere entrambi pari perché in tal caso sono divisibili fra di loro e quindi portabili ad una forma ridotta.
\end{note}
\begin{observation}
    \textbf{q pari:} Le funzioni con il $q$ pari ha dominio $ x \geq 0$ ed è invertibile sono come funzione $f: [0, +\infty) \longrightarrow [0, +\infty)$. È rappresentata in figura [\ref{fig:radice-pari}].
\end{observation}
\begin{observation}
    \textbf{q dispari:} Le funzioni con il $q$ positivo ha dominio $x \in \mathbb{R}$ ed è ugualmente invertibile su tutto $\mathbb{R}$, è inoltre una funzione dispari. È rappresentata in figura [\ref{fig:radice-dispari}].\\
\end{observation}

\subsubsection{Esponenziale}
\textbf{Funzione esponenziale:} $f(x) = a^x$ con $a \in \mathbb{R}$, \: \: $a > 0$, \: \: $a \neq 1$ \: \: $f: \mathbb{R} \longrightarrow (0, +\infty)$
\begin{figure}[h!]
    \begin{subfigure}{.5\textwidth}
        \centering
        \includegraphics[width=4cm]{esponenziale.png}
        \caption{con $a > 1$}
        \label{fig:esponenziale}
    \end{subfigure}
    \begin{subfigure}{.5\textwidth}
        \centering
        \includegraphics[width=4cm]{esponsenziale-base-minore.png}
        \caption{con $0 < a < 1$}
        \label{fig:esponsenziale-base-minore}
    \end{subfigure}
\end{figure}
\begin{note}
    La funzione esponenziale è sempre positiva.
\end{note}
\begin{observation}
    \textbf{$a > 1$:} La funzione è strettamente crescente, come in nell'immagine [\ref{fig:esponenziale}].
\end{observation}
\begin{observation}
    \textbf{$0 < a < 1$:} La funzione è decrescente, come in nell'immagine [\ref{fig:esponsenziale-base-minore}].
\end{observation}

\subsubsection{Logaritmo}
\textbf{Funzione logaritmo:} $f(x) = \log_a x$, \: \: $f: (0, +\infty) \longrightarrow \mathbb{R}$ \: \: (inversa dell'esponenziale).
\begin{figure}[h!]
    \begin{subfigure}{.5\textwidth}
        \centering
        \includegraphics[width=5cm,height=4cm]{logaritmo.png}
        \caption{con $a > 1$}
        \label{fig:logaritmo}
    \end{subfigure}
    \begin{subfigure}{.5\textwidth}
        \centering
        \includegraphics[width=5cm,height=4cm]{logaritmo-base-minore.png}
        \caption{con $0 < a < 1$}
        \label{fig:logaritmo-base-minore}
    \end{subfigure}
\end{figure}
\begin{observation}
    Casistica particolare - $f(x) = e^x$.\\
    In questa casistica se andiamo a ridurre il codominio la funzione esponenziale è invertibile. $f: \mathbb{R} \longrightarrow (0, +\infty)$.
    Il suo inverso è un caso particolare di logaritmo e di chiama \textbf{logaritmo naturale}. E si può scrive in due modi:
    \begin{itemize}
        \item $\ln{x}$: sarebbe logaritmo in base naturale.
        \item $\log x$: scrivendo il logaritmo senza la base intendiamo il logaritmo in base $e$.
    \end{itemize}
\end{observation}

\subsubsection{Seno e Arcoseno}
\textbf{Seno:} $f(x) = \sin x$, $f: \mathbb{R} \longrightarrow \mathbb{R}$. \hfill
\textbf{Arcoseno:} $f(x) = \arcsin x$, $f: [-1, 1] \longrightarrow [-\frac{\pi}{2}, \frac{\pi}{2}]$
\begin{figure}[h!]
    \begin{subfigure}{.5\textwidth}
        \centering
        \includegraphics[width=6cm]{seno.png}
        \caption{$\sin{x}$}
        \label{fig:seno}
    \end{subfigure}
    \begin{subfigure}{.5\textwidth}
        \centering
        \includegraphics[width=2cm, height=2.3cm]{arcoseno.png}
        \caption{$\arcsin{x}$ o $\sin{x}^{-1}$}
        \label{fig:arcoseno}
    \end{subfigure}
\end{figure}

\begin{observation}
    \textbf{Sin(x):} La funzione $\sin{x}$ (immagine [\ref{fig:seno}]) è periodica per $2\pi$ quindi possiamo scrivere $\sin{(x+2\pi)} = \sin x \: \forall x \in \mathbb{R}$. Inoltre è suriettiva per codominio [-1, 1]. Se invece definiamo $f: [-\frac{\pi}{2}, \frac{\pi}{2}] \longrightarrow [-1, 1]$ la funzione $\sin x$ è strettamente crescente e suriettiva, quindi anche invertibile, e la sua inversa è appunto $\arcsin{x}$.
\end{observation}
\begin{observation}
    \textbf{Arcsin(x):} La funzione $\arcsin{x}$ è l'inverso del seno e può essere scritta anche come $f(x) = \sin{x}^{-1}$, è rappresentata nell'immagine [\ref{fig:arcoseno}].
\end{observation}

\subsubsection{Coseno e Arcocoseno}
\textbf{Coseno:} $f(x) = \cos{x}$, $f: \mathbb{R} \longrightarrow \mathbb{R}$. \hfill
\textbf{Arcocoseno:} $f(x) =\arccos{x}$, $f: [-1, 1] \longrightarrow [0, \pi]$
\begin{figure}[h!]
    \begin{subfigure}{.5\textwidth}
        \centering
        \includegraphics[width=6cm]{coseno.png}
        \caption{$\cos{x}$}
        \label{fig:coseno}
    \end{subfigure}
    \begin{subfigure}{.5\textwidth}
        \centering
        \includegraphics[width=3.5cm, height=2.7cm]{arcocoseno.png}
        \caption{$\arccos{x}$ o $\cos{x}^{-1}$}
        \label{fig:arcocoseno}
    \end{subfigure}
\end{figure}
\vspace{-5pt}
\begin{observation}
    \textbf{Cos(x):} La funzione $\cos{x}$, rappresentata nell'immagine [\ref{fig:coseno}], è periodica per $2\pi$ quindi possiamo scrivere $\cos{(x+2\pi)} = \cos x \: \forall x \in \mathbb{R}$. Inoltre è suriettiva per codominio [-1, 1]. Se invece definiamo $f: [0, \pi] \longrightarrow [-1, 1]$ la funzione $\cos x$ è suriettiva, quindi anche invertibile, e la sua inversa è appunto $\arccos{x}$.
\end{observation}
\begin{observation}
    \textbf{Arccos(x):} La funzione $\arccos{x}$ è l'inverso del seno e può essere scritta anche come $f(x) = \cos{x}^{-1}$ ed è rappresentata nell'immagine [\ref{fig:arcocoseno}].
\end{observation}

\subsubsection{Tangente e Arcotangente}
\textbf{Tangente:} $f(x) = \tan{x}$, $f: \mathbb{R} \longrightarrow \mathbb{R}$ \hfill
\textbf{Arcotangente:} $f(x) = \arctan{x}$, $f: \mathbb{R} \longrightarrow [-\frac{\pi}{2}, \frac{\pi}{2}]$
\begin{figure}[h!]
    \begin{subfigure}{.5\textwidth}
        \vspace{-20pt}
        \centering
        \includegraphics[width=4.2cm]{tangente.png}
        \vspace{-20pt}
        \caption{$\tan{x}$}
        \label{fig:tangente}
    \end{subfigure}
    \begin{subfigure}{.5\textwidth}
        \centering
        \includegraphics[width=5cm]{arcotangente.png}
        \caption{$\arctan{x}$ o $\tan{x}^{-1}$}
        \label{fig:arcotangente}
    \end{subfigure}
\end{figure}
\begin{observation}
    \textbf{Tan(x):} La funzione $\tan{x}$, rappresentata nell'immagine [\ref{fig:tangente}], può essere scritta anche come $\frac{\sin{x}}{\cos{x}}$, ha come dominio $\{x \in \mathbb{R} \: | \:  x \neq \frac{\pi}{2} + k\pi, \: k \in \mathbb{Z}\}$. La funzione tangente è fatta da infiniti intervalli, è quindi periodica per $\pi$; è di base non invertibile, ma se la ristringiamo in $f: [-\frac{\pi}{2}, \frac{\pi}{2}] \longrightarrow \mathbb{R}$ diventa biunivoca ed accetta la funzione inversa che è $\arctan{x}$.
\end{observation}
\begin{observation}
    \textbf{Arctan(x):} La funzione $\arctan{x}$, rappresentate nell'immagine [\ref{fig:arcotangente}], è inversa della funzione $\tan{x}$, può quindi essere scritta anche con la forma $\tan{x}^{-1}$.
\end{observation}
\section{Massimi e minimi}
\subsection{Massimo e minimo intervalli}
\begin{definition}[Massimo]
Dato un insieme A tale che: \\$A \subseteq \mathbb{R}, \: A \neq \O, \: m \in \mathbb{R}$ m si dice \textbf{massimo} di A se $m \geq a \: \forall \: a \in A$ e $m \in A$
\end{definition}
\begin{definition}[Minimo]
Dato un insieme A tale che: \\$A \subseteq \mathbb{R}, \: A \neq \O, \: m \in \mathbb{R}$ m si dice \textbf{minimo} di A se $m \leq a \: \forall \: a \in A$ e $m \in A$
\end{definition}
\begin{example}
    Esempi massini e minimi intervalli:
    \begin{itemize}
        \item Dato $A = [0,1]$ il max(A) = 1 e il suo min(A) = 0
        \item Dato $B = [0, 1)$ il min(B) = 0 mentre B non ha massimo.
    \end{itemize}
\end{example}

\begin{wrapfigure}{r}{8cm}
    \vspace{10pt}
    \centering
    \includegraphics[width=7cm]{dimostrazione-massimo.png}
    \caption{Segmento B}
\end{wrapfigure}
\begin{demostration}
Dimostriamo questo esempio:
\end{demostration}
Supponiamo per assurdo che $m \in \mathbb{R}$ sia il max di B, con ovviamente $m \in B$. Se tale condizione è vera $m < 1$ perché 1 non è incluso nell'insieme B = [0, 1).\\ \\
Poniamo ora $\epsilon = 1 - m$, così facendo $\epsilon$ diventa la lunghezza dell'intervallo fra 1 ed m.\\ \\
Definiamo ora un $m_1 = m + \frac{\epsilon}{2}$. Creando questo valore $m_1$ vediamo che $m_1 \in B$ ma anche che $m < m_1$ che contrasta con la definizione di massimo di B che dovrebbe essere $m \geq b \: \forall \: b \in B$. Così dimostriamo la non esistenza di un valore massimo.

\subsection{Maggiorante e minorante intervalli}
\begin{definition}[Maggiorante]
$A \subseteq \mathbb{R}$, $A \neq \O$, $k \in \mathbb{R}$ si dice \textbf{maggiorante} di A se $k \geq a \: \: \forall \: \: a \in A$. L'insieme di tutti i maggioranti si indica con $M_A$.
\end{definition}
\begin{definition}[Minorante]
$A \subseteq \mathbb{R}$, $A \neq \O$, $k \in \mathbb{R}$ si dice \textbf{minorante} di A se $k \leq a \: \: \forall \: \: a \in A$. L'insieme di tutti i minoranti si indica con $m_A$.
\end{definition}
\begin{example}
A = [0,3] allora 3 è un maggiorante di A, quindi $3 \in M_A$. \\
Mentre $\frac{1}{4}$ non è un maggiorante, quindi $\frac{1}{4} \notin M_A$, perché $1 > A$ e $1 > \frac{1}{2}$.
\end{example}
\begin{observation}
    Se esiste un maggiorante di A allora ne esistono infiniti. Infatti se prendiamo un $k \in M_A$, m è un maggiorante di A $\forall \: \: m \geq k$.
    Questo discorso vale anche per i minoranti, infatti con $k \in m_A$, m è un minorante di A $\forall \: \: m \leq k$.
    \begin{example}
        Esempi per l'osservazione sopra:
        \begin{itemize}
            \item A = $\mathbb{R}$, A non ha maggioranti.
            \item A = [4, $+\infty$] non ha maggioranti ma ha minoranti.
        \end{itemize}
    \end{example}
\end{observation}

\subsection{Intervallo limitato}
\begin{definition}[Limitato superiormente]
    Dato un intervallo A, se $M_A \neq \O$ (insieme dei maggioranti) allora l'intervallo A si dice \textbf{limitato superiormente}
\end{definition}
\begin{definition}[Limitato inferiormente]
    Dato un intervallo A, se $m_A \neq \O$ (insieme dei minoranti) allora l'intervallo A si dice \textbf{limitato inferiormente}
\end{definition}
\begin{definition}[Limitato]
    $A \subset \mathbb{R}$, $A \neq \O$, se A è sia superiormente che inferiormente limitato allora A si dice semplicemente intervallo \textbf{limitato}.
\end{definition}
\begin{observation}
    A è limitato se e solo se $\exists \: \: h,k \in \mathbb{R}$ tale che $k \leq a \leq h \: \: \forall \: \: a \in A$
\end{observation}

\subsubsection{Estremi superiori ed inferiori}
\begin{theorem}[Estremo superiore]
    $A \subset \mathbb{R}$, $A \neq \O$ ed A è superiormente limitato, allora esiste il minimo dell'insieme dei maggioranti. Tale minimo si dice \textbf{estremo superiore} di A e si indica con sup(A).
\end{theorem}
\begin{theorem}[Estremo inferiore]
    $A \subset \mathbb{R}$, $A \neq \O$ ed A è inferiormente limitato, allora esiste il massimo dell'insieme dei minoranti. Tale massimo si dice \textbf{estremo inferiore} di A e si indica con inf(A).
\end{theorem}
\begin{example}
    Esempio estremi superiori ed inferiori:
    \begin{itemize}
        \item A = [0, 1) \hspace{.1cm} $M_A$ = [1, $+\infty$) e $m_A$ = ($-\infty$, 0] \hspace{.1cm} min($M_A$) = sup(A) = 1 \hspace{.2cm} max($m_A$) = inf(A) = 0
        \item B = [0, 1] \hspace{.1cm} $M_B$ = [1, $+\infty$) e $m_A$ = ($-\infty$, 0] \hspace{.1cm} min($M_B$) = sup(B) = 1 \hspace{.2cm} max($m_B$) = inf(B) = 0
    \end{itemize}
    \begin{observation}
        Se esiste max(A) allora max(A) = sup(A) e viceversa se esiste min(A) allora min(A) = inf(A)
    \end{observation}
\end{example}
\begin{note}
    Se A non è superiormente limitato scriviamo sup(A) = $-\infty$ e se non è inferiormente limitato inf(A) = $-\infty$.
\end{note}
\begin{observation}
    $A \neq \O$ e A è superiormente limitata, allora m = sup(A) se e solo se valgono 2 condizioni:
    \begin{enumerate}
        \item $a \leq m \: \: \forall \: \: a \in A$ \hspace{.3cm} Questo dice che m è un maggiorante
        \item $\forall \: \: \epsilon > 0 \: \: \exists \: \: \overline{a}$\footnote{$\overline{a}$ è un semplice metodo di notazione}$\in A \: \: | \: \: \overline{a} > m - \epsilon$ \hspace{.3cm} $m - \epsilon$ mi dice che non ci sono maggioranti più piccoli di m. 
    \end{enumerate}
    Se valgono queste 2 condizioni m è l'estremo sup e viceversa se m è sup(A) allora valgono queste condizioni.
    \begin{note}
        Questa considerazione vale anche per m = inf(A).
    \end{note}
\end{observation}
\begin{observation}
    La scrittura sup(A) $< +\infty$ vuol dire che l'estremo superiore di A è un numero reale, quindi A è superiormente limitato. Viceversa la scrittura inf(A) $> -\infty$ vuol dire che l'estremo inferiore di A è un numero reale, quindi A è inferiormente limitato.
\end{observation}

\subsection{Retta reale estesa}
\begin{definition}[Retta reale estesa]
    La retta reale estesa si indica con $\overline{\mathbb{R}} = \mathbb{R} \cup \{-\infty\} \cup \{+\infty\}$ in modo che valga: $-\infty \leq x \leq +\infty \: \: \forall x \in \overline{\mathbb{R}}$
\end{definition}
\begin{observation}
    Se $x \in \mathbb{R}$ (quindi $x \neq +\infty, x \neq -\infty$) allora $-\infty < x < +\infty$
\end{observation}

\subsubsection{Operazioni in $\overline{\mathbb{R}}$}
\begin{itemize}
    \item Se $x \neq +\infty$ allora $x + (-\infty) = -\infty$.
    \item Se $x \neq -\infty$ allora $x + (+\infty) = +\infty$.
    \item Se $x > 0$ allora $x(+\infty) = +\infty$ e $x(-\infty) = -\infty$.
    \item Se Se $x < 0$ allora $x(+\infty) = -\infty$ e $x(-\infty) = +\infty$.
    \item $(+\infty) + (-\infty)$ e viceversa \hspace{.3cm} $0(+\infty)$ o $0(\infty)$ \hspace{.3cm}\textbf{Sono vietate}
    \item $(+\infty)(+\infty) = +\infty$ \hspace{.2cm} $(+\infty)(-\infty) = -\infty$ \hspace{.2cm} $(-\infty)(-\infty) = +\infty$ \hspace{.2cm} \textbf{Sono consentite}
\end{itemize}
\begin{observation}
    Dato $A \subset \mathbb{Z}$ se A è superiormente limitato, A ha un massimo e se A è inferiormente limitato allora A ha un minimo.
\end{observation}

\subsection{Parte intera di un numero}
\begin{definition}
    Dato $x \in \mathbb{R}$ si dice \textbf{parte intera di x} e si indica con [x] il numero [x] = max$\{m \in \mathbb{Z}: m \leq x\}$
\end{definition}
\begin{wrapfigure}{r}{6cm}
    \vspace{-15pt}
    \centering
    \includegraphics[width=5cm]{parte-intera.png}
    \caption{Parte intera di x}
    \label{fig:my_label}
\end{wrapfigure}
Possiamo spiegarlo in maniera semplice che è il primo numero intero che troviamo alla sinistra di x.
\begin{example}
    $[\frac{25}{10}] = 2$ \hspace{.5cm} $[-\frac{25}{10}] = -2$\\ \\
\end{example}

\vspace{-20pt}
\subsubsection{Grafico di f(x) = [x]}
\begin{wrapfigure}[8]{l}{7cm}
    \vspace{-25pt}
    \centering
    \includegraphics[width=5cm]{funzione-parte-intera.png}
    \caption{Grafico f(x) = [x]}
    \label{fig:funzione-parte-intera}
\end{wrapfigure}
\vspace{20pt}
Possiamo vedere nell'immagine [\ref{fig:funzione-parte-intera}] che tutti numeri vanno a valere in y come il valore del primo intero a sinistra.
\begin{example}
    Esempio per f(x) = [x]:\\
    $f(\frac{1}{2}) = 0$ \hspace{.3cm} $f(\frac{3}{2}) = 1$\\ \\
    $f(\frac{10}{3}) = 3$ \hspace{.3cm} $f(\frac{4}{3}) = 1$\\ \\ \\
\end{example}

\subsection{Limiti, massimi e minimi su funzioni}
Andiamo a fare una serie di definizioni prendendo due insiemi A, B tale che $A \subseteq \mathbb{R}$ e $B \subseteq \mathbb{R}$ ed una funzione $f(A)$ definita come $f: A \longrightarrow B$.
\begin{definition}[Limitata superiormente, inferiormente]
    $f$ si dice limitata superiormente se $f(A)$ è limitata superiormente. Viceversa $f$ si dice limitata inferiormente se $f(A)$ è limitata inferiormente. Se $f$ è sia limitata superiormente che inferiormente si dice che $f$ è limitata.
\end{definition}
\begin{definition}[Massimo e minimo]
    $f$ ha massimo se la sua immagine $f(A)$ ha massimo. Si dice che $M$ è il massimo di $f$ e si scrive $M = max(f)$ se $M = max(f(A))$. Ugualmente $f$ ha minimo se la sua immagine $f(A)$ ha minimo. Si dice che $m$ è il minimo di $f$ e si scrive $m = min(f)$ se $m = min(f(A))$.
\end{definition}
\begin{definition}
    Se f non è limitata superiormente e si scrive $sup(f) = +\infty$. Ugualmente se f non è limitata inferiormente, e si scrive $inf(f) = -\infty$.
\end{definition}
\begin{note}
    Rircoda che sup($f$) corrisponde a scrivere sup($f(A)$) e ugualmente inf($f$) è uguale a inf($f(A)$).
\end{note}
\begin{definition}[Punti di massimo e minimo]
    Se $f$ ha massimo allora ogni $x_0 \in A$ tale che $f(x_0) = max(f)$ si dice punto di massimo per $f$. Similmente se $f$ ha minimo allora ogni $x_0 \in A$ tale che $f(x_0) = min(f)$ si dice punto di minimo per $f$.
\end{definition}
\begin{observation}
    Il massimo di $f$ è unico mentre i punti di massimo possono essere molti.
\end{observation}

\begin{example}
    $f: \mathbb{R} \longrightarrow \mathbb{R}$ \hspace{.3cm} $f(x) = \sin{x}$ [\ref{fig:massimo_sinx}]
\end{example}
max(f) = 1 \hspace{.3cm} $x_0 = \frac{\pi}{2} + 2k\pi, k \in \mathbb{Z}$\\
\begin{wrapfigure}[4]{r}{8cm}
    \vspace{-45pt}
    \centering
    \includegraphics[width=6.5cm]{images/massimo_es1.png}
    \caption{funzione $f(x) = \sin{x}$}
    \label{fig:massimo_sinx}
\end{wrapfigure}
In questo caso essendo la funzione periodica in ogni intervallo di $x_0 = \frac{\pi}{2} + 2k\pi, k \in \mathbb{Z}$ esisterà un punto di massimo mentre il massimo rimarrà sempre 1.
\begin{example}
    $f:(0, +\infty) \longrightarrow \mathbb{R}$ \hspace{.3cm} $f(x) = \frac{1}{x}$ [\ref{fig:massimo-minimo-frazione}]
\end{example}
In questa casistica $f$ non ha ne massimo ne minimo. Questo lo possiamo dimostrare andando ad immaginare una casistica dove esiste un massimo ed un minimo e facendo poi alcune considerazione. \\
\begin{wrapfigure}{r}{6cm}
    \vspace{-15pt}
    \centering
    \includegraphics[width=4.5cm]{images/massimo_es2.png}
    \caption{funzione $f(x) = \frac{1}{x}$}
    \label{fig:massimo-minimo-frazione}
\end{wrapfigure}
Innanzitutto prendiamo per assurdo che $f$ avesse massimo allora $\Longrightarrow \: \: \exists \: \: m$ tale che $f(x) \leq m \: \: \forall \: \: x \in (0, +\infty)$. \\ Se in questa casistica prendessimo un punto x e dicessima che quello è il massimo, $f(\frac{1}{x}) = m$, ma se poi prendiamo un punto che è $\frac{x}{2}$ esso apparitene sempre alla funzione e $f(\frac{x}{2}) = 2m$ e $2m > m$. Quindi vediamo come non è possibile determinare un massimo.\\ \\
Questa funzione non può nemmeno avere un minimo perché $f(x) > 0 \: \: \forall \: \: x$, quindi $inf(f) = 0$. Se $f$ avesse minimo dovrebbe essere $m(f) = inf(f) = 0$ ma questo presuppone che debba esiste un $x_0$ tale che $f(x_0) = 0$ cioè $\frac{1}{x_0} = 0$, ma questo è impossibile.\\
\begin{observation}
    Consideriamo un insieme A $\subset \mathbb{R}$ e una funzione $f: A \longrightarrow \mathbb{R}$, valgono per essi le seguenti osservazioni:
    \begin{itemize}
        \item Se A ha massimo e $f$ è debolmente crescente allora $f$ ha max e max($f$) = $f$(max(A)).
        \item Se A ha minimo e $f$ è debolmente crescente allora $f$ ha min e min($f$) = $f$(min(A)).
        \item Se A ha minimo e $f$ è debolmente crescente allora $f$ ha min e min($f$) = $f$(max(A)).
        \item Se A ha massimo e $f$ è debolmente crescente allora $f$ ha max e max($f$) = $f$(min(A)).
    \end{itemize}
\end{observation}
\begin{figure}[h!]
    \begin{subfigure}{.5\textwidth}
        \centering
        \includegraphics[width=6cm]{images/crescente-max-min.png}
        \caption{Punti max e min $f$ crescente}
        \label{fig:my_label}
    \end{subfigure}
    \begin{subfigure}{.5\textwidth}
        \centering
        \includegraphics[width=4cm]{images/descrescente-max-min.png}
        \caption{Punti max min $f$ decrescente}
        \label{fig:my_label}
    \end{subfigure}
\end{figure}
\begin{observation}
    Se $f: A \longrightarrow \mathbb{R}$ allora m = sup($f$) se e solo se valgono queste due condizioni:
    \begin{enumerate}
        \item $f(x) \leq m \: \: \forall \: \: x \in A$ \hspace{.3cm} Questo vuol dire che m deve essere maggiore o uguale di qualsiasi f(x)
        \item $\forall \: \: \epsilon > 0 \: \: \exists \: \:  \overline{x} \in A \: \: | \: \: f(\overline{x}) > m - \epsilon$ \hspace{.3cm} Questo vuol dire che per qualsiasi valore $\epsilon$ maggiore di 0 deve esistere un $\overline{x}$ appartenendo all'insieme A tale che, se sottraiamo il valore $\epsilon$ a m il risultato deve essere inferiore a $f(\overline{x})$ ciò vuol dire che non ci sono altri valori per il quale la funzione è sempre sotto.
    \end{enumerate}
\end{observation}

\newpage
\section{Valore assoluto}
\begin{definition}[Valore assoluto]
    Dato $x \in \mathbb{R}$ si dice valore assoluto di x il massimo valore fra x e -x e si indica con $|x|$.
    \begin{equation}
        |x| = max(\{x, -(x)\})
    \end{equation}
\end{definition}
\begin{example}
    Esempi valore assoluto:
    \begin{itemize}
        \item $|5| = max(\{5, -5\}) = 5$
        \item $|-3| = max(\{-3, -(-3)\}) = 3$
    \end{itemize}
\end{example}
\subsection{Proprietà valore assoluto}
\begin{table}[h!]
    \setlength{\tabcolsep}{7pt}
    \renewcommand{\arraystretch}{2}
    \centering
    \begin{tabular}{|c|c|}
        \hline
        (1) $x \leq |x| \: \: \forall \: \: x \in \mathbb{R}$ & (2) $|x| = x$ se $x \geq 0, |x| = -x$ se $x \leq 0$ \\
        (3) $|x| \geq 0 \: \: \forall \: \: x \in \mathbb{R}$ & (4) $|x| = 0 \Longleftrightarrow x = 0$ \\
        (5) $|-x| = |x|$ & (6) $-|x| \leq x \leq |x|$ \\
        (7) $|x| \leq M \Longleftrightarrow -M \leq x \leq M$ con $M \geq 0$ & (8) $|x| \geq M \Longrightarrow x \geq M$ oppure $x \leq -M$ \\ \hline
    \end{tabular}
    \caption{Proprietà valore assoluto}
    \label{tab:prop-valore-assoluto}
\end{table}
\subsubsection{Spiegazioni proprietà}
Se stabiliamo un punto M maggiore del valore assoluto la funzione si troverà compreso fra M e -M. Se invece stabiliamo un punto M minore del valore assoluto la funzione sarà maggiore di M e minore di -M. Spiegazione grafica nell'immagine [\ref{fig:prop-6-7}]
\begin{figure}[h!]
    \centering
    \includegraphics[width=10cm]{es-proprieta-valore-assoluto.png}
    \caption{Spiegazione proprietà 7 e 8}
    \label{fig:prop-6-7}
\end{figure}

\subsection{Disuguaglianza triangolare}
\begin{definition}[Disuguaglianza triangolare]
    Dati due valore $a$ e $b$ tali che $a, b \in \mathbb{R}$ risulta che:
    \begin{equation}
        (1)\:\:\:|a + b| \leq |a| + |b| \hspace{.7cm} (2)\:\:\:||a| + |b|| \leq |a - b| 
    \end{equation}
\end{definition}
\begin{demostration}
    Dimostrazione proprietà (1):\\
    Dati due valori $a$ e $b$ calcoliamo il valore assoluto, che per la proprietà (6) in tabella \ref{tab:prop-valore-assoluto}  possiamo scrivere nella seguente forma:
    \begin{equation}
            -|a| \leq a \leq |a| \hspace{.6cm} -|b| \leq b \leq |b|
    \end{equation}
    Ora facciamo una somma di disuguaglianze fra le forme riportate sopra:
    \begin{equation}
        - |a| - |b| \leq a + b \leq |a| + |b|
    \end{equation}
    Possiamo vedere la prima parte $-|a| - |b|$ come un -M, la parte $a + b$ come una $x$ e l'ultima parte $|a| + |b|$ come M. Utilizzando a questo punto la proprietà (7) in tabella \ref{tab:prop-valore-assoluto}, $|x| \leq M$ quindi:
    \begin{equation}
        |a + b| \leq |a| + |b|
    \end{equation}
\end{demostration}
\begin{observation}
Perché una disuguaglianza triangolare a 3 numeri, $|a + b + c| \leq |a| + |b| + |c|$, vale?\\ \\
Perché se $|a + b + c|$ lo dividiamo in $|(a + b) + c|$ possiamo applicare la propria triangolare su 2 valori considerando $(a+b)$ il primo e $c$ il secondo questo fa si che $|(a + b) + c| \leq |a + b| + |c|$ andando poi a riapplicare la disuguaglianza triangolare questa volta solo su $|a + b|$ vediamo che:
\begin{equation}
    |a + b + c| = |(a + b) + c| \leq |a + b| + |c| \leq |a| + |b| + |c|
\end{equation}
Da qui possiamo dedurre che la disuguaglianza triangolare vale indipendentemente dal numero di valori:
\begin{equation}
    |a_1, a_2, a_3, ..., a_n| \leq |a_1| + |a_2| + |a_3| + .... + |a_n|
\end{equation}
\end{observation}
\newpage
\section{Continuità}
\begin{definition}[Funzione continua]
    Dato un insieme A ed una funzione $f(x)$ tale che $A \subset \mathbb{R}$, $f: A \longrightarrow \mathbb{R}$, la funzione $f$ si dice \textbf{continua} in $x_0$ se $\forall \: \: \epsilon > 0 \: \: \exists \: \: \delta > 0$ tale che se data una $x \in A$:
    \begin{equation}\label{funzione-continua}
        |x - x_0| < \delta \Longrightarrow |f(x) - f(x_0)| < \epsilon
    \end{equation}
\end{definition}
La condizione scritta sopra [\ref{funzione-continua}] può essere scritta anche tramite due condizioni:
\begin{itemize}
    \item $|x - x_0| < \delta \Longleftrightarrow x_0 - \delta < x < x_0 + \delta$
    \item $|f(x) - f(x_0)| < \epsilon \Longleftrightarrow f(x_0) - \epsilon < f(x) < f(x_0) + \epsilon$
\end{itemize}
\begin{example}
Ora per capire meglio facciamo un esempio di funzione non continua:
\end{example}
Innanzitutto stabiliamo una $f(x)$ e verifichiamo che $f(x)$ in $x_0 = 0$ non è continua.\\ 
\begin{wrapfigure}{r}{8cm}
\vspace{-20pt}
    \centering
    \includegraphics[width=4.8cm, height=5cm]{images/esempio-non-continuita.png}
    \caption{funzione non continua}
    \label{fig:funzione-non-continua}
\end{wrapfigure}

$
  f(x)=\begin{cases}
    0 \: \: se & x \leq 0\\
    1 \: \: se & x > 0
  \end{cases}
$\\ \\ \\
Come prima cosa stabiliamo un $\epsilon = \frac{1}{2}$. Ora, qualunque sia $\delta > 0$ se andiamo a prendere una $x$ tale che $x \in (0, \delta) \Longrightarrow f(x) = 1$ quindi la disuguaglianza $f(x_0) - \epsilon > f(x) < f(x_0) + \epsilon$, che diventerebbe $0 - \frac{1}{2} < f(x) < 0 + \frac{1}{2}$, è falsa. \\ \\
Deduciamo quindi che in $x_0 = 0$ questa funzione non è continua. \\ \\

\begin{definition}
    Dato un insieme A ed una funzione $f(x)$ tale che $A \subset \mathbb{R}$, $f: A \longrightarrow \mathbb{R}$ ed un insieme $B \subset \mathbb{R}$ si dice che $f$ è continua in B e $f$ è continua in ogni punto $x_0 \in B$.
\end{definition}
Se invece si dice semplicemente che $f$ è continua senza specificare il sotto insieme B vuol dire che $f$ è continua in tutti i punti del suo dominio A.
\begin{example}
    Esempio basato sulla funzione vista sopra:\\\\
    $
      f(x)=\begin{cases}
        0 \: \: se & x \leq 0\\
        1 \: \: se & x > 0
      \end{cases}
    $ \hspace{1cm}
    $f$ è continua in $(-\infty, 0) \cup (0, +\infty)$ 
\end{example}

\subsection{Permanenza del segno}
\begin{theorem}[Permanenza del segno]\label{permanenza-segno}
    Dato un insieme A ed una funzione $f$ tale che $A \subset \mathbb{R}$, $f: A \longrightarrow \mathbb{R}$, $x_0 \in A$. Se $f$ è continua in $x_0$ e $f(x_0) > 0$ allora $\exists \: \: \delta > 0$ t.c. se $x \in A$ è $|x - x_0| < \delta \longrightarrow f(x) > 0$. Analogo risultato se $f(x_0) < 0$.
    \begin{demostration}
    Sappiamo che $f(x_0) > 0$. Ora scegliamo un $\epsilon = \frac{f(x_0)}{2}$ ed utilizziamolo nella definizione di continuità:
    \begin{equation}
        \exists \: \: \delta > 0 \: \: | \: \: x \in A, |x - x_0| < \delta \Longleftarrow |f(x) - f(x_0)| < \epsilon
    \end{equation}
    Ciò che risulta dalle condizioni poste dalla continuità è che:
    \begin{equation}
        f(x_0) - \epsilon < f(x) < f(x_0) + \epsilon
    \end{equation}
    Se prendiamo la prima parte $f(x_0) - \epsilon < f(x)$ e facciamo le dovute sostituzioni risulta che:
    \begin{equation}
        f(x_0) - \frac{f(x_0)}{2} < f(x)
    \end{equation}
    Visto che $f(x_0) - \frac{f(x_0)}{2}$ è sempre maggiore di 0 risulta anche che $f(x)$ è maggiore di 0. $\blacksquare$
    \end{demostration}
    \begin{corollaries}\label{collorartio-permanenza-segno}
        Se $f$ è continua in $x_0$ $f: A \longrightarrow \mathbb{R}$, $x_0 \in A$ e $f(x_0) > M$ con $M \in \mathbb{R}$, $x \in A$, $|x - x_0| < \delta \Longrightarrow f(x) > M$. (Vale anche con $f(x_0) < M \Longrightarrow f(x) < M$)
    \end{corollaries}
    \begin{demostration}[Dimostrazione del corollario \ref{collorartio-permanenza-segno}]
        La dimostrazione di questo corollario è immediata e si fa applicando al teorema precedente \ref{permanenza-segno} la funzione $g(x) = f(x) - M$, perché se la funzione $f(x) - M > 0$ è come dire $f(x) > M$. $\blacksquare$
    \end{demostration}
\end{theorem}

\subsection{Continuità con operazioni fra funzioni}
\begin{theorem}
    Prendendo due funzioni $f$ e $g$ continue in un punto $x_0$ allora le funzioni $f + g$, $f * g$ e $|f|$, se inoltre $f(x_0) \neq 0$ allora anche $\frac{1}{f}$ è continua.
    \begin{corollaries}
        Prendendo due funzioni $f$ e $g$ continue in un punto $x_0$ allora $\frac{f}{g}$ è continua se $g(x_0) \neq 0$
    \end{corollaries}
\end{theorem}

\subsection{Funzioni invertibili e continuità}
\begin{proposition}\label{proposizione-funzione-inversa}
    Prendendo due insiemi I (I deve essere un intervallo) e B tale che $I \subset \mathbb{R}$ e $B \subset \mathbb{R}$ ed una funzione $f: I \longrightarrow B$, se $f$ è continua in $I$ ed è invertibile allora $f^{-1}$ è continua in B.
\end{proposition}
\begin{observation}
    Possiamo osservare che ipotesi della proposizione \ref{proposizione-funzione-inversa} dice che il domino sia un intervallo, questo non può essere omesso. 
\end{observation}
\begin{example}
    Verifichiamo questa osservazione con un' esempio:\\
    Prendiamo una funzione $f(x)$ definita in $f: (-\infty, 1] \cup (2, +\infty) \longrightarrow \mathbb{R}$ \: \: \: $f(x) = 
    \begin{cases}
        x \: \: se & x \leq 1 \\
        x - 1 \: \: se & x > 1
    \end{cases}
    $\\
    Qui di seguito le rappresentazioni della funzione $f(x)$ e della sua inversa $f(x)^{-1}$
\end{example}
\begin{figure}[h!]
    \vspace{10pt}
    \begin{subfigure}{.5\textwidth}
        \centering
        \includegraphics[width=5.5cm]{images/esempio1-osservazione-prop1.png}
        \caption{Osservazione proposizione \ref{proposizione-funzione-inversa}, funzione $f(x)$}
        \label{fig:es1-prop1}
    \end{subfigure}
    \begin{subfigure}{.5\textwidth}
        \centering
        \includegraphics[width=4.2cm, height=5.2cm]{images/esempio2-osservzione-prop1.png}
        \caption{Osservazione proposizione \ref{proposizione-funzione-inversa}, funzione $f(x)^{-1}$}
        \label{fig:es2-prop1}
    \end{subfigure}
\end{figure}
\begin{itemize}
    \item \textbf{Domanda 1°:} $f$ è continua in $x_0 = 1$?\\
    La risposta a questa prima domanda è SI, essendo che noi andiamo a considerare solo i punti all'interno del dominio, quindi la parte compresa fra 1 e 2, dove la funzione presenta una discontinuità, non si considera.
    \item \textbf{Domanda 2°:} $f$ è continua in $x_0 = 2$?\\
La risposta in questo caso è che non ha senso considerare il punto $x_0 = 2$ visto che 2 non fa parte del dominio.
\end{itemize}
Quindi $f$ è continua in tutto il suo dominio. Essendo $f$ continua in tutto il suo dominio allora teoricamente $f^{-1}$ è una funzione invertibile. \\ \\
Possiamo però vedere che la funzione $f^{-1}$, figura [\ref{fig:es2-prop1}] non è continua in $x_0 = 1$ perché essendo la funzione inversa $f^{-1}$ è definita come $f^{-1}: \mathbb{R} \longrightarrow (-\infty, 1] \cup (2, +\infty)$, quindi dobbiamo considerare come dominio tutto $\mathbb{R}$, così facendo ci sono dei punti, in particolare con $x > 0$ che non rientrano nell'intervallo fra $f(x_0) - \epsilon$ e $f(x_0) + \epsilon$ . \\ \\
In conclusione da questo esempio deduciamo che, se $f$ non è definita in un intervallo potrebbe succedere che $f^{-1}$ non è continua anche se $f$ è continua.

\subsection{Continuità delle funzioni elementari}
$f(x) = x$ è una funzione continua. Da questa considerazione segue che tutte le funzioni con polinomi sono continue.
\begin{note}
    Ricorda che anche le funzioni costanti sono sempre continue
\end{note}
Definiamo in maniera generica così una funzione formata da polinomi continua:
\begin{equation}
    P(x) = a_n * x^n + a_{n-1} * x^{n-1} + .... + a_1 * x + a_0 \: \: con \: \: a_0, a_1, ..., a_n \in \mathbb{R}
\end{equation}
Quindi: $x^2 = x * x$ è continua \hspace{.5cm} $x^3 = x^2 * x$ è continua \hspace{.5cm} $x^n$ è continua $\: \: \forall x \in \mathbb{N}$ \\ \\
Le funzioni razionali sono continue nel loro insieme di definizione. Le funzioni razionali sono uguali a quoziente di polinomi:
$f(x) = \frac{p(x)}{q(x)}$ con $p,q$ polinomi, la funzione $f(x)$ è definita se $q(x) \neq 0$.\\ \\
Assumendo che $e^x$, $\sin{x}$, $\cos{x}$ sono funzioni continue quindi anche $\log{x}$, $\arcsin{x}$, $\arccos{x}$, $\tan{x}$, $\arctan{x}$ sono continue.

\subsection{Continuità fra composizione di funzioni}
\begin{theorem}
    Date due funzioni $f: A \longrightarrow \mathbb{R}$ e $g: B \longrightarrow \mathbb{R}$, ed un $x_0 \in A$, $y_0 = f(x_0) \in B$.
    Se $f$ è continua in $x_0$ e $g$ è continua in $y_0$ allora $g \bullet f$ è continua in $x_0$.
\end{theorem}
\begin{example}
    Facciamo un esempio usando la funzione $e^{\cos{x}}$.\\
    $e^{\cos{x}}$ è una funzione continua perché è la composizione di $f(x) = \cos{x}$, funzione continua, e $g(x) = e^y$, pure essa funzione continua.
\end{example}
\begin{observation}
    Data una $f: [a, b] \longrightarrow \mathbb{R}$ continua in $[a, b]$ allora sup($f(x)$) con $x \in (a, b)$ = sup($f(x)$) con $x \in [a, b]$. E ugualmente inf($f(x)$) con $x \in (a, b)$ = inf($f(x)$) con $x \in [a, b]$.
\end{observation}
\begin{example}
    $f(x) = x^2$ con $f: [0,1] \longrightarrow \mathbb{R}$\\
    sup($f(x)$) = $f(1) = 1$ con $x \in [0, 1]$ \hspace{.5cm} sup($f(x)$) = $f(1) = 1$ con $x \in (0, 1)$ \footnote{Ricorda che sup(imm($f(x)$)) = sup(0, 1) = 1}
\end{example}

\subsection{Teorema degli zeri}
\begin{theorem}[Teorema degli zeri]
    Data una $f: [a, b] \longrightarrow \mathbb{R}$ continua. Se $f(a) \cdot f(b) < 0$ allora $\exists c \in (a, b)$ tale che $f(c) = 0$
\end{theorem}
Questo teorema dice che prendendo una funzione, che deve essere obbligatoriamente continua, se i valori di $f(x)$ nei due estremi moltiplicati fra di loro risultano minori di 0 la funzione passa per 0 in un ponto $c$ e questo accade perché se il prodotto fra i due estremi torni inferiore a 0 vuol dire che hanno segno discorde.\\
\begin{example}
    Facciamo un esempio di un caso in cui la funzione NON è continua:
\end{example}
\begin{wrapfigure}{r}{5cm}
    \vspace{-10pt}
    \centering
    \includegraphics[width=4.5cm, height=2cm]{images/esempio-teorema-zeri.png}
    \caption{$f(x) = [x] + \frac{1}{2}$}
    \label{fig:esempio-teorema-zeri}
\end{wrapfigure}
Prendiamo $f(x) = [x] + \frac{1}{2}$ \: $f: [1, -1] \longrightarrow \mathbb{R}$\\ \\
Se ora prendiamo la f(x) nei due estremi e facciamo il prodotto torna che: \\
$f(1) \cdot f(-1) < 0$ ma $\nexists x \in [-1.1]$ t.c. $f(x) = 0$ come possiamo vedere nell'immagine \ref{fig:esempio-teorema-zeri}.\\ \\

\subsection{Teorema valori intermedi}
\begin{theorem}[Teorema dei valori intermedi]
    Prendendo un intervallo $I \subset R$, ed una funzione $f: I \longrightarrow \mathbb{R}$ continua, allora $f(I)$ è un intervallo.
\end{theorem}
Questo teorema dice che se il nostro dominio è un intervallo e la $f$ è continua all'ora anche il codominio o immagine di $f$ sarà un intervallo.
\begin{corollaries}
    Prendendo sempre un $I \subset R$, una funzione $f: I \longrightarrow \mathbb{R}$ continua, se $f$ assume $y_1$ e $y_2$ allora assume anche tutti i valori compresi fra $y_1$ e $y_2$.
\end{corollaries}

\subsection{Teorema di Weirstrass}
\begin{theorem}[Teorema di Weirstrass]
    Data una funzione $f: [a, b] \longrightarrow \mathbb{R}$ continua. Allora f ha massimo e minimo.
\end{theorem}
\begin{note}
    Notare che $a,b \in \mathbb{R}$ e non in $\overline{\mathbb{R}}$ perché $a, b \neq \pm \infty$ e gli estremi devono essere compresi.
\end{note}
\begin{example}
    Facciamo ora un esempio per confermare come il teorema di Weirstrass possa valore solo con un intervallo chiuso:\\
    Dato $f(x) = \frac{1}{x}$ con $f(x): (0, 1] \longrightarrow \mathbb{R}$ \\ \\in questo caso $f$ ha come dominio un intervallo non chiuso a sinistra\\
    $f$ è continua ma non ha max perché sup($f$) = $+\infty$
\end{example}
\begin{example}
    Facciamo ora un esempio per confermare come il teorema di Weirstrass possa valore solo con un intervallo limitato:\\
    Dato $f(x) = \arctan{x}$ con $f(x): \mathbb{R} \longrightarrow \mathbb{R}$ \\ \\in questo caso $f$ è una funzione continua definita come $-\frac{\pi}{1} < f(x) < \frac{\pi}{2}$\\
    Possiamo notare però che f non toccherà mai ne $-\frac{\pi}{2}$ ne $\frac{\pi}{2}$ e quindi non ha ne massimo ne minimo.
\end{example}
\newpage
\section{Limiti}
\subsection{Intorni}
\begin{definition}[Intorno]
    Dato $x_0 \in \mathbb{R}$ si dice \textbf{intorno} di $x_0$ un insieme del tipo $(x_0 - \epsilon, x_0 + \epsilon)$ dove $\epsilon \in \mathbb{R}$, e $\epsilon > 0$. Inoltre $\epsilon$ si dice raggio dell'intorno
\end{definition}
\begin{itemize}
    \item Un insieme del tipo $[x_0, x_0 + \epsilon]$ si dice \textbf{intorno destro} di $x_0$.
    \item Un insieme del tipo $[x_0 - \epsilon, x_0]$ si dice \textbf{intorno sinistro} di $x_0$.
\end{itemize}
\begin{definition}
    Se $x_0 = +\infty$ un intorno di $x_0$ è un insieme del tipo $(a, +\infty)$\footnote{$(a, +\infty)$ è una semiretta} dove $a \in \mathbb{R}$
\end{definition}
\begin{definition}[Punto di accumulazione]
    Dato $A \subset \mathbb{R}$ e $x_0 \in \overline{R}$ $x_0$ si dice \textbf{punto di accumulazione} per A se $\forall \: U$ intorno di $x_0$ risulta che $U \cap A \setminus \{x_0\} \neq 0$
\end{definition}
Questa definizione vuol dice che "vicino" a $x_0$ ci sono altri punti di A oltre a $x_0$ ($x_0$ potrebbe anche non appartenere ad A).
\begin{example}
    Prendiamo un intervallo A = (2, 3).
\end{example}
Se prendiamo un punto $x_0$ che appartiene a A, quindi $x_0 \in (a,b)$, allora ogni intorno di $x_0$ interseca A in infiniti punti, quindi $x_0$ è un punto di accumulazione di A.\\
\begin{definition}[Intorno bucato]
    Se invece non andiamo a considerare $x_0$ nel suo intorno si dice \textbf{Intorno bucato} e si scrive come $\{x_0 - \epsilon, x_0 + \epsilon\} \setminus \{x_0\}$
\end{definition}
\begin{wrapfigure}{r}{8cm}
    \centering
    \includegraphics[width=7cm]{images/esempio-intorno-su-estremi.png}
    \caption{Punto di acc $x_0 = 2$ dell'intervallo A}
\end{wrapfigure}

Ora andiamo a dimostrare come tutti i punti [2,3] $\in$ acc(A).
Se poniamo per esempio $x_0 = 2$. Se andiamo a prendere un intorno di $x_0$ nonostante il $\epsilon$ possa essere piccolissimo esisteranno sempre infiniti punti nell'intersezione fra $U$ intorno e $A$ ($U \cap A \setminus \{x_0\}$) perché qualsiasi sia l'epsilon $2 + \epsilon$ rientrerà sempre in A.\\
Questo anche con $x_0 = 3$. 
\begin{note}
Nota che oltre a tutto [2,3] $\in$ A non esisto altri punti di accumulazione di un intervallo A.
\end{note}

\begin{definition}[Punto isolato]
    Dato un insieme A, $x_0 \in A$ si dice \textbf{punto isolato} di A se esiste un $U$ intorno di $x_0$ tale che $U \cap A = \{x_0\}$
\end{definition}
\begin{example}
    Facciamo un osservazione con un intorno spezzato per vedere un caso di punto isolato.
\end{example}
\begin{wrapfigure}{l}{8cm}
    \centering
    \includegraphics[width=7.3cm, height=2.7cm]{images/esempio-intorno-sepezzato.png}
    \caption{Punto di acc $x_0 = 5$ dell'intervallo C}
\end{wrapfigure}

Se prendiamo un punto C = $(2,3) \cup \{5\}$ non possiamo dire che tutti i punti dell'intervallo C siano punti di accumulazione perché se prendiamo $x_0 = 5$ possono esistere dei casi in cui il suo intorno non interseca C (con U intorno di $x_0 = 5$, $U \cap C \setminus {5} = \O$).\\
Diciamo quindi che in questo caso acc(c) = [2,3]\\\\
\begin{example}
    Esempio in cui verifichiamo come, dato un insieme D = $(3, +\infty)$, sia $+\infty \in$ acc(D).\\
    Come prima cosa prendiamo un $U$ intorno di $x_0 = +\infty$. Quindi $U = (a, +\infty)$.\\
    Definiamo ora il punto maggiore fra 3 ed $a$, $b =$ max($3,a$), questo punto sarà l'estremo sinistro dei punti di accumulazione.
    Facciamo ora l'intersezione:
    \begin{center}
        $U \cap D \setminus \{x_0\} = (2, +\infty) \cap (a, +\infty) \setminus {+\infty} = (b, +\infty) \neq \O$.
    \end{center}
    Vediamo dunque che $+\infty$ è un punto i accumulazione di D, quindi acc(D) = $[b, +\infty]$.
\end{example}
\newpage
\begin{example}
    Esempio prendendo come insieme $E = \mathbb{N}$.
\end{example}
\begin{wrapfigure}{r}{7cm}
    \vspace{-15pt}
    \centering
    \includegraphics[width=6.2cm]{images/insieme-N.png}
    \caption{Insieme $\mathbb{N}$}
    \label{fig:insieme-N}
\end{wrapfigure}

Se osserviamo l'immagine \ref{fig:insieme-N} vediamo chiaramente come tutti gli elemento di $\mathbb{N}$ sia punti isolare e quindi non siano punti di accumulazione. Ma, per l'esempio visto sopra, $+\infty$ è l'unico punto di accumulazione di $\mathbb{R}$. Acc($\mathbb{N}$) = $+\infty$. \\
\begin{note}
    Allo stesso modo prendendo in considerazione l'insieme $\mathbb{Z}$ i suoi punti di accumulazione sono acc($\mathbb{Z}$) = $\{-\infty, +\infty\}$
\end{note}
\begin{definition}
    Dato un insieme $A \subset \mathbb{R}$, ed un $x_0 \in A$, si dice $x_0$ punto interno ad A se esiste un $U$ intorno di $x_0$ tale che $U \subset A$. L'insieme dei punti interni si indica con int(A).
\end{definition}
\begin{example}
    Dato un A = [3, 5] i punti intesi sono (3,5) e non [3,5] perché se prendiamo $x_0 = 3$ o $x_0 = 5$ essendo che l'intorno di $x_0$ è [$x_0 - \epsilon$, $x_0 + \epsilon$] rimarrà sempre una parte fuori, in particolare quella di sinistra per $x_0 = 3$, e quella di destra per $x_0 = 5$.
\end{example}

\subsubsection{Minimi e massimi locali}
\begin{definition}[Minimi e massimi locali e locali stretti]
    Dato un insieme $A \subset \mathbb{R}$, una funzione $f: A \longrightarrow \mathbb{R}$ ed un punto $x_0 \in A$ si dice che $x_0$ è:
    \begin{itemize}
        \item \textbf{Minimo locale} (o relativo) se esiste un $U$ intorno di $x_0$ tale che $f(x) \geq f(x_0) \: \forall \: x \in U \cap A$
        \item \textbf{Minimo locale stretto} se esiste un $U$ intorno di $x_0$ tale che $f(x) > f(x_0) \: \forall \: x \in U \cap A \setminus \{x_0\}$
        \item \textbf{Massimo locale} (o relativo) se esiste un $U$ intorno di $x_0$ tale che $f(x) \leq f(x_0) \: \forall \: x \in U \cap A$
        \item \textbf{Massimo locale stretto} se esiste un $U$ intorno di $x_0$ tale che $f(x) < f(x_0) \: \forall \: x \in U \cap A \setminus \{x_0\}$
    \end{itemize}
\end{definition}
Questa definizione vuol dire che se andiamo a prendere un intorno di $x_0$, il punto $x_0$ può essere definito minimo o massimo di quel determinato intorno se è il punto più "in basso" o più "in alto" rispetto a tutti gli altri punti dell'intorno.
\begin{figure}[h!]
    \begin{subfigure}{.5\textwidth}
        \centering
        \includegraphics[width=7.7cm]{images/min-max-locale.png}
        \caption{Minimo e massimo locale}
        \label{fig:min-max-locale}
    \end{subfigure}
    \begin{subfigure}{.5\textwidth}
        \centering
        \includegraphics[width=5.5cm]{images/min-max-locale-stretto.png}
        \caption{Minimo e massimo locale stretto}
        \label{fig:min-max-locale-stretto}
    \end{subfigure}
\end{figure}

Come si può vedere dalle immagini [\ref{fig:min-max-locale}] [\ref{fig:min-max-locale-stretto}] noi andiamo a considerare solo i punti all'interno dell'intorno di $x_0$, infatti esisterebbero altri punti esterni a $U$ intorno maggiori o minori, ma non li consideriamo.
\begin{note}
    Nota che se $x_0$ è punto di minimo allora è anche punto di minimo locale, qualsiasi sia l'intorno che prendiamo in considerazione.
\end{note}

\subsection{I limiti}
\begin{definition}[Limite]
    Dato un $A \subset \mathbb{R}$, una $f: A \longrightarrow \mathbb{R}$, ed un $x_0$ punto di accumulazione per A, si dice che $l \in \overline{\mathbb{R}}$ è il limite per $x$ che tende a $x_0$ di $f(x)$ se $\forall$ V intorno di $l$, $\exists \: U$ intorno di $x_0$ t.c. $x \in U \cap A \setminus \{x_0\} \Longrightarrow f(x) \in V$
\end{definition}
Questa definizione dice che un valore $l$ per essere definito come limite di una funzione con $x$ che tende a $x_0$ bisogna che per qualsiasi intorno che andiamo a prendere di $l$ deve esistere una intorno di $x_0$ chiamato U tale che, se una $x$ appartiene ad U allora la $f(x)$ apparterrà all'intorno di $l$. \\
Se ci rifacciamo alle definizioni di intorno vediamo che $x \in U \cap A \setminus \{x_0\}$ vuol dire che $|x-x_0| < \delta$ e che $f(x) \in V $ vuol dire che $l - \epsilon < f(x_0) < l + \epsilon$.\\
Questa definizione può essere scritta in altre parole dicendo che:
\begin{center}
    $\lim\limits_{x\to x_0}f(x) = l$\footnote{La notazione $\lim\limits_{x\to x_0}f(x)$ è quella con cui andiamo a scrivere i limiti e vuol dire limite di $f(x)$ con $x$ che tende a $x_0$ è uguale a $l$ valore del limite} $ \Longleftrightarrow \forall \epsilon > 0 \: \: \exists \delta > 0 $ tale che $x \in A, |x - x_0| < \delta \land x \neq x_0 \Longrightarrow |f(x) - f(x_0)| < \epsilon$
\end{center}
\begin{example}
Alcuni esempi di limiti:
\begin{itemize}
    \item $\lim\limits_{x\to x_0}f(x) = \pm \infty$ \hspace{.5cm} $V = (a, \pm \infty)$ \hspace{.5cm} $f(x) \in V$ se e solo se $f(x) > a$\\
    Il risultato di questo limite è $\pm \infty$ se $\forall a \in \mathbb{R} \: \exists \: \delta > 0$ t.c. $|x-x_0|<\delta, x \in A, x\neq x_0 \Longrightarrow f(x) > a$
    \item $\lim\limits_{x\to \pm \infty}f(x) = l$ \hspace{.5cm} se $l \in \mathbb{R}$ se e solo se $x \to \infty$\\
    Il risultato del limite è un valore appartenete a $\mathbb{R}$ se $\forall \epsilon > a \: \exists a \in \mathbb{R}$ t.c. $x > a \Longrightarrow |f(x) - l| < \epsilon$
    \item $\lim\limits_{x\to \pm \infty}f(x) = \pm \infty$ \:
    se e solo se $\forall a \in \mathbb{R} \exists b \in \mathbb{R}$ t.c. $x > b \Longrightarrow f(x) > a$
\end{itemize}
\end{example}
\begin{theorem}[Unicità dei limiti]
Se esiste un limite di $f$ con $x \to x_0$, questo limite è unico.
\end{theorem}

\subsection{Continuità con i limiti}
Rivediamo le definizioni di limiti (con il limite che sia un numero finito)  e continuità accanto:
\begin{enumerate}
    \item $\lim\limits_{x\to x_0}f(x) = l$ con $x_0 \in A$, $l \in \mathbb{R}$ è vera se e solo se $\forall\epsilon > 0 \: \: \exists \delta >0$ t.c. $x \in A, x \neq x_0$ è $|x - x_0| < \delta \Longrightarrow |f(x) - l| < \epsilon$
    \item $f$ è continua in $x_0$ se e solo se $\forall \epsilon > 0 \: \: \exists \delta > 0$ t.c. $|x - x_0| < \delta$ con $x \in A \Longrightarrow |f(x) - f(x_0)| < \epsilon$
\end{enumerate}
Notiamo subito che fra la definizione (1) e la (2) c'è come unica differenza che nella prima c'è $l$ mentre nella seconda c'è $f(x)$. Possiamo dunque trarre una serie di osservazioni.
\begin{observation}
Data una funzione $f(x)$ essa è continua in $x_0 \Longrightarrow \lim\limits_{x\to x_0}f(x) = l$
\end{observation}
\begin{observation}
Una funzione è sempre continua nei punti isolati.
\end{observation}
\begin{observation}
Nella definizione di limite non serve che $x_0$ sia nel dominio di una funzione, basta che sia un punto di accumulazione per il dominio.
\end{observation}
\begin{example}
Esempio di continuità con i limiti:
\end{example}
\begin{wrapfigure}{r}{6cm}
    \vspace{-50pt}
    \centering
    \includegraphics[width=5cm, height=4.7cm]{images/es-continuita-limiti.png}
    \caption{$\lim\limits_{x\to 0}f(x) = 3$}
\end{wrapfigure}

$f(x) = 
    \begin{cases}
        3 \: \: se & x \neq 0 \\
        2 \: \: se & x = 0
    \end{cases}
    $\\ \\
$\lim\limits_{x\to 0}f(x) = 3$, senza considerare f in $x = 0$.\\
Secondo la definizione di continuità di una funzione vista sopra (dove andiamo a guardare il valore del limite in $x_0$):
\begin{center}
    $|x - x_0| < \delta$, $x \in A$, $x \neq x_0$ allora $|f(x) - l| < \epsilon$.
\end{center}
Se andiamo però a vedere $\lim\limits_{x\to 0}f(x) = 3$ mentre $f(0) = 2$ e ovviamente $2 \neq 3$ quindi f non è continua in $x_0$.



\subsection{Limite destro e sinistro}
\begin{definition}[Limite destro e sinistro]
    Se dato un $A \subset \mathbb{R}$, un $x_0 \in Acc(A)$, un $x_0 \in \mathbb{R}$ ($x_0$ deve essere un numero finito), ed  $f: A \to \mathbb{R}$, allora si dice che $l \in \overline{\mathbb{R}}$ è il limite di $f(x)$ per x che tende a $x_0$ da \textbf{destra} (si scrive come $\lim\limits_{x\to x_0^+}f(x) = l$) se:
    \begin{center}
        $\forall \: V$ intorno di $l \: \exists \: \delta > 0$ t.c. $x_0 < x < x_0 + \delta$, $x \in A \Longrightarrow f(x) \in V$
    \end{center}
    Si dice limite \textbf{sinistro} (si scrive come $\lim\limits_{x\to x_0^-}f(x) = l$) se:
    \begin{center}
        $\forall \: V$ intorno di $l \: \exists \: \delta > 0$ t.c. $x_0 - \delta < x < x_0$, $x \in A \Longrightarrow f(x) \in V$
    \end{center}
\end{definition}
\begin{example}
Se prendiamo una $f: (-\infty, 0) \cup (0, +\infty) \to \mathbb{R}$, 
$f(x) = 
    \begin{cases}
        -1 \: \: se & x < 0 \\
        1 \: \: se & x > 0
    \end{cases}
    $ \\ 
    Il $\lim\limits_{x \to 0^+} f(x) = 1$ mentre $\lim\limits_{x \to 0^-} f(x) = -1$. Ciò perché andiamo nel caso del limite destro a guardare il valore "alla destra" di 0 e nel limite sinistro il valore "alla sinistra".
\end{example}
\begin{observation}
$\lim\limits_{x \to x_0^+} = l$ se e solo se $\lim\limits_{x \to x_0^+} = l_1$, $\lim\limits_{x \to x_0^-} = l_2$ e $l_1 = l_2$. Cioè per far in modo che il limite di una funzione che tende ad un valore $x_0$ sia unico bisogna che il limite destre e quello sinistro siano uguali. Nell'esempio precedente infatti possiamo notare che non esiste un unico limite perché i valori del destro e del sinistro sono diversi.
\end{observation}

\subsection{Limite da sopra e da sotto}
Dato un $A \subset \mathbb{R}$, una $f: A \to \mathbb{R}$, ed un $x_0 \in Acc(A)$
\begin{definition}
 Si dice che $\lim\limits_{x \to x_0}f(x) = l^+$ (con $l \in \mathbb{R}$) se $\lim\limits_{x\to x_0}f(x) = l$ ed esiste un $U$ intorno di $x_0$ t.c. $x \in U \cap A \setminus \{x_0\} \Longrightarrow f(x) > l$
\end{definition}
\begin{definition}
 Mentre analogamente si dice che $\lim\limits_{x \to x_0}f(x) = l^-$ (con $l \in \mathbb{R}$) se $\lim\limits_{x\to x_0}f(x) = l$ ed esiste un $U$ intorno di $x_0$ t.c. $x \in U \cap A \setminus \{x_0\} \Longrightarrow f(x) < l$
\end{definition}
Queste due definizione vogliono dire che la funzione può tendere ad un valore "da sopra" nel caso del + e "da sotto" nel caso del -.
\begin{figure}[h!]
    \begin{subfigure}{.5\textwidth}
        \centering
        \includegraphics[width=5.5cm]{images/limite-tende-sopra.png}
        \caption{Limite che tende da sopra}
    \end{subfigure}
    \begin{subfigure}{.5\textwidth}
        \centering
        \includegraphics[width=5.5cm]{images/lim-tende-sotto.png}
        \caption{Limite che tende sa sotto}
    \end{subfigure}
\end{figure}
\begin{example}
Un esempio è con $f(x) = \frac{1}{x}$ dove $\lim\limits_{x\to x_0}f(x) = 0^+$
\end{example}

\subsection{Permanenza del segno}
\begin{theorem}[Permanenza del segno]
Dato un $A \subset \mathbb{R}$, un $x_0 \in Acc(A)$ se esiste $\lim\limits_{x\to x_0}f(x) = l$, dove $l \in \overline{\mathbb{R}}$ e $l \neq 0$ allora esiste un intorno $U$ di $x_0$ t.c se $x \in U \cap A \setminus \{x_0\}$ allora $f(x)$ ha lo stesso segno di $l$.
\end{theorem}
\begin{example}
$f: (0, +\infty) \to \mathbb{R}$ \hspace{.5cm} $f(x) = \frac{1}{x}$ \hspace{.5cm} $\lim\limits_{x\to 0^+}f(x) = +\infty$ \\
Quindi visto che $+\infty > 0$ se prendiamo un intorno di $x_0$ qualsiasi $f(x)$ con $x$ appartenente all'intersezione fra il dominio e l'intorno (escluso $x_0$) tornerà che $f(x) > 0$. 
\end{example}

\subsection{Non esistenza di un limite}
Ci sono casistiche di funzioni nel quale un limite non esiste, e quindi no può essere calcolato. Per verificare ciò vediamo alcuni esempi.
\begin{example}
$\lim\limits_{x\to x_0} \sin(x)$ Non esiste. Vediamo perché.
\end{example}
\begin{wrapfigure}{l}{8cm}
    \vspace{-10pt}
    \centering
    \includegraphics[width=7cm, height=3cm]{images/es-limite-non-esiste.png}
    \caption{Limite che non esiste}
    \label{fig:limite-non-esiste}
\end{wrapfigure}

Supponiamo per assurdo che: \\$\lim\limits_{x\to x_0} \sin(x) = l$\\ \footnote{Ricorda che per la definizione di limite la f(x) deve essere compresa fra $f(x_0) + \epsilon$ e $f(x_0) - \epsilon$ qualsiasi sia il valore di $\epsilon$}Prendiamo ora un valore $\epsilon < \frac{1}{2}$. \\ \\
Se esistesse il limite $l \in \mathbb{R}$ allora dovrebbe esistere $a > 0$ t.c. $x > a \Longrightarrow l - \epsilon < \sin{x} < l + \epsilon$ ma questo assurdo perché vorrebbe dire che $\sin{x}$ oscilla con ampiezza minore di $2\epsilon$ mentre $\sin{x}$ oscilla con ampiezza 2. 
\begin{note}
Nota che nell'immagine \ref{fig:limite-non-esiste} le parti rosse escono dall''intervallo $[l-\epsilon, l+\epsilon]$.
\end{note}

\subsection{Continuità destra e sinistra}
\begin{definition}[Continuità destra e sinistra]\label{continuità-destra-sinistra}
Dato un $A \subset \mathbb{R}$, un $x_0 \in Acc(A)$:
\begin{itemize}
    \item se $\lim\limits_{x \to x_0^+}f(x) = f(x_0)$ allora si dice che $f$ è \textbf{continua a destra} in $x_0$.
    \item se $\lim\limits_{x \to x_0^-}f(x) = f(x_0)$ allora si dice che $f$ è \textbf{continua a sinistra} in $x_0$.
\end{itemize}
\end{definition}

\begin{example}
Data una $
f(x) = \begin{cases}
    1 \: \: se & x \geq 0 \\
    -1 \: \: se & x < 0
\end{cases}$\\
Il $\lim\limits_{0^+}f(x) = 1$ mentre $\lim\limits_{0^-}f(x) = -1$\\
Questo esempio ci dice, come spiegato nella definizione sopra (\ref{continuità-destra-sinistra}), che la funzione è continua a destra nel caso di $0^+$ mentre con $0^-$ la funzione non è continua a sinistra perché il risultato del limite $l \neq f(x_0)$. 
\end{example}
\begin{observation}
    Nel esempio sopra possiamo vedere che la funzione è continua in $x_0^+$ ma non in $x_0^-$. Sin può osservare infatti come una funzione $f$ è continua in un punto $x_0$ se e solo se è continua sia a destra che ha sinistra, perché ciò vorrebbe dire che entrambi i limiti, quello da $x_0^-$ e $x_0^+$, avrebbero uno stesso risultato:
    \begin{center}
        $\lim\limits_{x\to x_0^+}f(x) = l_1$ \:\:\: $\lim\limits_{x\to x_0^-}f(x) = l_2$ \:\:\: $l_1 = l_2 = f(x_0)$
    \end{center}
\end{observation}

\subsection{Teorema di confronto}
\begin{theorem}[Teorema di confronto]
    Dato un $A \subset \mathbb{R}$, un $x_0 \in Acc(x)$, e due funzioni $f,g: A \to \mathbb{R}$. Se esiste un $\lim\limits_{x\to x_0}f(x) = l_1$ e $\lim\limits_{x\to x_0}g(x) = l_2$ e se esiste un $U$ intorno di $x_0$ t.c. $x \in U \cap A \setminus \{x_0\}$ e $f(x) \leq g(x)$ allora $l_1 \leq l_2$.
\end{theorem}
Questo teorema in maniera sintetica dice che se una funzione "sta sotto" l'altra a sua volta anche il limite della prima starà sotto il secondo, detto in altre parole la disuguaglianza passa ai limiti:
\begin{center}
    Se $f(x) \leq g(x)$ allora $\lim\limits_{x\to x_0}f(x) \leq \lim\limits_{x\to x_0}g(x)$
\end{center}

\begin{observation}
    Se però esiste $f(x) < g(x)$ non potrei dire che $\lim\limits_{x\to x_0}f(x) < \lim\limits_{x\to x_0}g(x)$. Perché:
\end{observation}
Se prendiamo come esempio due funzioni una $f(x) = -\frac{1}{x}$ e una $g(x) = \frac{1}{x}$ vediamo che $f(x) < g(x)$ ma se calcoliamo i limiti $\lim\limits_{x\to x_0}f(x) = 0$ e $\lim\limits_{x\to x_0}g(x) = 0$ e quindi i limiti sono uguali. Possiamo dunque dire che le disuguaglianze passano al limite ma diventano sempre deboli:
\begin{center}
        Se $f(x) < g(x)$ allora $\lim\limits_{x\to x_0}f(x) \leq \lim\limits_{x\to x_0}g(x)$
\end{center}

\subsection{Teorema somma e prodotto}
\begin{theorem}[Teorema somma e prodotto]
    Dato un $A \subset \mathbb{R}$, un $x_0 \in Acc(A)$, e due funzioni $f,g: A \to \mathbb{R}$. Supponiamo che esistano i limiti $\lim\limits_{x\to x_0}f(x) = l_1$ e $\lim\limits_{x\to x_0}g(x) = l_2$ con $l_1, l_2 \in \overline{\mathbb{R}}$.
    \begin{itemize}
        \item Se ha senso $l_1 + l_2$ allora esiste $\lim\limits_{x\to x_0}(f + g)(x) = l_1 + l_2$.
        \item Se ha senso $l_1 \cdot l_2$ allora esiste $\lim\limits_{x\to x_0}(f + g)(x) = l_1 \cdot l_2$.
    \end{itemize}
\end{theorem}
\begin{note}
Sono esclusi i casi $l_1 = +\infty$ e $l_2 = -\infty$ (o viceversa) per il prodotto. Sono invece esclusi i casi $l_1 = 0$ e $l_2 = \pm\infty$ (o viceversa) per la somma. Questi casistiche sono dette indeterminate e non possono essere calcolate in maniera diretta.
\end{note}

\subsection{Teorema dei carabinieri}
\begin{theorem}[Teorema dei carabinieri]
    Dato un $A \subset \mathbb{R}$, un $x_0 \in Acc(A)$, e due funzioni $f,g,h: A \to \mathbb{R}$. Se esiste $\lim\limits_{x\to x_0}f(x) = l$ e $\lim\limits_{x\to x_0}h(x) = l$ (i due limiti hanno lo stesso risultato) e se esiste un intorno $U$ di $x_0$ t.c. $x \in A \cup U \setminus \{x_0\}$, se $f(x) \leq g(x) \leq h(x)$ allora esiste $\lim\limits_{x\to x_0}g(x) = l$.
\end{theorem}
Il teorema dei carabinieri dice in maniera sintetica che se due funzioni hanno lo stesso limite ed una è inferiore all'altra se esiste una $g(x)$ in mezzo a queste due funzioni avrà lo stesso limite per uno stesso $x_0$, quindi dall'esistenza dei limiti di $f$ e $h$ (uguali) deduco l'esistenza del limite di $g$
\begin{example}
Facciamo un esempio prendendo $\lim\limits_{x\to +\infty}\frac{2 + \sin{(x)}}{x}$.
Prendendo due funzioni $f(x) = \frac{1}{x}$ e $h(x) = \frac{3}{x}$ sapiamo che $\frac{1}{x} \leq \frac{2 + \sin{(x)}}{x} \leq \frac{3}{x}$. \\
Se poi andiamo a calcolare i limiti per $x \to +\infty$ di $f(x)$ e di $h(x)$ vediamo che $\lim\limits_{x\to +\infty}f(x) = 0$ e $\lim\limits_{x\to +\infty}h(x) = 0$.
Allora per il teorema dei carabinieri $\lim\limits_{x\to +\infty}\frac{2 + \sin{(x)}}{x} = 0$
\end{example}

Alcune conseguenze del teorema dei carabinieri visto sopra:
\begin{proposition}
Dato un $A \subset \mathbb{R}$, un $x_0 \in Acc(A)$, e due funzioni $f,g: A \to \mathbb{R}$:
\begin{itemize}
    \item Se $f$è lim. inferiormente in intorno di $x_0$ e $\lim\limits_{x\to x_0}g(x) = +\infty \Longrightarrow \lim\limits_{x\to x_0}(f + g)(x) = +\infty$.
    \item Se $f$è lim. superiormente in intorno di $x_0$ e $\lim\limits_{x\to x_0}g(x) = -\infty \Longrightarrow \lim\limits_{x\to x_0}(f + g)(x) = -\infty$.
    \item Se $f$è limitata in un intorno di $x_0$ e $\lim\limits_{x\to x_0}g(x) = 0 \Longrightarrow \lim\limits_{x\to x_0}(f \cdot g)(x) = 0$.
\end{itemize}
\end{proposition}

\begin{example}
Prendiamo il $\lim\limits_{x\to +\infty}x + \sin(x)$\\
$\lim\limits_{x\to +\infty}x = +\infty$ \hspace{.5cm} $\lim\limits_{x\to +\infty}\sin(x)$ non esiste.\\
Data l'inesistenza del secondo limite non posso applicare il teorema sul limite della somma ma $\sin(x)$ è limitata inferiormente quindi:
Per il teorema dei carabinieri $x - 1 \leq x + \sin(x) \leq x + 2$, e visto che $\lim\limits_{x\to +\infty}x - 1 = +\infty$ e $\lim\limits_{x\to +\infty}x + 2 = +\infty$ possiamo dire che $\lim\limits_{x\to +\infty}\sin(x) = +\infty$
\end{example}

\subsection{Limitatezza funzione con i limiti}
\begin{theorem}
    Dato un $A \subset \mathbb{R}$, un $x_0 \in Acc(A)$, e $f: A \to \mathbb{R}$. Se esiste $\lim\limits_{x\to x_0}f(x) = l$ e $l \in \mathbb{R}$ (quindi $l$ non è $\pm\infty$) allora $f$ è limitata in un intorno di $x_0$ cioè $\exists \: U$ intorno di $x_0$ e $\exists M \in \mathbb{R}$ con $M > 0$ t.c. $x \in U \cap A \Longrightarrow |f(x)| \leq M$.
\end{theorem}
Questo teorema dice che se prendiamo una funzione che ha un limite per $x\to x_0$ che è un valore diverso da $\pm\infty$ e prendiamo un intorno di $x_0$ esisterà un valore M dove per qualsiasi $x \in U \cap A$ il $|f(x)| \leq M$ che corrisponderebbe a $-M \leq f(x) \leq M$ quindi la funzione sarà limitata nell'intorno selezionato.
\begin{example}
Se prendiamo $f(x) = \frac{1}{x}$ è limitata in un intorno di $+\infty$ perché $\lim\limits_{x\to x_0}f(x) = 0$.
\end{example}

\begin{definition}
Dato un $A \subset \mathbb{R}$, un $x_0 \in Acc(A)$, e $f: A \to \mathbb{R}$ possiamo dire che:
\begin{itemize}
    \item Se $\lim\limits_{x\to x_0}f(x) = 0$ allora si dice che $f$ è \textbf{infinitesima} per $x$ che tende a $x_0$.
    \item Se $\lim\limits_{x\to x_0}f(x) = +\infty$ allora si dice che $f$ è \textbf{diverge positivamente} per $x$ che tende a $x_0$.
    \item Se $\lim\limits_{x\to x_0}f(x) = -\infty$ allora si dice che $f$ è \textbf{diverge negativamente} per $x$ che tende a $x_0$.
    \item Se $\lim\limits_{x\to x_0}f(x) = l$ ed $l \in \mathbb{R}$ ($l$ è finito) allora si dice che $f$ è \textbf{converge} in $l$ per $x$ che tende a $x_0$.
\end{itemize}
\end{definition}

\subsection{Forme indeterminate}
\begin{table}[h!]
    \setlength{\tabcolsep}{7pt}
    \renewcommand{\arraystretch}{1.5}
    \centering
    \begin{tabular}{|c c c|}
        \hline
        $[1]$ $(+\infty) + (-\infty)$ & $[2]$ $(-\infty) + (+\infty)$ & $[3]$ $0 \cdot (\pm \infty)$ \\
        $[4]$ $(\pm \infty)^0$ & $[5]$ $(0^+)^0$ & $[6]$ $(1)^{\pm \infty}$\\ 
        \hline
    \end{tabular}
    \caption{Forme indeterminate}
\end{table}
\begin{demostration}
Dimostriamo come la forma [1] e la [2] siano indeterminate (facciamo un esempio considerandone una, ma sono equivalente).\\
Prendiamo un $f(x) = 2x$ e $g(x) = -x$ e facciamo i limiti di entrambi, ed il limite della somma.\\\\
$\lim\limits_{x\to +\infty}f(x) = +\infty$ e $\lim\limits_{x\to +\infty}g(x) = -\infty$, la somma $\lim\limits_{x\to +\infty}(f + g)(x) = 2x - x = x = +\infty$\\
In questo cosa il limite di $(+\infty) + (-\infty)$ torna $+\infty$.\\ \\
Ora prendiamo invece altre due funzioni $f(x) = \frac{x}{2}$ e $g(x) = -x$ e calcoliamo come prima i limiti di entrambi ed il limite della loro somma.\\\\
$\lim\limits_{x\to +\infty}f(x) = +\infty$ e $\lim\limits_{x\to +\infty}g(x) = -\infty$, la somma $\lim\limits_{x\to +\infty}(f + g)(x) = (\frac{x}{2} - x) = -\frac{x}{2} = -\infty$\\
In questo caso invece il limite di $(+\infty) + (-\infty)$ torna $-\infty$.\\\\
Alla domanda, quale scegliamo? La risposta è nessuna delle due, infatti non potendo avere un risultato fisso diciamo che questa è una forma indeterminata.\\
Nota che questa dimostrazione è valida anche per la forma $0 \cdot (\pm \infty)$.
\end{demostration}
Per le forme [4], [5] e [6] possiamo tramite dei calcoli algebrici spiegarle riconducendoci alle prime 3 forme.\\\\
Possiamo infatti vederle come $f(x)^{g(x)} = e^{\log(f(x)^{g(x)}}) = e^{g(x) \cdot \log(f(x))}$ e quindi possiamo analizzare i casi in cui $\lim\limits_{x\to x_0}g(x) \cdot \lim(f(x))$ è indeterminato:
\begin{enumerate}
    \setcounter{enumi}{3}
    \item Con $g\to 0$ e $f\to +\infty \Longrightarrow \log(f(x)) \to +\infty = 0 \cdot +\infty$ (quindi $(+\infty)^0$ è indeterminata).
    \item Con $g\to 0$ e $f\to +0^+ \Longrightarrow \log(f(x)) \to -\infty = 0 \cdot -\infty$ (quindi $(0^+)^0$ è indeterminata).
    \item Con $g\to \pm\infty$ e $f\to 1 \Longrightarrow \log(f(x)) \to 0 = 0 \cdot \pm\infty$ (quindi $(1)^{\pm\infty}$ è indeterminata).
\end{enumerate}

\subsection{Calcolo dei limiti}
\begin{proposition}
Dato un $A \subset \mathbb{R}$, un $x_0 \in Acc(A)$, e $f: A \to \mathbb{R}$ possiamo vedere che nel calcolare alcuni limiti si verificano delle situazioni ricorrenti:
\begin{itemize}
    \item Se $\lim\limits_{x\to x_0}f(x) = 0^+ \Longrightarrow \lim\limits_{x\to x_0}\frac{1}{f(x)} = +\infty$.
    \item Se $\lim\limits_{x\to x_0}f(x) = 0^- \Longrightarrow \lim\limits_{x\to x_0}\frac{1}{f(x)} = -\infty$.
    \item Se $\lim\limits_{x\to x_0}f(x) = +\infty \Longrightarrow \lim\limits_{x\to x_0}\frac{1}{f(x)} = 0^+$.
    \item Se $\lim\limits_{x\to x_0}f(x) = -\infty \Longrightarrow \lim\limits_{x\to x_0}\frac{1}{f(x)} = 0^-$.
    \item Se $\lim\limits_{x\to x_0}f(x) = l$ con $l \neq 0, \pm\infty \Longrightarrow \lim\limits_{x\to x_0}\frac{1}{f(x)} = \frac{1}{l}$.
\end{itemize}
\end{proposition}
\begin{note}
Nota che se abbiamo $\lim\limits_{x\to x_0}f(x) = 0$ (non $0^+$ o $0^-$) non si conclude nulla su $\lim\limits_{x\to x_0}\frac{1}{f(x)}$
\end{note}

\begin{proposition}
Dati due valori $a,b \in \overline{\mathbb{R}}$, una $f:(a,b) \to \mathbb{R}$ con $f$ debolmente crescente. Allora esistono $\lim\limits_{x\to a^+}f(x) = inf(f(x))$ quando $x \in (a,b)$ e $\lim\limits_{x\to b^-}f(x) = sup(f(x))$ con $x \in (a,b)$. (Analogamente con $f$ debolmente crescente)
\end{proposition}

\begin{example}
$f:(9, -\infty)\to \mathbb{R}$ con $f(x) = -\frac{1}{x}$\\
Se calcoliamo i limiti viene che $\lim\limits_{x\to 0^+}-\frac{1}{x} = +\infty = sup(f)$ \hspace{.1cm} mentre $\lim\limits_{x\to 0^-}-\frac{1}{x} = 0 = inf(f)$
\end{example}

\subsubsection{Limiti fondamentali}
\begin{table}[h!]
    \setlength{\tabcolsep}{7pt}
    \renewcommand{\arraystretch}{1.5}
    \centering
    \begin{tabular}{|c c|c|}
        \hline
        $\lim\limits_{x\to +\infty}x^n = +\infty$ & $\lim\limits_{x\to +\infty}\frac{1}{x^n} = \frac{1}{+\infty} = 0$ & $\lim\limits_{x\to +\infty}a^x = +\infty$ e $\lim\limits_{x\to -\infty}a^x = 0^+$ se $a \geq 1$ \\\hline
        $\lim\limits_{x\to +\infty}e^x = +\infty$ & $\lim\limits_{x\to -\infty}e^x = 0^+$ & $\lim\limits_{x\to +\infty}a^x = 1$ e $\lim\limits_{x\to -\infty}a^x = 1$ se $a = 1$  \\\hline
        $\lim\limits_{x\to 0^+}\log(x) = -\infty$ & $\lim\limits_{x\to +\infty}\log(x) = +\infty$ & $\lim\limits_{x\to +\infty}a^x = 0^+$ e $\lim\limits_{x\to -\infty}a^x = +\infty$ se $0 < a < 1$ \\
        \hline
    \end{tabular}
    \vspace{-5pt}
    \caption{Limiti fondamentali}
\end{table}
Questi limiti scritti sopra sono alcuni dei limiti fondamentali (considera quando c'è $n$ come $n\in \mathbb{N}$)

\subsubsection{Limiti di polinomi}
Se prendiamo una funzione generale così definitiva:
\begin{center}
    $p(x) = a_nx^n + a_{n-1}x^{n-1} + ... + a_1x + a_0$ con $a_0, a_1, ..., a_n \in \mathbb{R}$, $n$ è il grado del polinomio $n \in N$
\end{center}
è possibile trovare una standardizzazione per la risoluzione di $\lim\limits_{x\to +\infty}p(x)$
\begin{example}
Prendiamo in $\lim\limits_{x\to +\infty}3x^2 -7x + 1$.\\
Questa è una forma indeterminata $\lim\limits_{x\to +\infty}3x^2 -7x + 1 = +\infty -\infty + 1$, per risolvere si raccogliere:
\begin{center}
    $\lim\limits_{x\to +\infty}3x^2(1 - \frac{7x}{3x^2} + \frac{1}{3x^2}) = \lim\limits_{x\to +\infty}+\infty \cdot (1 - \frac{7x}{+\infty} + \frac{1}{+\infty}) = \lim\limits_{x\to +\infty}+\infty \cdot (1 - 0 + 0) = +\infty$
\end{center}
\end{example}
\hspace{-15pt}Come regola generale presa la funzione $p(x)$ scritta sopra risolviamo il limite tendente a $\pm\infty$ raccogliendo:
\begin{center}
    \vspace{-10pt}
    $\lim\limits_{x\to \pm\infty}p(x) = \lim\limits_{x\to \pm\infty}a_nx^n (1 + \frac{a_{n-1}}{a_n} \cdot \frac{x^{n-1}}{x^n} + ... + \frac{a_{1}}{a_n} \cdot \frac{x}{x^n} + \frac{a_{0}}{a_n} \cdot \frac{1}{x^n})$
\end{center}
Poi visto che i vari $\frac{x^{n-1}}{x^n}$, $\frac{x}{x^n}$ ecc. si annullano e quindi:
\begin{center}
    $\lim\limits_{x\to \pm\infty} a_nx^4 + a_{n-1}x^{n-1} + ... + a_1x + a_0 = \lim\limits_{x\to \pm\infty}a_nx^n$
\end{center}

\subsubsection{Funzioni razionali}
Se prendiamo una situazione $\frac{p(x)}{q(x)}$ con $p,q$ due polinomi quindi
\begin{center}
    $p(x) = a_nx^n + ... + a_1x + a_0$ \hspace{1cm} $q(x) = b_mx^m + ... + b_1x + b_0$
\end{center}
Possiamo sviluppare il limite seguendo la logica vista nei singoli limiti di polinomi:
\begin{center}
    $\lim\limits_{x\to \pm\infty} = \lim\limits_{x\to \pm\infty} \frac{a_nx^n (1 + \frac{a_{n-1}}{a_n}\cdot\frac{x^{n-1}}{x^n} + ... + \frac{a_0}{a_n}\cdot\frac{1}{x^n})}{b_nx^n (1 + \frac{b_{n-1}}{b_n}\cdot\frac{x^{n-1}}{x^n} + ... + \frac{b_0}{b_n}\cdot\frac{1}{x^n})} = \lim\limits_{x\to \pm\infty}\frac{a_nx^n}{b_mx^n}$
\end{center}

\begin{example}
$\lim\limits_{x\to +\infty}\frac{7x^4 + 5x^2}{-2x^3 + x} = \lim\limits_{x\to +\infty}\frac{7x^4}{-2x^3} = \lim\limits_{x\to +\infty}\frac{7x}{-2} = -\infty$
\end{example}

\subsubsection{Limiti notevoli}
In tabella \ref{tab:limiti-notevoli} alcuni limiti notevoli, cioè limiti che all'apparenza possono sembrare il risultato ma che in realtà tornano un risultato finito.
\begin{table}[h!]
    \centering
    \setlength{\tabcolsep}{10pt}
    \renewcommand{\arraystretch}{2.5}
    \begin{tabular}{|c|c|}
        \hline
        $\lim\limits_{x\to 0}\frac{\sin(x)}{x} = 1$ & $\lim\limits_{x\to 0} \frac{1-\cos(x)}{x^2} = \frac{1}{2}$ \\\hline
        $\lim\limits_{x\to 0}\frac{e^x-1}{x} = 1$ & $\lim\limits_{x\to 0}\frac{\log(1+x)}{x} = 1$\\
        \hline
    \end{tabular}
    \caption{Limiti notevoli}
    \label{tab:limiti-notevoli}
\end{table}
\begin{demostration}
Dimostriamo $\lim\limits_{x\to 0} \frac{1-\cos(x)}{x^2} = \frac{1}{2}$.
\begin{enumerate}
    \item $\lim\limits_{x\to 0} \frac{1-\cos(x)}{x^2} = \frac{1}{2} = \frac{(1-\cos(x)) \cdot (1+\cos(x))}{x^2 \cdot (1+\cos(x))}$ \hspace{.7cm} Moltiplico e divido per $(1+\cos(x))$.
    \item $\lim\limits_{x\to 0} \frac{(1-\cos(x)) \cdot (1+\cos(x))}{x^2 \cdot (1+\cos(x))} = \frac{1-\cos^2(x)}{x^2 \cdot (1 + \cos(x))} = \frac{\sin^2(x)}{x^2 \cdot (1 + \cos(x))}$ \hspace{.7cm} Utilizzo le formule goniometriche.
    \item $\lim\limits_{x\to 0}\frac{\sin^2(x)}{x^2 \cdot (1 + \cos(x))} = \lim\limits_{x\to 0}\frac{\sin(x)}{x} \cdot \frac{\sin(x)}{x} \cdot \frac{1}{1 + \cos(x)}$ \hspace{.7cm} Spezziamo la divisioni in 3 parti.
    \item $\lim\limits_{x\to 0}\frac{\sin(x)}{x} = 1$ \: \: $\lim\limits_{x\to 0}\frac{\sin(x)}{x} = 1$ \: \: $\lim\limits_{x\to 0}\frac{1}{1 + \cos(x)} = \frac{1}{1 + 1}$ \hspace{.7cm} Facciamo il limite dei singoli pezzi.
    \item $\lim\limits_{x\to 0}\frac{1-\cos(x)}{x^2} = 1 \cdot 1 \cdot \frac{1}{2} = \frac{1}{2}$ \hspace{.7cm} Dimostrazione finita. $\blacksquare$
\end{enumerate}
\end{demostration}

\subsubsection{Logaritmi e potenze}
Vediamo una serie di casi di calcolo di limiti con logaritmi e potenze.
\begin{itemize}
    \item $\lim\limits_{x\to +\infty}\frac{\log(x)}{x} = \frac{+\infty}{+\infty}$ forma indeterminata.\\
    Eseguiamo un cambio di variabili con $y = \log(x)$ e $x = e^y$. Se $x\to +\infty \Longrightarrow y = \log(x) \to +\infty$\\\\
    Torna che $\lim\limits_{x \to +\infty}\frac{\log(x)}{x} = \lim\limits_{y\to +\infty}\frac{y}{e^y} = 0$
    \item $\lim\limits_{x\to +\infty}\frac{(\log(x))^\beta}{x^\alpha}$ con $\alpha, \beta \in \mathbb{R}$ e $\alpha, \beta > 0$\\
    Possiamo risolvere con un cambio di variabile $y = \log(x)$, $x = e^y$ e se $x \to +\infty \Longrightarrow y\to +\infty$\\\\
    Quindi $\lim\limits_{x\to +\infty}\frac{(\log(x))^\beta}{x^\alpha} = \lim\limits_{y \to +\infty}\frac{y^\beta}{(e^y)^\alpha} = \lim\limits_{y \to +\infty}\frac{y^\beta}{e^{y\cdot\alpha}} = 0$  (l'esponenziale cresce più velocemente).
    \item $\lim\limits_{x\to 0^+}x\log(x) = 0 \cdot (-\infty)$ forma indeterminata.\\
    Facciamo il cambio di variabile $y = \log(x)$, e $x = e^y$ con $x\to 0^+ \Longrightarrow y\to -\infty$.\\\\
    $\lim\limits_{x\to 0^+}x\log(x) = \lim\limits_{y\to -\infty}e^y \cdot y = 0^+ \cdot (-\infty)$ ancora indeterminata.\\
    Possiamo fare un altro cambio di varibile con $z = -y$, e $y = -z$ e se $y \to -\infty \Longrightarrow z \to +\infty$\\\\
    $\lim\limits_{y\to -\infty}e^y \cdot y = \lim\limits_{z\to +\infty}e^{-z} \cdot (-z) = \frac{-z}{e^z} = 0$
    \item $\lim\limits_{x\to 0^+}x^\alpha \cdot \log(x)$ con $\alpha > 0$.\\
    Cambio di variabile con $y = x^\alpha$, e $x = y^{\frac{1}{\alpha}}$ e con $x\to 0^+ \Longrightarrow y\to^+$\\\\
    $\lim\limits_{x\to 0^+}x^\alpha \cdot \log(x) = \lim\limits_{y\to 0^+}y \cdot \log(y^{\frac{1}{\alpha}}) = \lim\limits_{y\to 0^+}\frac{y}{\alpha} \cdot \log(y) = \frac{1}{\alpha}\lim\limits_{y\to 0^+} y \cdot \log(y) = 0$ per l'esempio sopra.
\end{itemize}

\subsection{Limite della composizione di funzioni}
\begin{theorem}[Limite della composizione di funzioni]
    Dati $A,B \subset \mathbb{R}$, una $f: A \to B$, ed una $g: B \to \mathbb{R}$, un punto $x_0 \in Acc(A)$. Se esiste $\lim\limits_{x\to x_0}f(x) = y_0$ e $y_0 \in Acc(B)$ e $\exists \lim\limits_{x\to x_0}g(y) = l \in \overline{\mathbb{R}}$ e se verifichiamo almeno delle seguenti ipotesi:
    \begin{enumerate}
        \item $y_0 \in B$ e g è continua in $y_0$.
        \item Esiste $U$ intorno di $x_0$ t.c. se $x \in U \cap A \setminus \{x_0\} \Longrightarrow f(x) \neq y_0$
    \end{enumerate}
    Allora $\lim\limits_{x\to x_0}(g \circ f)(x) = l$. Cioè:
    \begin{center}
        \vspace{-5pt}
        $\lim\limits_{x\to x_0}(g \circ f)(x) = \lim\limits_{y\to y_0}g(y)$
    \end{center}
\end{theorem}
\begin{example}
Facciamo un esempio andando a calcolare il $\lim\limits_{x\to -\infty}\arctan(x^2)$.\\
Questo limite è una composizione fra $f(x) = x^2$ e $g(y) = \arctan(y)$, che può essere scritto come $(g \circ f)(x) = g(f(x)) = g(x^2) = \arctan(x^2)$.\\
Noi abbiamo che $x_0 = -\infty$ mentre $t_0 = \lim\limits_{x\to x_0}f(x) = \lim\limits_{x\to -\infty}x^2 = +\infty$.\\
Vediamo dunque che l'ipotesi (1) non è verificata perché $y_0 = +\infty$ e non appartiene al dominio di $g$.\\
Mentre possiamo vedere che l'ipotesi (2) è ovviamente verificata perché chiedo che $f(x) \neq y_0$ cioè $f(x) \neq +\infty$ che è ovviamente sempre vero. Possiamo dunque applicare il teorema:\\
$\lim\limits_{y\to y_0}g(y) = \lim\limits_{y\to +\infty}\arctan(y) = \frac{\pi}{2} \Longrightarrow \lim\limits_{x\to -\infty}\arctan(x^2) = \frac{\pi}{2}$
\end{example}
\begin{observation}
    Quello che osserviamo nel teorema del limite della composizione di funzioni + un teorema di cambiamento di variabili. Infatti andando a prendere l'esempio di prima vediamo che:
    \begin{center}
        Da $\lim\limits_{x\to +\infty}\arctan(x^2)$ cambiamo variabile e ponto $y = x^2$, $\lim\limits_{y\to +\infty}\arctan(y) = \frac{\pi}{2}$
    \end{center}
    Nel caso $x\to -\infty$ dobbiamo vedere a quanto tende $y$, quindi $\lim\limits_{x\to -\infty} = \lim\limits_{x\to -\infty}x^2 = +\infty$
\end{observation}
\begin{observation}
    Un altra osservazione è del perché è inserita l'ipotesi (2) nel teorema. Facciamo un esempio per capire il suo scopo.\\
    Prendiamo $f: \mathbb{R} \to \mathbb{R}$, definita come $f(x) = 1 \forall x \in \mathbb{R}$.\\
    Poi prendiamo anche una $g: \mathbb{R} \to \mathbb{R}$ definita come $g(x) = \begin{cases}
        3 se & y = 1\\
        5 se & y \neq 1\\
    \end{cases}$. Facciamo la composizioni di queste due funzioni e valutiamo il limite con $x\to 0$.\\
    $(g \circ f)(x) = g(f(x)) = g(1) = 3 \forall x \in \mathbb{R} \Longrightarrow \lim\limits_{x\to 0}(g \circ f)(x) = 3$.
    Ma  $\lim\limits_{y \to y_0}g(y) = \lim\limits_{y\to 1}g(y) = 5$.\\
    $y_0 = \lim\limits_{x \to x_0}f(x) = \lim\limits_{x\to 0}f(x) = 1$.\\
    Vediamo dunque che $\lim\limits_{x\to x_0} \neq \lim\limits_{y \to y_0}g(y)$.\\
    Ma infatti in questo esempio non abbiamo considerato che non vale l'ipotesi (2) e nemmeno la (1).
\end{observation}

\subsection{Teorema di Weirstrass generalizzato}
\begin{theorem}[Teorema di Weirstrass generalizzato]
Siano $a,b \in \overline{\mathbb{R}}$ e $f: (a,b) \to \mathbb{R}$ continua t.c. $\exists \: \lim\limits_{x \to a}f(x) = l_1$ e $\exists \: \lim\limits_{x \to b}f(x) = l_2$, valgono i seguenti risultati:
\begin{enumerate}
    \item $f$ è limitata inferiormente $\Longleftrightarrow$ $l_1 \neq -\infty$ e $l_2 \neq -\infty$.
    \item $f$ è limitata superiormente $\Longleftrightarrow$ $l_1 \neq +\infty$ e $l_2 \neq +\infty$.
    \item $f$ è limitata  $\Longleftrightarrow$ $l_1 \in \mathbb{R}$ e $l_2 \in \mathbb{R}$.
    \item $f$ ha minimo $\Longleftrightarrow \: \exists x_0 \in (a,b)$ t.c. $f(x_0) \leq min\{l_1, l_2\}$.
    \item $f$ ha massimo $\Longleftrightarrow \: \exists x_0 \in (a,b)$ t.c. $f(x_0) \geq max\{l_1, l_2\}$.
 \end{enumerate}
\end{theorem}
\begin{observation}
I risultati precedenti valgono anche nel caso $a \in \mathbb{R}$ e $f: [a,b) \to \mathbb{R}$ oppure $b\in \mathbb{R}$ e $f: (a,b] \to \mathbb{R}$ (f sempre continua).
\end{observation}
\begin{wrapfigure}[9]{l}{9cm}
    \vspace{-10pt}
    \centering
    \includegraphics[width=8cm]{images/es-weirstrass-generalizzato.png}
    \vspace{-7pt}
    \caption{Massimi e minimi con Weirstrass}
    \label{fig:werstrass-generalizzato}
\end{wrapfigure}

Come possiamo vedere nella figura \ref{fig:werstrass-generalizzato} se la funzione sale sopra il limite maggiore dovrà necessariamente scendere e quindi si andrà a creare un massimo.\\\\
Ugualmente se la funzione scende sotto il limite minore vuol dire che poi risalirà creando dunque un minimo.\\\\\\

\begin{example}
Prendiamo $f(x) = \frac{1}{x - x^2}$ definita in $f:(0,1) \to \mathbb{R}$ e calcoliamo il limite agli estremi:\\ \\
$\lim\limits_{x\to 0^+}\frac{1}{x \cdot (1 - x)} = \frac{1}{0^+ \cdot 1} = \frac{1}{0^+} = +\infty$ \hspace{.7cm}
$\lim\limits_{x\to 1^-}\frac{1}{x \cdot (1 - x)} = \frac{1}{1 \cdot (1 - 1^-)} = \frac{1}{1 \cdot 0^+)} = \frac{1}{0^+} = +\infty$\\ \\
In questo caso per il teorema visto la funzione $f(x)$ ha minimo.
\end{example}
\begin{example}
Con $f(x) = \frac{x^2 + x|x| + x}{1 + x^2}$ che va da $f: \mathbb{R}\to \mathbb{R}$ verifichiamo se c'è massimo e o minimo.\\
$f(x) = \begin{cases}
    \frac{2x^2 + x}{1 + x^2}& se \: \: x \geq 0\\
    \frac{x}{1 + x^2}& se \: \: x < 0\\
\end{cases}$ \hspace{.7cm} $\lim\limits_{x\to +\infty}\frac{x^2 + x|x| + x}{1 + x^2} = 2 $ \hspace{.3cm} $\lim\limits_{x\to -\infty}\frac{x^2 + x|x| + x}{1 + x^2} = 0 $\\\\
Quello che ci dobbiamo domandare è se $\exists x_0$ t.c. $f(x) \leq 0$ e o $f(x) \geq 2$.\\\\
Se $x<0 \Longrightarrow f(x) = \frac{2x^2 + x}{1 + x^2} < 0 \forall x < 0$ quindi $f$ ha minimo.\\
Mentre se $x \geq 0 \Longrightarrow f(x) = \frac{x}{1 + x^2} \geq 0 \Longrightarrow 2x^2 + x \geq 2 + 2x^2 \Longrightarrow x \geq 2$ quindi $f$ ha anche massimo.
\end{example}
\newpage
\section{Infinitesimi}
\subsection{O-piccolo}
\begin{definition}[O-piccolo]
Prendiamo $A \subset \mathbb{R}, x_0 \in Acc(A)$, $f,g: A \to \mathbb{R}$ ($x_0 \in \overline{\mathbb{R}}$). Si dice che $f$ è \textbf{o-piccolo} di $g$ per x che tende a $x_0$, e si scrive $f(x) = o(g(x))$ per $x \to x_0$ se esiste una funzione $\omega(x)$ t.c. $\lim\limits_{x \to x_0} \omega(x) = 0$ e $f(x) = g(x) \cdot \omega(x)$.
\end{definition}
\begin{observation}
Se esiste un intorno $U$ di $x_0$ t.c. $g(x) \neq 0 \forall x \in U \setminus \{x_0\}$ allora $f(x) = o(g(x)) \Longleftrightarrow \lim\limits_{x\to x_0}\frac{f(x)}{g(x)}=0$ (vuol dire che $f(x) = \omega(x) \cdot g(x) = \frac{f(x)}{g(x)} = \omega(x) \to 0$), possiamo infatti scrivere:
\begin{center}
    \vspace{-8pt}
    $\lim\limits_{x\to 0}\frac{f(x)}{g(x)} = 0$ allora $f(x) = o(g(x))$
\end{center}
\end{observation}
Intuitivamente possiamo dire anche che se $f(x) = o(g(x))$ vuol dire che $f(x)$ è infinitesimamente più piccola di $g(x)$ per $x\to x_0$.
\begin{example}
Se prendiamo una $f(x) = x^3$ e $g(x) = x^2$, $f(x) = o(g(x))$ per $x\to 0$.\\
Infatti $\frac{f(x)}{g(x)} = \frac{x^3}{x^2} = x \to 0$ per $x\to 0$.\\
Possiamo vedere l'applicazione della definizione con $f(x) = g(x) \cdot \omega(x)$ con $\omega(x) = x$ e visto $\omega(x) \to 0$.
\end{example}

\subsection{Proprietà o-piccolo}
Dato un $A \subset \mathbb{R}$, un $x_0 \in Acc(A)$, e due funzioni $f,g: A \to \mathbb{R}$ e con tutti gli o-piccoli che si intendono per $x\to x_0$, valgono le seguenti proprietà.
\begin{enumerate}
    \item $f(x) \cdot o(g(x)) = o(f(x) \cdot g(x))$.
    \item Se $k \in \mathbb{R}$, e $k \neq 0 \Longrightarrow o(k \cdot g(x)) = o(g(x))$.
    \item $o(g) + o(g) = o(g)$. \footnote{Scrivere $o(g(x))$ oppure $o(g)$ è equivalente}
    \item Se $\lim\limits_{x\to x_0}f(x) = 0 \Longrightarrow f(x) \cdot g(x) = o(g(x))$.
    \item Se $\lim\limits_{x\to x_0}f(x) = 0 \Longrightarrow o(g) + o(f \cdot g) = o(g)$.
    \item $o(o(g)) = o(g)$.
    \item $o(f + g) = o(f) + o(g)$.
    \item $o(g) \cdot o(f) = o(f \cdot f)$.
\end{enumerate}

\begin{observation}
Facciamo un osservazione relativa alla proprietà (3) e di essa valga anche nel caso $o(g) - o(g)$.\\
$o(g) - o(g) = o(g) + (-1)\cdot o(g) = o(g) + o(-1 \cdot g) = o(g) + o(g) = o(g)$. \\
Vediamo dunque che la proprietà (2) comprende anche i casi con il meno.
\end{observation}

\begin{example}
Facciamo un esempio per capire meglio l'osservazione sopra. \\
Prendiamo $f(x) = x^3$, $g(x) = x^2$ e $h(x) = x^4$ , vediamo che $x^3 = o(x^2)$ e $x^3 = o(x^2)$ ma che $x^3 - x^4 \neq 0$.
\end{example}

\begin{observation}
Una casistica molto frequente e quella con $g = $ potenza di $x$ (o di $x - x_0$).\\\\
Infatti se prendiamo $\alpha, \beta \in \mathbb{R}$ con $\alpha > \beta \Longrightarrow x^\alpha = o(x^\beta)$ perché $x^\alpha = x^\beta \cdot x^{\alpha - \beta}$.\\
Quindi quando $\omega(x) = x^{\alpha-\beta} \to 0$ perché $\alpha > \beta$.
Mentre quando $\omega(x) = \frac{x^\alpha}{x^\beta} \to 0$ sempre perché $\alpha > \beta$.
\end{observation}

\begin{example}
Prendiamo $f(x) = \tan(x) \cdot \sin(x)$ e dico che $f(x) = o(x)$ per $x\to 0$.
Infatti $\lim\limits_{x\to 0}\frac{f(x)}{x} = \lim\limits_{x\to 0}\frac{\tan(x) \cdot \sin(x)}{x} = \lim\limits_{x\to 0}\tan(x) \cdot \lim\limits_{x\to 0}\frac{\sin(x)}{x} = 0 \cdot 1 = 0$ (ricorda il limite notevole $\lim\limits_{x\to0}\frac{\sin(x)}{x} = 1$)
\end{example}

\newpage
\subsection{Sviluppi al primo ordine}
\begin{itemize}
    \item Dai limiti notevoli sappiamo che $\lim\limits_{x\to 0}\frac{\sin(x)}{x} = 1 \Longrightarrow \lim\limits_{x\to 0}\frac{\sin(x)}{x} - 1 = 0$.\\
    Possiamo dunque dire che $\lim\limits_{x\to 0}\frac{\sin(x) - x}{x} = 0$ quindi per definizione:
    \begin{center}
        \vspace{-5pt}
        $\sin(x) - x = o(x)$ \:\:\: e che \:\:\: $\sin(x) = x - o(x)$ per $x\to 0$
    \end{center} 
    \item Dal limite notevole $\lim\limits_{x\to 0}\frac{1 - \cos(x)}{x^2} = \frac{1}{2}$ ottengo, come prima, che:
    \begin{center}
        \vspace{-5pt}
        $1 - \cos(x) - \frac{1}{2} = o(x^2)$ \:\:\: e che \:\:\: $\cos(x) = 1 - \frac{x^2}{2} + o(x^2)$
    \end{center}
    \item $\lim\limits_{x\to 0}\frac{\tan(x)}{x} = \lim\limits_{x\to 0}\frac{\sin(x)}{\cos(x)} \cdot \frac{1}{x} = \lim\limits_{x\to 0}\frac{\sin(x)}{x} \cdot \frac{1}{\cos(x)} = 1 \cdot \frac{1}{1} = 1 \Longrightarrow \tan(x) = x + o(x)$
    \item $\lim\limits_{x\to x_0}\frac{e^x - 1}{x} = 1 \Longrightarrow e^x = 1 + x + o(x)$
    \item $\lim\limits_{x\to 0} \frac{\log(1 + x)}{x} = 1 \Longrightarrow \log(1 + x) = x + o(x)$
\end{itemize}

\begin{example}
    Esempio risolvendo $(\tan(x))^2$ in termini di o-piccoli. Sappiamo che $\tan(x) = x + o(x)$.\\\\
    $\tan(x)^2 \:\: = \:\: (x + o(x))^2 = x^2 + 2x \cdot o(x) + (o(x))^2 \:\: = \:\: x^2 + o(2x^2) + o(x^2) \:\: = \:\: x^2 + o(x^2) + o(x^2) \:\: = \:\: x^2 + o(x^2)$\\\\
    Quindi il risultato è che $\tan(x)^2 = x^2 + o(x^2) $
\end{example}

\begin{example}
    Proviamo a risolvere $\lim\limits_{x\to 0}\frac{\cos(\sin^2(x)) - 1}{x^4}$. Ricorda che $\sin(x) = x + o(x)$, quindi \\\\
    Ricorda che $\sin(x) = x + o(x)$, quindi $\sin^2(x) = (x + o(x))^2 = x^2 + o(x^2)$ \\\\
    $\cos(\sin^2(x)) - 1 = \cos(x^2 + o(x^2)) - 1$ poniamo $t = x^2 + o(x^2)$\\\\
    Abbiamo quindi che in termini di o-piccolo $\cos(t) = 1 + \frac{t^2}{2} + o(t^2)$ con $t\to 0$\\\\
    Possiamo fare questa sostituzione perché $\cos(t) = 1 + \frac{t^2}{2} + o(t^2)$ vale con $t\to 0$, se $t = x^2 + o(x^2)$ ottengo che se $x\to 0$ allora $x^2 + o(x^2) \to 0$ quindi $t\to 0$.\\\\
    Ri-sostituendo la $t$ abbiamo che $\cos(t) = 1 - \frac{t^2}{2} + o(t^2) = 1 - \frac{(x^2 + o(x^2))^2}{2} + o((x^2 + o(x^))^2) =$\\\\
    $= 1 - \frac{x^4 + 2x^2 \cdot o(x^2) + (o(x^2))^2}{2} + o(x^4 + 2x^2 \cdot o(x^2) + o(x^2)^2) = 1 - \frac{x^4 + o(x^4) + o(x^4)}{2} + o(x^4 + o(x^4) + o(x^4)) =$\\\\
    $= 1 - \frac{x^4}{2} + o(x^4) + o(x^4) = 1 - \frac{x^4}{2} + o(x^4)$ quindi abbiamo che:\\\\
    $\frac{\cos(\sin^2(x)) - 1}{x^4} = \frac{1 - \frac{x^4}{2} + o(x^4) -1}{x^4} = \frac{-\frac{x^4}{2} + o(x^4)}{x^4} = -\frac{1}{2} + \frac{o(x^4)}{x^4}$\\\\
    Visto che $\frac{o(x^4)}{x^4}$ tende a 0 abbiamo che $\lim\limits_{x\to 0}\frac{\cos(\sin^2(x)) - 1}{x^4} = -\frac{1}{2}$
\end{example}

\subsection{O-grande}
\begin{definition}[O-grande]
    Dato $A \subset \mathbb{R}$, $x_0 \in Acc(A)$, e $f,g: A \to \mathbb{R}$. Se $\exists M \in \mathbb{R}$ t.c. $|f(x)| \geq M \cdot |g(x)|    \forall x \in U \cap A \setminus \{x_0\}$ dove $U$ è un intorno di $x_0$, allora si dice che $f$ è O-grande di $g$ per $x$ che tende a $x_0$ e si scrive $f(x) = O(g(x))$ per $x\to x_0$.
\end{definition}

\begin{observation}
Se $g$ non si annulla in un intorno di $x_0$ allora possiamo scrivere che:
    \begin{center}
        $f(x) = O(g) \Longleftrightarrow |\frac{f(x)}{g(x)}| \geq M$ in un intorno di $x_0$
    \end{center}
\end{observation}

\begin{example}
    Facciamo un esempio prendendo $f(x) = x\sin(x)$ e $g(x) = x$.\\
    Vediamo che $|\frac{f(x)}{g(x)}| = |\frac{x\sin(x)}{x}| = |\sin(x)| \geq 1$ quindi $f(x) = O(g(x))$ per $x\to x_0$ per qualunque $x_0 \to \overline{\mathbb{R}}$
\end{example}

\begin{definition}
    Dato $A \subset \mathbb{R}$, $x_0 \in Acc(A)$, e $f,g: A \to \mathbb{R}$ infinitesime per $x\to x_0$ (cioè $\lim\limits_{x\to x_0}f(x) = 0$ e $\lim\limits_{x\to x_0}g(x) = 0$). Se esistono $L, \alpha \in \mathbb{R}$ con $L \neq 0$ t.c. $f(x) = L \cdot (g(x))^\alpha + o((g(x))^\alpha)$ per $x\to x_0$ si dice che $f$ è infinitesima di ordine $\alpha$ rispetto a $g$ con parte principali $L(g(x))^\alpha$ per x che tende a $x_0$.\\
    Stessa definizioni del caso in. cui $f$ e $g$ siano divergenti (cioè $\lim\limits_{x\to x_0}f(x) = \pm\infty$ e $\lim\limits_{x\to x_0}g(x) = \pm\infty$)
\end{definition}

\begin{example}
    Prendiamo $f(x) = 3\sin(x) + x^2$ e $g(x) = x$ con $x_0 = 0$.\\
    $f$ è di ordine 1 rispetto a $g$ per $x\to 0$ con parte principale $3x$. Infatti $3\sin(x) + x^2 = 3x + o(x)$.\\
    (Perché $\sin(x) = x + o(x) \Longrightarrow 3\sin(x) + x^3 = 3x + o(x) + x^2 = 3x + o(x)$)
\end{example}

\begin{example}
    Prendiamo il caso con $f(x) = 5x^4 + (2\sin(x)) \cdot x^2 + 3x$ e $g(x) = x$.\\
    $f$ è di ordine 4 rispetto a $x$ per $x\to +\infty$ con parte principale $5x^4$\\
    Questo perché $(2\sin(x)) \cdot x^2 + 3x = o(x^4)$ quindi, $f(x) = 5x^4 + o(x^4)$ infatti $\frac{(2\sin(x)) \cdot x^2 + 3x}{x^4} \to 0$
\end{example}

\begin{example}
    Guardiamo un esempio con $f(x) = \log(e^{3x} + x^2)$ per $x\to +\infty$\\\\
    $\log(e^{3x} + x^2) = \log(e^{3x} \cdot (1 + \frac{x^2}{e^{3x}}) = \log(e^{3x}) + \log(1 + \frac{x^2}{e^{3x}}) = 3x + \log(1 + \frac{x^2}{e^{3x}})$\\
    Abbiamo che $\frac{x^2}{e^{3x}}\to 0$ per $x\to +\infty$. Possiamo dunque dire che $f(x)$ è di ordine 1 rispetto a $x$ con parte principale $3x$ per $x\to +\infty$. Quindi $f(x) = 3x + o(x)$
\end{example}
\newpage
\section{Asintoti}

\subsection{Asintoto orizzontale}
\begin{definition}[Asintoto orizzontale]
Data una $f: A \to \mathbb{R}$, un $a \in \mathbb{R}$. Se esiste $\lim\limits_{x\to x_0}f(x) = l \in \mathbb{R}$ (finito) si dice che $f$ ha un \textbf{asintoto orizzontale} di equazione $y = l$ per x che tende a $\pm\infty$.
\end{definition}
\begin{example}
Prendiamo $f(x) = e^x$ con $f:\mathbb{R}\to \mathbb{R}$.
\end{example}
\begin{wrapfigure}[7]{l}{7cm}
    \vspace{-5pt}
    \centering
    \includegraphics[width=5.5cm]{images/asintoto-esponenziale.png}
    \caption{Asintoto orizzontale di $e^x$}
\end{wrapfigure}

Andiamo come prima cosa a calcolare il limite: $\lim\limits_{x\to -\infty}f(x) = 0$.
Possiamo così vedere che $f$ ha un asintoto orizzontale di equazione $y=0$ per $x\to -\infty$. \\\\
Come possiamo notare nell'immagine a fianco (asintoto segnato dalla linea blu in basso).\\\\\\\\\\

\begin{example}
Facciamo un altro esempio prendendo questa volta $f(x) = \arctan(x)$ con $f: \mathbb{R}\to \mathbb{R}$
\end{example}
\begin{wrapfigure}[9]{r}{7.5cm}
    \vspace{-5pt}
    \centering
    \includegraphics[width=5.5cm]{images/asintoto-arctan.png}
    \caption{Asintoto orizzontale di $\arctan(x)$}
\end{wrapfigure}

Anche qui compre prima cosa calcoliamo il limite sia vero $+\infty$ che verso $-\infty$ della funzione:\\ $\lim\limits_{x \to +\infty}\arctan(x) = \frac{\pi}{2}$
$\lim\limits_{x \to -\infty}\arctan(x) = -\frac{\pi}{2}$\\\\
Vediamo dunque due asintoti con equazioni $y=\frac{\pi}{2}$ e $y=-\frac{\pi}{2}$ rispettivamente con $x\to +\infty$ e $x\to -\infty$.
Possiamo vedere i due asintoti nell'immagine a fianco (rette in blu).\\

\subsection{Asintoto verticale}
\begin{definition}[Asintoto verticale]
Dato un $A \subset \mathbb{R}$, $x_0\in Acc(A)$, $x_0 \in \mathbb{R}$, una $f:A\to \mathbb{R}$. Se $f$ diverge per $x$ che tende a $x_0$ da destra o da sinistra (o da entrambe le parti) si dice che f ha un \textbf{asintoto verticale} di equazione $x=x_0$.
\end{definition}

\begin{example}
Prendiamo la funzione $f(x) = \frac{1}{x}$ definita come $f: \mathbb{R} \setminus \{0\} \to \mathbb{R}$.
\end{example}
\begin{wrapfigure}[9]{r}{7.5cm}
    \vspace{-10pt}
    \centering
    \includegraphics[width=3cm]{images/asintoto-verticale-1.png}
    \caption{Asintoto verticale di $\frac{1}{x}$}
\end{wrapfigure}

Andiamo a calcolare nel punto di discontinuità, che è lo 0, il limite sia da destra che da sinistra:\\\\
$\lim\limits_{x\to 0^+}\frac{1}{x} = +\infty$, $\lim\limits_{x\to 0^-}\frac{1}{x} = -\infty$.\\\\
Vediamo dunque la $f$ ha un asintoto verticale di equazione $x=0$. Possiamo vedere l'asintoto nell'immagine a fianco (asintoto verticale segnato in blu).\\

\begin{observation}
Una funzione al massimo ha 2 asintoti orizzontali (uno a $+\infty$ ed uno a $-\infty$) ma può anche avere $\infty$ asintoti verticali, come nel caso di $f(x) = \tan(x)$ che ha $\infty$ asintoti verticali.
\end{observation}

\subsection{Asintoto obliquo}
\begin{definition}[Asintoto obliquo]
    Data una $f:(a, +\infty) \to \mathbb{R}$. Se esiste $\lim\limits_{x\to +\infty}\frac{f(x)}{x} = m$ con $m \in \mathbb{R}$ e $m\neq 0$, e se esiste anche $\lim\limits_{x\to +\infty}f(x) - mx = q$ con $q \in \mathbb{R}$ allora si dice che $f$ ha un \textbf{asintoto obliquo} di equazione $y = mx + q$ per $x\to +\infty$. Lo stesso vale con $x \to -\infty$.
\end{definition}

\begin{example}
Facciamo un esempio di calcolo dei asintoto obliquo con $f(x) = \frac{2x^2 + 3x +2}{x-5}$.\\
$\lim\limits_{x\to +\infty}\frac{f(x)}{x} = \frac{2x^2 + 3x +2}{x^2-5x} = 2$, quindi $m=2$\\
$\lim\limits_{x\to +\infty}f(x) - mx = \frac{2x^2 + 3x +2}{x-5} - 2x = \lim\limits_{x\to +\infty}\frac{2x^2 + 3x +2 - 2x(x-5)}{x-5} = \lim\limits_{x\to +\infty} \frac{3x + 2 + 10x}{x-5} = \lim\limits_{x\to +\infty}\frac{13x + 2}{x-5} = 13$\\\\
Abbiamo dunque che esiste un asintoto obliquo di equazione $y=2x +13$ per $x\to +\infty$
\end{example}

\begin{observation}
Una funzione può avere al massimo 2 asintoti obliqui (uno a $+\infty$ ed uno a $-\infty$). Inoltre non può avere contemporaneamente un asintoto orizzontale ed uno obliquo "dalla stessa parte".
\end{observation}

\begin{example}
Prendiamo $f(x) = 3x + 5\log(x)$ definita come $f: (0,+\infty) \to \mathbb{R}$. Proviamo ora a calcolare l'asintoto obliquo.\\\\
$\lim\limits_{x\to +\infty}\frac{f(x)}{x} = \lim\limits_{x\to +\infty} \frac{3x + 5\log(x)}{x} = 3 + \lim\limits_{x\to +\infty}\frac{5\log(x)}{x} = 3 + 0 = 3$ quindi $m=3$.\\
$\lim\limits_{x\to +\infty}f(x) - mx = \lim\limits_{x\to +\infty} 3x + 5\log(x) -3x = 5\log(x) = +\infty$.\\\\
Visto che la $q$ non torna un numero finito vediamo che questa funzione non ha asintoto obliquo.
\end{example}
\input{derivate}
\newpage
\section{Sviluppi di Taylor}

\subsection{Fattoriale}
\begin{definition}[Fattoriale]
Dato un $n \in \mathbb{N}$ con $n \geq 1$ definiamo un fattoriale come il prodotto dei primi n numeri naturali:
\begin{center}
    $n! = 1 \cdot 2 \cdot 3 \cdot 4 \cdot ... \cdot n$
\end{center}
\end{definition}
\begin{note}
Nota che $0! = 1$ per definizione.
\end{note}
\begin{example}
$1! = 1$ \hspace{.5cm} $2! = 1 \cdot 2 = 2$ \hspace{.5cm} $3! = 1 \cdot 2 \cdot 3 = 6$ \hspace{.5cm} $4! = (1 \cdot 2 \cdot 3) \cdot 4 = 24$
\end{example}
\hspace{-15pt}Possiamo definire un uguaglianza per definire il fattoriale:
\begin{center}
    $(n+1)! = n! \cdot (n+1)$ dove $(n+1)! = [1 \cdot 2 \cdot ... \cdot n] \cdot (n+1) = n! \cdot (n+1)$
\end{center}

\subsection{Sommatorie}
Supponiamo di avere dei numeri naturali indicizzati con un numero naturale.
\begin{center}
    $a_1, a_2, ..., a_n$ e $a_j \in \mathbb{R}$ con $j \in \mathbb{N}$
\end{center}
Per esempio si potrebbe prendere $a_j = \frac{1}{j}$ quindi: $a_1 = \frac{1}{1}$, $a_2 = \frac{1}{2}$, $a_3 = \frac{1}{3}$, ecc.
Oppure possiamo $a_j = \sqrt{j}$ quindi: $a_1 = \sqrt{1}$, $a_2 = \sqrt{2}$, ecc.

\begin{definition}[Sommatoria]
Definisco sommatoria degli $a_j$ per $j$ che va da $m$ ad $n$ dove $m,n \in \mathbb{N}$ e $m \leq n$, e si scrivere\footnote{Usiamo $j$ per convenzione ma è possibile utilizzare qualsiasi variabile}:
\begin{center}\vspace{-5pt}
    $\sum\limits_{j = m}^n a_j = a_m + a_{m+1} + a_{m+2} + ... + a_n$
\end{center}
\end{definition}
\begin{example}
$\sum\limits_{j = 1}^5 \frac{1}{j} = \frac{1}{1} + \frac{1}{2} + ... + \frac{1}{5}$
\end{example}
\begin{example}
$\sum\limits_{j = 0}^3 j^2 = 0^2 + 1^2 + 2^2 + 3^2 = 1 + 4 + 9 = 14$
\end{example}

\subsection{Formula di Taylor}
\subsubsection{Taylor con resto di Peano}
Supponiamo di avere una funzione $f$ derivabile nel punto $x_0 \in (a,b)$, allora abbiamo visto che posso scrivere $f(x) = f(x_0) + f'(x_0) \cdot (x-x_0) + o(x - x_0)$ per $x\to x_0$. Abbiamo dunque un polinomi di grado 1 ungule a $f(x_0) + f'(x_0) \cdot (x-x_0)$ ed un resto $o(x - x_0)$, $f$ quindi differisce dal polinomio per un resto che è infinitesimo rispetto a $x- x_0$ cioè $\lim\limits_{x\to x_0} = \frac{o(x - x_0)}{x - x_0} = 0$.\\\\
Posso precisare meglio la quantità di $o(x - x_0)$ ma $f$ deve essere derivabile più volte nel punto $x_0$.

\begin{definition}[Formula di Taylor con resto di Peano]
Dato una funzione $f:(a,b) \to \mathbb{R}$ e $x_0 \in (a,b)$. Se $f$ è derivabile n volte in $x_0$ ed almeno $n-1$ volte nel resto dell'intervallo (a,b) (cioè in $(a,b) \setminus \{x_0\}$) allora esiste un unico polinomio $P_n(x)$ di grado $\leq n$ ed una funzione $R_n(x)$ tale che:
\begin{center}
    $f(x) = P_n(x) + R_n(x)$ e $R_n(x) = o(x - x_0)^n$ per $x\to x_0$
\end{center}
Il polinomio $P_n(x)$ ha la seguente forma:
\begin{center}\vspace{-5pt}
    $P_n(x) = \sum\limits_{j=0}^n \frac{f^{(j)}(x_0)}{j!} \cdot (x - x_0)^j$
\end{center}
\end{definition}

Scritto in maniera esplicita:\\
$P_n = f(x_0) + f'(x_0) \cdot (x - x_0) + f''(x_0)\frac{f''(x_0)}{2} \cdot (x-x_0)^2 + ... + \frac{f^{(n)}(x_0)}{n!} \cdot (x - x_0)^n$

\begin{observation}
Il grado massimo del polinomio è correlato all'ordine di infinitesimo del resto. Cioè $P_n$ è di grado n e $R_n = o(x - x_0)^n$. Questo vuol dire che: $f(x) - P_n(x) = o((x-x_0)^n)$, $o((x-x_0)^n)$ è la differenza fra la funzione ed il polinomio che l'approssima.
\end{observation}

\subsubsection{Taylor con resto di Lagrange}
\begin{definition}[Formula di Taylor con resto di Lagrange]
Dato una funzione $f:(a,b) \to \mathbb{R}$ e $x_0 \in (a,b)$ e $f$ derivabili in $n+1$ volte in $(a,b) \setminus \{x_0\}$ e n volte in $x_0$. Allora $f(x) = P_n + R_n(x)$ ed esiste $z$ compreso tra $x$ e $x_0$ tale che:
\begin{center}
    $R_n(x) = \frac{f^{n+1}(z) \cdot (x - x_0)^{n+1}}{(n+1)!}$
\end{center}
\end{definition}
\hspace{-15pt}Dico un punto compreso fra $x$ e $x_0$ perché a priori non so quali dei due valori sta a destra e quale sta a sinistra, quindi parlo semplicemente di punto compreso.

\subsubsection{Esempi di formula di Taylor}
\begin{example}
$f(x) = e^x$ e $f'(x) = e^x$, $f''(x) = e^x$, ... $f^{(j)}(x) = e^x \: \forall j \in \mathbb{N}$. La calcolo in $x_0 = 0$ \footnote{Si dice che in questo caso si fa centrato in 0}.\\
$f(0) = 1$, $f'(0) = 1$, ..., $f^{j}(0) = 1$. Quindi $e^x = (\sum\limits_{j=0}^n\frac{x^j}{j!}) + o(x^n) = (\sum\limits_{j=0}^n\frac{f^{(j)}(0)}{j!}\cdot (x-0)^j) + o(x^n)$\\
$e^x = 1 + x + \frac{x^2}{2} + \frac{x^3}{3!} + \frac{x^4}{4!} + ... + \frac{x^n}{n!} + o(x^n)$.\\
Per esempio in ordine 2: $e^x = 1 + x + \frac{x^2}{2} + o(x^2)$, se lo confrontiamo con il limite notevole $e^x = 1 + x + o(x)$ vediamo che $o(x)$ (che è $R_1(x)$)in realtà è $\frac{x^2}{2} + o(x^2)$ (che è $R_2(x)$).
\end{example}
\begin{observation}
$R_2(x)$ in particolare è un $o(x)$ perché se faccio $\frac{R_2(x)}{x} = \frac{\frac{x^2}{2} + o(x^2)}{x} = \frac{x}{2} + o(x) \to 0$ se $x\to 0$.
Quella con il grado 2 è più precisa di quella con il grado 1.
\end{observation}

\begin{example}
$f(x) = \sin{x}$, $f'(x) = \cos{x}$, $f''(x) = -\sin{x}$, $f'''(x)=-\cos{x}$.\\
$f(0) = 0$, $f'(0) = 0$, $f''(0) = 0$, $f'''(0) = -1$. $\sin{x} = \sum\limits_{i=0}^n\frac{f^{(i)}(0)}{j!} \cdot x^j + R_n(x)$.\\
$\sin{x} = 0 + \frac{x}{1} + 0 \cdot \frac{x^2}{2} - \frac{x^3}{3! + o(x^3)} = x - \frac{x^3}{6} + o(x^3)$. Ordine $n=3$.\\
In questo caso $P_3(x) = x - \frac{x^3}{6}$ e $R_3(x) = o(x^3)$.\\
Proviamo con ordine 4: $\sin{x} = 0 + 1 \cdot x + 0 \cdot \frac{x^2}{2} - 1 \frac{x^3}{3!} + o \cdot \frac{x^4}{4!} + o(x^4) = x - \frac{x^3}{6} + o(x^4)$. \\
In questo caso invece $P_4(x) = x - \frac{x^3}{6}$ e $R_4(x) = o(x^4)$, vediamo che in questo caso $P_3(x) = P_4(x)$.\\\\
Ora confrontiamo:\\
$\sin{x} = x - \frac{x^3}{6} + o(x^3)$ ordine 3 \hspace{.5cm} $\sin{x} = x - \frac{x^3}{6} + o(x^4)$ ordine 4.\\
Possiamo vedere che sono vere entrambi ma la seconda è più precisa perché ha un resto più piccolo.\\
Allo stesso modo $\sin{x} = x + o(x)$ ma visto che sappiamo che la derivata seconda del seno calcolato in 0 è 0 possiamo scrivere in maniera più precisa $\sin{x} = x + o(x^2)$.
\end{example}

\subsection{Taylor per le funzioni elementari}
Possiamo dunque ora scrivere le varie formule di Taylo per delle funzioni ricorrenti.\\

\hspace{-15pt}\textbf{Formula seno:} $\sin{x} = (\sum\limits_{j = 0}^n \frac{(-1)^j \cdot x^{2j +1}}{(2j + 1)!}) + o(x^{2n+2})$

\begin{example}
Proviamo questa formula con $n=2$.\\\\
$\frac{(-1)^0 \cdot x^{2 \cdot 0 + 1}}{(2 \cdot 0 + 1} + \frac{(-1)^1 \cdot x^{2 \cdot 1 + 1}}{(2 \cdot 1 + 1)!} + \frac{(-1)^2 \cdot x^{2 \cdot 2 \cdot 1}}{(2 \cdot 2 + 1)!} + o(x^{2 \cdot 2 + 2}) = x - \frac{x^3}{3!} + \frac{x^5}{5!} + o(x^6)$.\\
\end{example}

\hspace{-15pt}\textbf{Formula coseno:} $\cos{x} = (\sum\limits_{j = 0}^n \frac{(-1)^j \cdot x^{2j +1}}{(2j)!}) + o(x^{2n+1})$
\begin{example}
Formula di Taylor di grado 7 per il coseno:\\
$\cos{x} = 1 - \frac{x^2}{2} + \frac{x^4}{4!} - \frac{x^6}{6!} + o(x^7)$.\\
\end{example}

\hspace{-15pt}\textbf{Formula logaritmo:} $\log(1+x) = (\sum\limits_{j=1}^n(-1)^{j+1}\frac{x^j}{j}) + o(x^n)$
\begin{example}
Facciamo un esempio con $n=4$ della formula del logaritmo:\\
$\log(1+x) = x - \frac{x^2}{2} + \frac{x^3}{3} - \frac{x^4}{4} + o(x^4)$
\end{example}

\begin{note}
Nota che il coseno è una funzione peri ed il polinomio dalla funzione di Taylor contiene sempre potenze pari mentre il seno essendo dispari contiene solo dispari.\\
\end{note}

\hspace{-15pt}\textbf{Formula tangente:} per la tangente la formula è molto complicata quindi scriviamo semplicemente:\\
$\tan(x) = x + o(x^2)$ e $\tan(x) = x + \frac{x^3}{3} + \frac{2x^5}{15} + o(x^6)$.\\

\hspace{-15pt}\textbf{Formula Arcotangente:} $\arctan(x) = (\sum\limits_{j = 0}^n (-1)^j \frac{x^{2j +1}}{2j +1}) + o(x^{2n+2})$
Quindi sviluppata al settimo grado:\\
$\arctan(x) = x - \frac{x^3}{3} + \frac{x^5}{5} - \frac{x^7}{7} + o(x^8)$
\begin{note}
Nota che anche nell'arcotangente come nel logaritmo non c'è il fattoriale.\\
\end{note}

\hspace{-15pt}\textbf{Formula Binomiale:} dato $\alpha \in \mathbb{R}$ possiamo scrivere:\\\\
$(1+ \alpha) = 1 + \alpha x + \frac{\alpha(\alpha -1)}{2} \cdot x^2 + \frac{\alpha(\alpha -1)(\alpha -2)}{3!}\cdot x^3 + ... + \frac{\alpha(\alpha -1)(\alpha -2)...(\alpha - n+1)}{n!}\cdot x^n + o(x^n)$.

\begin{example}
Con $\alpha = \frac{1}{2}$ quindi $\sqrt{1 + x} = (1 + x)^{\frac{1}{2}}$.\\
$(1 + x)^{\frac{1}{2}} = 1 + \frac{1}{2}x + \frac{\frac{1}{2}(\frac{1}{2}-1)}{2} \cdot x^2 + o(x^2) = 1 + \frac{x}{2} - \frac{1}{8}x^2 + o(x^2)$. 
\end{example}

\begin{example}
Con invece $\alpha = -1 $ quindi con $(1 + x)^{-1} = \frac{1}{1+x}$.\\
$\frac{1}{1+x} = 1 - x + \frac{(-1)(-2)}{2!} \cdot x^2 + \frac{(-1)(-2)(-3)}{3!} \cdot x^3 + o(x^3) = 1 - x + \frac{2}{2}x^2 - \frac{3!}{3!}\cdot x^3 + o(x^3) = 1 - x + x^2 + x^3 + o(x^3)$.\\\\
Quindi se sostituiamo $x = -t$ abbiamo che:\\
$\frac{1}{1-t} = 1 - (-t) + (-t)^2 - (-t^3) + o(t^3) = 1 + t + t^2 + t^3 + o(t^3)$, generalizzando possiamo scrivere:\\
$\frac{1}{1-t} = 1 - (-t) + (-t)^2 - (-t^3) + ... + t^n + o(t^n)$
\end{example}

\begin{table}[h!]
    \setlength{\tabcolsep}{5pt}
    \renewcommand{\arraystretch}{2.2}
    \centering
    \begin{tabular}{|c|c|}
        \hline
        $e^x$ & $1 + x + \frac{x^2}{2!} + \frac{x^3}{3!} + \frac{x^4}{4!} + ... + \frac{x^n}{n!} + o(x^n)$  \\
        $\log(1+x)$ & $x - \frac{x^2}{2} + \frac{x^3}{3} - \frac{x^4}{4} + \frac{x^5}{5} + ... + (-1)^{n-1}\frac{x^n}{n} + o(x^n)$ \\
        $\sin(x)$ & $x - \frac{x^3}{3!} + \frac{x^5}{5!} - \frac{x^7}{7!} + ... + (-1)^n \frac{x^{2x+1}}{(2n+1)!} + o(x^{2n+2})$ \\
        $\cos(x)$ & $1 - \frac{x^2}{2!} + \frac{x^4}{4!} - \frac{x^6}{6!} + ... + (-1)^n\frac{x^2n}{(2n)!} + o(x^{2n+1})$ \\
        $\tan(x)$ & $x + \frac{x^3}{3} + \frac{2}{15}x^5 + o(x^6)$\\
        $\arctan(x)$ & $x - \frac{x^3}{3} + \frac{x^5}{5} - \frac{x^7}{7} + ... + (-1)^n\frac{x^{2x+1}}{(2n + 1)} + o(x^{2n+2})$\\
        $\arcsin{x}$ & $x + \frac{x^3}{6} + \frac{3}{40}x^5 + o(x^6)$\\
        $\sqrt{1+x}$ & $1 + \frac{1}{2}x - \frac{1}{8}x^2 + \frac{1}{16}x^3 + o(x^3)$\\
        $(1+x)^{\alpha}$ & $1 + \alpha x + \frac{\alpha(\alpha - 1)}{2}x^2 + \frac{\alpha(\alpha - 1)(\alpha - 2)}{6}x^3 + o(x^3)$\\
        \hline
    \end{tabular}
    \caption{Formule di taylor}
\end{table}


\subsection{Utilizzo di Taylor nei limiti}
\begin{example}
Calcolare $\lim\limits_{x\to 0}\frac{\sin{x} - x}{e^x - \log(1 + x) - 1}$. Si può utilizzare gli o-piccoli:\\
$\sin{x} = x + o(x^2)$ \hspace{.5cm} $e^x = 1 + x + o(x)$ \hspace{.5cm} $\log(1 + x) = x + o(x)$\\\\
$\frac{\sin{x} - x}{e^x - \log(1 + x) - 1} = \frac{x + o(x^2) - x}{1 + x + o(x) - (x + o(x)) - 1} = \frac{o(x^2)}{o(x)}$ ma anche questo è indeterminato.\\\\
Dobbiamo quindi andare un po' avanti negli sviluppi del numeratore e del denominatore.\\\\
$\sin{x} = x - \frac{x^3}{6} + o(x^4)$ \hspace{.5cm} $e^x = 1 + x + \frac{x^2}{2} + o(x^2)$ \hspace{.5cm} $\log(1 + x) = x -\frac{x^2}{2} o(x^2)$\\\\
$\frac{\sin{x} - x}{e^x - \log(1 + x) - 1} = \frac{x - \frac{x^3}{6} + o(x^4) - x}{1 + x + \frac{x^2}{2} + o(x^2) - (x - \frac{x^2}{2} + (x^2)) - 1} = \frac{-\frac{x^3}{6} + o(x^4)}{\frac{x^2}{2} + \frac{x^2}{2} + o(x^2)} = \frac{-\frac{x^3}{6} + o(x^4)}{x^2 + o(x^2)} = \frac{-\frac{x}{6} + o(x^4)}{1 + o(x^2)} = \frac{0}{1} = 0$
\end{example}

\begin{example}
$\lim\limits_{x\to 0}\frac{(\sin{x})^2 - \sin{x^2}}{x^4}$\\
$\sin{t} = t + o(t^2)$ \hspace{.5cm} $t= x^2$\\
$\sin{x}^2 = (x + o(x^2))^2 = x^2 + 2x \cdot o(x^2) + (o(x^2))^2 = x^2 + o(x^3) + o(x^4) = x^2 + o(x^3)$ \hspace{.3cm}$\sin{x^2} = x^2 + o(x^4)$\\\\
$\frac{(\sin{x})^2 - \sin{x^2}}{x^4} = \frac{x^2 + o(x^3) - x^2 + o(x^4)}{x^4} = \frac{o(x^2)}{x^4} = \frac{o(x^2)}{x^3} \cdot \frac{1}{x} = 0 \cdot \infty$ \\
Questa è una forma indeterminata perché $\frac{o(x^2)}{x^3} \to 0$ e $\frac{1}{x} \to \infty$. Quindi aumentiamo il grado dell'approssimazione andando a migliorare $(\sin{x})^2$. $\sin{x} = x - \frac{x^3}{6} + o(x^4)$\\
$(\sin{x})^2 = (x - \frac{x^3}{6} + o(x^4))^2 = x^2 + \frac{x^6}{36} + (o(x^4))^2 - 2x \cdot \frac{x^2}{6} + 2x \cdot o(x^4) - 2 \cdot \frac{x^3}{6} \cdot o(x^4) = x^2 \frac{x^6}{36} + o(x^8) -  \frac{x^4}{3} + o(x^5) + o(x^7) = x^2 - \frac{x^4}{3} + o(x^5)$\\
$\frac{(\sin{x})^2 - \sin{x^2}}{x^4} = \frac{x^2 - \frac{x^4}{3} + o(x^5) - x^2 + o(x^4)}{x^4} = \frac{x^2 - \frac{x^4}{3} - x^2 + o(x^4)}{x^4} = \frac{-\frac{x^4}{3} + o(x^4)}{x^4} = \frac{-\frac{1}{3} + o(1)}{1} \to -\frac{1}{3} + 0 = -\frac{1}{3}$. (Divido sopra e sotto per $x^4$)
\end{example}


\newpage
\section{Convessità}
\subsection{Funzione convessa}
\begin{definition}[Convessa]
Dato un $I \subset \mathbb{R}$ intervallo\footnote{Si parla sempre di intervalli quando si parla di convessità perché la convessità non ha senso sennò} ed una $f: I \to \mathbb{R}$. $f$ si dice \textbf{convessa} in I se, presi due punti qualsiasi sul grafico di $f$ il segmento che li unisce è sopra il grafico di $f$.
\end{definition}
\begin{wrapfigure}[6]{r}{6cm}
    \vspace{-10pt}
    \centering
    \includegraphics[width=3.8cm]{images/convessa.png}
    \caption{Funzione convessa}
\end{wrapfigure}
\hspace{-15pt}In formule si esprime dicendo che: $f$ si dice convessa in $I$ se $\forall x_1, x_2 \in I$ con $x_1 < x_2$ e $\forall t \in (0,1)$ risulta che:
\begin{center}
    $f(x_1 + t(x_2 - x_1)) \leq f(x_1) + t(f(x_2) - f(x_1))$
\end{center}
Se la stessa disuguaglianza vale con il $<$ (minore stretto) allora $f$ si dice strettamente convessa.

\subsection{Funzione concava}
\begin{definition}[Concava]
$f$ si dice concava se $-f$ è convessa. Strettamente concava se $-f$ è strettamente convessa.
\end{definition}
\begin{wrapfigure}[6]{l}{6cm}
    \vspace{-10pt}
    \centering
    \includegraphics[width=3.8cm]{images/concava.png}
    \caption{Funzione concava}
\end{wrapfigure}
Se andiamo a scrivere in formule una funzione concava è uguale a:
\begin{center}
    $f(x_1 + t(x_2 - x_1)) \geq f(x_1) + t(f(x_2) - f(x_1))$
\end{center}
\begin{note}
Nota che, come per la concavità, se andiamo scrivere $>$ (maggiore stretto) allora $f$ si dice strettamente concava.
\end{note}

\vspace{15pt}
\subsection{Calcolo della convessità}
\begin{proposition}
Dato $I \subset \mathbb{R}$ intervallo, $f: I \to \mathbb{R}$ derivabile 2 volte. Sono equivalenti:
\begin{enumerate}
    \item $f$ è convessa (strettamente convessa).
    \item $f'$ è debolmente crescente (strettamente crescente).
    \item $f'' \geq 0$ ($f'' > 0$).
\end{enumerate}
\end{proposition}

\begin{note}
La proposizione è uguale per la concavità ma con il segno scambiato.
\end{note}

\begin{example}
$f(x) = x^2$ da $f:\mathbb{R} \to \mathbb{R}$.\\
$f'8x) = 2x$, $f''(x) = 2 > 0 \: \forall x \in \mathbb{R} \Longrightarrow f$ è convessa (anche strettamente) in tutto $\mathbb{R}$.
\end{example}

\begin{example}
$f(x) = e^x$ e $f'(x) = e^x$, $f''(x) = e^x > 0$ sempre $\Longrightarrow f:\mathbb{R} \to \mathbb{R}$ è strettamente convessa.
\end{example}

\begin{example}
$f(x) = \log(x)$ con $f:(0,+\infty) \to \mathbb{R}$.\\
$f'(x) = \frac{1}{x}$, $f''(x) = \frac{1}{x^2} < 0 \: \forall x > 0 \Longrightarrow f$ è strettamente concava. 
\end{example}

\subsection{Interpretazione geometrica}
\begin{wrapfigure}[7]{l}{6cm}
    \vspace{-13pt}
    \centering
    \includegraphics[width=5.5cm]{images/interpretazione-geometrica-convessita.png}
\end{wrapfigure}
Dire che $f'$ è crescente vuol dire che diciamo che il coefficiente angolare sulla tangente cresce, e questo vuol dire che se noi pensiamo alla retta tangente come un punto che tocca il grafico e mano a mano si sposta sul grafico e così facendo va a cambiare inclinazione ruotando, quindi possiamo dire che "la tangente ruota in senso antiorario".\\
\newpage
\begin{example}
Esempio di funzione concava e convessa solo in sotto intervalli del dominio.
\end{example}
\begin{wrapfigure}[4]{r}{8cm}
    \vspace{-10pt}
    \centering
    \includegraphics[width=7cm]{images/seno-concavo-convesso.png}
\end{wrapfigure}

$f(x) = \sin{x}$, $f: [0, 2\pi]$. $f'(x) = \cos{x}$ e $f''(x) = -\sin{x}$.\\
$-\sin{x} \geq 0. \Longleftrightarrow \sin{x} \leq 0 \Longleftrightarrow x \in [\pi, 2\pi]$.\\
$f''(x) \geq 0 \Longleftrightarrow x \in [\pi, 2\pi]$ \hspace{.5cm} $f''(x) \leq 0 \Longleftrightarrow x \in [0, \pi]$\\\\

\begin{proposition}
Prendiamo un $I \subset \mathbb{R}$ intervallo, una $f: I \to \mathbb{R}$ derivabile. Allora $f$ è convessa in $I$ se e solo se $\forall \: x_0 \in I$ il grafico di $f$ è sopra la retta tangente nel punto $(x_0, f(x_0))$ cioè, $\forall \: x_0, x\in I$:
\[f(x) \geq f(x_0) + f'(x_0)(x-x_0)\]
Concava se vale il $\leq$. Stret. convessa se vale $>$ con $x \neq x_0$ e stret. concava se vale $<$ con $x\neq x_0$.
\end{proposition}

\begin{note}
Il grafico  di $f(x_0) + f'(x_0)(x-x_0)$ è la retta tangente.
\end{note}

\begin{example}
$f(x) = e^{-|x|}$, questa è una funzione pari e $f(x) = e^{-x}$ se $x \geq 0$.\\
Questa funzione non è ne concava ne convessa in tutto $\mathbb{R}$, perché ci sono dei tratti dove $f$ sta sotto altri dove sta sopra.\\\\
$f(x) = e^{-|x|} = \begin{cases}e^{-x} & \text{ se } x\geq 0\\e^{x} & \text{ se } x < 0\end{cases}$ \\\\
Quindi se $x>0$ $f'(x) = -e^{-x}$ e $f''(x) = e^{-x} > 0 \Longrightarrow f$ è convessa sull'insieme $\{x>0\}.$\\
Mentre se $x<0$ $f'(x) = e^{-x}$ e $f''(x) = e^{x} > 0 \Longrightarrow f$ è convessa sull'insieme $\{x\leq0\}.$
\end{example}
\hspace{-15pt}Da questo esempio vediamo che se prendiamo $f$ in due intervalli separati, in entrambi questi intervalli è convessa ma nell'unione dei due intervalli $f$ smette di essere convessa. Il motivo è che abbiamo un punto in $x=0$ di non derivabilità.

\begin{example}
Se invece prendiamo $f(x) = e^{|x|}$ quindi $f(x) = \begin{cases}e^x & \text{ se } x\geq 0 \\e^{-x} & \text{ se } x< 0 \end{cases}$.\\\\
In questo caso $f$ è convessa in $(-\infty, 0]$ ed è convessa anche in $[0, +\infty)$ e in questo caso $f$ è convessa anche in tutto $\mathbb{R}$.
\end{example}

\hspace{-15pt}Possiamo notare che nel secondo esempio se calcoliamo $f'_-(0) = -1$ e $f'_+(0) = 1$ mentre se vediamo l'esempio prima $f'_-(0) = 1$ e $f'_+(0) = -1$.

\begin{proposition}
Prendiamo un $I \subset \mathbb{R}$ intervallo, $x_0$ punto interno di $I$, $f:\mathbb{R} \to \mathbb{R}$ derivabile in $I \setminus \{x_0\}$. Siano $I_1 = \{x \in I \: |\: x<x_0\}$ e $I_2 = \{x\in I \: |\: x > x_0\}$ abbiamo che se $f$ è convessa in $I_1$ e $I_2$ e $x_0$ è un punto angoloso per $f$ allora $f$ è convessa in $I$ se e solo se $f'_-(x_0) \leq f'_+(x_0)$.
\end{proposition}

\hspace{-15pt}Questa cosa perché, se noi prendiamo una funzione che presenta un angolo e tracciamo la tangente, data dalla derivata, a sinistra notiamo che mano a mano che ci spostiamo verso destra questa tangente "ruoterà" sul grafico, nel punto $x_0$ avremo due tangenti una dalla derivata destra ed una dalla sinistra, possiamo notare che se la funzione rimane concava o convessa questa tangente continuerà a "ruotare" nello stesso verso senza fare "uno scatto" nel suo andamento, in caso contrario allora non manterrà la concavità o la convessità.

\subsection{Flessi}
\begin{definition}[Flesso]
Dato un $I\subset \mathbb{R}$ intervallo, $f: I\to \mathbb{R}$, $x_0$ punto interno ad $I$ si dice punto di flesso se $f$ è derivabile in $x_0$ ed esiste un intorno $U \subset I$ di $x_0$ t.c. la quantità
\[\frac{f(x) - (f(x_0) + f'(x_0)(x-x_0))}{x-x_0} \text{ non cambia segno in }U \setminus \{x_0\}\]
\end{definition}

\hspace{-15pt}Dire che $\frac{f(x) - (f(x_0) + f'(x_0)(x-x_0))}{x-x_0}$ non cambia segno vuol dire che il grafico della funzione passa da sopra a sotto la tangente (o viceversa).

\begin{definition}[Flesso a tangente verticale]
Se invece $f'(x) = \pm\infty$ ($f$ non è derivabile), $f$ è continua in $x_0$, e se $f$ è convessa in un intorno destro di $x_0$ e concava in un intorno sinistro di $x_0$ (o viceversa) allora $x_0$ si dice punto di flesso a tangente verticale. 
\end{definition}

\hspace{-15pt}Un flesso verticale è un cambiamento di convessità con un flesso verticale.

\begin{observation}
Se avete una funzione $f: I \to \mathbb{R}$, $I$ intervallo ed $f$ derivabile due volte in $I$. Allora se $f''(x_0)=0$ e $f$ cambia segno in $x_0$ allora $x_0$ è punto di flesso.
\end{observation}

\hspace{-15pt}Cambia segno vuol dire che $f''(x) \leq 0$ se $x\leq x_0$ e $f''(x) \geq 0$ se $x\geq x_0$ (o viceversa), con $x \in U$ intorno di $x_0$.

\begin{example}
Calcoliamo il flesso di $f(x) = x^3$, $f'(x) = 3x^2$, $f''(x) =6x$.\\
Vediamo dall'immagine che esiste un flesso in $x=0$, infatti:\\
$f''(x) = 0$, $f''(x) \leq 0$ se $x \leq 0$ e $f''(x) \geq 0$ se $x \geq 0$.
\end{example}

\begin{observation}
$f''(x_0) = 0$ non è sufficiente per aver un flesso
\end{observation}

\begin{example}
Prendiamo per verificare l'osservazione $f(x) = x^4$, $f'(x) = 4x^3$, $f''(x) = 12x^2$.\\
Anche se $f(0)=0$ abbiamo che $f''(x) \geq 0 \forall x \in \mathbb{R} \Longrightarrow $ f è convessa in $\mathbb{R}$.
\end{example}

\begin{observation}
Ci possono essere punti di flesso dove non esiste la derivata seconda.
\end{observation}

\begin{example}
$f(x) = x \cdot |x|$\hspace{.3cm} $f(x) = \begin{cases}x^2 & \text{ se } x \geq 0\\ -x^2 & \text{ se } x<0\end{cases}$\hspace{.3cm} $f'(x) = \begin{cases}2x & \text{ se } x>0 \\ -2x & \text{ se } x<0\end{cases}$\\\\
Possiamo vedere che $x_0 = 0$ è punto di flesso, infatti $f$ è derivabile in $x_0 = 0$ infatti $f'(0) = \lim\limits_{x\to 0}\frac{f(x) - f(0)}{x-0} = \lim\limits_{x\to 0}|x| = 0$.
La retta tangente in $x=0$ è $y=0$. $f$ passa da sopra la tangente in $x_0 = 0$, quindi $x_0$ è un punto di flesso. Però non esiste la derivata seconda in $x_0 = 0$ perché in questo punto c'è un punto angoloso.
\end{example}

\begin{observation}
Se abbiamo una funzione $f: I \to \mathbb{R}$, con $I \subset \mathbb{R}$, $f$ convessa nei punti interni di $I$, ed $f$ continua in tutto $I \Longrightarrow f$ è convessa in $I$.\\
Quindi se abbiamo $f:[a,b] \to \mathbb{R}$ convessa in $(a,b)$ ed $f$ continua in $[a,b] \Longrightarrow f$ è convessa in $[a,b]$.
\end{observation}

\newpage
\section{Studio di funzione}
\subsection{Punti da seguire}
Data una funzione $f(x)$ bisogna andare ad eseguire una serie di passi. $f(x)$ viene di solito assegnata senza specificare il dominio.
\begin{enumerate}
    \item Determinare l'insieme di definizione di $f$.
    \item Determinare l'insieme di continuità di $f$.
    \item Determinare l'insieme di derivabilità di $f$.
    \item Vedere eventuali asintoti orizzontali, verticali o obliqui.
    \item Studiare la monotonia della funzione.
    \item Trovare punti di massimo o di minimo locali.
    \item Determinare massimo e minimo d $f$ oppure estremo sup. ed inf.
    \item Studiare la convessità di $f$ (con eventuali punti di flesso).
\end{enumerate}

\subsection{Esempio studio di funzione}
\begin{example}
Studiamo la funzione $f(x) = \log|x| - \frac{x^2 - 1}{4x}$.
\begin{enumerate}
    \item $|x| > 0 \Longleftrightarrow x\neq 0$ e $4x \neq 0 \longleftarrow x \neq 0$. \textbf{Insieme di definizione} è $\mathbb{R} \setminus \{0\}$.
    \item La $f$ è \textbf{continua} in tutto $\mathbb{R} \setminus \{0\}$ (composizione funzioni continue e prodotto e sottrazioni funzioni continue).
    \item $f$ \textbf{derivabile} in tutto $\mathbb{R} \setminus \{0\}$ (sempre perché tutte queste funzioni sono derivabili in tutto il loro insieme di definizione, il valore assoluto non è derivabile in 0 ma non lo si considera).
    \item Per vedere gli \textbf{asintoti} dobbiamo fare i limiti ai bordo e sui punti non interi al dominio:\\
    $\lim\limits_{x\to -\infty}f(x) = \log|x| - \frac{x^2 -1}{4x} = \log|x| - \frac{x}{4} + \frac{1}{4x}$ \hspace{.5cm} $\lim\limits_{x\to -\infty}\log|-\infty| - \frac{-\infty}{4} + \frac{1}{4(-\infty)} = + \infty$.\\
    $\lim\limits_{x\to 0^-}\log|0^-| - \frac{0^-}{4} + \frac{1}{4(0^-)} = -\infty - 0 - \infty$. \hspace{.5cm} 
    $\lim\limits_{x\to 0^+}\log|0^+| - \frac{0^+}{4} + \frac{1}{4(0^+)} = -\infty - 0 + \infty$\\
    $\lim\limits_{x\to 0^+}f(x) = \lim\limits_{x\to 0^+}(-\frac{x}{4}) + \lim\limits_{x\to 0^+}\log|x| + \frac{1}{4x} = 0 + \lim\limits_{x\to 0^+}\frac{4x\log|x| + 1}{4x} = 0 + \frac{0+1}{4 \cdot 0^+} = +\infty$\\
    $\lim\limits_{x\to +\infty} \log|+\infty| - \frac{\infty}{4} + \frac{1}{4 \cdot \infty} = \infty - \infty + 0$\\
    $\lim\limits_{x\to +\infty} x(\frac{\log|x|}{4} - \frac{1}{4}) + \lim\limits_{x\to +\infty} \frac{1}{4x} = \infty(0 - \frac{1}{4}) + 0 = -\infty.$
    Abbiamo quindi un asintoto verticale di equazione $x=0$ e non ci sono asintoti orizzontali. Vediamo se ci sono asintoti obliqui:
    $\lim\limits_{x\to +\infty}\frac{f(x)}{x} = \lim\limits_{x\to +\infty}(\log|x| -\frac{x}{4} + \frac{1}{4x}) \cdot \frac{1}{x} = 0 - \frac{1}{4} + 0 = -\frac{1}{4}$, quindi $m= -\frac{1}{4}$\\
    $\lim\limits_{x\to +\infty}f(x) -mx = \lim\limits_{x\to +\infty} \log|x| - \frac{x}{4} + \frac{1}{4x} + \frac{1}{4}\cdot x = \infty + 0 = \infty$. \\
    Non c'è asintoto obliquo per $x\to +\infty$ e neanche a $x\to -\infty$ perché i conti sono uguali.
    \item Studiamo ora la \textbf{monotonia} di $f$.
    
    $\log|x| = \begin{cases}\log(x) & \text{ se } x>0 \\ \log(-x) & \text{ se } x<0\end{cases}$\hfill
    $D(\log|x|) = \begin{cases}D(\log(x)) = \frac{1}{x} & \text{ se } x>0 \\D(\log(-x)) = \frac{1}{-x} \cdot (-1) = \frac{1}{x} & \text{ se } x<0\end{cases}$
    
    Quindi possiamo notare che $D(\log |x|) = \frac{1}{x}$.\\
    $f(x) = \log|x| - \frac{x}{4} + \frac{1}{4x}$ \hspace{.3cm} $f'(x) = \frac{1}{x} - \frac{1}{4} -\frac{1}{4x^2} = \frac{-x^2 + 4x -1}{4x^2}$ conferma che è derivabile ovunque tranne che in $x = 0$.\\\\
    Il denominatore è $> 0$ in tutto il dominio, allora il segno di $f'$ è lo sesso del numeratore. Per trovare il segno bisogna trovare dove si annulla il numeratore.\\
    $-x^2 + 4x - 1 = 0 \Longleftrightarrow x^2 - 4x + 1 = 0$ \hspace{.3cm} $x = 2 \pm \sqrt{4-1} = 2 \pm \sqrt{3}$.\\
    $f$ è decrescente in $(-\infty,0)$, decrescente in $(0,2-\sqrt{3}]$, crescente in $[2-\sqrt{3}, 2+\sqrt{3}]$, decrescente in $[2+\sqrt{3},+\infty)$. Questa separazione va fatta perché il teorema di Lagrange prevede intervalli e lo 0 interrompeva l'intervallo.
    \item Vedendo la monotonia possiamo anche dire i punti di \textbf{massimo e minimo locali}. \\
    $x = 2 - \sqrt{3}$ è punti di minimo locale. \hspace{.3cm} $x = 2 + \sqrt{3}$ è punti di massimo locale.\\
    Per calcolare esattamente questi punti dove si collocano nel grafico basta sostituirli in $f(x)$.
    \item Dal fatto che $\lim\limits_{x\to +\infty} = +\infty$ otteniamo che $sup(f) = +\infty \Longrightarrow f$ non ha \textbf{massimo}. Dal fatto che $\lim\limits_{x\to 0^-}=-\infty$ otteniamo che $inf(f) = -\infty \Longrightarrow f$ non ah \textbf{minimo}.
    \item Come ultima calcoliamo al derivata seconda e troviamo la \textbf{convessità}.\\
    $f'(x) = \frac{1}{x} - \frac{1}{4} - \frac{1}{4x^2}$ \hspace{.5cm} $f''(x) = -\frac{1}{x^2} +\frac{1}{2x^3} = \frac{-2x + 1}{2x^2}$\\
    Segno del numeratore $-2x + 1 > 0 \Longleftrightarrow 1 > 2x \Longleftrightarrow x < \frac{1}{2}$.\\
    Segno del denominatore $2x^3 > 0 \Longleftrightarrow x > 0$.\\
    $f$ è concava in $(-\infty, 0)$ \hspace{.3cm} convessa in $(0,\frac{1}{2}]$ \hspace{.3cm} concava in $[\frac{1}{2},
    +\infty)$.\\
    Il punti di ascissa $x=\frac{1}{2}$ è punto di flesso visto che c'è un cambio di convessità.
\end{enumerate}
\end{example}
% !TeX spellcheck = it_IT
\newpage
\section{Integrali}
\begin{wrapfigure}[5]{r}{5.5cm}
    \vspace{-25pt}
    \centering
    \includegraphics[width=4cm]{images/area-sottografico.png}
\end{wrapfigure}
In questo corso tratteremo gli integrali detti \textbf{di Riemain}.
Sia $f: [a,b] \to \mathbb{R}$, limitata. (ad esempio una funzione continua). L'idea della definizione è che l'integrale (definito) di $f(x)$ in $[a,b]$ rappresenta l'area del sotto grafo di $f$  (questo è vero se $f \geq 0$ su $[a,b]$).

\begin{definition}[Suddivisione di un intervallo]
Una \textbf{suddivisione} di [a,b] è un insieme di $A = \{x_0, x_1, ..., x_n\}$ con $a = x_0 < x_1 < x:2 < ... < x_n = b$.
\end{definition}
\begin{observation}
Le lunghezze degli intervalli $[x_{i-1}, x_u]$ non sono necessariamente uguali.\\
Inoltre $\sum\limits_{i=1}^n(x_i - x_{i-1}) = b - a = $ lunghezza di $[a,b]$.
\end{observation}

\begin{definition}[Somma inferiore]
Dato una suddivisione di un intervallo A, si dice somma inferiore di $f$ relativa alla suddivisione di A 
\vspace{-5pt}
\[S'(f,A) = \sum^n\limits_{i=1} \big( \inf(f(x))_{x \in [x_{i-1}, x_i]} \big) \cdot (x_i - x_{i - 1})\]
\end{definition}
E la somma delle aree dei rettangoli rossi. Approssima l'area del sotto grafico di $f(x)$ per difetto.
\begin{definition}[Somma superiore]
Dato una suddivisione di un intervallo A, si dice somma superiore di $f$ relativa alla suddivisione di A 
\vspace{-5pt}
\[S'(f,A) = \sum^n\limits_{i=1} \big( \sup(f(x))_{x \in [x_{i-1}, x_i]} \big) \cdot (x_{i-1} - x_i)\]
\end{definition}
Somma delle aree dei rettangoli rossi. Questa volta è un'approssimazione per eccesso dell'area del sotto grafico.
\begin{observation}
Non server che $f$ sia continua per dare tutte queste definizione, ma soltanto che sia limitata.
\end{observation}
\begin{definition}[Somme indipendente dalle suddivisioni]
Le somme inferiori e superiori indipendenti dalle suddivisioni si definiscono come:
\begin{itemize}
    \item $S'(f) = sup\{S'(f,A) \:\: |$ A suddivisione di $[a,b]\}$ si dice somma inferiore di $f$.
    \item $S''(f) = inf\{S'(f,A) \:\: |$ A suddivisione di $[a,b]\}$ si dice somma superiore di $f$.
\end{itemize}
\end{definition}
Aggiungendo punti le somme inferiori crescono (e le somme superiori calano).

\begin{figure}[h!]
    \begin{subfigure}{.3\textwidth}
        \centering
        \includegraphics[width=4cm]{somma-superiore.png}
        \caption{Somma superiore}
    \end{subfigure}
    \begin{subfigure}{.3\textwidth}
        \centering
        \includegraphics[width=4.5cm]{somma-inferiore.png}
        \caption{Somma inferiore}
    \end{subfigure}
    \begin{subfigure}{.3\textwidth}
        \centering
        \includegraphics[width=4.5cm]{somma-indipendente.png}
        \caption{Somma indipendente}
    \end{subfigure}
    \caption{Somme delle sezioni}
\end{figure}

\begin{definition}[Integrabile secondo Rieman]
Se $S'(f) = S''(f)$ si dice che $f$ è \textbf{integrabile secondo Rieman} su $[a,b]$ e il valore comune si dice integrale di $f$ su $[a,b]$ e si indica come:
\[\int_{a}^b f(x)\:dx \: \: = S'(f) = S''(f)\]
\end{definition}
\newpage
\begin{observation}
Questa definizione ha senso anche quando $f$ può prendere anche valori negativi.
\end{observation}
\begin{wrapfigure}[4]{r}{6cm}
    \vspace{-15pt}
    \centering
    \includegraphics[width=5cm]{images/area-positiva-negativa.png}
\end{wrapfigure}
Se $f \leq 0 \Longrightarrow \int_a^b f(x)\:dx \leq 0$ ed è l'opposto dell'area in figura.
In generale $\int_a^b f(x)\:dx$ è la somma algebrica delle aree in figura (si sommano le aree dove l'integrale è positivo e si sottraggono quelle dove è negativo). \\

\begin{theorem}
Se $f: [a,b] \to \mathbb{R}$ è continua, allora è integrabile.
\end{theorem}

\begin{observation}
Ci sono anche funzioni non continue che sono integrabili, ad esempio una funzione con un punto in cui c'è un salto.
\end{observation}

\begin{definition}
Una $f: [a,b] \to \mathbb{R}$ è generalmente continua se è limitata e ha eventualmente un numero finito di punti di discontinuità.
\end{definition}

\begin{example}
Funzione non generalmente continua. $f(x) = \begin{cases}\frac{1}{x} & x\neq 0 \\ 0 & x=0\end{cases}$ con $f: [-1,1] \to \mathbb{R}$
\\C'è un solo punto di discontinuità, ma $f$ non è limitata $\Longrightarrow$ non è generalmente continua.
\end{example}

\begin{theorem}
Se $f:[a,b] \to \mathbb{R}$ è generalmente continua, allora f è integrabile.
\end{theorem}

\begin{example}
$f(x) = \begin{cases}\sin\frac{1}{x} & x\neq 0 \\ 0 & x=0\end{cases}$ con $f: [0,1] \to \mathbb{R}$.\\
$f(x)$ non è continua ma è generalmente continua $\Longrightarrow$ integrabile
\end{example}

\begin{example}
Esempio di una funzione non integrabile. (Esempio con la funzione di Dirichlet).
\end{example}
\begin{wrapfigure}[7]{r}{5cm}
    \vspace{-25pt}
    \centering
    \includegraphics[width=5cm]{images/funzione-dirichlet.png}
\end{wrapfigure}
$f(x) = \begin{cases}1 & x\in\mathbb{Q} \\ 0 & x \notin \mathbb{Q} \end{cases}$ con $f: [0,1] \to \mathbb{R}$.\\
Per qualsiasi intervallo $[x_{i-1}, x_i] \subseteq [0,1]$ si ha che:\\\\
$sup(f(x))_{x \in [x_{i-1}, x_i]} = 1$ e $inf(f(x))_{x \in [x_{i-1}, x_i]} = 0$. \\
Segue che $S'(f,A) = 0 \:\: \forall A$ suddivisione di $[0,1] \Longrightarrow S'(f) = 0$ e $S''(f,A) = 1 \:\: \forall A $ suddivisione di $[0,1] \Longrightarrow S''(f) = 1$. Quindi $S'(f) \neq S''(f) \Longrightarrow f$ non è integrabile.\\\\

\begin{wrapfigure}[3]{l}{5cm}
    \vspace{-30pt}
    \centering
    \includegraphics[width=4cm]{images/differenze-aree-integrale.png}
\end{wrapfigure}

Se $f$ è integrabile, $S''(f,A) - S'(f,A)$ (la differenza, l'area della regione verde nell'immagine) "tende a 0" al raffinarsi delle suddivisioni.
\vspace{15pt}
\subsection{Calcolo degli integrali}
\begin{theorem}
Siano $f,g$ integrabili su $[a,b]$ e un numero $k \in \mathbb{R}$, allora: $f+g, k\cdot, |f|$ sono integrabili, e si ha che:
\begin{enumerate}
    \item $\int_a^b (f+g)\:dx = \int_a^b f(x)\:dx + \int_a^b g(x)\:dx$.
    \item $\int_a^b (k\cdot f)\:dx = k \cdot \int_a^b f(x) \:dx$.
    \item Se $f(x) \leq g(x) \forall x \in [a,b]$ allora $\int_a^b f(x) \:dx \leq \int_a^b g(x)\:dx$.
    \item $\big|\int_a^b f(x) \:dx \big| \leq \int_a^b |f(x)|\:dx$.
    \item Se $a < c < b$ allora $\int_a^b f(x) \:dx = \int_a^c f(x) \:dx + \int_c^b f(x)\:dx$.
\end{enumerate}
\end{theorem}

\newpage
\begin{figure}[h!]
    \centering
    \begin{subfigure}{.3\textwidth}
        \centering
        \includegraphics[width=4.5cm]{images/teorema-calcolo-integrali-1.png}
        \caption{Caso 1°}
    \end{subfigure}
    \begin{subfigure}{.3\textwidth}
        \centering
        \includegraphics[width=4.5cm]{images/teorema-calcolo-integrali-2.png} 
        \caption{Caso 2°}
    \end{subfigure}
    \begin{subfigure}{.3\textwidth}
        \centering
        \includegraphics[width=4.5cm]{images/teorema-calcolo-integrali-3.png}
        \caption{Caso 3°}
    \end{subfigure}
\end{figure}

\begin{observation}
Osserviamo anche che se $f: [a,b] \to \mathbb{R}$ è constante, cioè $f(x) = k \:\: \forall x \in [a,b]$ allora $\int_a^b f(x) \:dx = k \cdot (b-a)$
\end{observation}

\subsection{Media Integrabile}
\begin{definition}[Media Integrabile]
Se $f: [a,b] \to \mathbb{R}$ integrabile, si dice \textbf{media integrabile} di $f$ su $[a,b]$.
\vspace{-10pt}
\[m = \frac{1}{b-a} \cdot \int_a^b f(x) \:dx\]\\
\end{definition}
\begin{wrapfigure}[2]{l}{5cm}
\vspace{-45pt}
    \centering
    \includegraphics[width=4cm]{images/media-integrabile.png}
\end{wrapfigure}
\vspace{-10pt}
Graficamente, m è l'altezza di un rettangolo di base $b-a$, con la stessa area del sotto grafico di $f$.
\vspace{15pt}
\begin{theorem}[Teorema della media integrale]
Sia $f:[a,b] \to \mathbb{R}$ integrabile, allora:
\vspace{-5pt}
\[inf(f(x))_{[a,b]} \leq \frac{1}{b-a} \cdot \int_a^b f(x) \:dx \leq sup(f(x))_{[a,b]}\]
Se $f$ è continua, allora $\exists x \in [a,b]$ tale che:
$f(z) = \frac{1}{b-a} \cdot \int_a^b f(x) \:dx$
\end{theorem}

\begin{demostration}
$\forall x \in [a,b]$ abbiamo $inf(f(x))_{[a,b]} \leq f(x) \leq sup(f(x))_{[a,b]}$. Integriamo questa disuguaglianza usando la proprietà (3) del teorema, e otteniamo:\\
$\int_a^b inf(f(x))_{[a,b]}\:dx \leq \int_a^b f(x)\:dx \leq \int_a^b sup(f(x))_{[a,b]}\:dx$. Sia $\int_a^b inf(f(x))_{[a,b]}\:dx$ che $\int_a^b sup(f(x))_{[a,b]}\:dx$ sono costanti $\Longrightarrow \big( inf(f(x))_{[a,b]} \big)(b-a) \leq \int_a^b f(x)\:dx \leq \big( sup(f(x))_{[a,b]} \big)(b-a)$.\\
Dividendo per $(b-a)$ ottengo proprio: $inf(f(x))_{[a,b]} \leq \frac{1}{b-a} \cdot \int_a^b f(x) \:dx \leq sup(f(x))_{[a,b]}$.\\\\
Se $f$ è continua, allora per il teorema di Weirstrass $inf(f) = min(f)$ e $sup(f) = max(f)$. Inoltre per il teorema dei valor intermedi $f$ prende tutti i valori compresi tra il $min(x)$ e $max(f)$. La media integrale è un tale valore per quanto visto, quindi $\exists z \in [a,b]$ tale che $f(z) = \frac{1}{b-a} \cdot \int_a^b f(x)\:dx$. $\blacksquare$
\end{demostration}

\begin{observation}
Se $b<a$, definiamo $\int_a^b f(x)\:dx = - \int_b^a f(x)\:dx$, e definiamo anche $\int_a^a f(x) = 0$.
\end{observation}

\begin{example}
$\int_2^1 x^3\:dx = -\int_1^2 x^3\:dx$
\end{example}

\hspace{-15pt}Le proprietà viste precedentemente valgono anche con i valori scambiati come nell'esempio sopra.
\begin{observation}
La media integrale ha senso anche quando gli estremi sono scambiati. Se $b < a$, allora $\frac{1}{b-a}\int_a^b f(x)\:dx = (\frac{1}{b-a})\big(-\int_a^b f(x)\:dx \big) = \frac{1}{a-b}\int_a^bf(x)\:dx$
\end{observation}

\begin{definition}[Primitiva]
Prendiamo un $I\subseteq \mathbb{R}$ intervallo, $f: I \to \mathbb{R}$, una funzione $F: I \to \mathbb{R}$ si dice primitiva di $f$ se $F$ è derivabile in $I$ e vale che $F'(x) = f(x) \:\: \forall x \in I$.
\end{definition}

\begin{example}
$f(x) = 2x$. Una primitiva è $F(x) = x^2$. Non è l'unica primitiva, $G(x) = x^2 + k$, $k\in \mathbb{R}$ ho comunque $G'(x) = 2x + 0 = f(x)$ quindi queste funzioni sono tutte primitive di $f(x) = 2x$. 
\end{example}
\hspace{-15pt}In generale, se $F$ è primitiva di $f$, tutte le funzioni $G(x) = F(x) + k$ con $k \in \mathbb{R}$ sono pure primitive di $f(x)$.

\begin{observation}
In effetti due primitive di $f(x)$ differiscono sempre per una costante.
\end{observation}

\begin{demostration}
Siano F e G due primitive di $f$. Allora ho che $F' = f$, $G' = f$. Quindi $(F - G)' = F' - G' = f - f = 0$. Visto che siamo su un intervallo, concludo che $F - G$ è costante $K \in \mathbb{R} \Longrightarrow F(x) = G(x) + k \:\: \forall x \in I$.
\end{demostration}

\begin{definition}[Integrale indefinito]
\textbf{L'integrale indefinito} di $f(x)$ è l'insieme di tutte le primitive di $f(x)$ e si indica con $\int f(x)\:dx$ (senza gli estremi).
\end{definition}

\begin{observation}
$\int f(x)\:dx$ non indica una singola funzione, ma un insieme di funzioni.
\[\int f(x)\:dx = \{F: I \to \mathbb{R} \:\: | \:\: F \: derivabile \: e \: F'=f\}\]
\end{observation}

\begin{example}
Se prendiamo per esempio $\int 2x\:dx = \{x^2 + k \:\:|\:\: k \in \mathbb{R}\}$ di solito si abbrevia scrivendo $\int 2x\:dx = x^2 + k$. 
\end{example}

\hspace{-15pt}L'integrale di Riemainn $\int_a^b f(x)\:dx$ invece è un numero reale, e rappresenta l'area del sotto grafico di $f$, e si dice \textbf{integrale definito} e $a,b$ sono gli \textbf{estremi di integrazione} ("a" è inferiore e "b" superiore). 

\subsection{Formule per integrali indefiniti}
Dalle formule per le derivate seguono formule per le primitive di una funzione f(x). Vedere la tabella di seguito.
\begin{table}[h!]
    \centering
    \setlength{\tabcolsep}{6pt}
    \renewcommand{\arraystretch}{1.5}
    \begin{tabular}{|c||c|}
        \hline
        $\int e^x=e^x + k$ & $\int \frac{1}{x}\:dx=\log|x| + k$\\
        
        $\int \cos(x)\:dx=\sin{x} + k$ & $\int \sin{x}\:dx=-\cos(x) + k$ \\
        
        $\int \frac{1}{1+x^2}\:dx=\arctan{x} + k$ & $\int \frac{1}{\sqrt{1 - x^2}}\:dx=\arcsin{x}$\\
        
        $\int \frac{1}{(\sin{x})^2}\:dx=-\cot{x}$ & $\int \frac{1}{(\cos{x})^2}\:dx=\tan{x}$ \\
        
        $\int x^n \:dx=\frac{1}{n+1}x^{n+1} + k$ & $\int -\frac{1}{x^2}\:dx=\frac{1}{x}$\\
        \hline
    \end{tabular}
    \caption{Formule primitive}
\end{table}
\vspace{-10pt}
\subsection{Teorema fondamentale del calcolo integrale}
\begin{theorem}[Teorema fondamentale del calcolo integrale]
Sia $I \subseteq \mathbb{R}$ un intervallo, $a \in I$, $f: I \to \mathbb{R}$ continua. Allora la funzione $F(x) = \inf_a^x f(t) \:dt$ (chiamata anche funzione integrale) è una primitiva di $f$, cioè $F(x)$ è derivabile e $F'(x) = f(x)$.
\end{theorem}

\begin{demostration}
Mostriamo che $F$ è derivabile calcolandone il rapporto incrementale in $x_0 \in I$ arbitrario, e poi facendo il limite.\\\\
$\frac{F(x) - F(x_0)}{x - x_0} = \frac{1}{x - x_0}\big( \int_a^x f(y) \:dt - \int_a^{x_0} f(y) \:dt \big) = \frac{1}{x - x_0} \int_{x_0}^x f(t) \:dt$. In risultato è la media integrale di $f$ sull'intervallo di estremi $x$ e $x_0$.\\\\
Visto che $f$ è continua, per il teorema della media integrale $\exists \: z(x)$ compreso tra $x_0$ e $x$ tale che $f(z(x)) = \frac{1}{x - x_0} \int_{x_0}^x f(t) \:dt$.\\
Quindi $F'(x_0) = \lim\limits_{x\to x_0}\frac{F(x) - F(x_0)}{x - x_0} = \lim\limits_{x\to x_0}f(z(x))$. Cambio variabile e prendo $y = z(x)$. Devo capire a cosa tende $y$ quando $x\to x_0$. So che $z(x)$ è compreso tra $x_0$ e $x$ (ad esempio se $x \leq x_0$, so che $x \leq z(x) \leq x_0$) quindi per il teorema dei carabinieri ho che $\lim\limits_{cx \to x_0}y = x_0$.\\\\
Segue che $\lim\limits_{x \to x_0}f(z(x)) = \lim\limits_{y \to x_0} f(y) = f(x_0)$ (questo per la continuità di f). Questo dimostra che $F'(x_0) = f(x_0)$m quindi $F'(x) = f(x) \:\: \forall x \in I$. $\blacksquare$
\end{demostration}

\newpage
\subsection{Teorema di Torricelli}
\begin{theorem}[Teorema di Torricelli]
$I \subseteq \mathbb{R}$ intervallo, $f: I \to \mathbb{R}$ funzione continua, $a \in I$. Se G è una primitiva di $f$ in I, allora $\exists k \in \mathbb{R}$ tale che $G(x) = \int_a^x f(t) \:dt + k$ e $\forall \alpha, \beta \in I$ abbiamo che $\int_{\alpha}^{\beta}f(t) \:dt = G(\beta) - G(\alpha)$.
\end{theorem}

\hspace{-15pt}A livello di notazioni si va a scrivere: $[G(x)]_{\alpha}^{\beta} = G(\beta) - G(\alpha)$

\begin{example}
Prendiamo $\int_1^3 x \:dx$. Una primitiva di $f(x) = x$ è $G(x) = \frac{x^2}{2}$. \\
Quindi $\int_1^3 x \:dx = [\frac{x^2}{2}]_1^3 = \frac{9}{2} - \frac{1}{2} = \frac{8}{2} = 4$. (Se prendiamo un'altra primitiva ad esempio $F(x) = \frac{x^2}{2} + 1$, trovato $\int_1^3 x \:dx = [\frac{x^2}{2} + 1]_1^3 = \frac{8}{2} = 4$)
\end{example}

\subsection{Integrali con estremi variabili}
\begin{theorem}
Dato un $I \subseteq \mathbb{R}$ intervallo, $f: I \to \mathbb{R}$ continua. Abbiamo poi $A\subseteq \mathbb{R}$, e $\alpha,\beta:A \to I$ derivabili. Sia $G(x) = \int_{\alpha(x)}^{\beta(x)} f(t) \:dt$. Allora $G(x)$ è derivabile e si ha:
\[G'(x) = f(\beta(x)) \cdot \beta'(x) - f(\alpha(x)) \cdot \alpha'(x)\]
In particolare se $\alpha(x) = a$ constante e $\beta(x) = x$, si ha $G(x) = \int_a^x f(t)\:dt$, e la formula scritta sopra è uguale a $f(x) \cdot \ - f(a) \cdot 0 = f(x)$. (Come della conclusione del teorema fondamentale)
\end{theorem}

\begin{example}
$G(x) = \int_{x^2}^{\sin{x}}e^t \cdot \arctan(t) \:dt$ \hfill $f(t) = e^t \arctan(t)$, $\alpha(x) = x^2$, $\beta(x) = \sin(x)$.\\
Abbiamo $G'(x) = f(\beta(x)) \cdot \beta'(x) - f(\alpha(x)) \cdot \alpha'(x) = e^{\sin{x}} \cdot \arctan(\sin(x)) \cdot \cos(x) - e^{x^2} \cdot \arctan(x^2) \cdot 2$.\\
Applicazione: $\lim\limits_{x\to 0}\frac{\int_0^{x^2} e^t \cdot \arctan(x) \:dt}{\sin(x^4)} = \frac{\int_0^0 (...)}{\sin(0)} = \frac{0}{0}$. Usiamo de l'hopital.\\
$\lim\limits_{x\to 0} \frac{e^{x^2}\cdot \arctan(x^2) \cdot 2x}{\cos(x^4) \cdot 4x^3} = \lim\limits_{x\to 0}\frac{e^{x^2}}{\cos(x^4)}\cdot\frac{\arctan(x^2) \cdot x}{2x^3} = \lim\limits_{x\to 0} \frac{e^{x^2}}{\cos(x^4)}\cdot \frac{\arctan(x^2)}{2x^2} = \lim\limits_{x\to 0}\frac{e^{x^2}}{\cos(x^4)}\cdot\frac{x^2 + o(x^2)}{2x^2} = 1 \cdot \frac{1}{2}$
\end{example}

\subsection{Metodi di calcolo per integrali indefiniti}
\subsubsection{Integrazione per parti}
Prendiamo $f,g: I \to \mathbb{R}$ con $I\subseteq \mathbb{R}$ intervallo, $f$ continua e $g$ di classe $C^1$ ($g$ e derivabile e la derivata è continua). Se $F$ è una primitiva di $f$ allora:
\vspace{-5pt}
\[\int f\cdot g\:dx = F \cdot g - \int F \cdot g' \:dx\]

\begin{demostration}
Se faccio la derivata del prodotto $(F \cdot g)' = F'\cdot g + F\cdot g' = fg + F'g$. \\
(Se due funzioni sono uguali anche gli integrali indefiniti delle due funzioni sono uguali)Integrando ambo i membri ottengo che $\int (Fg)'\:dx = \int (fg)\:dx + \int F\cdot g' \:dx = \int F\cdot g\:dx = \int (fg)\:dx + \int F\cdot g' \:dx$. Abbiamo così dimostrato la formula. $\blacksquare$
\end{demostration}

\hspace{-15pt}Esempi ed esercizi guarda i lucidi delle lezioni (gli appunti del professore).

\begin{observation}
Se il ho $\log(f(x))' = \frac{f'(x)}{f(x)}$ (sto supponendo che $f(x) > 0$), quindi segue che $\int \frac{f'(x)}{f(x)} \:dx = \log(f(x)) + k$.
\end{observation}

\subsubsection{Integrazione per sostituzione}
Supponiamo di avere $I,J \subseteq \mathbb{R}$ intervalli, $f: I \to \mathbb{R}$ continua. Prendiamo poi $\phi: J \to I$ di classe $C^1$. Se $F$ è una primitiva di $f$, allora $\int (f \circ \phi) \cdot \phi' \:dx = (F \circ \phi) + k$.

\begin{demostration}
$(F \circ \phi)' = (F'(\phi)) \cdot \phi' = (f \circ \phi) \cdot \phi'$ per la regola di derivazione di funzioni composte, Integrando trovo che: $\int (f \circ \phi) \cdot \phi' \: dx = \int (F \circ \phi)' \:dx = (F \circ \phi) + k$. $\blacksquare$
\end{demostration}

\begin{example}
Prendiamo $\int xe^{x^2}\:dx$. Pongo $t=x^2$ (funzione di x), $\frac{dt}{dx} = dx$ quindi $dt = 2xdx$, $\frac{dt}{2} = xdx$.\\
$\int e^t \cdot \frac{dt}{2} = \frac{1}{2}\int e^t \:dt = \frac{1}{2} e^t +c$ e poi si torna in $x$ sostituendo $t=x^2$ quindi torna $\frac{1}{2}e^{x^2} + k$.\\\\
Per gli integrali definiti possiamo fare in due modi. Prendiamo $\int_0^2 xe^{x^2}\:dx$:
\begin{enumerate}
    \item Calcolare $\int xe^{x^2}\:dx$. Abbiamo che che è $\frac{1}{2}e^{x^2} + k$. Poi $\int_0^2 xe^{x^2}\:dx = [\frac{1}{2}e^{x^2} + k]_0^2 = \frac{1}{2}(e^4-1)$.
    \item Possiamo usare la sostituzione, ricordandosi di cambiare gli estremi: $\int_0^2 xe^{x^2}\:dx$ pongo come prima  $dt = 2xdx$, $\frac{dt}{2} = xdx$.\\
    Quindi $\int_0^2 xe^{x^2}\:dx = \int \frac{e^t}{2}\:dt$ e bisogna calcolare gli estremi vedendo quanto vale t negli estremi.\\
    $x= 0$ quindi $t = 0^2 = 0$ e $x = 2$ quindi $t = 2^2 = 4$. Alla fine avremo  $\int_0^4 xe^{x^2}\:dx$, da qui poi si va avanti come prima.
\end{enumerate}
\end{example}

\begin{example}
$\int \sqrt{1-x^2}\:dx$. $x = \sin(t)$, $t = \arcsin(x)$ e $\frac{dx}{dt} = \cos(t)$ quidi $dx = \cos(t) \:dt$.\\
$= \int \sqrt{1-\sin(t)^2} \cdot \cos(t) \:dt = \int \sqrt{\cos(t)^2} \cdot \cos(t) \:dt = \int |\cos(t)| \cdot \cos(t) \:dt$ (il valore assoluto si toglie visto che $\cos(t) \geq 0$ nell'intervallo in cui stiamo integrando che è fra $-\frac{\pi}{2}$ e $\frac{\pi}{2}$).\\
$\int \cos(t)^2 \:dt = \frac{t + \sin(t)\cdot\cos(t)}{2} + c = \frac{\arcsin(x) + x \cdot \sqrt{1 - x^2}}{2} + c$.
\end{example}

\hspace{-15pt}Se andiamo a fare l'integrale di $f(x) = \sqrt{1-x^2}$ si va a calcolare l'area del cerchio unitario.\\
Infatti $4 \int_0^1 \sqrt{1-x^2}\:dx = 4\big[\frac{\arcsin(x) + x\sqrt{1-x^2}}{2}\big]_0^1 = 4 \cdot \frac{\arcsin(1)}{2} = 2 \frac{\pi}{2} = \pi$.\\\\
Se volessimo calcolare $\int \frac{1}{\sqrt{1 - x^2}}\:dx = \arcsin(x) + k$ visto che $(\arcsin(x))' = \frac{1}{\sqrt{1 - x^2}}$.
\begin{observation}
Ho anche $(\arccos(x))' = -\frac{1}{\sqrt{1-x^2}}$, quindi $\int \frac{1}{\sqrt{1-x^2}}\:dx = -\int -\frac{1}{\sqrt{1-x^2}}\:dx = -\arccos(x) + k'$. Segue che $\arcsin(x) - (-\arccos(x))$ è costante. Per vedere quanto vale basta calcolare in $x=0$, e trovo $\arcsin(0) + \arccos(0) = 0 + \frac{\pi}{2}$. Quindi  $\arcsin(x) + \arccos(x) = \frac{\pi}{2} \:\: \forall x \in [-1,1]$.
\end{observation}

\subsection{Integrali di funzioni razionali}
Consideriamo integrali nella forma $\int \frac{p(x)}{q(x)} \:dx$ dove $p(x)$ e $q(x)$ sono polinomi in x ed il grado di $q(x) \leq 2$, $\deg(q(x)) \leq 2$.
\begin{itemize}
    \item Caso con denominatore ha grado 1, $\deg(q(x)) = 1$.\\
    Esempio caso particolare con numeratore costante con $\int \frac{1}{ax + b}\:dx$. In questo caso usiamo la sostituzione $y = ax+b$ e $dy = a \cdot dx$.\\
    $= \int \frac{1}{y} \cdot \frac{dy}{a} = \frac{1}{a} \int \frac{1}{y}\:dy = \frac{1}{a} \log|ax+b| + c$.\\\\
    Caso con $\deg(p(x)) > 0$. Usiamo il caso precedente ma facendo prima la divisione di polinomi di $p(x)$ per $q(x) = ax + b$. Cioè scriviamo:\\
    $p(x) = (ax + b) \cdot Q(x) + R(x)$ dove $Q(x)$ e $R(x)$ sono polinomi e $\deg R(x) < \deg(ax + b) = 1$, (allora R(x) è una costante ed è uguale a $p(-\frac{b}{a})$).\\\\
    C'è un algoritmo per fare la divisione in maniera veloce:
    \begin{example}
    Prendiamo $\int \frac{2x^2 + 1}{x+1}\:dx$. $\frac{p(x)}{ax + b} = \frac{2x^2 +1}{x+1}$, quindi $a=1$ e $b=1$.\\
    Divido $2x^2 + 1$ per $x+1$. (Fare la divisione, vedere gli appunti delle lezioni per il modo preciso).\\
    Il risultato è: $\int \frac{(x+1)(2x-2)+3}{x+1}\:dx = \int \frac{(x+1)(2x-2)}{(x+1)}dx + \int \frac{3}{x+1} = \int (2x-2)dx + 3\log|x+1| + c = x^2 -2x + 3\log|x+1| + c$
    \end{example}
    
    \item Caso con grado denominatore uguale a 2, $\deg(q(x)) = 2$.\\
    Il primo passaggio è sempre quello di fare la divisione scrivendo $p(x) = (ax^2 + bx + c) \cdot Q(x) + R(x)$ dove $\deg R(x) <2$ cioè $R(x) = cx + d$.\\
    Quindi $\int \frac{p(x)}{q(x)}dx = \int \frac{(ax^2 + bx + c) \cdot Q(x) + R(x)}{ax^2 + bx + c} dx = \int Q(x)dx + \int \frac{R(x)}{ax^2 + bx + c}dx$, dove $R(x) = cx+d$.\\
    Per calcolare gli integrali di questa forma rimane da vedere come calcare $\int \frac{cx + d}{ax^2 + bx + c}dx$.\\
    Ci sono usa serie di casi particolare da analizzare, a seconda del numero di radici reali del denominatore:
    \begin{enumerate}
        \item Due radici coincidenti e numeratore costante. $\int \frac{dx}{(x-a)^2}$, si fa una sostituzione del tipo $y= x-a$ e $dy= dx$.
        $\int \frac{dy}{y^2} = \int y^{-2}\:dy = \frac{1}{-1} \cdot y^{-1}+c = -\frac{1}{y} + c = -\frac{1}{x-a} + c$.
        \item Due radici reali distinte e numeratore costante. $\int \frac{dx}{(x-a)(x-b)}$ con $a\neq b$. Si cercano due numeri reali A e B tali che valga:\\
        $\frac{1}{(x-a)(x-b)} = \frac{A}{(x-a)} + \frac{B}{(x-b)} = \frac{A(x-b) + B(x-a)}{(x-a)(x-b)} = \frac{x(A+B) - Ab - Ba}{(x-a)(x-b)}$. Se voglio che valga questa uguaglianza, per il principio di identità dei polinomi deve essere che:\\\\
        $\begin{cases}A+B=0\\-Ab-Ba = 1\end{cases}=$\hspace{.3cm}$\begin{cases}B = -A\\-Ab + Aa = 1\end{cases}=$\hspace{.3cm}$\begin{cases}A+B=0\\A=\frac{1}{a-b}\end{cases}=$\hspace{.3cm} $\begin{cases}B = -\frac{1}{a-b}\\A=\frac{1}{a-b}\end{cases}$\\\\
        A questo punto posso sostituire con l'espressione trovata sopra:\\
        $\int \frac{dx}{(x-a)(x-b)} = \int (\frac{1}{a-b} \cdot \frac{1}{(x-a)} - \frac{1}{a-b}\cdot\frac{1}{(x-b)})dx = \frac{1}{a-b} (\log|x-a) - \log|x-b|) + c$
    \end{enumerate}
    
    \item Denominatore senza radici reali e numeratore costante.\\
    $\int \frac{dx}{1+x^2}dx = \arctan(x) + c$. Generalizzando $\int \frac{dx}{k^2 + x^2}$ con $k \in \mathbb{R}$ e $k\neq 0$.\\
    $\int \frac{dx}{k^2 + x^2} = \frac{1}{k^2}\cdot \int \frac{dx}{1 + (\frac{x}{k})^2}$ facciamo poi una sostituzione con $y= \frac{x}{k}$ e $dy = \frac{dx}{k}$.\\
    $\frac{1}{k^2} \int \frac{1}{a+y^2} \cdot k \: dy = \frac{1}{k} \cdot \arctan(y) + c = \frac{1}{k} \cdot \arctan(\frac{x}{k}) + c$.\\
    Il casi generale con il denominatore come $ax?2 +bx + c$ senza radici reali, cioè $\Delta < 0$. In realtà posso supporre che $a = 1$:
    $\int \frac{1}{ax^2 + bx + c}dx = \frac{1}{a}\cdot \int \frac{1}{x^2 + \frac{b}{a}x + \frac{c}{a}}dx$. Quindi guardo polinomi della forma $x^2 + bx + c$ con $\Delta < 0$. Io posso fare $x^2 + bx + c = (x^2 + bx + \frac{b^2}{4}) - \frac{b^2}{4} + c = (x + \frac{b}{2})^2 + \frac{1}{4}(-b^2 + 4c)$.\\
    $\int \frac{dx}{x^2 + x + c} = \int \frac{dx}{(x + \frac{b}{2})^2 + k^2}$, con $k^2 = \frac{1}{4}(-b^2 + 4c) > 0$. Se poi andiamo a sostituire con $y = x + \frac{b}{2}$ e $dy = dx$ abbiamo $\int \frac{1}{y^2 + k^2}dx$ e questo lo sappiamo fare perché visto sopra ed è $\frac{1}{k}\arctan(\frac{x + \frac{b}{2}}{k})+c$.
    
    \begin{example}
    $\int \frac{dx}{x^2 + 2x + 10} = \int \frac{dx}{x^2 + 2x + 1 - 1 + 10} = \int \frac{dx}{(x+1)^2 + 9} = \int \frac{dy}{y^2 + 9} = \frac{1}{3}\arctan(\frac{x+1}{3})+c$.\\
    (se si fosse scelto $k=-3$ invece che $k=3$ sarebbe venuto lo stesso risultato perché $-\frac{1}{3}\arctan(-\frac{y}{3}) + c = \frac{1}{3}\arctan(\frac{y}{3})$)
    \end{example}
    
    \item Caso nel quale il denominatore non è costante, cioè ha grado 1, bisogna vedere come comportarsi con il numeratore.\\
    $\int \frac{ax + b}{x^2 + cx + d}dx = \frac{a}{2} \int \frac{2x + \frac{2b}{a}}{x^2 + cx + d}dx = \frac{a}{2}\int \frac{2x + c - c + \frac{2b}{a}}{x^2 + cx + d} = \frac{a}{2}\int \frac{2x + c}{x^2 + cx + d}dx + \frac{a}{2}\int \frac{-c \frac{2b}{a}}{x^2 + cx + d}dx$ ora per il primo integrale il numeratore è la derivata del denominatore, mentre nel secondo essendoci una costante al numeratore lo sappiamo fare.\\
    $\frac{a}{2}\log|x^2 + cx + d| + \frac{a}{2}\int \frac{-c + \frac{2b}{a}}{x^2 + cx + d}dx$.
    \begin{example}
    $\int \frac{4x + 5}{x^2 + 2x - 1}dx = 2\int \frac{2x + \frac{5}{2} + 2 - 2}{x^2 + 2x - 1}dx = 2 \int \frac{2x + 2}{x^2 + 2x - 1}dx + 2 \int \frac{\frac{1}{2}}{x^2 + 2x - 1}dx =$\\ $=2\log|x^2 + 2x - 1| + \int \frac{1}{x^2 + 2x - 1}$
    \end{example}
\end{itemize}

\subsection{Integrali impropri}
Gli Integrali impropri o generalizzati estendono la definizione di integrale definito al caso in cui l'integrale non è limitato, oppure l'intervallo di integrazione non è limitato.
\begin{example}
Dobbiamo dare un senso per esempio a $\int_0^{+\infty}e^{-x}\:dx$.
\end{example}
\begin{wrapfigure}[6]{r}{5cm}
    \vspace{-25pt}
    \centering
    \includegraphics[width=4.5cm]{images/esempio-integrale-improprio-1.png}
\end{wrapfigure}
Intuitivamente rappresenta l'area di tutto il sotto grafico sopra $(0,+\infty)$.
Formalmente definiremo un limite: \\\\
$\lim\limits_{M\to +\infty}\int_0^Me^{-x}\:dx = \lim\limits_{M\to + \infty}[-e^{-x}]^M_0=\lim\limits_{M\to +\infty}-e^{-M}+1 = 1$.\\
In questo caso il sotto grafico $[0,+\infty)$ ha area finita uguale a 1.

\begin{example}
Se invece prendiamo $\int_0^{+\infty}\frac{1}{1+x}dx = \lim\limits_{M\to +\infty}\int_0^{M}\frac{1}{1+x}dx = \lim\limits_{M\to +\infty}[\log(1+x)]_0^M = \lim\limits_{M\to +\infty}(\log(1+M)-0) = +\infty$. In questo caso l'area del sotto grafico è infinito.
\end{example}
\newpage
\begin{example}
Facciamo un' altro esempio di integrale improprio con $\int_0^1 \frac{1}{\sqrt{x}}dx$.
\end{example}
\begin{wrapfigure}[5]{l}{4.5cm}
    \vspace{-25pt}
    \centering
    \includegraphics[width=3.7cm]{images/esempio-intergale-improprio-2.png}
\end{wrapfigure}
$\int_0^1 \frac{1}{\sqrt{x}}dx$ notiamo che la funzione $\frac{1}{\sqrt{x}}$ non è limitata sull'intervallo compreso fra $[0,1)$.\\\\
$\lim\limits_{M\to 0^+}\int_M^1 \frac{1}{\sqrt{x}}dx = \lim\limits_{M\to 0^+}[2\sqrt{x}]_M^1 = \lim\limits_{M\to 0^+}(2-2\sqrt{M}) = 2$, l'area del sotto grafico di $\frac{1}{\sqrt{x}}$ sopra a $[0,1]$.
\vspace{10pt}
\begin{example}
$\int_0^1 \frac{1}{x}dx = \lim\limits_{M\to 0^+}\int_0^1 \frac{1}{x}dx = \lim\limits_{M\to 0^+}[\log(x)]_M^1 = \lim\limits_{M\to 0^+} (0-\log(M)) = + \infty$.\\
Quindi in questo caso il sotto grafico ha area infinita.
\end{example}
\begin{definition}[Integrali impropri o generalizzati]
Dati due punti $a\in \mathbb{R}$ e $b \in \mathbb{\overline{R}}$, $a<b$ e $f:[a,b)\to \mathbb{R}$ che sia integrabile in tutti gli intervalli $[a,M]$ con $a<M<b$. Se esiste $\lim\limits_{M\to b^-}\int_a^M f(x)\:dx = L$, definiamo $\int_a^b f(x)\:dx = L$. 
\begin{itemize}
    \item Se L è reale finito si dice che l'integrale di $f(x)$ su $[a,b)$ converge (oppure che $f(x)$ è integrabile "in senso generalizzato su $[a,b)$").
    \item Se L è uguale a $+\infty$ si dice che l'integrale diverge positivamente (o "a $+\infty$").
    \item Se L è uguale a $-\infty$ si dice che l'integrale diverge negativamente (o "a $-\infty$").
\end{itemize}
\end{definition}
\hspace{-15pt}Vedendo gli esempi visti sopra possiamo dire che:\\
$\int_0^{+\infty}e^{-x}\:dx$ converge \hfill $\int_0^{+\infty}\frac{1}{1+x}\:dx$ diverge pos. \hfill $\int_0^1 \frac{1}{\sqrt{x}}\:dx$ converge \hfill $\int_0^1\frac{1}{x}\:dx$ diverge pos.

\begin{example}
Esempio in cui il limite non esiste: 
\end{example}
\begin{wrapfigure}[2]{l}{4.5cm}
    \vspace{-25pt}
    \centering
    \includegraphics[width=4cm]{images/esempio-integrale-improprio-3.png}
\end{wrapfigure}
$\int_0^{+\infty}\cos(x)\:dx = \lim\limits_{M\to +\infty}\int_0^M\cos(x)\:dx = \lim\limits_{M\to +\infty}[\sin(x)]_0^M = \lim\limits_{M\to +\infty}(\sin(M)-0)$ e questo non esiste.\\\\

\hspace{-15pt}Analogamente si definisce $\int_a^b f(x)\:dx$ quando $f:(a,b] \to \mathbb{R}$ con $a \in \mathbb{\overline{R}}$, $b\in \mathbb{R}$ e $f$ integrabile su $[M,B] \forall a < M < b$ come $\lim\limits_{M\to a^+}\in_M^b f(x) \:dx$ (se esiste).\\\\
Se però abbiamo $f:(a,b)\to \mathbb{R}$ questa funzione "ha un problema" in entrambi a e b, ad esempio $\int_{-\infty}^{+\infty}\frac{1}{1+x^2}\:dx$ oppure $\int_{-1}^1 \frac{1}{1-x^2}\:dx$.

\begin{definition}
Sia $f: (a,b)\to \mathbb{R}$ con $a,b \in \mathbb{\overline{R}}$ che sia integrabile su $[M_1,M_2]$ con $a<M_1<M_2<b$. Scegliamo arbitrariamente $c\in (a,b)$, se esistono entrambi $\int_a^c f(x)\:dx$ e $\int_c^b f(x)\:dx$ allora si definisce:
\[\int_a^b f(x)\:dx = \int_a^c f(x)\:dx + \int_c^b f(x)\:dx \text{ Se la somma non è indeterminata (cioè non è} +\infty-\infty)\]
E in questo caso si diche che $f$ è integrabile in senso improprio su (a,b)
\end{definition}

\begin{observation}
L'esistenza e il valore di $\int_a^b f(x) \:dx$ non dipende dalla scelta di $c\in (a,b)$.\\
Se scelgo $d\in (a,b)$ ho $\int_a^b f(x)\:dx = \int_a^c f(x)\:dx + \int_c^b f(x)\:dx$ e $\int_d^b f(x)\:dx = \int_d^c f(x)\:dx + \int_c^b f(x)\:dx$.\\
Sommando queste due equazioni ottengo $\int_a^b f(x)\:dx + \int_d^b f(x)\:dx = \int_a^c f(x)\:dx + \int_c^b f(x)\:dx + \int_c^d f(x)\:dx + \int_d^c f(x)\:dx$ la seconda somma $\int_c^d f(x)\:dx + \int_d^c f(x)\:dx = 0$ quindi vediamo che il risultato non cambia.
\end{observation}

\begin{example}
$\int_{-\infty}^{+\infty} \frac{1}{1+x^2}\:dx$. Scelgo $c=0$.\\
$\int_{-\infty}^0 \frac{1}{1+x^2}\:dx = \lim\limits_{M\to -\infty} \int_M^0\frac{1}{1+x^2}\:dx = \lim\limits_{M\to -\infty} [\arctan(x)]_{-M}^0 = \lim\limits_{M\to -\infty}[0 -\arctan(M)] = +\frac{\pi}{2}$\\\\
$\int_0^{+\infty}\frac{1}{1+x^2}=$ (stessi conti di prima) $= \frac{\pi}{2}$.
Quindi $\int_{-\infty}^{+\infty}\frac{1}{1+x^2}\:dx = \frac{\pi}{2}+\frac{\pi}{2} = 2$ (converge)
\end{example}

\begin{example}
$\int_0^{+\infty}\frac{1}{x^2}dx$, scegliamo $c=1$.\\
$\int_0^1 \frac{1}{x^2}\:dx = \lim\limits_{M\to 0^+} \int_M^1\frac{1}{x^2}\:dx = \lim\limits_{M\to 0^+}[-x^{-1}]_M^1 = \lim\limits_{M\to 0^+}[-1+\frac{1}{M}]_M^1 = +\infty$ (diverge positivamente).\\
$\int_1^{+\infty}\frac{1}{x^2}\:dx = \lim\limits_{M\to +\infty}\int_1^M \frac{1}{x^2}\:dx = \lim\limits_{M\to +\infty}[-x^{-1}]_1^M = \lim\limits_{M\to +\infty}[-\frac{1}{M} + 1] = 1$.\\
Quindi $\int_0^+\infty \frac{1}{x^2}\:dx = +\infty + 1 = +\infty$ quindi il grafico diverge positivamente.
\end{example}

\begin{example}
Prendiamo $\int_{-1}^1 \frac{1}{x}\:dx$ in questo caso spezziamo in $c=0$.\\
$\int_{-1}^1 \frac{1}{x}dx = \int_{-1}^0 \frac{1}{x}\:dx + \int_0^1\frac{1}{x}\:dx$.\\
$\int_{-1}^0 \frac{1}{x}\:dx = \lim\limits_{M\to 0^-}\int_{-1}^M =  \lim\limits_{M\to 0^-} [\log(-x)]_{-1}^M =  \lim\limits_{M\to 0^-} \log(-M) = -\infty$.\\
$\int_0^1\frac{1}{x}\:dx = \lim\limits_{M\to 0^+}\int_M^1\frac{1}{x}\:dx = \lim\limits_{M\to 0^+} [\log(x)]_M^1 = \lim\limits_{M\to 0^+} -\log(M) = +\infty$.\\
Se vado a fare la soma ho che la somma è indeterminata $\int_{-1}^1 \frac{1}{x}\:dx = +\infty - \infty$ e dunque non esiste.\\\\
Attenzione a non fare $\int_{-1}^1 \frac{1}{x}\:dx = [\log|x|]_{-1}^1 = \log(1) - \log(1) = 0$ perché è sbagliato, il teorema di Torricelli non si applica perché $f$ non è integrabile su $[-1,1]$. Bisogna trattarlo come integrale improprio. 
\end{example}

\begin{observation}
I potrebbe pensare che ha senso dire che $\int_{-1}^1 \frac{1}{x}\:dx = 0$ visto che $\frac{1}{x}$ è dispari, e le aree sopra e sotto si sovrappongono perfettamente. Si preferisce dire comunque che l'integrale non esiste.\\\\
Si potrebbe sommare $\int_{-1}^a\frac{1}{x}\:dx + \int_b^1\frac{1}{x}\:dx$ e far tendere $a\to 0^-$ e $b\to 0^+$.\\
Il problema è che il risultante del limite dipende da come viene fatto questo limite.
\begin{example}
$\lim\limits_{b\to 0^+}(\int_{-1}^{-b}\frac{1}{x}\:dx + \int_b^1\frac{1}{x}\:dx) = \lim\limits_{b\to 0^+}(\log(b)-\log(b))=0$. \\
Ma per esempio se prendiamo $-2b$ invece che b abbiamo $\lim\limits_{b\to 0^+}(\int_{-1}^{-2b}\frac{1}{x}\:dx + \int_b^1\frac{1}{x}\:dx) = \lim\limits_{b\to 0^+}(\log(2b)-\log(b)) = \lim\limits_{b\to 0^+} \log(\frac{2b}{b}) = \log(2)$ ed il risultato è diverso.
\end{example}
\end{observation}

\hspace{-15pt} Se ci sono "più problemi" sull'intervallo di integrazione si spezza in tanti intervalli quanto basta per ricondursi a integrali impropri in cui c'è solo un problema.
\begin{example}
$\int_{-\infty}^{+\infty}\frac{1}{x^4-1}dx$ ci sono problemi sia agli estremi, più ha due asintoti a -1 e 1.\\
Quindi si spezza come $\int_{-\infty}^{+\infty} = \int_{-\infty}^{-2} + \int_{-2}^{-1} + \int_{-1}^{0} + \int_{0}^{1} + \int_{1}^{2} + \int_{2}^{+\infty}$ e la somma ha senso se hanno senso (cioè i limiti esistono) e non è indeterminata.
\end{example}

\begin{observation}
In questi casi si scrive comunque $\int_{-1}^1 \frac{1}{x}\:dx$ e non $\int_{[-1,0)\cup(0,1]}\frac{1}{x}\:dx$.
\end{observation}

\begin{proposition}
Data una $f:[a,b)\to \mathbb{R}$ integrabile su $[a,M] \forall \: a<M<b$ e supponiamo che $f$ abbia segno costante. Allora esiste (finito o infinito) $\int_a^b f(x)\:dx$. Ed esiste un enunciato analogo per il caso simmetrico $f:(a,b]\to \mathbb{R}$.
\end{proposition}

\begin{demostration}
Supponiamo ad esempio che $f\geq 0$ su $[a,b)$. Mostriamo che $F(x) = \int_a^x f(t)\:dt$ è debolmente crescente. Seguirà che $\exists \lim\limits_{x\to b^-}F(x)$ che è proprio $\int_a^b f(t)\:dt$. Infatti se $x_1 < x_2$, allora $F(x_2) = \int_a^{x_2}f(t)\:dt = \int_a^{x_1} + \int_{x_1}^{x^2}f(t)\:dt \geq \int_a^{x_1}F(x_1)$.
Il pezzo $\int_{x_1}^{x^2} \geq 0$ perché $f(t) \geq 0$ e $x_2 > x_1$. $\blacksquare$
\end{demostration}

\subsubsection{Integrali impropri notevoli}
Con la forma: $\int_1^{+\infty}\frac{1}{x^{\alpha}}$ con $\alpha \in \mathbb{R}$.
\begin{itemize}
    \item Se $\alpha = 1$, $\int \frac{1}{x}\:dx = \log|x| \longrightarrow \int_1^{+\infty}\frac{1}{x}\:dx = \lim\limits_{M\to +\infty}[\log(x)]_1^M = \lim\limits_{M\to +\infty}(\log(M)) = +\infty$, diverge.
    \item Se $\alpha \neq 1$, $\int \frac{1}{x} = \int x^{-\alpha}\:dx = \frac{1}{1-\alpha}x^{1-\alpha} + c$.\\
    Quindi $\int_1^{+\infty}\frac{1}{x^{\alpha}} = \lim\limits_{M\to +\infty} [\frac{1}{1-\alpha}x^{1-\alpha}]_1^M = \lim\limits_{M\to +\infty} (\frac{1}{1-\alpha}M^{1-\alpha} - \frac{1}{1-\alpha})$.
    \item Se $1-\alpha > 0$, cioè $\alpha < 1$, il limite è $+\infty$.
    \item se $1-\alpha < 0$, cioè $\alpha > 1$, il limite è finito e vale $-\frac{1}{1-\alpha} = \frac{1}{\alpha-1}>0$.
\end{itemize}

\begin{example}
$\int_1^{+\infty}\frac{1}{x^2}\:dx$ converge, e $\int_1^{+\infty}\frac{1}{\sqrt{x}}\:dx$ diverge a $+\infty$.
\end{example}

\hspace{-15pt}Con la forma: $\int_0^1\frac{1}{x^{\alpha}}$ con $\alpha \in \mathbb{R}$.
\begin{itemize}
    \item Se $\alpha = 1$, $\int_0^{1}\frac{1}{x}\:dx = \lim\limits_{M\to 0^+}[\log(x)]_M^1 = \lim\limits_{M\to 0^+}(\log(M)) = +\infty$ diverge.
    \item Se $\alpha \neq 1$, $\int \frac{1}{x} = \int x^{-\alpha}\:dx = \frac{1}{1-\alpha}x^{1-\alpha} + c$.\\
    Quindi $\int_0^1\frac{1}{x^{\alpha}} = \lim\limits_{M\to 0^+} [\frac{1}{1-\alpha}x^{1-\alpha}]_M^1 = \lim\limits_{M\to 0^+} (\frac{1}{1-\alpha} - \frac{1}{1-\alpha}M^{1-\alpha})$.
    \item Se $1-\alpha > 0$, cioè $\alpha < 1$, il limite è finito e vale $-\frac{1}{1-\alpha} >0$.
    \item se $1-\alpha < 0$, cioè $\alpha > 1$, il limite è finito e vale $+\infty$.
\end{itemize}

\begin{observation}
Quindi questo implica che $\int_0^{+\infty}\frac{1}{x^{\alpha}}\:dx = +\infty \:\: \forall \alpha \in \mathbb{R}$.
\end{observation}

\subsection{Criteri per la convergenza di integrali impropri}
\subsubsection{Criterio del confronto}
Prendiamo un $a\in \mathbb{R}$, $b\in \mathbb{\overline{R}}$ (deve essere $+\infty$), e $f,g: [a,+\infty)\to \mathbb{R}$ integrabile in ogni $[a,M] \: \forall \: a<M<b$. Se $\exists \: U$ intorno sinistro di $b$ tale che $0 \leq f(x) \leq g(x) \:\: \forall \: x\in U \cap [a,b)$.
\begin{enumerate}
    \item Se $\int_a^b g(x)\:dx$ converge, allora anche $\int_a^b f(x)\:dx$ converge.
    \item Se $\int_a^b f(x)\:dx$ diverge $(a,+\infty)$, allora anche $\int_a^b g(x)\:dx$ diverge $(a,+\infty)$.
\end{enumerate}
C'è un enunciato analogo se $f,g: (a,b]$.

\begin{example}
$\int_1^{+\infty}\frac{dx}{x^4+3x^3+x+1}$, chiamiamo $f(x) = \frac{dx}{x^4+3x^3+x+1}$ è continua in $[1,+\infty)$ perché $x^4+3x^3+x+1 > 0 \:\: \forall x \geq 1$.\\
Inoltre $0\leq f(x) \leq \frac{1}{x^4} \forall x\in [1,+\infty)$. Visto che $\int_1^{+\infty}\frac{1}{x^4}\:dx$ converge, per confronto concludiamo che $\int_1^{+\infty}f(x)\:dx$ converge.
\end{example}

\subsubsection{Criterio del confronto asintotico o C.A.}
Prendiamo $a\in \mathbb{R}$, $b\in \mathbb{\overline{R}}$, e $f,g: [a,b)\to \mathbb{R}$ integrabile in ogni $[a,M] \: \forall a<M<b$. Se $\exists U$ intorno sinistro di $b$ tale che $f(x) \geq 0$, $g(x) \geq 0 \forall x \in U \cap [a,b)$ e $\lim\limits_{x\to b^-}\frac{f(x)}{g(x)}=l$. Allora:
\begin{itemize}
    \item Se $l\neq 0,+\infty$, $\int_a^b f(x)\:dx$ converge $\Longleftrightarrow \int_a^b g(x)\:dx$ converge.
    \item Se $l = 0$ e $\int_a^b g(x)\:dx$ converge $\Longrightarrow \int_a^b f(x)\:dx$ converge.
    \item Se $l = +\infty$ e $\int_a^b f(x)\:dx$ converge $\Longrightarrow \int_a^b g(x)\:dx$ converge.
\end{itemize}
C'è un enunciato analogo se $f,g: (a,b]$.\\
Esempio: nel secondo caso $\lim\limits_{x\to b^-}\frac{f(x)}{g(x)}=0 \Longrightarrow$ per $x$ vicine a b vale $\frac{f(x)}{g(x)}\leq 1 \Longrightarrow f(x) \leq g(x)$ vicino $b$.

\begin{observation}
Le implicazioni di questi criteri non si invertono.
\end{observation}

\begin{example}
$\frac{1}{x^2} \leq \frac{1}{x}$ ($f(x) \leq g(x)$) per $x\geq 0$ e $\int_1^{+\infty}\frac{1}{x}\:dx$ diverge non si può concludere che $\int_1^{+\infty}\frac{1}{x^2}\:dx$ diverge. Il criterio del confronto non vale in maniera inversa.
\end{example}

\begin{example}
$\int_0^1 \frac{1}{x-\sin(x)}\:dx$, prendiamo $f(x) = \frac{1}{x-\sin{x}}$ è continua in $(0,1]$ e $f(x) > 0$ in $(0,1]$.\\
Il metodo è usare Taylor per confrontare la $f(x)$ con una certa forma $\frac{1}{x^{\alpha}}$. Sviluppiamo il denominatore in 0 (il punto "problematico")\\
$x-\sin{x} = x - (x-\frac{x^3}{6} + o(x^3)) = \frac{x^3}{6} + o(x^3)$. $f(x) = \frac{1}{x-\sin{x}} = \frac{1}{\frac{x^3}{6}}$ attorno a 0.\\
Uso il criterio del confronto asintotico con $g(x) = \frac{1}{x^3}$. $\lim\limits_{x\to 0^+}\frac{f(x)}{\frac{1}{x^3}} = \lim\limits_{x\to 0^+}\frac{x^3}{x-\sin(x)} = \lim\limits_{x\to 0^+} \frac{x^3}{x-\sin{x}} = \lim\limits_{x\to 0^+}\frac{x^3}{\frac{x^3}{6} + o(x^3)} = \frac{1}{\frac{1}{6}} = 6$. Per il C.A. concludo che $\int_0^1 f(x)\:dx$ si comporta come $\int_0^1 \frac{1}{x^3}$ che sappiamo diverge. Quindi $\int_0^1 \frac{1}{x-\sin(x)}\:dx$ diverge.
\end{example}

\begin{observation}
I criteri del confronto e del confronto asintotico si possono usare anche per funzioni negative, cambiando opportunamente le conclusioni.\\
Ad esempio: se $g(x) \leq f(x) \leq 0$ per $x\in [a,b)$ allora:
\begin{itemize}
    \item Se $\int_a^b g(x)\:dx$ converge allora anche $\int_a^b f(x)\:dx$ converge.
    \item Se $\int_a^b f(x)\:dx$ diverge (a $-\infty$ per forza) allora anche $\int_a^b g(x)\:dx$ diverge (a $-\infty$).
\end{itemize}
\end{observation}

\subsubsection{Criterio dell'assoluta convergenza}
Questo criterio si applica a funziono a segno variabile.
\begin{definition}
$f$, integrabile su ogni intervallo chiuso $[a,b]\subseteq I$, si dice assolutamente integrabile su I se $|f|$ è integrabile (eventualmente in senso generalizzato) su I, cioè $\int_I |f(x)|\:dx$ converge.
\end{definition}

\begin{definition}[Parte positiva e negativa]
Prendiamo un $x\in \mathbb{R}$. Definiamo:
\begin{itemize}
    \item La \textbf{parte positiva} di $x$ è $x^+ = max{x,0}$ cioè è x se $x \geq 0$ ed è 0 se $x < 0$. 
    \item Mentre la \textbf{parte negativa} di $x$ è $x^- = -min{x,0}$ che è $-x$ quando $x\leq 0$ e 0 se $x>0$.
\end{itemize}
\end{definition}

\begin{example}
$4^+ = 4$, $4^- = 0$, $(-3)^+ = 0$, $(-3)^- = 3$
\end{example}

\begin{observation}
Ogni $x = x^+ - x^-$ mentre $|x| = x^+ + x^-$. Segue che $x^+ = \frac{|x| + x}{2}$ e $x^- = \frac{|x|-x}{2}$.\\
Analogamente, se $f(x)$ è una funzione ho $f(x) = (f(x))^+ - (f(x))^-$, e $|f(x)| = (f(x))^+ + (f(x))^-$.
\end{observation}

\begin{proposition}[Criterio dell'assoluta integrabilità]
Se $f$ è assolutamente integrabile su $I$ allora $f$ è integrabile (in senso generalizzato) su I. 
\end{proposition}

\hspace{-15pt}Per questa proposizione non vale il viceversa.

\begin{demostration}
$|f(x)| = (f(x))^+ + (f(x))^-$ quindi:\\
$0 \leq (f(x))^+ \leq |f(x)|$ e $0 \leq (f(x))^- \leq |f(x)|$\\\\
Per confronto, supponendo che $\int_I |f|\:x$ converga, concludo che convergono $\int_I f(x)^+\:x$ e $\int_I f(x)^-\:x$.\\
Visto che $f(x) = (f(x))^+ - (f(x))^-$, concludo che: \\
$\int_I f(x)\:dx = \int_I (f(x)^+ - f(x)^-)\:dx = \int_I f(x)^+\:dx - \int_I f(x)^- \:dx$.\\\\
Ad esempio se $I = [a,b)$, abbiamo che:\\
$\int_a^M f(x)\:dx = \int_a^M (f(x)^+ - f(x)^-)\:dx = \int_a^M f(x)^+\:dx - \int_a^M f(x)^- \:dx$, passando al limite per $M\to b^-$ so che i limiti di $\int_a^M f(x)^+\:dx - \int_a^M f(x)^- \:dx$ esistono, quindi esiste anche $\lim\limits_{M\to b^-}\int_a^M f(x)\:dx$. $\blacksquare$
\end{demostration}

\begin{corollaries}
$f,g: [a,b)\to \mathbb{R}$ con $a \in \mathbb{R}$ e $b \in \mathbb{\overline{R}}$ integrabili in $[a,M] \forall a < M < b$. Se $\exists U$ intorno sinistro di $b$ tale che $|f(x)| \leq g(x) \forall x \in U \cap [a,b)$ e se $\int_a^b g(x) \:dx$ converge $\Longrightarrow \int_a^b f(x)\:dx$ converge. (Confronto + assoluta integrabilità)
\end{corollaries}

\begin{example}
$\int_1^{+\infty}\frac{\sin{x}}{x^2}$. $f(x) = \frac{\sin{x}}{x^2}$ a segno variabile su $[1,+\infty)$.\\
$|f(x)| = \frac{|\sin{x}|}{x^2} \leq \frac{1}{x^2}$, prendo $g(x) = \frac{1}{x^2}$ nel corollario di sopra.\\
Visto che $\int_1^{+\infty}\frac{1}{x^2}\:dx$ converge, concludo che $\int_1^{+\infty}\frac{\sin{x}}{x^2}$ converge.
\end{example}

\begin{example}
$\int_1^{+\infty}\frac{\sin{x}}{x}$. Procedendo alla stesso modo di sopra $f(x) = \frac{\sin{x}}{x}$ a segno variabile.\\
$|f(x)| = \frac{|\sin{x}|}{x^2} \leq \frac{1}{x}$ prendo $g(x) = \frac{1}{x^2}$. Questa volta però $\int_1^{+\infty}\frac{1}{x}\:dx$ diverge.\\
Quindi non posso concludere niente su $\int_1^{+\infty}\frac{\sin{x}}{x}$. In questo caso possiamo:
$\int_1^{+\infty}\frac{\sin{x}}{x} = \lim\limits_{M\to +\infty}\int_1^{M}\frac{\sin{x}}{x} = $ (integro per parti) $= \int_1^{M}\sin{x}\frac{1}{x}\:dx = [-\frac{\cos{x}}{x}]_1^M - \int_1^M \frac{\cos{x}}{x^2}\:dx =  \lim\limits_{M\to +\infty}(-\frac{\cos{M}}{M} + \frac{\cos{1}}{1} - \int_1^M \frac{\cos{x}}{x^2})\:dx =  \lim\limits_{M\to +\infty}(-\frac{\cos{M}}{M} + \cos{1}) - \lim\limits_{M\to +\infty}\int_1^M \frac{\cos{x}}{x^2}\:dx$.\\
Il risultato finale è uguale a $\int_1^M \frac{\cos{x}}{x^2}\:dx = \int_1^{+\infty}\frac{\cos{x}}{x^2}$ che converge come il caso con seno (visto nell'esempio prima). Mentre la parte $-\frac{\cos{M}}{M}$ tende a 0, quindi $\int_1^{+\infty}\frac{\sin{x}}{x}$ converge.
\end{example}

\begin{observation}
Stesso discorso per $\int_1^{+\infty}\frac{\cos{x}}{x}$ che converge.
\end{observation}

\begin{example}
Vediamo come $\int_1^{+\infty}\frac{|\sin{x}|}{x}$ diverge(questo da un esempio di $f(x)$ tale che $\int_1^{+\infty}f(x)\:dx$ converge, ma $\int_1^{+\infty}|f(x)|\:dx$ diverge).\\\\
Osserviamo che $|\sin{x}| \geq (\sin{x}^2)$ (perché $-1 \leq \sin{x} \leq 1$). Quindi $\int_1^{M}\frac{|\sin{x}|}{x} \geq \int_1^{M}\frac{\sin{x}^2}{x}\:dx = \int_1^{M}\frac{(1-\cos{2x})}{2x} = \int_1^{M}\frac{1}{2x} - \int_1^{M}\frac{\cos{sx}}{2x} = \int_1^{M}\frac{1}{2x} - \frac{1}{2}\int_2^{2M}\frac{\cos{t}}{t}\:dt$ con $t=2x$ e $dt=2dx$. Il primo integrale diverge ed il secondo converge perché si ritorna ad un caso visto prima ($\int_2^{+\infty}\frac{\cos(t)}{t}\:dt$). \\
Quindi in conclusione la somma diverge a $+\infty$ quindi $\int_1^{+\infty}\frac{|\sin{x}|}{x}$ diverge a $+\infty$.
\end{example}

\subsubsection{Integrali impropri ricorrenti}
\underline{\textbf{TIPO 1°}}. Vediamo ora gli integrali del tipo $\int_2^{+\infty}\frac{1}{x^{\alpha}\log(x)^{\beta}}\:dx$ con $\alpha, \beta \in \mathbb{R}$.
\begin{itemize}
    \item Caso con $\alpha > 1$: possiamo prendere un $\gamma \in \mathbb{R}$ tale che $\alpha > \gamma > 1$.\\
    $f(x) = \frac{1}{x^{\alpha}(\log(x)^{\beta})}$ e $g(x) = \frac{1}{x^{\gamma}}$. $f(x), g(x) \geq 0$ e $\lim\limits_{x\to +\infty}\frac{f(x)}{g(x)} = \lim\limits_{x\to +\infty} \frac{x^{\gamma}}{x^{\alpha}(\log(x)^{\beta})} = \lim\limits_{x\to +\infty} \frac{1}{x^{\alpha-\gamma}(\log(x))^{\beta}}$ questo limite è 0.\\
    Quindi visto che $\gamma > 1$, quindi $\int_2^{+\infty}\frac{1}{x^{\gamma}}\:dx$ converge e per C.A. concludiamo che $\int_2^{+\infty}\frac{1}{x^{\alpha}(\log(x)^{\beta})}\:dx$ converge.
    \item Caso con $\alpha < 1$: possiamo prendere un $\gamma \in \mathbb{R}$ tale che $\alpha < \gamma < 1$.\\
    $f(x) = \frac{1}{x^{\alpha}(\log(x)^{\beta})}$ e $g(x) = \frac{1}{x^{\gamma}}$. $f(x), g(x) \geq 0$.\\
    Questa volta $\lim\limits_{x\to +\infty}\frac{f(x)}{g(x)} = \lim\limits_{x\to +\infty}\frac{x^{\gamma-\alpha}}{(\log(x)^{\beta})} = +\infty$.\\
    Visto che $\gamma < 1$, $\int_2^{+\infty}\frac{1}{x^{\gamma}}\:dx$ diverge per C.A. Possiamo quindi concludere che $\int_2^{+\infty}\frac{1}{x^{\alpha}(\log(x)^{\beta})}\:dx$ diverge $\forall \beta \in \mathbb{R}$.
    \item Caso con $\alpha = 1$: $\int_2^{+\infty}\frac{1}{x^{\alpha}(\log(x)^{\beta})}\:dx$.\\
    $\int_2^{M}\frac{1}{x^{\alpha}(\log(x)^{\beta})}\:dx$ = con $t = \log(x)$ e $dt = \frac{1}{x}\:dx = \int_{\log(2)}^{\log(M)}\frac{1}{t^{\beta}}\:dt = \lim\limits_{M\to +\infty}\frac{1}{x^{\alpha}(\log(x)^{\beta})} = \int_{\log(2)}^{+\infty}\frac{1}{t^{\beta}}\:dt$ che converge se $\beta > 1$ e diverge a $+\infty$ se $\beta \leq 1$.
\end{itemize}
\underline{\textbf{TIPO 2°}}. Analogamente studiamo $\int_0^{\frac{1}{2}}\frac{1}{x^{\alpha}|\log(x)|^{\beta}}\:dx$ con $\alpha, \beta \in \mathbb{R}$.\\
$\int_M^{\frac{1}{2}}\frac{1}{x^{\alpha}|\log(x)|^{\beta}}\:dx =$ con $t=\frac{1}{x}$ quindi $x = \frac{1}{t}$, e $dx = -\frac{1}{t^2}\:dt \int_{\frac{1}{M}}^2 \frac{-dt}{t^2 \cdot t^{-\alpha} |l-\log(t)|^{\beta}} =$ (se $M\to 0^+$ allora $\frac{1}{M}\to +\infty$) $=\int_2^{\frac{1}{M}} \frac{dt}{t^{2-\alpha}\cdot|\log(t)|^{\beta}} = \lim\limits_{M\to 0^+}\frac{dt}{t^{2-\alpha}\cdot|\log(t)|^{\beta}} = \int_2^{\frac{1}{M}} \frac{dt}{t^{2-\alpha}\cdot|\log(t)|^{\beta}}$ e questo l'abbiamo appena studiato.
Segue che $\int_0^{\frac{1}{2}}\frac{1}{x^{\alpha}|\log(x)|^{\beta}}\:dx$ abbiamo che:
\begin{itemize}
    \item $2-\alpha > 1$ ($\alpha < 1$) converge $\forall \beta \in \mathbb{R}$.
    \item $2-\alpha < 1$ ($\alpha > 1$) diverge a $+\infty$ $\forall \beta \in \mathbb{R}$.
    \item $2-\alpha = 1$ ($\alpha = 1$) con $\beta > 1$ converge.
    \item $2-\alpha = 1$ ($\alpha = 1$), $\beta \leq 1$ diverge a $+\infty$.
\end{itemize}

\hspace{-15pt}\underline{\textbf{TIPO 3°}}. Vediamo come ultimo gli integrali della forma $\int_a^{+\infty} \frac{1}{x^{\alpha}}$.
\begin{itemize}
    \item Questo integrale converge se $a > 0$ e $\alpha > 1$.
    \item Invece l'integrale diverge a $+\infty$ se $a > 0$ e $\alpha \leq 1$-
\end{itemize}

\subsubsection{Esempi riassuntivi}
\begin{example}
Primo esempio riassuntivo: $\int_{x_0}^{x_0+1}\frac{dx}{x-x_0}$
\begin{itemize}
    \item Converge se $\alpha < 1$.
    \item Diverge a $+\infty$ se $\alpha \geq 1$.
\end{itemize}
Dato M tale che $x_0 < M < x_0 + 1$. $\int_M^{x_0+1}\frac{dx}{(x-x_0)^{\alpha}}$ = con $t = x - x_0$ e $dt = dx = \int_{M-x_0}^1 \frac{dt}{t^{\alpha}}$.\\
$\lim\limits_{M\to x_0^+}\int_M^{x_0+1}\frac{dx}{(x-x_0)^{\alpha}} = \lim\limits_{M\to x_0^+}\int_{M-x_0}^1 \frac{dt}{t^{\alpha}} = \int_0^1 \frac{dt}{t^{\alpha}}$ e sappiamo che questo converge se $\alpha < 1$ e diverge a $+\infty$ se $\alpha \geq 1$.
\end{example}

\begin{example}
Secondo esempio riassuntivo: $\int_0^2 \frac{x^3 + 1}{x^2 -4}\:dx$. $f(x) = \frac{x^3 + 1}{x^2 -4}$ è definita e continua in $[0,2)$ quindi integrale va trattato come integrale improprio. Bisogna notare anche che $f(x) < 0$ (sempre positiva) in tutto $[0,2)$ perché $x^3+1 > 0$ per $x>0$ e $x^2 -4 < 0$ per $0 \leq x < 2$.\\
Avendo segno costante si possono usare i criteri del confronto e del confronto asintotico.\\
$f(x) = \frac{x^3+1}{x^2-4} = \frac{x^3 +1}{(x-2)(x+2)}$ il pezzo problematico è $g(x) = \frac{1}{x-2}$.\\
Usiamo C.A. con $g(x) = \frac{1}{x-2}$ (notare $g(x) < 0$ in $[0,2)$). Poi facciamo $\lim\limits_{x\to 2}\frac{f(x)}{g(x)}$:\\
$\lim\limits_{x\to 2} \frac{x^3 +1}{(x-2)(x+2)}\cdot(x-2) = \frac{9}{4} \neq 0, +\infty$. Per C.A. $\int_0^2 f(x)\:dx$ ha lo stesso comportamento di $\int_0^2 \frac{1}{x-2}\:dx$ che sappiamo diverge negativamente (sostituzione $t=2-x$ per ricondursi a $\int \frac{dt}{t}$).\\
Quindi $\int_0^2 \frac{x^3 + 1}{x^2 -4}\:dx = -\infty$ (si scrive sono $-\infty$ che vuol dire che diverge negativamente)
\end{example}

\begin{example}
Terzo esempio riassuntivo: $\int_0^{+\infty}\frac{\log(1+x^2)}{\sqrt{1+x^2}}\:dx$. $f(x) = \frac{\log(1+x^2)}{\sqrt{1+x^2}}$ è definita e continua in $\mathbb{R}$ ($1 + x^2 > 0 \forall x \in \mathbb{R}$). Infine $f(x) \geq 0 \forall x \in \mathbb{R}$. Quindi l'unico problema è a $+\infty$.\\
Per $x$ grandi $\frac{\log(1+x^2)}{\sqrt{1+x^2}}$ sarà circa $\frac{\log(x^2)}{\sqrt{x^2}} = \frac{2\log(x)}{x} \geq \frac{1}{x}$ e sappiamo che $\int_1^{+\infty}\frac{1}{x}\:dx$ diverge, quindi probabilmente anche il nostro divergerà.\\
Facciamo C.A. con $g(x) = \frac{1}{x}$. $\lim\limits_{x\to +\infty}\frac{f(x)}{g(x)} = \lim\limits_{x\to +\infty} \frac{\log(1+x^2)}{\sqrt{1+x^2}}\cdot x = \lim\limits_{x\to +\infty} \frac{\log(1+x^2)}{\sqrt{1+\frac{1}{x^2}}} = \frac{+\infty}{1} = +\infty$.\\
Quindi visto che $\int_1^{+\infty}\frac{1}{x}\:dx$ diverge, concludo che $\int_1^{+\infty}f(x)\:dx$ diverge positivamente. Segue che $\int_0^{+\infty}f(x)\:dx = \int_0^{1}f(x)\:dx + \int_1^{+\infty}f(x)\:dx = +\infty$ cioè il nostro integrale diverge positivamente.
\end{example}

\begin{example}
Quarto esempio riassuntivo: $\int_0^{+\infty}\frac{x^2}{(2+3x^4)\cdot\arctan(x^{\frac{5}{2}})}\:dx$. $f(x) = \frac{x^2}{(2+3x^4)\cdot\arctan(x^{\frac{5}{2}})}$ definita e continua su $(0,+\infty)$ e $f(x) > 0$ su $(0,+\infty)$, ci sono 2 problemi, in 0 e a $+\infty$ quindi spezziamo in $\int_0^{+\infty}f(x)\:dx = \int_0^1 f(x)\:dx + \int_1^{+\infty}f(x)\:dx$ e studiamo i due pezzi.
\begin{itemize}
    \item Caso $\int_0^1$: per $x\to 0^+$ si ha $\arctan(x^{5/2}) = x^{5/2} + o(x^{5/2})$ quindi $f(x) = \frac{x^2}{(2+3x^4)(x^{\frac{5}{2}} + o(x^{5/2}))} = \frac{x^2}{2x^{\frac{5}{2}} + o(x^{\frac{5}{2}})} = \frac{1}{2x^{\frac{1}{2}} + o(x^{\frac{1}{2}})}$ prendo $g(x) = \frac{1}{x^{\frac{1}{2}}} = \frac{1}{\sqrt{x}}$.\\
    Ho $\lim\limits_{x\to 0^+}\frac{f(x)}{g(x)} = \lim\limits_{x\to 0^+}\frac{1}{2x^{\frac{1}{2}} + o(x^{\frac{1}{2}})} = \frac{1}{2} \neq 0, +\infty$. Visto che $\int_0^1 \frac{1}{\sqrt{x}}\:dx$ converge per C.A. concludo che converge anche $\int_0^1 f(x)\:dx$.
    \item Caso $\int_1^{+\infty}$: per $x\to +\infty$, $f(x) = \frac{x^2}{(2+3x^4)\cdot\arctan(x^{\frac{5}{2}})} = \frac{x^2}{x^4(\frac{2}{x^4}+3)\cdot\arctan(x^{\frac{5}{2}})} = \frac{1}{x^2} \cdot \frac{1}{(\frac{2}{x^4} + 3)\cdot \arctan(x^{\frac{5}{2}})}$ la seconda parte per $x\to +\infty$ fa $\frac{1}{3\cdot \frac{\pi}{2}} = \frac{2}{3\pi}$, prendo quindi $g(x) = \frac{1}{x^2}$.\\
    $\lim\limits_{x\to +\infty}\frac{f(x)}{g(x)} = \lim\limits_{x\to +\infty}\frac{x^2}{x^2\cdot(\frac{2}{x^4} + 3)\cdot \arctan(x^{\frac{5}{2}})} = \frac{2}{3\pi} \neq 0, +\infty$. Vito che $\int_1^{+\infty}\frac{1}{x^2}\:dx$ converge, per C.A. converge anche $\int_1^{+\infty}f(x)\:dx$.
\end{itemize}
In conclusione, anche $\int_0^{+\infty}f(x)\:dx = \int_0^1 f(x)\:dx + \int_1^{+\infty}f(x)\:dx$ converge.
\end{example}

\begin{example}
Quinto esempio (con segno variabile): $\int_0^{+\infty}\frac{\sin{x}}{x^{3/2}(x^2+1)}\:dx$. $f(x) = \frac{\sin{x}}{x^{3/2}(x^2+1)}$ definita e continua in $(0,+\infty)$, problemi in $x=0$ e $x=+\infty$, $f(x)$ a segno variabile.\\
Spezziamo $\int_0^{+\infty}f(x)\:dx = \int_0^1f(x)\:dx + \int_1^{+\infty}f(x)\:dx$
\begin{itemize}
    \item Caso $\int_0^1$: osserviamo che $f(x) \geq 0$ per $0 \leq x \leq 1$ perché ($\sin(x) \geq 0$ per $0 \leq x \leq 1 < \frac{\pi}{2}$). Quindi posso usare confronto e C.A. per $x\to 0^+$ $f(x) =\frac{\sin{x}}{x^{3/2}(x^2+1)} = \frac{x+o(x)}{x^{3/2} + o(x^{3/2})} = \frac{1 + o(1)}{x^{1/2}+o(x^{1/2})}$, prendo quindi $g(x) = \frac{1}{x^{1/2}}$.\\
    $\lim\limits_{x\to 0^+}\frac{f(x)}{g(x)} = \lim\limits_{x\to 0^+} \frac{1 + o(1)}{x^{1/2}+o(x^{1/2})} \cdot x^{1/2} = 1 \neq 0,+\infty$. Visto che $\int_0^1 \frac{1}{\sqrt{x}}\:dx$ converge, per C.A. concludiamo che $\int_0^1 f(x)\:dx$ converge.
    \item Caso $\int_1^{+\infty}$ qui $f(x)$ non è costante ma oscilla tra valori positivi e negativi. Proviamo ad usare assoluta convergenza:\\
    $|f(x)| = \frac{|\sin(x)|}{x^{3/2}(x^2+1)} \leq \frac{1}{x^{3/2}(x^2 + 1)} \leq \frac{1}{x^{3/2}}\cdot \frac{1}{x^2} = \frac{1}{x^{7/2}}$. Visto che $\int_1^{+\infty}\frac{1}{x^{7/2}}\:dx$ converge, per confronto ho che $\int_1^{+\infty}|f(x)|\:dx$ converge e per il criterio dell'assoluta integrabilità segue che $\int_1^{+\infty}f(x)$ converge.
\end{itemize}
In conclusione, anche $\int_0^{+\infty}f(x)\:dx = \int_0^1 f(x)\:dx + \int_1^{+\infty}f(x)\:dx$ converge.
\end{example}
\newpage
\section{Successioni}
\begin{definition}[Successione]
Una successione\footnote{Nelle successioni si è soliti scrivere n al posto di x come simbolo per la variabile ess. $f(n)$} è una funzione $f: S\to \mathbb{R}$ dove S è una semiretta di $\mathbb{N}$, cioè $S = \{n \in \mathbb{R}\:|\: x\geq n_0\}$ per qualche $n_0$.
\end{definition}

\begin{example}
Consideriamo $f(n) = n^2$ con $S = \mathbb{N}$.
\end{example}
\begin{wrapfigure}[6]{l}{6cm}
    \vspace{-20pt}
    \centering
    \includegraphics[width=5cm]{images/esempio-successione-1.png}
\end{wrapfigure}
Da questa funzione posso calcolare tutti i valori: $f(0) = 0^2 = 0$, $f(1) = 1^2 = 1$, $f(2) = 2^2 = 4$\\\\
E possibile disegnare un grafico di una successione che è composto da una serie di punti sparsi.\\

\begin{example}
$f(n) = \frac{1}{n}$, come S non posso prendere tutti i naturali perché con 0 non ha senso quindi $S = \{n \in \mathbb{N} \: |\: n \geq 1\}$.  $f(1) = \frac{1}{1}=1$, $f(2) = \frac{1}{2}$, $f(3) = \frac{1}{3}$.
\end{example}

\subsection{Notazione}
Nelle successioni invece di scrivere $f(n)$ di solito una successione si denota con $a_n$. Negli esempio di prima si sarebbe: $a_n = n^2$, $a_n = \frac{1}{n}$.\\
L'intera successione si denota con $\{a_n\}$ oppure $\{a_n\}_{n\in \mathbb{N}}$, $\{a_n\}_{n\in S}$.
\begin{example}
$a_n = \frac{1}{n-5}$. La formula ha senso per $n\neq 5$, quindi si può prendere $S = \{n \in \mathbb{N} \:|\: n\geq 6\}$ (avrei anche potuto prendere $n \geq 7$ o $n \geq 8$).
\end{example}

\begin{example}
$a_n = \sqrt{5 - n}$. La formula ha senso se $5-n \geq 0$ cioè $n \leq 5$. Nessuna semiretta va bene perché in una successione n diventa sicuramente più grande ad un certo punto quindi non definisce una successione.
\end{example}

\subsection{Limiti di Successioni}
Come per le funzioni bisogna guardare come si comporta la successioni all'avvicinarsi ad un limite. L'unico limite che ha senso è il limite per $n\to +\infty$, perché $+\infty$ è l'unico punto di accumulazione di tutto il dominio (perché $S \subseteq \mathbb{N}$).
\begin{definition}[Limite di successione]
Si ha che $\lim\limits_{n\to +\infty}a_n = l$ se $\forall \:\: U$ intorno di l si ha che $\exists \: \overline{n}\in \mathbb{N}$ tale che $a_n \in U \:\: \forall \: n\geq \overline{n}$.\\
Si dice che $a_n$ converge a $l$ se $\lim\limits_{n\to +\infty}a_n = l$ e $l\in \mathbb{R}$ e che diverge a $\pm \infty$ se $\lim\limits_{n\to +\infty}a_n = \pm \infty$.
\end{definition}

\begin{figure}[h!]
\centering
\begin{subfigure}{.45\textwidth}
    \vspace{-15pt}
    \centering
    \includegraphics[width=5cm]{images/limite-successione-1.png}
    \caption{Graficamente se il limite è in $\mathbb{R}$ quindi $l \in \mathbb{R}$}
\end{subfigure}
\begin{subfigure}{.45\textwidth}
    \centering
    \includegraphics[width=4cm]{images/limite-successione-2.png}
    \caption{E con $l = +\infty$}
\end{subfigure}
\end{figure}

Esiste una \textbf{Terminologia} quando si parla di queste cose: se $P(n)$ è un predicato la cui verità dipende da $n\in \mathbb{N}$ (esempio: $P(n) =$ "n è pari") si dice che $P(n)$ è vero definitivamente se $\exists \: \overline{n}\in \mathbb{N}$ tale che $P(n)$ è vero $\forall n \geq \overline{n}$.\\
Quindi $\lim\limits_{n\to +\infty} a_n = l$ se $\forall \: U$ introno di l si ha che $a_n \in U$ definitivamente.

\subsection{Sottosuccessioni (estratte)}
\begin{definition}[Sottosuccessione]
Dato $a_n: S \to \mathbb{R}$ una successione, consideriamo $k_n: \mathbb{N} \to S$ strettamente crescente (cioè $k_n > k_m$ quando $n>m$), possiamo considerare la composizione $a_{k_n}$. Questa è una nuova successione detta sottosuccessione di $\{a_n\}$ (In pratica scegliamo solo un certo sottoinsieme di indici, in modo crescente).
\end{definition}

\begin{example}
Prendiamo la successione $a_n = \frac{1}{n}$.
\end{example}
\begin{wrapfigure}[4]{r}{6cm}
    \vspace{-40pt}
    \centering
    \includegraphics[width=5cm]{images/esempio-sottosuccessioni.png}
\end{wrapfigure}
Per avere una sottosuccessione prendo $k_n: \mathbb{N}\to S$,e prendo $n \mapsto 2n+1$. Abbiamo $a_{k_n} = \frac{1}{k_n} = \frac{1}{2n+1}$.\\
Quindi graficamente: \\
$a_{k_0} = \frac{1}{0+1} = 1$, $a_{k_1} = \frac{1}{2\cdot 1 +1} = \frac{1}{3} = a_3$, $a_{k_2} = \frac{1}{2 \cdot 2+1} = \frac{1}{5} = a_5$.\\\\
\begin{theorem}
Data una successione $\lim\limits_{n\to +\infty}a_n = l$ se e solo se vale $\lim\limits_{n\to +\infty}a_{k_n} = l$ per ogni sottosuccessione di $\{a_n\}$.
\end{theorem}
\hspace{-15pt} A volta si può usare per dimostrare che una successione non ha limite.
\begin{example}
$a_n= (-1)^h = \begin{cases}-1 & \text{se n è pari} \\ -1 & \text{se n è dispari} \end{cases}$ \\\\
Questo successione non ha limite e si dimostra con il teorema visto sopra. Infatti, consideriamo le sottosuccessioni $\{a_{2n}\}$ e $\{a_{2n+1}\}$ date da indici pari e dispari. \\
Abbiamo che $a_{2n} = (-1)^{2n} = (1)^n = 1$ che converge a 1 mentre, $a_{2n+1} = (-1)^{2n+1} = -1$ e quindi converge a -1. Visto che questi limiti esistono e sono diversi, segue dal teorema che $\{a_n\}$ non può avere limite.
\end{example}

\begin{observation}
Per i limiti di successioni valogono molti dei teoremi visti per le funzioni, ad esepio:
\begin{itemize}
    \item Formule per limiti di somme, prodotti, quozienti, esponenziali etc.
    \item Teorema di permanenza del segno.
    \item Teorema dei carabinieri.
    \item Teorema del confronto, ed altri...
\end{itemize}
\end{observation}

\begin{example}
Per esempio il teorema della permanenza del segno per le successioni dice: se abbiamo una successione che $\lim\limits_{n\to +\infty} a_n = l > 0$, allora $a_n > 0$ definitivamente.
\end{example}

\subsection{Monotonia}
\begin{definition}[Monotonia]
Una successione $\{a_n\}$ essa si dice:
\begin{itemize}
    \item \textbf{Debolmente crescente} se $n>m \Longrightarrow a_n \geq a_n$.
    \item \textbf{Strettamente crescente} se $n > m \Longrightarrow a > a_m$.
    \item \textbf{Debolmente decrescente} se $n > m \Longrightarrow a_n \leq a_m$.
    \item \textbf{Strettamente decrescente} se $n > m \Longrightarrow a_n < a_m$.
\end{itemize}
Successione è monotona quando vale una di queste 4 proprietà.
\end{definition}

\begin{observation}
$\{a_n\}$ è debolmente crescente se e solo se vale $a_{n+1} \geq a_n \forall \: n \in S$ (basta guardare termini successivi).\\
Infatti, se so che $a_{n+1} \geq a_n \forall \: n \in \mathbb{N}$, poi se $n > m$ allora $a_n \geq ... \geq a_{m+2} \geq a_{m+1} \geq a_{m}$.
\end{observation}

\begin{example}
Prendiamo $a_n=n^2$ e controlliamo che è strettamente crescente: vediamo che $a_{n+1} > a_n$. Infatti $a_{n+1} = (n+1)^2 = n^2 + 2n + 1$ e $a_n = n^2$ e quindi $n^2 + 2n + 1 > n^2 \Longleftrightarrow 2n+1 > 0$ che è vero $\forall \:n \in \mathbb{N}$.
\end{example}

\begin{theorem}
Se $\{a_n\}$ è monotona (cioè debolmente crescente o decrescente) allora ammette limite.
Se è debolmente crescente, il limite non può essere $-\infty$ e se Se è debolmente decrescente, il limite non può essere $+\infty$
\end{theorem}

\subsection{Limitatezza}
\begin{definition}[Limitatezza]
Una successione $\{a_n\}$ è \textbf{limitata superiormente} se $\exists\: M \in \mathbb{R}$ tale che $a_n \subseteq M \:\forall\: \in S$ e \textbf{limitata inferiormente} se $\exists \:m \in \mathbb{R}$ tale che $a_n \geq m \forall \: n \in S$ e \textbf{limitata} se è limitata sia inferiormente e superiormente. (immagine \ref{limitatezza-successioni})
\end{definition}

\begin{observation}
Una successione convergente (che ha limite finito) è limitata. Questo non è vero per funzioni di variabile reale.
\end{observation}

\begin{example}
$f(x) = \frac{1}{x}$, $f: (0,+\infty)\to \mathbb{R}$ abbiamo $\lim\limits_{x\to +\infty}f(x) = 0$ ma f non è limitata, perché $\lim\limits_{x\to 0^+}f(x) = +\infty$ però $a_n = \frac{1}{n}$ invece è limitata.
\end{example}

\begin{theorem}
Se $\lim\limits_{n\to +\infty}a_n = +\infty$, allora $\{a_n\}$ ha minimo (cioè $\exists \:n_{min} \in \mathbb{N}$ tale che $a_n \geq a_{n_{min}} \: \forall \:n \in S$). Se invece $\lim\limits_{n\to +\infty}a_n = -\infty$ allora $a_n$ ha massimo.  (immagine \ref{teorema-minimo})
\end{theorem}
\begin{figure}[h!]
\centering
\begin{subfigure}{.45\textwidth}
    \vspace{-25pt}
    \centering
    \includegraphics[width=5cm]{images/limitatezza-successioni.png}
    \caption{Graficamente definizione di limiti inf, sup}
    \label{limitatezza-successioni}
\end{subfigure}
\begin{subfigure}{.45\textwidth}
    \vspace{-15pt}
    \centering
    \includegraphics[width=4cm]{images/teorema-minimo.png}
    \caption{Graficamente teorema minimo massimo}
    \label{teorema-minimo}
\end{subfigure}
\caption{Raffigurazione di definizione limitatezza e teorema minimo massimo}
\end{figure}
\hspace{-15pt}Ci si può chiedere come domanda se una successione $\{a_n\}$ è limitata, necessariamente massimo e minimo? La risposte è no.
\begin{example}
Se prendiamo $a_n = \frac{1}{n}$ è limitata: $1 \geq \frac{1}{n} > 0$ ma non ha minimo. $max\{a_n\} = 1$ e $inf\{a_n\} = 0$ (uguale a $\lim\limits_{n\to +\infty} a_n$). Non ha minimo perché non esiste $n \in \mathbb{N}$ tale che $\frac{1}{n} = 0$
\end{example}
\hspace{-15pt}Inoltre è possibile chiedersi se $\{a_n\}$ è limitata, esiste almeno uno tra massimo minimo? E la risposta anche in questo caso è no.
\begin{example}
Prendiamo $a_n = (1-\frac{1}{n})(-1)^n = \begin{cases}1-\frac{1}{n} & \text{per n pari} \\ -(1 - \frac{1}{n}) & \text{per n dispari}\end{cases}$
\end{example}
\begin{figure}[h!]
\centering
\begin{subfigure}{.3\textwidth}
    \vspace{-25pt}
    \centering
    \includegraphics[width=4.5cm]{images/esempio-mas-min-succesioni-1.png}
    \caption{n pari}
\end{subfigure}
\begin{subfigure}{.3\textwidth}
    \centering
    \includegraphics[width=4cm]{images/esempio-mas-min-succesioni-2.png}
    \caption{n dispari}
\end{subfigure}
\begin{subfigure}{.3\textwidth}
    \centering
    \includegraphics[width=4cm]{images/esempio-mas-min-succesioni-3.png}
    \caption{Complessivamente}
\end{subfigure}
\end{figure}
\hspace{-15pt}Complessivamente possiamo vedere la la successione oscilla avvicinandosi con $sun\{a_n\} = 1$ e $inf\{a_n\} = -1$, e non esistono massimo e minimo, anche se $a_n$ è limitata, visto che $-1 < a_n < 1$.

\newpage
\begin{example}
Prendiamo $a_n = \frac{(-1)^n}{n}$ e ci chiediamo se ha limite e sa ha massimo e o minimo.
\end{example}
\begin{wrapfigure}[4]{l}{6cm}
    \vspace{-15pt}
    \centering
    \includegraphics[width=5cm]{images/esempio-mas-min-successioni-4.png}
\end{wrapfigure}
Abbiamo che $\lim\limits_{n\to +\infty} = 0$. Infatti abbiamo che $-\frac{1}{n} \leq a_n \leq \frac{1}{n}$ e visto che $\lim\limits_{n\to +\infty}-\frac{1}{n}=\lim\limits_{n\to +\infty}\frac{1}{n} = 0$ per il teorema dei carabinieri abbiamo che $\lim\limits_{n\to +\infty}a_n = 0$. Quindi ha massimo e minimo il massimo è in $n=2$ ed il minimo in $n=1$.\\\\

\begin{theorem}
Se ho usa successione che converge $\lim\limits_{n\to +\infty}a_n = l$ finito allora:
\begin{itemize}
    \item $\exists \: \overline{n}\in \mathbb{N}$ tale che $a_{\overline{n}} \geq l \Longrightarrow \{a_n\}$ ha massimo.
    \item $\exists \: \overline{n}\in \mathbb{N}$ tale che $a_{\overline{n}} \leq l \longrightarrow \{a_n\}$ ha minimo.
\end{itemize}
\end{theorem}

\subsection{Legame tra limiti di funzione e successioni}
\begin{theorem}
Prendiamo una funzione definita in $A \subseteq \mathbb{R}$ sottoinsieme $f:A \to \mathbb{R}$, e $x_0 \in acc(A)$. Allora abbiamo che $\lim\limits_{x\to x_0}f(x) = l$ se e solo se $\lim\limits_{n\to +\infty}f(a_n) = l$ per ogni successione $\{a_n\}\subseteq A$ tale che $\lim\limits_{n\to +\infty}a_n = x_0$ e $a_n \neq x_0$ definitivamente.
\end{theorem}
\begin{wrapfigure}[2]{r}{6cm}
    \vspace{-35pt}
    \centering
    \includegraphics[width=4.7cm]{images/legame-lim-successioni-funzioni.png}
\end{wrapfigure}
Questo teorema a volte si può utilizzare per dimostrare che non esiste $\lim\limits_{x\to x_0}f(x)$.\\\\
\begin{example}
Dimostriamo che non esiste $\lim\limits_{x\to +\infty}\sin(x)$.\\
Esibiamo due successioni $a_n, b_n$ che tendono a $+\infty$,  tali che $\lim\limits_{n\to +\infty}\sin(a_n)$ e $\lim\limits_{n\to +\infty}\sin(b_n)$ esistono, ma sono diversi.\\
Prendo $a_n = n\pi$. Abbiamo $\lim\limits_{a\to +\infty}a_n = n\pi = +\infty$. Inoltre $\lim\limits_{n\to +\infty}\sin(a_n)= \lim\limits_{n\to +\infty}\sin(n\pi) = 0$ e $b_n = \frac{\pi}{2} + 2n\pi$. Di nuovo, $\lim\limits_{n\to +\infty}b_n = +\infty$ ma questa volta $\lim\limits_{n\to +\infty}\sin(b_n) = \sin(\frac{\pi}{2} + 2n\pi) = 1$.
\end{example}
\begin{wrapfigure}[4]{l}{6cm}
    \vspace{-15pt}
    \centering
    \includegraphics[width=5cm]{images/esempio-dim-con-legame-succ-fun.png}
\end{wrapfigure}

Per il teorema concludo che non esiste il $\lim\limits_{x \to +\infty}\sin(x)$
In particolare il teorema implica che se $\lim\limits_{x\to +\infty}f(x) = l$, allora $\lim\limits_{n\to +\infty}f(n) = l$. Attenzione che non è vero il viceversa.\\\\
\begin{example}
$f(x) = \sin(x\pi)$. Abbiamo $f(n) = \sin(n\pi) = 0$. Quindi $\lim\limits_{n\to +\infty}f(n) = 0$, ma non esiste $\lim\limits_{x\to +\infty}\sin(x\pi)$.
\end{example}

\subsection{Calcolo dei limiti di successioni}
\begin{theorem}
Se abbiamo due successioni $a_n \to l$ e $b_n \to l'$ allora $a_n + b_n \to l + l'$, $a_n \cdot b_n \to l \cdot l'$, $\frac{a_n}{b_n} \to \frac{l}{l'}$ (se $l' \neq 0$ e $b_n \neq 0$ definitivamente), $a_n^{n^n} \to l^{l'}$ (se $l>0$ e $a_n > 0$ definitivamente), se $a_n = c \forall\: n \in \mathbb{N}$ allora $\lim\limits_{n\to +\infty}a_n = c$.
\end{theorem}
\hspace{-15pt}Questo teorema vale solo se supponiamo che non vengono forme indeterminate che sono le stesse viste con le funzioni.
\begin{theorem}
Se $f: A \to \mathbb{R}$ e $x_0 \in acc(A)$ e $\lim\limits_{x\to x_0}f(x) = l$ e $a_n: S \to A$ tale che $a_n \to x_0$ e $a_n \neq x_0$ definitivamente allora $\lim\limits_{n\to +\infty}f(a_n) = l$. In particolare se $\lim\limits_{x\to +\infty}f(x) = l$, allora $\lim\limits_{n\to +\infty}f(n) = l$.
\end{theorem}

\begin{example}
Alcuni esempi di calcolo dei limiti con successioni:
\begin{itemize}
    \item $\lim\limits_{n\to +\infty}(n^2 + 2n)$. Partendo da $\lim\limits_{n\to +\infty}n = +\infty$ troviamo $(\lim\limits_{n\to +\infty}n^2) + \lim\limits_{n\to +\infty}(2n) = (\lim\limits_{n\to +\infty}n)(\lim\limits_{n\to +\infty}n) + 2 \lim\limits_{n\to +\infty}n = (+\infty) \cdot (+\infty) + 2(+\infty) = +\infty$.
    \item $\lim\limits_{n\to +\infty}(n^2 - 2n) = +\infty - \infty$ possiamo fare $n^2 - 2n = n(n-2) \to +\infty \cdot (+\infty) = +\infty$. Si poteva anche dire $f(x) = x^2 - 2x$ visto che $\lim\limits_{x\to +\infty} (x^2 - 2x)=+\infty$ allora $\lim\limits_{n\to +\infty}f(n) = +\infty$.
    \item $\lim\limits_{n\to +\infty}\frac{n^2 - 2n}{n} = +\frac{+\infty}{+\infty}$ possiamo però fare $\frac{n^2 - 2n}{n} = n-2 \to +\infty$.
    \item $\lim\limits_{n\to +\infty}e^n$, consideriamo $f(x) = e^x$, so che $\lim\limits_{x\to +\infty}e^x = +\infty$ quindi $\lim\limits_{n\to +\infty}f(n) = +\infty$.
    \item $\lim\limits_{n\to +\infty}n\cdot \sin\frac{1}{n} = +\infty \cdot 0$, pongo $f(x) = x \cdot \sin\frac{1}{x}$ e calcoliamo $\lim\limits_{x\to +\infty}x\cdot \sin\frac{1}{x} = \frac{\sin\frac{1}{x}}{\frac{1}{x}}$ e $\frac{1}{x} \to 0$ quando $x\to +\infty$, poniamo $t = \frac{1}{x}$ e viene $\lim\limits_{x\to +\infty}\frac{\sin{t}}{t} = 1$.\\\\
    Altro modo utilizzando taylor: poniamo $\sin{t} = t + o(t)$ per $t\to 0$ sostituisco $t=\frac{1}{n}$ (infatti $\frac{1}{n}\to 0$ quando $n\to +\infty$), po $\sin{\frac{1}{n}} = \frac{1}{n} + o(\frac{1}{n})$ quindi $n\cdot \sin{\frac{1}{n}} = n \cdot (\frac{1}{n} + o(\frac{1}{n})) = 1 + o(1) \to 1$ per $x\to +\infty$.
\end{itemize}
\end{example}

\begin{observation}
$f(n)$ può avere limite anche se $f(x)$ non c'è l'ha infatti per esempio:
$f(x) = \sin{\pi x}$ non ha limite per $x\to +\infty$ ma $f(n) = \sin{\pi n}$ ha limite. \\
Quindi il metodo di utilizzare la funzione può non sempre funzionare.
\end{observation}

\begin{example}
Ci chiediamo se esiste $\lim\limits_{n\to +\infty}\sin(n)$. Vediamo che il limite non esiste:\\
Chiediamo quando $\sin(x) \geq \frac{1}{2}$ in $[0,\pi]$ succede esattamente per $x \in [\frac{\pi}{6},\frac{5}{6}\pi]$. L'intervallo ha lunghezza $\frac{5}{6}\pi - \frac{1}{6}\pi = \frac{4}{6}\pi = \frac{2}{8}\pi > 2$. Quindi l'intervallo contiene almeno due numeri interi (in $\mathbb{N}$) e lo stesso vale per tutti gli traslati di multipli di $2\pi$.\\
Questo ci permette di costruire una successione crescete $h_n$ di numeri naturali tale che $\sin(h_n) \geq \frac{1}{2} \: \forall \: n \in \mathbb{N}$.
Questo mi dice che se esiste $\lim\limits_{n\to +\infty}\sin(n) = l$, allora sicuramente $l \geq \frac{1}{2}$ (conseguenza della permanenza del segno). \\
Posso fare lo stesso discorso partendo da $\sin(x) \leq -\frac{1}{2}$, e trovo che $l \leq -\frac{1}{2}$. Questo è assurdo, e mi dimostra che non esiste $\lim\limits_{n\to +\infty}\sin(n)$.
\end{example}

\begin{example}
$\lim\limits_{n\to +\infty}n^2 \cdot \sin{n}$, ci chiediamo se esiste il limite.\\
Considerando la successione dell'esempio precedente $h_n$, troviamo una sottosuccessione $h_n^2 \cdot \sin(h_n)$, $\sin(h_n) \geq \frac{1}{2}$ quindi $h_n^2 \cdot \sin(h_n) \geq \frac{1}{2}\cdot h_n^2 \to +\infty$. Se $k_n$ è una successione di naturali tale che $\sin(k_n) \leq -\frac{1}{2} \forall \: n$, abbiamo una sottosuccessione $k_n^2 \cdot \sin(k_n) \leq -\frac{1}{2}kn^2 \to +\infty$. Quindi ho due sottosuccessioni di $n^2\sin{n}$ che hanno limiti diversi. Segue che non esiste $\lim\limits_{n\to +\infty}n^2 \cdot \sin{n}$.
\end{example}

\begin{theorem}
Sia $\{a_n\}_{n \in \mathbb{N}}$ (nota \footnote{In questa scrittura ci sono tutti i numeri naturali}) una successione,e $\{a_{h_n}\}$ e $\{a_{k_n}\}$ due sottosuccessioni tale che $\{h_n \: | \: n \in \mathbb{N}\} \cup \{k_n \: | \: n \in \mathbb{N}\} = \mathbb{N}$. (si dice che le due sottosuccessioni "saturano tutti gli indici").\\
Se $\exists \: \lim\limits_{n\to +\infty}\_{h_n}$ e $\exists \lim\limits_{n\to +\infty}a_{k_n}$ e sono uguali, allora esiste anche $\lim\limits_{n\to +\infty}a_n$ ed è uguale agli altri due.
\end{theorem}
\hspace{-15pt}Un caso tipico in cui si utilizza questo teorema è quando si prendono gli indici pari e dispari.
\begin{example}
$\Large{\lim\limits_{n\to +\infty}\frac{(\log{n+1})^{(-1)^h}}{n^3}}$. Guardiamo gli indici pari $k_n = 2n$ con il quale ho $\frac{(\log(2n+1))^1}{(2n)=3} \to 0$, e poi guardiamo gli indici dispari $h_n = 2n+1$ dove viene $\frac{(\log(2n+1))^{-1}}{(2n+1)^3} = \frac{1}{(2n+1)^3\log(2n+1)} = \frac{1}{+\infty} = 0$ quindi le sottosuccessioni saturano tutti gli indici.\\
Usando il teorema concludiamo che $\Large{\lim\limits_{n\to +\infty}\frac{(\log{n+1})^{(-1)^h}}{n^3}} = 0$.
\end{example}

\subsubsection{Criterio del rapporto}
\begin{theorem}[Criterio del rapporto]
Sia $\{a_n\}$ una successione.  Se $a_n > 0$ definitivamente, e se esiste $\lim\limits_{n\to +\infty}\frac{a_{n+1}}{a_n} = l$ allora:
\begin{enumerate}
    \item Se $0 \leq l \leq 1$, allora $\lim\limits_{n\to +\infty} a_n = 0$.
    \item Se $l > 1$, allora $\lim\limits_{n\to +\infty}a_n = +\infty$.
\end{enumerate}
\end{theorem}

\begin{observation}
Se $l = 1$, non si può dire niente sul comportamento di $a_n$.
\end{observation}

\begin{example}
Esempi del criterio del rapporto con $l=1$:
\begin{itemize}
    \item Prendo $a_n = 1 \forall \: n \in \mathbb{N}$. Allora $\frac{a_{n+1}}{a_n} = \frac{1}{1} = 1 \to 1$ quindi $l=1$ e $a_n$ converge a 1.
    \item Con $a_n = n$. Allora $\frac{a_{n+1}}{a_n} = \frac{n+1}{n} \to 1$ quindi $l=1$ e $a_n \to +\infty$.
    \item Con $a_n = \frac{1}{n}$, di nuovo $\frac{a_{n+1}}{a_n} = \frac{n}{n+1} \to 1$ sempre $l=1$ e $a_n \to 0$.
\end{itemize}
\end{example}

\begin{example}
Esempi di applicazioni del criterio del rapporto:
\begin{itemize}
    \item $a_n = (\frac{1}{2})^n$. Usando il criterio $\frac{a_{n+1}}{a_n} = \frac{(\frac{1}{2})^{n+1}}{(\frac{1}{2})^n} = \frac{2^n}{2^{n+1}} = \frac{1}{2} = l$ e ho $0 \leq l \leq 1$. Quindi $a_n \to 0$.
    \item $a_n = 2^n$ (si può usare $f(x) = 2^x$ e il fatto che $\lim\limits_{x\to +\infty}2^x = +\infty$). Usiamo il criterio del rapporto quindi $\frac{a_n+1}{a_n} = \frac{2^{n+1}}{2^n} = 2 = l$ e con $l>2$ concludo che $a_n \to +\infty$.
    \item $a_n = n!$. Criterio del rapporto che $\frac{a_{n+1}}{a_n} = \frac{(n+1)!}{n!} = \frac{(n+1)n!}{n!} = n+1 \to +\infty = l$ quindi $l>1$ e quindi $a_n \to +\infty$. In questo caso si poteva anche osservare che $n! > n$ e $n \to +\infty$ quindi per confronto segue che $n! \to +\infty$.
\end{itemize}
\end{example}

\hspace{-15pt}Confronto di $n!$ con $n^k$, $b^n$, $n^n$:
\begin{itemize}
    \item \textbf{Potenza} ($n^k$) con ($k\geq 1$). Vogliamo guardare che $\lim\limits_{n\to +\infty}\frac{n!}{n^k} = \frac{+\infty}{+\infty}$ forma indeterminata.\\
    Usiamo quindi il criterio del rapporto per $a_n = \frac{n!}{n^k}$. Ho $\frac{a_{n+1}}{a_n} = \frac{(n+1)!}{(n+1)^k} \cdot \frac{n^k}{n!} = \frac{(n+1)!}{n!} \cdot \frac{n^k}{(n+1)^k} = (n+1) \cdot (\frac{n}{n+1})^k \to +\infty \cdot (1)^k = +\infty = l$ quindi $l > 1$.\\
    Segue che $\frac{n!}{n^k} \to +\infty$, quindi $n!$ "tende a $+\infty$ più velocemente di $n^k$".
    \item \textbf{Esponenziale} ($b^n$) con $b > 1$. $\lim\limits_{n\to +\infty} \frac{n!}{b^n} = \frac{+\infty}{+\infty}$ forma indeterminata. Guardiamo quindi il rapporto, per $a_n = \frac{n!}{b^n}$. Quindi abbiamo $\frac{a_{n+1}}{a_n} = \frac{(n+1)!}{b^{n+1}} \cdot \frac{b^n}{n!} = \frac{(n+1)!}{n!} \cdot \frac{b^n}{b^{n+1}} = (n+1)\cdot \frac{1}{b} \to +\infty = l > 1$. Segue che $\frac{n!}{b^n} \to +\infty$, quindi $n!$ tende a $+\infty$ più velocemente di $b^n$.
    \item \textbf{Esponenziale potentissimo} ($n^n$). Notare che $n^n \to +\infty$ ad esempio perché $n^n \geq n$ e $n\to +\infty$. Facciamo $\lim\limits_{n\to +\infty}\frac{n^n}{n!} = \frac{+\infty}{+\infty}$ forma indeterminata. Usiamo il criterio del rapporto per $a_n = \frac{n^n}{n!}$. $\frac{a_{n+1}}{a_n} = \frac{(n+1)^{n+1}}{(n+1)!}\cdot \frac{n!}{n^n} = \frac{n!}{(n+1)!} \cdot \frac{(n+1)^{n+1}}{n^n} = \frac{1}{n+1} \cdot \frac{(n+1)^{n+1}}{n^n} = (\frac{n+1}{n})^n = (1 + \frac{1}{n})^n$ che è un limite notevole che $\to e > 1$. (per vederlo ad esempio si può scrivere $(1 + \frac{1}{n})^n = e^{\log(1+\frac{1}{n})^n} = e^{n\cdot \log(1+\frac{1}{n})} = e^{n\cdot(\frac{1}{n} + o(\frac{1}{n}))} = e^{1 + o(1)} \to e^1 = e$). Quindi segue che $\frac{n^n}{n!} \to +\infty$ quindi $n^n$ tende a $+\infty$ più velocemente di $n!$.
\end{itemize}


\subsubsection{Criterio della radice}
\begin{theorem}
Se $a_n > 0$ definitivamente, e $\exists \lim\limits_{n\to +\infty}\sqrt[n]{a_n} = l$, allora:
\begin{enumerate}
    \item Se $0 \leq l < 1$, allora $\lim\limits_{n\to +\infty} a_n = 0$.
    \item Se $l > 1$, allora $\lim\limits_{n\to +\infty}a_n = +\infty$.
\end{enumerate}
\end{theorem}

\begin{observation}
Se $l=1$ non si può dire niente riguardo al comportamento di $a_n$ come sul criterio del rapporto.
\end{observation}

\begin{demostration}
Dimostrazione dei due casi del criterio della radice.
\begin{enumerate}
    \item Suppongo che $0 \leq l \leq 1$ e fisso un $m\in \mathbb{R}$ tale che $l < m < 1$. Visto che $\sqrt[n]{a_n}\to l$ definitivamente avrò $\sqrt[n]{a_n} < m$, quindi $a_n < m^n$. Ora visto che $m < 1$ abbiamo visto che $m^n \to 0$, quindi visto che $0 < a_n < m^n$ per il teorema dei carabinieri segue che $a_n \to 0$.
    \item Questo punto si fa analogo, se invece $l > 1$ scelto $m \in \mathbb{R}$ tale che $1 < m < l$. Visto che $\sqrt[n]{a_n} \to l$ avrò $\sqrt[n]{a_n} > m$ definitivamente segue che definitivamente ho $a_n > m^n$ e visto che $n > 1$ ho $m^n \to +\infty$. Per confronto segue che $a_n \to +\infty$. $\blacksquare$
\end{enumerate}
\end{demostration}

\subsubsection{Relazione fra criteri del rapporto e della radice}
\begin{theorem}[Relazione fra rapporto e radice]
Se $a_n > 0$ definitivamente e se $\exists \lim\limits_{n\to +\infty}\frac{a_{n+1}}{a_n} = l$, allora $\exists \lim\limits_{n\to +\infty}\sqrt[n]{a_n}$ ed è uguale a l.
\end{theorem}

\begin{observation}
Questo teorema è vero anche con $l=1$.
\end{observation}

\begin{observation}
Potrebbe esiste $\lim\limits_{n\to +\infty}\sqrt[n]{a_n}$ e non esistere il $\lim\limits_{n\to +\infty} \frac{a_{n+1}}{a_n}$ (quindi questo teorema vale solo per un verso e non il viceversa).
\end{observation}

\begin{example}
Alcuni esempi utilizzando quest'ultimo teorema.
\begin{itemize}
    \item Fissiamo un $a > 0$. Proviamo a calcolare $\lim\limits_{n\to +\infty} \sqrt[n]{a}$. (Si può fare in diversi modi come $\sqrt[n]{a} = a^{\frac{1}{n}} \to a^0 = 1$).
    Usiamo l'ultimo teorema $a_n = a$ successione costante. Abbiamo quindi $\frac{a_{n+1}}{n} = \frac{a}{a} = 1$. Per il teorema segue che $\sqrt[n]{a_n} = \sqrt[n]{a} = 1$.
    \item Proviamo a fare $\lim\limits_{n\to +\infty}\sqrt[n]{n}$. usiamo il teorema con $a_n = n$. Abbiamo quindi $\frac{a_{n+1}}{a_n} = \frac{n+1}{n} \to 1$. Quindi segue che $\sqrt[n]{a_n} = \sqrt[n]{n} \to 1$.
    \item Nello stesso modo dell'esempio sopra si vede che $\sqrt[n]{p(n)}\to 1$ dove $p(n)$ è un polinomio in n.
\end{itemize}
\end{example}

\begin{example}
Esiste $\lim\limits_{n\to +\infty}\sqrt[n]{a_n}$ ma non esiste $\lim\limits_{n\to +\infty}\frac{a_{n+1}}{a_n}$.
Prendiamo $a_n = \begin{cases}1 & \text{ se n è pari} \\ 2 & \text{ se n è dispari }\end{cases}$\\\\
Abbiamo $\sqrt[n]{1} \leq \sqrt[n]{a_n} \leq \sqrt[n]{2} \: \forall \: n \in \mathbb{N}$. Abbiamo appena visto che $\sqrt[n]{1} \to 1$ e $\sqrt[n]{2} \to 1$, per il teorema dei carabinieri segue che $\sqrt[n]{2} \to 1$.\\\\
Ora $\frac{a_{n+1}}{a_n} = \begin{cases}\frac{2}{1}=2 & \text{ se n è pari} \\ \frac{1}{2} & \text{ se n è dispari }\end{cases}$ \hspace{.5cm} e questa successione non ha limite.
\end{example}

\begin{example}
Calcoliamo $\lim\limits_{n\to +\infty}\sqrt[n]{2 + \sin{n}}$. Usare il rapporto non sembra promettente perché se $a_n = 2 + \sin{n}$, sarebbe $\frac{a_{n+1}}{a_n} = \frac{2 + \sin{n+1}}{2 + \sin{n}}$. Visto che $-1 \leq \sin{n} \leq 1$ abbiamo che $\sqrt[n]{1} \leq \sqrt[n]{2 + \sin{n}} \leq \sqrt[n]{3}$, in questo caso sia $\sqrt[n]{1} \to 1$ che $\sqrt[n]{3} \to 1$ quindi per il teorema dei carabinieri, il limite è 1.
\end{example}

\hspace{-15pt}In riferimento all'esempio di prima possiamo dire più in generale, che se $a_n$ è limitata $m \leq a_n \leq M$ definitivamente (definitivamente limitata), con $m > 0$ allora ho $\sqrt[n]{m} \leq \sqrt[n]{a_n} \leq \sqrt[n]{M}$ e come sopra visto che $\sqrt[n]{1} \to 1$ e $\sqrt[n]{3} \to 1$ concludo che $\sqrt[n]{a_n} \to 1$.

\begin{example}
$\lim\limits_{n\to +\infty}\sqrt[n]{n!}$, pongo $a_n = n!$ e $\frac{a_{n+1}}{a_n} = \frac{(n+1)!}{n!} = n+1 \to +\infty$. Dall'ultimo teorema visto segue che $\sqrt[n]{a_n} = \sqrt[n]{n!} \to +\infty$.
\end{example}
\newpage
\section{Serie numeriche}
Sia $\{a_n\}$ una successione $S\to \mathbb{R}$ una successione $S\to \mathbb{R}$. Vogliamo definire $\sum\limits_{n\in S}a_n$, la somma di tutti i termini della successione.
\begin{example}
Dato $a_n = \frac{1}{2^n}$, con $S= \{n \geq 1\}$. Voglio definire $a_1 + a_2 + a_3 + ... + a_n$.\\
Questo sarà uguale a $\frac{1}{2} + \frac{1}{4} + \frac{1}{8} + \frac{1}{16} + ... + \frac{1}{2^n}$.
\end{example}
\begin{wrapfigure}[4]{r}{6cm}
    \vspace{-25pt}
    \centering
    \includegraphics[width=4.7cm]{images/serie-numeriche.png}
\end{wrapfigure}
 Notiamo che aggiungendo termini sembra che la somma si avvicini sempre di pi a 1. In effetti si ha che $\frac{1}{2} + \frac{1}{4} + \frac{1}{8} + \frac{1}{16} + ... + \frac{1}{2^n} = 1 - \frac{1}{2^n}$.\\
Prendendo il limite per $n\to +\infty$ sembra ragionevole che $\frac{1}{2} + \frac{1}{4} + \frac{1}{8} + \frac{1}{16} + ... + \frac{1}{2^n} = 1$.\\

\begin{definition}
Dato $\{a_n\}: \mathbb{N} \to \mathbb{R}$, definiamo $s_n = \sum^n_{j=0}a_j = a_0 + a_i + ... + a_n$ (somma parziale n-esima), se $\{s_n\}_{n\in \mathbb{N}}$ è una nuova successione. Definiamo $\sum_{n}a_n$ (questa è la serie associata alla successione $\{a_n\}$) come $s = \lim\limits_{n\to +\infty}s_n$, se questo esiste. 
\begin{itemize}
    \item Se il limite non esiste, si dice che la serie è indeterminata.
    \item Altrimenti, se $s \in \mathbb{R}$, si dice che la serie è convergente.
    \item Mentre se $s = +\infty$ si dice che la serie diverge positivamente.
    \item Mentre se $s = -\infty$ si dice che la serie diverge negativamente.
\end{itemize}
\end{definition}

\begin{example}
Vediamo alcuni esempi di serie.
\begin{itemize}
    \item $a_n = 0 \: \forall \: n \in \mathbb{N}$, $s_n = a_0 + a_1 + ... + a_n = 0 + ... + 0 = 0$ e $s = \lim\limits_{\to +\infty}s_n = \lim\limits_{n\to +\infty}0 = 0$
    \item $a_n = 1 \: \forall \: n \in \mathbb{N}$, $s_n = a_0 + a_1 + ... + a_n = 1 + ... + 1 = n+1$ e $ s = \lim\limits_{\to +\infty}s_n = \lim\limits_{n\to +\infty}(n+1) = +\infty$.
    \item $a_n = n$, $s_n = 0 + 1 + 2 + ... + n = \frac{n(n+1)}{2}$ e quindi $\sum\limits_{n\in \mathbb{N}}a_n = \lim\limits_{n\to +\infty}s_n = \lim\limits_{n\to + \infty}\frac{n^2 + n}{2} = +\infty$.
\end{itemize}
\end{example}

\subsection{Serie geometrica}
Prendiamo un $\alpha \in \mathbb{R}$, $a_n = \alpha^n$ (l'esempio di sopra è $\alpha = \frac{1}{2}$). Proviamo ora a calcolare $\sum\limits_{n\in \mathbb{N}}a_n = \sum\limits_{n\in \mathbb{N}}\alpha^n$.\\\\
Per farlo dobbiamo calcolare le serie parziali $s_n = \sum\limits_{j=0}^n a_j = 1 + \alpha + \alpha^2 + ... + \alpha^n = \frac{\alpha^{n+1} - 1}{\alpha -1}$. Questa può essere dimostrata per induzione oppure usando la seguente uguaglianza: $x^{n+1} - y^{n+1} = (x - y)(x^n + x^{n-1}y + x^{n-1}y^2 + ... + xy^{n-1} + y^n)$.\\
Facciamo $\lim\limits_{n\to +\infty}s_n = \lim\limits_{n\to +\infty}\frac{a^{n+1}-1}{\alpha - 1}$ e questo può fare:
\begin{itemize}
    \item Se $|\alpha| < 1$, abbiamo $\alpha^{n+1}\to 0$, quindi $\sum_{n\in \mathbb{N}}\alpha^n = \lim\limits_{n\to +\infty}s_n = \lim\limits_{n\to +\infty}\frac{a^{n+1}-1}{\alpha - 1} = \frac{-1}{\alpha - 1} = \frac{1}{1 - \alpha}$. Quindi converge.
    \item Se $|\alpha| > 1$, allora $\alpha^{n+1}\to +\infty$, quindi $\sum_{n\in \mathbb{N}}\alpha^n = \lim\limits_{n\to +\infty}s_n = \lim\limits_{n\to +\infty}\frac{a^{n+1}-1}{\alpha - 1} = +\infty$, quindi diverge positivamente.
    \item Se $\alpha = 1$ allora $a_n = \alpha^n = 1^n = 1 \forall \: n \in \mathbb{N}$, quindi $\sum_{n\in \mathbb{N}}\alpha^n = \sum_{n\in \mathbb{N}}1 = +\infty$.
    \item Se $\alpha = 0$, $a_n = \alpha^n = 0 \: \forall \: n \geq 1$, quindi $\sum_{n\in \mathbb{N}}\alpha^n = \sum_{n\in \mathbb{N}}0 = 0$ quindi converge.
    \item Se $\alpha < -1$, $\alpha^{n+1}$ non ha limite e questo perché se n è pari (quindi n+1 è dispari) $\alpha^{n+1} < 0$ tende a $-\infty$ (perché $|\alpha| >1$).\\
    Se n è dispari (quindi n+1 è pari) abbiamo che $\alpha^{n+1} > 0$ e tende a $+\infty$. Abbiamo quindi due sottosuccessioni $d_{2n} \to -\infty$ e $d_{2n+1}\to -\infty$ e segue per i teoremi precedentemente visti che $b_n = \alpha^{n+1}$ non ha limite. Quindi anche $s_n = \frac{a^{n+1}-1}{\alpha - 1}$ non ha limite. \\
    $s_{2n} = \frac{b_{2n} - 1}{\alpha - 1} \to \frac{-\infty}{\alpha - 1} = +\infty$, $s_{2n+1} = \frac{b_{2n+1} - 1}{\alpha - 1} \to \frac{+\infty}{\alpha - 1} = -\infty$. Dunque $s_n$ non ha limite e $\sum_{n\in \mathbb{N}}\alpha^n$ è indeterminata se $\alpha < -1$.
    \item $\alpha = -1$, $\alpha^n = (-1)^n = \begin{cases}1 & \text{ n pari}\\ -1 & \text{ n dispari}\end{cases}$\\
    $s_0 = a_0 = (-1)^0 = 1$, $s_1 = a_0 + a_1 = 1 + (-1)^1 = 0$, $s_2 = a_0 + a_1 + a_2 = 1 + (-1)^1 + 1 = 1$, $s_3 = a_0 + a_1 + a_2 + a_3 = 1 + (-1) + 1 + (-1) = 0$ ...\\\\
    $s_n = \begin{cases}1 & \text{ n pari}\\ 0 & \text{ n dispari}\end{cases}$ non ha limite, anche in questo caso $\sum_{n\in \mathbb{N}}(-1)^n$ è indeterminata.
\end{itemize}
Riassumendo questi esempio possiamo dire che $\sum\limits_{n=0}^{+\infty}\alpha^n$:
\vspace{-15pt}
\begin{itemize}
    \item Se $|\alpha| < 1$ allora fa $\frac{1}{1 - \alpha}$.
    \item Se $\alpha \geq 1$ allora fa $+\infty$.
    \item Se $\alpha \leq 1$ allora è indeterminata.
\end{itemize}
Ora chiediamoci cosa fa $\sum_{n=k}^{+\infty}\alpha^n = \alpha^k + \alpha^{k+1} + \alpha^{k+2} + ...$ (per $|\alpha| < 1$ con $\alpha \neq 0$) \\$= \alpha^k (1 + \alpha + \alpha^2 + ...) = \frac{\alpha^k}{1 - \alpha}$. Ad esempio se $\alpha = \frac{1}{2}$ e $k=1$, quindi guardo $\frac{1}{2} + \frac{1}{4} + \frac{1}{8} + ... = \frac{\alpha^k}{1 - \alpha} = \frac{(\frac{1}{2})^1}{1 -\frac{1}{2}} = \frac{\frac{1}{2}}{\frac{1}{2}} = 1$ (come ci asoettavamo).\\\\
Invece $\sum_{n=0}^{+\infty}(\frac{1}{2})^4 = 1 + \frac{1}{2} + \frac{1}{4} + ... = 1 + 1 = 2$ ($= \frac{1}{1 - \alpha}$ in questo caso $\alpha = \frac{1}{2}$ quindi $= \frac{1}{1 - \frac{1}{2}} = \frac{1}{\frac{1}{2}} = 2$).

\begin{example}
Prendiamo $\alpha = -\frac{1}{3}$, e con $k = 0$, quindi ho $\sum_{n=0}^{+\infty}(-\frac{1}{3})^n = \frac{1}{1 + \frac{1}{3}} = \frac{3}{4}$.
\end{example}

\begin{observation}
Se $-1 < \alpha < 0$, la somma $\sum_{n=0}^{+\infty}\alpha^n = \frac{1}{1 - \alpha}$. Vediamo che $0 < -\alpha < 1 \Longrightarrow 1 < 1 - \alpha < 2$ quindi la somma è compresa tra $\frac{1}{2}$ e 1. \\
Quindi $\sum_{n=0}^{+\infty} \alpha^n = 1 + \alpha + \alpha^2 + \alpha^3 + \alpha^4 + ...$, i vari elementi sono tutti $\alpha > 0, \alpha^2 > 0, \alpha^3 > 0 \alpha^4 > 0$.
\end{observation}

\begin{example}
Un caso per calcolare il valore preciso di una serie è quando si può usare gli sviluppi di taylor. Vediamo per esempio $\sum_{n}\frac{1}{n!}$ che converge ed è uguale a $e$.\\
Partiamo da $e^x = \sum_{j=0}^n \frac{x^j}{j!} + R_n(x)$ sviluppo di taylor di $e^x$ con io resto di lagrange che in generale è $R_n = \frac{f^{n+1}(z)}{(n+1)!}(x - x_0)^{n+1}$ con z compresa tra $x$ e $x_0$.\\
Nel nostro caso $R_n(x) = \frac{e^z}{(n+1)!}(x-0) = \frac{e^z}{(n+1)!}\cdot x^n$ con z compreso tra 0 e x. Ora specifichiamo $x=1$, troviamo $e^1 = \sum_{j^0}^n \frac{1}{j!} + R_n(1) = \frac{e^z}{(n+1)}\cdot 1$. (Ricordiamo che $\sum_{j^0}^n = s_n$ per $\sum_n \frac{1}{n!}$).\\
Quindi ricavo che $|e - s_n| = \frac{e^z}{(n+1)!}$ (la z dipende da $n!$ ma è sempre $0 < z < 1$) quindi posso dire che $\frac{e^z}{(n+1)!} < \frac{e}{(n+1)!}$ Prendo il limite per $n\to +\infty$, visto che $\frac{e}{(n+1)!} \to 0$ e quindi concludo che $s_n \to e$. Quindi concludo che $\sum_{n=0}^{+\infty} \frac{1}{n!} = e$.
\end{example}

\subsection{Condizione necessaria per l'esistenza di una serie}
\begin{theorem}[Condizione necessaria di una serie]
Se $a_n$ è una successione qualsiasi, e $\sum_n a_n$ converge, allora concludo che $\lim\limits_{n\to +\infty}a_n = 0$.
\end{theorem}

\begin{demostration}
$s_{n+1} = a_0 + a_1 + a_2 + ... + a_n + a_{n+1} = s_n + a_{n+1}$. Quindi se $s_{n+1} - s_n = a_{n+1}$. Se suppongo che $\sum_n a_n = l \in \mathbb{R}$ allora $s_{n+1} - s_n \to (l - l) = 0$, ma differenza $s_{n+1} - s_n = a_n$ quindi segue che $a_n \to 0$. $\blacksquare$
\end{demostration}

\hspace{-15pt}La conseguenza pratica di questo teorema è che se ho una successione $\{a_n\}$ e controllo che $\lim\limits_{n\to +\infty}a_n$ non è 0 (quindi può non esistere oppure essere $\pm \infty$ o essere un numero $\neq 0$) allora sicuramente $\sum_n a_n$ non converge.

\begin{example}
Alcuni esempio in cui è utile usare questo teorema.
\begin{itemize}
    \item $a_n = 1 \: \forall \: n \in \mathbb{N}$. $\lim\limits_{n \in \mathbb{N}}a_n = \lim\limits_{n\to +\infty} 1 = 1$, quindi $\sum_{n\in \mathbb{N}}$ non converge.
    \item $a_n = n$, $\lim\limits_{n\to +\infty} a_n = +\infty \Longrightarrow \sum_{n\in \mathbb{N}}n$ non converge.
\end{itemize}
\end{example}

\hspace{-15pt}Attenzione che se $\lim\limits_{n\to +\infty}a_n = 0$, non è detto che $\sum_n a_n$ converga.

\subsection{Valore della somma di sue serie}
\begin{theorem}
se $a_n$ e $b_n$ sono due successioni e $\sum_{n}a_n$ e $\sum_n b_n$ hanno senso (cioè non sono indeterminate) allora anche $\sum_n (a_n + b_n)$ ha senso e vale $\sum_n (a_n + b_n) = \sum_n a_n + \sum_n b_n$ questo supponendo che la somma non sia una forma indeterminata.\footnote{Le forme indeterminate possibili sono $+\infty - \infty$ o $-\infty + \infty$}
\end{theorem}

\begin{example}
Alcuni esempi di utilizzo di questo teorema.
\begin{itemize}
    \item $a_n = (\frac{1}{2})^n$, $b_n = (\frac{1}{3})^n$. Abbiamo $\sum^{+\infty}_{n=0}a_n = \frac{1}{1 - \frac{1}{2}} = 2$, $\sum^{+\infty}_{n=0}b_n = \frac{1}{1 - \frac{1}{3}} = \frac{3}{2}$. \\
    Quindi $\sum_{n=0}^{+\infty}(a_n + b_n) = \sum_{n=0}^{+\infty}((\frac{1}{2})^n + (\frac{1}{3})^n) = 2 + \frac{3}{2} = \frac{7}{2}$.
    \item $a_n = 1$, $b_n = -1$, ho $\sum_n a_n = +\infty$, $\sum_n b_n = -\infty$ quindi $\sum_n (a_n + b_n)$ non si può sapere tramite il teorema perché non si applica. Però $a_n + b_n = 1 - 1 = 0$, quindi $\sum_n (a_n + b_n) = 0$.
    \item $a_n = n^2$, $b_n = -n$, ho quindi $\sum_n a_n = +\infty$, $\sum_n b_n = -\infty$. Questa volta $a_n + b_n = n^2 - n \to +\infty$ (perché $n^2 - n = n(n-1)$) e segue dalla condizione necessaria che $\sum_n (a_n + b_n)$ non converge (ma diverge positivamente, visto che $a_n + b_n \to +\infty$).
\end{itemize}
\end{example}

\begin{observation}
Non c'è un teorema analogo riguardo a $\sum_n (a_n \cdot b_n)$. In particolare non è vero che $\sum_n(a_n \cdot b_n) = (\sum_n a_n) \cdot (\sum_n b_n)$.
\end{observation}

\begin{example}
Possiamo vedere del perché di questa osservazione prendendo $a_n = (\frac{1}{2})^n$, $b_n = (\frac{1}{3})^n$. $\sum_n a_n = 2$, $\sum_n b_n = \frac{3}{2}$. $a_n \cdot b_n = (\frac{1}{6})^2$ e $\sum_n a_n \cdot b_n = \sum_n (\frac{1}{6})^n = \frac{1}{1 - \frac{1}{6}} = \frac{6}{5}$ e $\frac{6}{5} \neq 2 \cdot \frac{3}{2}$.
\end{example}

\hspace{-15pt}Può anche succedere che $\sum_n a_n$ e $\sum_n b_n$ convergano ma $\sum_n a_n \cdot b_n$ non converge.

\subsection{Serie definitivamente a termini positivi}
\begin{theorem}
Se ho $a_n \geq 0$ definitivamente \footnote{Questo vuol dire che da un certo punto in poi è sempre positiva} allora $\sum_n a_n$ converge oppure diverge positivamente (non pu essere indeterminata o andare a $-\infty$).
\end{theorem}

\begin{demostration}
Come prima abbiamo visto che $s_{n+1} = s_n + a_{n+1}$. Se $a_n \geq 0$ definitivamente, ho che $s_{n+1} \geq s_n$ definitivamente. Quindi $\{s_n\}$ è definitivamente (debolmente) crescente, quindi ammette limite, che può essere un numero reale, oppure $+\infty$ (non $-\infty$ perchè ho una successione che sta crescendo). $\blacksquare$
\end{demostration}

\begin{observation}
Se $a_n \leq 0$ definitivamente, analogamente si può dire che $\sum_n a_n$ converge oppure diverge negativamente.
\end{observation}

\subsection{Criterio del confronto}
\begin{theorem}[Criterio del confronto]
Se $o \leq a_n \leq b_n$ definitivamente. Allora:
\begin{enumerate}
    \item Se $\sum_n b_n$ converge $\Longrightarrow \sum_n a_n$ converge.
    \item Se $\sum_n a_n$ diverge $\Longrightarrow \sum_n b_n$ diverge. 
\end{enumerate}
\end{theorem}

L'idea è che se $0 \leq a_n \leq b_n \: \forall \: \in \mathbb{N}$, allora $0 \leq \sum_n a_n \leq \sum_n b_n$

\begin{example}
Alcuni esempi su questo teorema.
\begin{itemize}
    \item Sapendo che $\sum_n 1 = +\infty$, posso concludere che $\sum_{n=0}^{+\infty}n = +\infty$ (perché $0 \leq 1 \leq n \forall \: n \geq 1$) e anche $\sum_{n=0}^{+\infty}n^2 = +\infty$ (perché $0 \leq 1 \leq n^2 \: \forall \:n \geq 1$)
    \item Voglio sapere cosa fa $\sum_n \frac{\sin{n}^2}{2^2}$. $a_n = \frac{\sin{n}^2}{2^n} \leq \frac{1}{2^n} = b_n$. So che $\sum_n b_n$ converge e sappiamo calcolare la somma, dunque per il teorema anche questa $\sum_n a_n$ converge.
    \item Cosa fa $\sum_n n!$. Abbiamo $n! \geq n \: \forall \:n \geq 1$, e sappiamo che $\sum_n n = +\infty$, quindi concludiamo che $\sum_n n! = +\infty$
\end{itemize}
\end{example}

\subsection{Criterio del confronto asintotico}
\begin{theorem}[Criterio del confronto asintotico]
Prendiamo $\{a_n\}, \{b_n\}$ successioni, tale che $a_n > 0$ e $b_n > 0$ definitivamente, e supponiamo che $\lim\limits_{n\to +\infty}\frac{a_n}{b_n}= l \in \overline{\mathbb{R}}$. Allora si può dire che:
\begin{enumerate}
    \item Se $l \in (0, +\infty)$, allora $\sum_n a_n$ e $\sum_n b_n$ hanno lo stesso comportamento (cioè entrambe convergono o entrambe divergono a $+\infty$).
    \item Se $l = 0$ e $\sum_n b_n$ converge allora $\sum_n a_n$ converge. ("infatti" $\frac{a_n}{b_n}\to 0 \Longrightarrow \frac{a_n}{b_n} <1$ definitivamente $\Longrightarrow a_n < b_n$ definitivamente e da qui è chiaro che $se \sum_n b_n$ converge allora anche $\sum_n a_n$)
    \item Se $l = +\infty$ e $\sum_n b_n$ diverge, allora $\sum_n a_n$ diverge. ($\frac{a_n}{b_n}\to +\infty \Longrightarrow \frac{a_n}{b_n} > 1$ definitivamente $\Longrightarrow a_n > b_n$ definitivamente)
\end{enumerate}
\end{theorem}

\begin{observation}
Ad esempio nel punto (2), se $\sum_n b_n = +\infty$, non posso concludere niente riguardo a $\sum_a a_n$.
\end{observation}

\begin{example}
$\sum_n \frac{1}{2^n - \log(n)}$. $a_n = \frac{1}{2^n - \log(n)}$, definitivamente $> 0$ perché $2^n > \log(n)$ definitivamente. L'idea qui è che per n grande, $\log(n)$ "conta molto meno di $2^n$" quindi faccio confronto asintotico con $b_n = \frac{1}{2^n}$.\\
Abbiamo $\lim\limits_{n\to +\infty}\frac{a_n}{b_n} = \lim\limits_{n\to +\infty} \frac{\frac{1}{2^n - \log(n)}}{\frac{1}{2^n}} = \lim\limits_{n\to +\infty}\frac{2^n}{2^n - \log(n)} = \lim\limits_{n\to +\infty} \frac{1}{1 - \frac{\log(n)}{2^n}} = 1$ questo è l. Quindi in questo caso $l \in (0, +\infty)$, quindi $\sum_n a_n$ ha lo stesso comportamento di $\sum_n b_n = \sum_n (\frac{1}{2})^n$ che converge. Quindi $\sum_n a_n$ converge.
\end{example}

\subsection{Criterio della radice}
\begin{theorem}[Criterio della radice]
Prendo una $\{a_n\}$ una successione tale che $a_n > 0$ definitivamente. Se $\exists \: \lim\limits_{n \to +\infty}\sqrt[n]{a_n} = l \in \overline{\mathbb{R}}$.
\begin{enumerate}
    \item Se $0 \leq l \leq 1$, allora $\sum_n a_n$ converge. ($\Longrightarrow$ per la condizione necessaria $\lim\limits_{n\to +\infty}a_n = 0$).
    \item Se $l > 1$, allora $\sum_n a_n$ diverge.
\end{enumerate}
\end{theorem}

\begin{demostration}
Dimostriamo i due casi del teorema.
\begin{enumerate}
    \item Se $l < 1$, scelgo $\alpha \in \mathbb{R}$ tale che $l < \alpha < 1$, e visto che $\sqrt[n]{a_n}\to l$, definitivamente avrò $\sqrt[n]{a_n} < \alpha$ quindi $a_n < \alpha^n$ definitivamente. Per confronto, visto che $\sum_n \alpha^n$ converge, concludo che anche $\sum_n a_n$ converge.
    \item Discorso simile anche per questo punto, quindi prendo $< \alpha < l$, e poi definitivamente $\alpha < \sqrt[n]{a_n}$ quindi $\alpha^n < a_n$ definitivamente, e ora però $\sum_n \alpha^n = +\infty$ perché $\alpha > 1$, quindi anche $\sum_n a_n$ diverge. $\blacksquare$
\end{enumerate}
\end{demostration}

\begin{observation}
Come per le successioni quando $l=1$ no si può concludere niente.
\end{observation}

\begin{example}
$\sum_n \frac{n}{3^n}$, $a_n = \frac{n}{3^n}$, e $\sqrt[n]{a_n} = \frac{\sqrt[n]{n}}{3} \to \frac{1}{3} = l$ quindi $l < 1$, e quindi la serie converge.
\end{example}

\newpage
\subsection{Criterio del rapporto}
\begin{theorem}[Criterio del rapporto]
Prendo $\{a_n\}$ successione, $a_n > 0$ definitivamente. Se $\exists \lim\limits_{n \to +\infty}\frac{a_{n+1}}{a_n} = l \in \overline{\mathbb{R}}$.
\begin{enumerate}
    \item Se $0 \leq l < 1$, allora $\sum_n a_n$ converge.
    \item Se $l > 0$, allora $\sum_n a_n$ diverge.
\end{enumerate}
\end{theorem}

\begin{demostration}
Sappiamo che se $\exists \: \lim\limits_{n\to +\infty}\frac{a_{n+1}}{a_n} = l$, allora esiste anche $\lim\limits_{n\to +\infty}\sqrt[n]{a_n}$, ed è uguale a l. Quindi la conclusione segue dal criterio della radice (appena visto).
\end{demostration}

\begin{example}
$\sum_n \frac{n^2}{n!}$, $a_n = \frac{n^2}{n!}$. Usiamo il criterio del rapporto.\\
$\frac{a_{n+1}}{a_n} = \frac{(n+1)^2}{(n+1)!} \cdot \frac{n!}{n^2} = \frac{(n+1)^2}{(n+1)n!}\cdot \frac{n!}{n^2} = \frac{n+1}{n^2} \to 0 = l$. Quindi visto che $l = 0$, concludo che la serie converge.
\end{example}

\begin{observation}
Questi criteri per successioni definitivamente positive si applicano anche a successioni \textbf{definitivamente negative}. Infatti se $a_n < 0$ definitivamente allora $-a_n > 0$ definitivamente, quindi applico i criteri visti alla successioni $\{-a_n\}$ e poi $\sum_{j=0}^n a_j = -\sum_{j=0}^n(-a_j)$ dunque $\sum_{n=0}^{+\infty}a_n = -\sum_{n=0}^{+\infty}(-a_n)$ (se i limiti esistono).
\end{observation}

\subsection{Legami con gli integrali impropri}
Una serie $\sum_n a_n$ si può scrivere come integrale improprio. Considero una $f: [0,+\infty) \to \mathbb{R}$ data da $f(x)=a_{[x]}$ ($[x]$ parte intera di un x).\\
\begin{wrapfigure}[4]{r}{6cm}
    \vspace{-33pt}
    \centering
    \includegraphics[width=5cm]{images/legame-serie-integrali-impropri.png}
\end{wrapfigure}
Si crea dunque una funzione a gradini. Si ha $\sum_{j=0}^n a_j = \int_0^{n+1}f(x)\:dx$. Quindi prendendo il limite per $n\to +\infty$, trovo $\sum_n a_n = \int_0^{+\infty}f(x)\:dx$ (se i limiti hanno senso).\\\\
\begin{wrapfigure}[5]{l}{6cm}
    \vspace{-30pt}
    \centering
    \includegraphics[width=5.5cm]{images/legame-serie-integrali-impropri-2.png}
\end{wrapfigure}

Viceversa, partendo da  $f: [0,+\infty)\to \mathbb{R}$, posso considerare la successione $a_n = f(n)$ e la serie $\sum_n a_n = \sum_n f(n)$ (in questo caso la serie $\sum_n a_n$ è la somma delle aree dei rettangoli blu). Questa volta $\sum_n a_n$ e $\int_0^{+\infty}f(x) \:dx$ non saranno proprio uguali.\\
\begin{theorem}[Criterio dell'integrale]
Fissiamo $\overline{n} \in \mathbb{N}$, e $f: [\overline{n}, +\infty) \to \mathbb{R}$ che sia debolmente crescente, continua, con $f(x) \geq 0 \: \forall x \in [\overline{n}, +\infty)$, e poniamo $a_n = f(n)$. Allora $\sum_n a_n$ e $\int_{\overline{n}}^{+\infty}f(x)\:dx$ hanno lo stesso comportamento, e $\sum_{n=\overline{n}+1}^{+\infty}$.
\end{theorem}
Questo teorema può essere usato per entrambi i versi.
\begin{example}
Vediamo alcuni esempi del criterio.
\begin{itemize}
    \item $\sum_n \frac{1}{n^{\alpha}}$. Serie armonica generalizzata ($\alpha = 1 \longrightarrow \sum_n \frac{1}{n}$ serie armonica).\\
    Converge se $\alpha > 1$, e dunque se $\alpha \leq 1$. Infatti se prendo $f(x) = \frac{1}{x^{\alpha}}$ è decrescente e continua.
    Quindi abbiamo che $\int_1^{+\infty}\frac{1}{x^{\alpha}}$:
    \begin{itemize}
        \item Converge se $\alpha > 1$.
        \item Diverge a $+\infty$ se $\alpha \leq 1$
    \end{itemize}
    Quindi applicando il criterio dell'integrale si conclude quello scritto sopra.
    \begin{observation}
    Se $\alpha \leq 0$, $\sum_n\frac{1}{x^{\alpha}}$ diverge perché non è soddisfatta nemmeno la condizione necessaria.
    \end{observation}
    \item Calcoliamo $\sum_{n=2}^{+\infty}\frac{1}{n^{\alpha}(\log(n))^{\beta}}$. Usiamo il criterio dell'integrale con $f(x) = \frac{1}{x^{\alpha}(\log(x))^{\beta}}$.\\
    $\int_2^{+\infty}\frac{1}{x^{\alpha}(\log(x)^{\beta}}$ posiamo notare che:
    \begin{itemize}
        \item Converge se $\alpha > 1$, $\beta \in \mathbb{R}$.
        \item Diverge se $\alpha < 1$, $\beta \in \mathbb{R}$.
        \item Converge se $\alpha = 1$, $\beta > 1$.
        \item Diverge se $\alpha = 1$, $\beta \leq 1$.
    \end{itemize}
    (Questo come visto in precedenza). La serie si comporta allo stesso modo.
\end{itemize}
\end{example}

\begin{example}
Prendiamo $\sum_{n=1}^{+\infty}(e^{\frac{1}{n}} - 1)$. $a_n = e^{\frac{1}{n}} - 1 > 0$ (essendo maggiore di zero posso usare in seguito il confronto asintotico).\\
$\lim\limits_{n\to +\infty}a_n = \lim\limits_{n\to +\infty}(e^{\frac{1}{n}}-1) = e^0 - 1 = 1 - 1 = 0$. Quindi la condizione necessaria è soddisfatta e la serie può convergere. Possiamo usare lo sviluppo di talyor con $e^t = 1 + t + o(t)$ per $t\to 0$, quindi $e^{\frac{1}{n}} = 1 + \frac{1}{n} + \frac{1}{n}$ (t = $\frac{1}{n}$).
In termini di "importante" sarà $\frac{1}{n}$. In questi casi pongo $b_n = \frac{1}{n}$ e uso il confronto asintotico:\\
$\lim\limits_{n \to +\infty}\frac{a_n}{b_n} = \lim\limits_{n \to +\infty}\frac{e^{\frac{1}{n}}-1}{\frac{1}{n}} = \lim\limits_{n \to +\infty} \frac{\frac{1}{n} + o(\frac{1}{n})}{\frac{1}{n}} = \lim\limits_{n \to +\infty}(1 + o(1)) = 1$. \\
Per confronto asintotico, concludo che $\sum_n a_n$ ha lo stesso comportamento di $\sum_n b_n = \sum_n \frac{1}{n}$ (che è la serie armonica) che sappiamo diverge. Quindi $\sum_n a_n$ diverge a $+\infty$.
\end{example}

\subsection{Convergenza assoluta}
Prendiamo $\{a_n\}$ un successione qualsiasi (quindi non supponiamo che sia ne definitivamente positivo ne def. negativo).
\begin{definition}[Convergenza assoluta]
Diamo che $\sum_n a_n$ \textbf{converge assolutamente} se $\sum_n |a_n|$ converge.
\end{definition}

\begin{theorem}[Criterio dell'assoluta convergenza]
Se la serie $\sum_n a_n$ converge assolutamente, allora converge, e $|\sum_n a_n| \leq \sum_n |a_n|$.
\end{theorem}

\begin{demostration}
Dimostrazione che segue quella dell'analogo per gli integrali impropri. Vediamo innanzitutto che $a_n = a_n^+ - a_n^-$, mentre $|a_n| = a_n^+ + a_n^-$ e $0 \leq a_n^+ \leq |a_n|$, $0 \leq a_n^- \leq |a_n|$ e se $\sum_n |a_n|$ converge, per confronto convergono anche $\sum_n a_n^+$ e $\sum_n a_n^-$. Quindi converge anche $\sum_n a_n = \sum_n a_n^+ - \sum_n a_n^-$. Per la disuguaglianza triangolare se prendo $|\sum_{j=0}^n a_j| \leq \sum_{j=0}^n |a_j|$ e prendendo il limite per $n\to +\infty$ trovo $|\sum_n a_n| \leq \sum_n |a_n|$. $\blacksquare$
\end{demostration}

\begin{example}
Prendo $\sum_{n=1}^{+\infty}\frac{\sin{n}}{n^2}$. $a_n = \frac{\sin{n}}{n^2}$ è a segno variabile. \\
$|a_n| = |\frac{\sin{n}}{n^2}| = \frac{|\sin{n}|}{n^2} < \frac{1}{n^2}$. Visto che $\sum_n \frac{1}{n^2}$ converge (serie armonica generalizzata con $\alpha = 2 > 1$) per confronto segue che $\sum_n |\frac{\sin{n}}{n^2}|$ converge, quindi per il criterio di assoluta convergenza, concludo che anche $\sum_n \frac{\sin{n}}{n^2}$ converge (notiamo che però non sappiamo a che numero converge sappiamo solo che converge).
\end{example}

\begin{observation}
Se $\sum_n |a_n|$ diverge, non si può dire niente riguardo a $\sum_n a_n$ (cioè la $\sum a_n$ potrebbe converge o divergere).
\end{observation}

\subsection{Criterio di Leibnitz}
\begin{definition}[Serie a segno alterno]
Una serie a segno alterno è una serie della forma $\sum_n (-1)^n \cdot a_n$, dove $\{a_n\}$ è una successione a segno costante.
\end{definition}

\begin{example}
$\sum_n \frac{(-1)^n}{n^3}$ è a segno alterno. $\sum_n (-1)^n(-\frac{1}{n})$ è a segno alterno.. $\sum_n (-1)^n \sin{n}$ non è a segno alterno.
\end{example}

\begin{theorem}[Criterio di Leibnitz]
Se ho $\{a_n\}$ definitivamente $\geq 0$ e debolmente crescente e tale che $\lim\limits_{n\to +\infty}a_n = 0$, allora $\sum_n (-1)^n a_n$ converge.
E $\big|\sum_{j=0}^{+\infty}(-1)^ja_j - \sum_{j=0}^{n}(-1)^j a_j \big| \leq a_{n+1}$.
\end{theorem}

\begin{example}
Vediamo alcuni esempi di questo criterio.
\begin{itemize}
    \item $\sum_n \frac{(-1)^n}{n}$ converge, perché $a_n = \frac{1}{n}$ è $\geq 0$ e debolmente decrescente e $\lim\limits_{n\to +\infty}\frac{1}{n} = 0$.\\
    Notare che la serie dei valori assoluti è $\sum_n |\frac{(-1)^n}{n}| = \sum_n \frac{1}{n} = +\infty$. Questo è un esempio in cui $\sum_n |b_n|$ diverge ma $\sum_n b_n$ converge.
    \item $\sum_n \frac{(-1)^{n+1}}{n}$ vediamo se si può applicare Laibnitz. $b_n = \frac{(-1)^{n+1}}{n} = -\frac{(-1)^n }{n}$ converge, quindi converge anche $\sum_n \frac{(-1)^{n+1}}{n} = -\sum_n \frac{(-1)^n}{n}$.
\end{itemize}
\end{example}

\begin{example}
Vediamo alcuni esempi particolari "di avvertimento".
\begin{itemize}
    \item Può essere che $\sum_n a_n$ e $\sum_n b_n$ convergano, ma $\sum_n a_nb_n$ non converga.\\
    $a_n = \frac{(-1)^n}{n}$, $b_n = \frac{(-1)^n}{\log(n)}$. $\sum_n a_n$ converge. $\sum_n b_n = \sum_{n \geq 0} \frac{(-1)^n}{\log(n)}$ converge per Leibnitz.\\
    $a_nb_n = \frac{(-1)^n}{n} \cdot \frac{(-1)^n}{\log(n)} = (-1)^{2n}\cdot \frac{1}{n\log(n)} = \frac{1}{n\log(n)}$ e $\sum_n a_nb_n = \sum_n \frac{1}{n\log(n)}$ diverge (visto prima).
    \item Il confronto asintotico non funziona se il segno della successione non è definitivamente costante.\\
    $a_n = \frac{(-1)^n}{\sqrt{n}}$, $b_n = \frac{(-1)^n}{\sqrt{n}} + \frac{1}{n} = \frac{(-1)^n \sqrt{n} + 1}{n}$. Si ha $\lim\limits_{n\to +\infty}\frac{a_n}{b_n} = \lim\limits_{n\to +\infty}\frac{\frac{(-1)^n}{\sqrt{n}}}{\frac{(-1)^n\sqrt{n}+1}{n}} = \lim\limits_{n\to +\infty} \frac{(-1)^n \sqrt{n}}{(-1)^n \sqrt{n} + 1} = \lim\limits_{n\to +\infty}\frac{1}{1 + \frac{1}{(-1)^n \sqrt{n}}} = 1$ (perché $ \frac{1}{(-1)^n \sqrt{n}} \to 0$).\\
    Quindi il confronto asintotico (se funzionasse) mi diverge che $\sum a_n$ e $\sum b_n$ hanno lo stesso comportamento. Ma $\sum_n a_n = \sum_n \frac{(-1)^n}{\sqrt{n}}$ converge per Leibnitz e $\sum_n b_n = \sum_n b_n = \sum_n \frac{(-1)^n}{\sqrt{n}} + \sum_n \frac{1}{n}$ ($\sum_n \frac{(-1)^n}{\sqrt{n}}$ converge e $\sum_n \frac{1}{n} = +\infty$) quindi $\sum_n b_n$ diverge.
\end{itemize}
\end{example}
\newpage
\section{Introduzione analisi in più variabili}
Fin'ora abbiamo visto: insiemi $\subset \mathbb{R}$, funzioni da $\mathbb{R}\to \mathbb{R}$ con i imiti di funzioni ed il calcolo differenziale, teoria dell'integrazione per funzioni di una variabile, successioni e serie numeriche. L'ultima parte del corso l'obbiettivo è lavorare con più variabili in particolare con $\mathbb{R}^n$, vorremmo definire l'insieme nel piano (nello spazio), derivare delle curve e studiare funzioni in più variabili $f: \mathbb{R}^n \to \mathbb{R}$ quindi definire un limine la continuità, le derivare e studiare la funzione individuando massimi e minimi.
\subsection{Struttura euclidea di $\mathbb{R}^n$}
\begin{definition}
Possiamo definire $\mathbb{R}^n$ come uno spazio vettoriale dove possiamo fare somma, prodotto per un numero, combinazioni lineare indipendenza lineare, la base ed i sottospazi. Se $x \in \mathbb{R}^n \rightarrow x = (x_1, x_2, ..., x_n)$.
\end{definition}
\hspace{-15pt}Per esempio $x = (x_1, x_2)\in \mathbb{R}^2$, mentre $x = (x_1, x_2, x_3) \in \mathbb{R}^3$. Alle livello di notazioni si può scrivere $\mathbb{R}^2 = (x,y) \in \mathbb{R}^2$. \\
$\mathbb{R}^2$ sarà un piano mentre $\mathbb{R}^3$ sarà uno spazio dove possiamo definire.
\begin{itemize}
    \item Somma di 2 vettori: $x = (x_1, ..., x_n)$, $y = (y_1, ..., y_n)$, e $x + y = (x_1 + y_1, ..., x_n + y_n)$.
    \item Prodotto per un numero $\lambda \in \mathbb{R}$: $\lambda \cdot x = (\lambda \cdot x_1, \lambda \cdot x_2, ..., \lambda\cdot x_n)$.
\end{itemize}

\subsection{Operazioni sullo spazio}
Definiamo una la struttura di questo spazio, e lo facciamo definendo delle operazioni.
\begin{definition}[Prodotto scalare]
Il \textbf{prodotto scalare} è un operazione che ha come input 2 vettori, dato $x,y \in \mathbb{R}^n$ definiti come $x= (x_1, ..., x_n), y = (y_1, ..., y_n)$ il prodotto scalare è:
\vspace{-10pt}
\[<x, y> = x \cdot y = (x, y) = x_1y_2 + x_2y_2 + x_3y_3 + ... + x_ny_n = \sum\limits_{i=1}^n x_iy_i\]
\end{definition}
\begin{definition}[Norma]
La \textbf{norma} è un operazioni che ah come input un solo vettore e come risultato un numero maggiore o uguale a 0. Dato $x \in \mathbb{R}^n$, $x = (x_1, x_2, ..., x_n)$ la norma è:
\[||x|| = |x| = \sqrt{<x,x>} = \sqrt{x_1^2 + x_2^2 + ... + x_n^2}\]
La norma di x è la lunghezza del vettore x.
\end{definition}
\begin{definition}[Distanza]
La \textbf{distanza} è un'operazione che ha come input due vettori e come risultato un numero maggiore o uguale a 0. Dato un $x,y \in \mathbb{R}^n$ definiamo la distanza come:
\[dist(x,y) = d(x,y) = ||x - y|| = \sqrt{(x_1 - y_1)^2 + (x_2 - y_2)^2 + ... + (x_n - y_n)^2}\]
\end{definition}

\hspace{-15pt}Notare che con queste operazioni abbiamo definito in $\mathbb{R}^n$:\\
(misurare gli angoli)$<,> \xrightarrow[]{\text{induce}}$ (lunghezza di un vettore) $|| . || \xrightarrow[]{\text{induce}}$ (distanza tra due elementi) $d$
Il prodotto scalare serve a misurare gli angoli perché geometricamente $<x,y> = ||x|| \cdot ||y|| \cdot \cos{\Theta}$ dove $\Theta$ è l'angolo compreso fra i vettori $x,y$.
\begin{note}
Una piccola digressione per dire che possiamo dire sia punti $(x_1, x_2)$ oppure il punto $x$ è uguale dove in $\mathbb{R}^2$ se scrivo $(x_1, x_2)$ questo rappresenta sia il punto che il vettore applicato nell'origine come punto della freccia in $(x_1, x_2)$.
\end{note}
\begin{wrapfigure}[4]{r}{6cm}
    \vspace{-20pt}
    \centering
    \includegraphics[width=5cm]{images/ess-vettore.png}
\end{wrapfigure}
In questa rappresentazione la differenza tra i due vettori P, Q dove $P \leftrightarrow x = (x_1, x_2)$ e $Q \leftrightarrow y = (y_1, y_2)$, è $x-y = (x_1 - y_1, x_2 - y_2) = \vec{P} - \vec{Q}$ (penso il vettore differenza come applicato un Q con punta della freccia in P). Quindi $d(x,y) = d(P,Q) = ||x-y||$.

\subsection{Insiemi nello spazio $\mathbb{R}^n$}
Data la nozione di distanza possiamo definire gli insiemi in uno spazio $\mathbb{R}^n$.
\begin{definition}[Palla]
Dato $x \in \mathbb{R}^n$ e dato $r > 0, r \in \mathbb{R}$ si dice \textbf{palla} di centro x e raggio r:
\[B(x,r) = B_r(x) = \{y \in \mathbb{R}^n \::\: d(x,y) < r\}\]
La palla di centro x e raggio r può essere chiamato anche intorno sferico di x di raggio r.
\end{definition}
\hspace{-15pt}Per esempio in caso di $\mathbb{R}^2$, $B(x,r) = B_r(x) = \{y \in \mathbb{R}^2 \::\: d(x,y) < r\}$, sarà una circonferenza escluso il bordo. Nel caso di $\mathbb{R}^3$ invece sarà una sfera piena privata del bordo.\\
Se torniamo al caso $\mathbb{R}$ la $B_r(x) = \{\in \mathbb{R} \::\: d(x,y) < r\}$ dove $d(x,y) = |x-y|$, quindi come il caso $\mathbb{R}^n$ solo che nel caso di $\mathbb{R}$ abbiamo un valore assoluto mentre nel caso $\mathbb{R}^n$ abbiamo una norma.

\begin{figure}[h!]
\centering
\begin{subfigure}{.45\textwidth}
    \centering
    \includegraphics[width=6cm]{images/sfera-piena-priva-bordo.png}
    \caption{Sfera piena priva di bordo}
\end{subfigure}
\begin{subfigure}{.45\textwidth}
    \centering
    \includegraphics[width=4cm]{images/circonferenza-esclusa.png}
    \caption{Circonferenza esclusa}
\end{subfigure}
\end{figure}

\begin{definition}[Sfera]
Dato $x \in \mathbb{R}^n$. e dato $r > 0, r \in \mathbb{R}$ si dice \textbf{sfera} di centro x e raggio r l'insieme:
\[S(x,r) = S_r(x) = \{y \in \mathbb{R}^n \::\: d(x,y) = r\}\]
\end{definition}
\hspace{-15pt}Nel caso vedessimo la sfera in $\mathbb{R}$ avremmo $S_r(x) = \{x-r\} \cup \{x + r\}$.

\begin{figure}[h!]
    \centering
    \includegraphics[width=9cm]{images/ess-sfera.png}
    \vspace{-20pt}
    \caption{Esempi sfera}
\end{figure}

\vspace{-10pt}
\subsection{Proprietà di $\mathbb{R}^n$}
Ricordiamo come notazione che se $E$ è un insieme $E \subseteq \mathbb{R}^n$ allora indichiamo ocn $E^c = \mathbb{R}^n \setminus E =$ complementare di E rispetto a tutto $\mathbb{R}^n$.
\begin{definition}[Punto interno, esterno, di frontiera]
Sia $E \subseteq \mathbb{R}^n$ un punto $x_0 \in \mathbb{R}^n$ si dice:
\begin{itemize}
    \item \textbf{Punto interno ad E} se esiste una palla di centro $x_0$ e raggio $r > 0$ contenuta in E, cioè se esiste $r > 0$ tale che $B_r(x_0) \subset E$, si dice che $x_0$ è un punto interno ad E.
    \item \textbf{Punto esterno ad E} se esiste una palla di centro $x_0$ e raggio $r$ tutta contenuta in $E^x$ cioè se $\exists r > 0$ tale che $B_r(x_0) \subset E^c = R^n \setminus E$.
    \item \textbf{Punto di frontiera per E} se non è ne interno ne esterno. $\forall r > 0$ $V_r(x_0) \cap E \neq \O$ e $B_r(x_0) \cap E^c \neq \O$.
\end{itemize}
\end{definition}

\begin{observation}
Alcune osservazioni su queste proprietà:
\begin{itemize}
    \item Se $x_0$ è un punto interno ad E $\Longrightarrow x_0 \in E$.
    \item Se $x_0$ è punto esterno ad E $\Longrightarrow x_0 \notin E$.
    \item Se $x_0$ è punto di frontiera $\Longrightarrow x_0 \in E$ oppure $x_0 \notin E$.
\end{itemize}
\end{observation}
\hspace{-15pt}A livello di notazione si indica $\mathring{E}$ l'insieme dei punti interni, e con $\delta E$. l'insieme dei punti di frontiera di E.

\begin{example}
Dato E = $\{x \in \mathbb{R}^2 \::\: x = (x_1, x_2) \::\: x_2> 0 \}$. Ci chiediamo quali siano $\mathring{E}$, $\delta E$ e l'insieme dei punti esterni.
\end{example}

\begin{figure}[h!]
    \centering
    \vspace{-10pt}
    \includegraphics[width=8cm]{images/ess-punti-interni.png}
\end{figure}

\vspace{-10pt}
\begin{itemize}
    \item L'insieme dei punti interni $\mathring{E} = E$. Per dimostrare prendo $(x_1, x_2) \in E$ appartenendo ad E ho che $x_2 > 0$ quindi scelgo $r < \frac{x_2}{2} \Longrightarrow B_r(x_0) \subset E \rightarrow E x_0 \in \mathring{E}$ quindi ho dimostrato che $E \subset \mathbb{E}$ e quindi $E = \mathring{E}$. In questo caso tutti i punti di E sono punti interni ad E (il viceversa è sempre vero).
    \item $\delta E = \{(x_1, x_2) \in \mathbb{R}^2 \::\: x_2 = 0\}$ questo è vero perché qualsiasi punto i prenda fra i punti di frontiera andrò ad intersecare sia E che il suo complementare. 
    \item A questo punto l'insieme dei punti esterni $= A = \{(x_1, x_2) \in \mathbb{R}^2 \::\: x_2 < 0\}$. Per verificare di chiediamo se i punti esterni $\in E^c$, voglio dimostrare che se $A \subset$ punti esterni allora $(x_1, x_2) \in A$, $x_2 < 0 \rightarrow$ scelgo $r < \frac{|x_2|}{2}$, quindi $B_r (x_1, x_2)\subset E^c$ ed allora ho dimostrato che $(x_1, x_2)$ è punto esterno. 
\end{itemize}

\begin{example}
Prendo $E = \{(x_1, x_2) \in \mathbb{R}^2 \::\: 1 < x_1^2 + x_2^2 \leq 4\}$. L'esercizio consiste nel calcolare $\mathring{E}$, $\delta E$ e l'insieme dei punti esterni di E.
\end{example}

\subsection{Punto di accumulazione}
\begin{definition}[Punto di accumulazione]
\end{definition}
\begin{wrapfigure}[4]{r}{6cm}
    \vspace{-45pt}
    \centering
    \includegraphics[width=4cm]{images/punti-accumulazione.png}
\end{wrapfigure}
\hspace{-15pt}Dato $E \subset \mathbb{R}^n$ un punto $x \in \mathbb{R}^n$ si dice \textbf{punto di accumulazione} per E se ogni palla di centro x esiste un punto di E diverso da x. Questo è vero se e solo se $\forall r > 0, B_r(x) \cap E \setminus \{x_0\} \neq \O$.

\begin{observation}
Osserviamo che se un punto è interno $\Longrightarrow$ è punto di accumulazione.
\end{observation}

\begin{definition}[Punto isolato]
Se un punto di E non di accumulazione per E allora si dice \textbf{punto isolato}.
\end{definition}

\begin{definition}[Insieme aperto e chiuso]
Possiamo definire dato un insieme E $\subseteq \mathbb{R}^n$ che 
\begin{itemize}
    \item Questo insieme si dice \textbf{aperto} se ogni $x \in E$ è punto interno ad E cioè $E= \mathring{E}$.
    \item Questo insieme si dice \textbf{chiuso} se $E^c$ è aperto.
\end{itemize}
\end{definition}

\begin{example}
Alcuni esempi di insiemi aperti e chiusi.
\begin{itemize}
    \item $E= \{(x_1, x_2) \in \mathbb{R}^2 \::\: x_1 > 0\}$ è aperto.
    \item $E = \{(x_1, x_2) \in \mathbb{R}^2 \::\: x_1 \geq 0\}$ è chiuso.
\end{itemize}
\end{example}

\begin{theorem}
Se E contiene tutto $\delta E \Longleftrightarrow E$ è chiuso. Quindi diciamo che $\delta E \subset E \Longrightarrow E$ chiuso.
\end{theorem}

\begin{example}
Qualche altro esempio.
\begin{itemize}
    \item $E= \{(x_1, x_2) \in \mathbb{R}^2 \::\: x_1^2 + x_2^2 < 1\}$ è aperto. La frontiera è $\gamma E = \{(x_1, x_2) \::\: x_1^2 + x_2^2 = 1\}$.
    \item $E= \{(x_1, x_2) \in \mathbb{R}^2 \::\: 2x_1 + 3x_2 -1 = 0\}$, $2x_1 + 3x_2 -1 = 0$ è una retta quindi $E = \gamma E$ ed è chiuso.
\end{itemize}
\end{example}

\subsection{Proprietà insiemi aperti e chiusi}
Alcune proprietà degli insiemi aperti di $\mathbb{R}^n$:
\begin{itemize}
    \item Sia $\O$ che $\mathbb{R}^n$ sono considerati insiemi aperti.
    \item L'unione (anche numerabile) è aperta. Quindi se prendo $E_1, ..., E_n, ...$ aperti $\cup_{n \in \mathbb{N}}E_n = E$ è aperta.
    \item L'intersezione (finita) di insiemi aperti è un insieme aperto.
\end{itemize}
Alcune proprietà degli insiemi chiusi di $\mathbb{R}^n$:
\begin{itemize}
    \item Sia $\O$ che $\mathbb{R}^n$ sono considerati anche chiusi.
    \item L'unione finita di insiemi chiusi è un insieme chiuso.
    \item L'intersezione (anche numerabile) di un insiemi chiusi è chiusa.
\end{itemize}

\subsection{Insieme limitato}
\begin{definition}[Insieme limitato]
Un insieme $E \subseteq \mathbb{R}^n$ si dice \textbf{limitato} se esiste un palla di centro l'origine che contiene tutto E. Ovvero se $\exists r > 0$ tale che $E \subset B_r(0)$.
\end{definition}

\begin{example}
Ad esempio se prendiamo un quadrato $\subseteq \mathbb{R}^2$ è limitato. Mentre invece una retta $\subseteq \mathbb{R}^2$ non è limitata.
\end{example}

\begin{definition}[Intorno]
Un intorno di raggio r sferico di $\infty$ (o la palla di centro $\infty$ e raggio r) è il complementare della palla chiusa di $\mathbb{R}^n$ con centro l'origine e raggio r.
\end{definition}

\begin{example}
Prendiamo per esempio $\mathbb{R}^2$, si ha $B_r(x) = \{y \in \mathbb{R}^n \::\: d(x,y)<r\}$.\\
Se prendo invece centro $x = \infty$, si ha $B_r(\infty) = \mathbb{R}^n \setminus \overline{B_r(0)}$. $\forall \overline{B_r(0)}$ trovo un intorno sferico di $\infty$ definito come $\mathbb{R}^n \setminus B_r(0)$.
\end{example}

\begin{observation}
Avendo introdotto $\mathbb{R}^n$ e ricordando la definizione di punto di accumulazione cioè $E \subseteq \mathbb{R}^n$, x si dice di accumulazione per E se in ogni palla di centro x esiste un punto E diverso da x. Un insieme non è limitato se e solo se $\infty$ è punto di accumulazione.
\end{observation}
\begin{wrapfigure}[4]{r}{6cm}
    \vspace{-25pt}
    \centering
    \includegraphics[width=5.5cm]{images/oss-punto-accumulo.png}
\end{wrapfigure}
Questo perché $\infty$ è punto di accumulazione $\Longleftrightarrow$ comunque grande io prenda la palla di centro 0 quando vado a prendere il suo complementare continuo a trovare punti che si intersecano con E. Vuol dire che E non può essere chiuso in una palla di centro 0.

\subsection{Oggetti su un piano $\mathbb{R}^2$}
\begin{wrapfigure}[4]{l}{5cm}
    \vspace{-20pt}
    \centering
    \includegraphics[width=4cm]{images/piano_R2.png}
\end{wrapfigure}
Ricordiamo ora che se prendiamo un piano $\mathbb{R}^2$, e due vettori $P_1 = x = (x_1, x_2)$, $P_2 = y = (y_1, y_2)$, la distanza $d(P_1, P_2) = \sqrt{(x_1 - x_2)^2 + (y_1-y_2)^2}$, e questo viene dal teorema di Pitagora dove $(P_1P_2)^2 = (P_1H)^2 + (P_2H)$.

\subsubsection{Retta per 2 punti}
\begin{wrapfigure}[4]{r}{5cm}
    \vspace{-20pt}
    \centering
    \includegraphics[width=4.5cm]{images/retta-2-punti.png}
\end{wrapfigure}
Prendiamo un piano con $P_1 = (x_1, y_1) = v$, $P_2 = (x_2, y_2) = w$. Noi vogliamo scrivere l'equazione di una retta che passa per due punti del piano. Per descrivere questa retta chiamiamo $u = w - v$, la retta è l'insieme dei punti che ottengo partendo da $P_1$ e spostandomi in direzioni di u. Analiticamente:
\[Retta = \{P_{t\in \mathbb{R}} = v + tu\} \text{ dove t è un paramento} = \Bigg\{\begin{pmatrix}x\\y\end{pmatrix} = \begin{pmatrix}x_1\\y_y\end{pmatrix} + t\begin{pmatrix}x_2 - x_1\\y_2 - y_y\end{pmatrix}\Bigg\}\]
Questa forma in cui ho descritto la retta si chiama \textbf{forma parametrica} della retta passante per $P_1$ e $P_2$, perché usiamo un parametrica p. Da qui possiamo scrivere la forma cartesiana:
\[\begin{cases}x = x_1 + t(x2 - x_1)\\ y = y_1 + t(y_2 - y_1)\end{cases} = \begin{cases}t = \frac{x - x_1}{x_2 - x_1}\\t = \frac{y - y_1}{y_2 - y_1}\end{cases} = \frac{x-x_1}{x_2 - x_1} = \frac{y - y_1}{y_2 - y_1}\]
Questa ultima forma senza t si chiama appunto \textbf{forma cartesiana} della retta passante per $P_1$ e $P_2$.
\[x-x_1 = \frac{(y - y_1)(x_2 - x_1)}{y_2 - y_1} = y\frac{(x_2 - x_1)}{y_2 - y_1} - \frac{y_1(x_2 - x_1)}{y_2 - y_1}\]
\[x - y\frac{x_2 - x_1}{y_2 - y_1} + \frac{y_1(x_2 - x_1)}{y_2 - y_1} - x_1 = 0 \rightarrow ax + by + c = 0\:\:  \text{ almeno uno tra a e b deve} \neq 0\]
\begin{observation}
Nella forma $ax + by + c = 0$ due equazioni rappresentato la setta retta $\Longleftrightarrow$ sono l'una multipla dell'altra.
\end{observation}

\subsubsection{Retta perpendicolare a v passante per l'origine}
Quindi siamo sempre nel piano ed abbiamo un vettore $v = (a,b)$, (in questo caso al posto di $x_1, x_2$ uso a,b). Individuo il vettore $w \perp v$ passante per l'origine e descrivo $r: \{P = t\cdot w, t \in \mathbb{R}\}$. \\
\begin{wrapfigure}[8]{r}{6cm}
    \vspace{-20pt}
    \centering
    \includegraphics[width=4.5cm]{images/retta-perpendicolare-in-R3.png}
\end{wrapfigure}
Per trovare w partiamo dal fatto che abbiamo visto che $v \perp w \Longleftrightarrow <v,w> = 0$ se do' a w due componenti $w = (w_1, w_2)$ sto cercando $w_1$ e $w_2$ tali che $w \perp v$, a questo punto posso imporre la condizione $<(a,b), (w_1, w_2)> = 0$ che è $a \cdot w_1 + b \cdot w_2 = 0$, mi accorgo che posso scegliere $w_1 = -b$ e $w_2 = a$ ed in questo modo ho $<v,w> = a \cdot (-b) + b \cdot a = 0$ quindi ricapitolando w è determinato da v ed è $w = (-b,a)$ tale che $w \perp v$ (anche $-w = (a, -b) \perp v$) quindi adesso:
\[r = \{P = t \cdot w\ = \{\begin{pmatrix}x\\t\end{pmatrix}\in \mathbb{R}^2 \::\: \begin{pmatrix}x\\y\end{pmatrix}=t \cdot \begin{pmatrix}-b\\a\end{pmatrix}\}\]
Questa è detta forma parametrica perché abbiamo appunto un parametro t. Da questa forma possiamo ricavare la forma cartesiana.
\[\begin{cases}x = -tb \\ y = ta\end{cases} = \begin{cases}t = \frac{y}{a}\\ y= -\frac{y}{a}\cdot b\end{cases} = ax + by = 0\]
\begin{observation}
Vediamo una serie di osservazioni.
\begin{enumerate}
    \item $ab + by = 0$ è una forma cartesiana della retta passante per l'origine $\Longleftrightarrow c = 0$, $e \perp a$, $v = (a,b)$.
    \item Data una retta $ax + by + c = 0 \rightarrow v = (a,b)$ è $\perp r$.
\end{enumerate}
\end{observation}

\subsubsection{Retta tangente ad un grafico}
Siamo sempre in $\mathbb{R}^2$ e supponiamo di avere una funzione $f: \mathbb{R}\to \mathbb{R}$ continua, derivabile e con derivate continue in tutto $\mathbb{R}$.\\
\begin{wrapfigure}[7]{l}{6cm}
    \vspace{-10pt}
    \centering
    \includegraphics[width=5cm]{images/retta-tangente-in-R3.png}
\end{wrapfigure}
Il suo grafico $graf(f) \subset \mathbb{R}^2$. dato un $x_0 \in \mathbb{R}$ chiamo $y_0 = f(x_0)$ possiamo fare lo sviluppo di Taylor di f in $x_0$ di primo ordine, l'obbiettivo qui è scrivere l'equazione della retta che passa per $x_0, y_0$ tangente a $f$. Lo sviluppo è: $f(x) = f(x_0) + f'(x_0)(x-x_0) + o(x-x_0)$.\\ \\
Sappiamo che $y = f(x)$ ci da il grafico di $f$, quindi ci chiediamo $y = f(x_0) + f'(x_0)(x-x_0)$ che grafico sia. Possiamo vedere che:
\begin{enumerate}
    \item Come prima cosa si tratta di una retta perché $f'(x_0)x + y - f'(x_0)x_0 + f(x_0) = 0$.
    \item Poi osservo che questa retta passa per $(x_0, y_0)$, perché se sostituisco $x=x_0$ ho $y = f(x_0) = y_0$ che sta sul grafico di $f$.
    \item Inoltre grazie allo sviluppo di Taylor posso concludere che la differenza fra i due grafici $f(x) - f(x_0) - f'(x_0)(x-x_0) = o(x-x_0)$.
\end{enumerate}
Date tutte queste considerazioni possono concludere che $y=f(x_0) + f'(x_0)(x-x_0)$ è la retta tangente al grafico $y=f(x)$ in $(x_0, y_0)$. Proviamo ora a riscriverla nella forma $ax + by + c = 0$.
\[y - f(x_0) - f'(x_0)\cdot(x-x_0) = 0 \rightarrow y - f(x_0) f'(x_0)x + f'(x_0)x_0 = 0 \rightarrow -f'(x_0)x + y + f'(x_0)x_0 - f(x_0) = 0\]
Questa è la forma cartesiana della retta r dove $a = -f'(x_0)$ e $b = 1$.\\
$v = (a,b) = (-f'(x_0), 1)$ è perpendicolare a r $\Longrightarrow w = (1, f'(x_0))$ è vettore tangente ad r tale che $<v,w> = 0$. Abbiamo quindi il vettore $w = (1, f'(x_0)$ che è tangente alla retta, posso allora dimostrare che questa retta è tangente. Prima di tutto sappiamo che la retta tange al grafico di $f(x)$ in $(x_0, y_0)$ per definizione è la retta passante per il punto $(x_0, y_0)$ che ha coefficiente angolare $= f'(x_0)$.\\
La retta r di equazione $y - f'(x_0)x + f'(x_0)x_0 - f(x_0) = 0$ che ha come vettore tangente w ha come coefficiente angolare $f'(x_0)$. Ora metto insieme tutte le informazione:
\begin{itemize}
    \item La retta r passa per $(x_0, y_0)$.
    \item La retta r ha come coefficiente angolare $f'(x_0)$
\end{itemize}
Quindi possiamo vedere che r è la retta tangente a $y=f(x)$ nel punto $(x_0, y_0)$. Possiamo riscriverla nella seguente forma cartesiana:
\[y = f(x_0) + f'(x_0)(x-x_0)\]
Mentre la forma parametrica è la seguente.
\[r: \Bigg\{ \begin{pmatrix}x\\y\end{pmatrix} = \begin{pmatrix}x_0\\y_0\end{pmatrix} + t \begin{pmatrix}1\\f'(x_0)\end{pmatrix}\Bigg\}\]
In $\mathbb{R}^2$ una retta è descritta da una sola equazione perché in $\mathbb{R}^2$ ho due gradi di liberta che sono le due variabili mentre su una retta posso solo muovermi sulla retta.

\subsection{Spazio cartesiano in $\mathbb{R}^3$}
Nel caso ci trovassimo in un $\mathbb{R}^3$ abbiamo un vettore scritto come $v = (x_1, y_1, z_1)$. $\mathbb{R}^3$ ha 3 gradi di liberà (x,y,z) mentre la retta ha 1 grado di libertà quindi ci aspettiamo che per descrivere una retta in uno spazio sarà descritta in 2 equazioni perché dai 3 gradi ne devo vincolare 2.

\subsubsection{Retta passante per due punti}
\begin{wrapfigure}[6]{r}{5cm}
    \vspace{-15pt}
    \centering
    \includegraphics[width=4.2cm]{images/retta-passante-2-punti-R3.png}
\end{wrapfigure}
Una retta passante per 2 punti in $\mathbb{R}^3$ possiamo prendere due punti con i vettori associati $P_1 = (x_1, y_1, z_1)$ e $P_2 = (x_2, y_2, z_2) = w$, noi vogliamo scrivere l'equazione della retta che passa per $P_1, P_2$. 
Chiamiamo $u = w-v = (x_2 - x_1, y_2-y_1, z_2-z_1)$, ottengo la forma parametrica di r scrivendo r come:
\[r = \{P = v + tu \::\: t \in \mathbb{R}\} = \Bigg\{ \begin{pmatrix}x\\y\\z\end{pmatrix}\in \mathbb{R}^3 \::\: \begin{pmatrix}x\\y\\z\end{pmatrix} =  \begin{pmatrix}x_1\\y_1\\z_1\end{pmatrix} = t\begin{pmatrix}x_2 - x_1\\y_2 - y_1\\z_2 - z_1\end{pmatrix}\Bigg\}\]
La forma cartesiana di r invece la possiamo scrivere ricavando il paramento:
\[\begin{cases}x=x_1+t(x_2-x_1)\\y=y_1 +t(y_2 - y_1)\\z=z_1 + t(z_2-z_1)\end{cases} = \begin{cases}t = \frac{x-x_1}{x_2 - x_1}\\t = \frac{y-y_1}{y_2 - y_1}\\t=\frac{z-z_1}{z_2-z_1}\end{cases} = \begin{cases}\frac{x-x_1}{x_2-x_1} = \frac{y-y_1}{y_2-y_1}\\\frac{y-y_1}{y_2-y_1} = \frac{z-z_1}{z-z_2}\end{cases}\]

\vspace{20pt}
\subsubsection{Piano in $\mathbb{R}^3$ passante per 3 punti}
\begin{wrapfigure}[5]{l}{5cm}
    \vspace{-10pt}
    \centering
    \includegraphics[width=4.5cm]{images/retta-passante-3-punti-R3.png}
\end{wrapfigure}
La prima cosa è scrivere l'equazione su $\mathbb{R}^3$ passante per 3 punti. Prendo innanzitutto 3 punti $P_1 = (x_1, y_1, z_1), P_2 = (x_2, y_2 z_2), P_3 = (x_3 y_3, z_3)$. Vediamo poi che in un piano ci sono 3 gradi di libertà, mentre nello spazio abbiamo 3 gradi di liberà, quindi ci aspettiamo 1 equazione lineare per definire un piano dello spazio. \\\\
Chiamiamo i vettori per i 3 punti $u = P_1, v = P_1, w = P_3$. Poi definiamo il piano per questi 3 punti che chiamiamo $\pi$, scriviamo poi:\\ $P_2 - P_1 = v-u = (x_1 - x_1, y_2 - y_1, z_2 - z_1)$ \hspace{.3cm} $P_3 - P_1 = w- u = (x_3 - x_1, y_3 - y_1, z_3 - z_1)$\\
La forma parametrica di $\pi$ può essere scritta come:
\[\pi = \Bigg\{\begin{pmatrix}x\\y\\z\end{pmatrix}\in \mathbb{R}^3 \::\: \begin{pmatrix}x\\y\\z\end{pmatrix} = P_1+t(v-u)+s(w-v) \::\: t,s \in \mathbb{R}\Bigg\} = \Bigg\{\begin{pmatrix}x\\y\\z\end{pmatrix} = \begin{pmatrix}x_1\\y_1\\z_1\end{pmatrix} + t\begin{pmatrix}x_2-x_1\\y_2-y_1\\z_2-z_1\end{pmatrix}+s\begin{pmatrix}x_3 - x_1\\y_3 -y_1\\z_3 - z_1\end{pmatrix}\Bigg\}\]
Se passo alla forma cartesiana vedo che l'equazione lineare che ottengo è una sola.
\[\begin{cases}x = x_1 + t(x_2 - x_1) + s(x_3 - x_1)\\ y = y_1 + t(y_2 - x_1) + s(y_3 - y_1)\\z = z_1 + t(z_2 - z_1) + s(z_3 - z_1)\end{cases} = \begin{cases}\frac{x - x_1}{x_2 - x_1} = t + s \frac{x_3 - x_1}{x_2 - x_1}\\ \frac{y-y_1}{y_2-y_1} = t + s\frac{y_3-y_1}{y_2-y_1}\\ \frac{z-z_1}{z_2-z_1} = t + s\frac{(z_3 - z_1)}{z_2 - z_1}\end{cases} = \text{1°eq - 2°eq}  = \]
\[= \begin{cases}\frac{x-x_1}{x_2-x_1} - \frac{y-y_1}{y_2-y_1} = s(\frac{x_3-x_1}{x_2-x_1} - \frac{y_3 - y_1}{y_2-y_1})\\\frac{x-x_1}{x_2-x_1} - \frac{z - z_1}{z_2-z_1} = s(\frac{x_3-x_1}{x_2-x_1} - \frac{z_3 - z_1}{z_2-z_1})\end{cases} = \begin{cases}s = \frac{(\frac{x-x_1}{x_2 - x_1} - \frac{y-y_1}{y_2-y_1})}{(\frac{x_3-x_1}{x_2-x_1} - \frac{y_3 - y_1}{y_2 - y_1})}\\s = \frac{(\frac{x-x_1}{x_2-x_1} - \frac{z -z_1}{z_2 - z_1})}{(\frac{x_3 - x_1}{x_2 - x_1} - \frac{z_3-z_1}{z_2-z_1})}\end{cases} = \frac{\frac{x-x_1}{x_2-x_1} - \frac{y - y_1}{y_2-y_1}}{\frac{x_3-x_1}{x_2-x_1} - \frac{y_3-y_1}{y_2-y_1}} = \frac{\frac{x-x_1}{x_2-x_1} - \frac{z-z_1}{z_2-z_2}}{\frac{x_3 - x_1}{x_2-x_1} - \frac{z_3-z_1}{z_2-z_1}}\]
Questa è 1 equazione linerare che è l'equazione cartesiana di $\pi$ piano passante per $P_1, P_2, P_3$.

\subsection{Disegno di insiemi nel piano}
Il primo obbiettivo e dunque quello di disegnare insiemi di $\mathbb{R}^2$ descritti da equazioni e o disequazioni. Per farlo vediamo alcuni esempi per vedere come visualizzare nel piano.

\begin{example}\label{ess-1}
Prendiamo $x, y \in \mathbb{R}$, e disegnammo $\{(x,y) \in \mathbb{R}^2 \::\: (x+y) \ge1 0\}$. Sappiamo che $x+y \geq 0 \Longleftrightarrow y \geq -x$ e questa è una retta della forma $y + y = 0$ ma noi prendiamo solo i punti sopra visto che usiamo un maggiore uguale. 
\end{example}

\begin{example}\label{ess-2}
Disegnammo $\{(x,y) \in \mathbb{R}^2 \::\: x-y \leq 0 \}$, quindi abbiamo $y = x$ e visto che abbiamo il minore uguale prendiamo tutti i punti sotto.
\end{example}

\begin{example}\label{ess-3}
Ora prendiamo $\{(x,y) \in \mathbb{R}^2 \:\: 3 \leq x+y \leq 5\}$, in questo caso però ci sono due condizioni che devono essere verificate contemporaneamente quindi mettiamo tutto come un sistema:\\
$\begin{cases}x+y \geq 3 \\ x+y \leq 5\end{cases} = \begin{cases}y \geq 3-x & \text{ Si in individua una retta della forma } y=3-x \\ y \leq 5-x & \text{ Si in individua una retta della forma } y=5-x\end{cases}$\\
Disegniamo queste due rette e poi prendiamo i punti compresi fra entrambe. La soluzione è dunque l'intersezione fra i due insiemi.
\end{example}

\begin{figure}[h!]
\centering
\begin{subfigure}{.3\textwidth}
    \centering
    \includegraphics[width=4cm]{images/disegno-R3-ess1.png}
    \caption{Esempio \ref{ess-1}}
\end{subfigure}
\begin{subfigure}{.3\textwidth}
    \centering
    \includegraphics[width=4cm]{images/disegno-R3-ess2.png}
    \caption{Esempio \ref{ess-2}}
\end{subfigure}
\begin{subfigure}{.3\textwidth}
    \centering
    \includegraphics[width=4cm]{images/disegno-R3-ess3.png}
    \caption{Esempio \ref{ess-3}}
\end{subfigure}
\end{figure}

\begin{example}\label{ess-4}
Consideriamo $\{(x,y) \in \mathbb{R}^2 \::\: x \cdot y \geq 0\}$ affinché il prodotto di $x, y$ si maggiore e uguale di zero dobbiamo vedere i due casi mettendoli a sistema:\\
$\begin{cases}x \geq 0 \\ y \geq 0\end{cases}$ e $\begin{cases}x \leq 0 \\ y \leq 0\end{cases}$
Infatti affinché $x \cdot y \geq 0$ x ed y devono essere concordi. Quindi prendiamo le parti del piano che soddisfano la proprietà di avere x ed y concordi.
\end{example}

\begin{example}\label{ess-5}
Dato $\{(x,y) \in \mathbb{R}^2 \::\: x^2 - x \geq 0\}$. Sappiamo che $x^2 - x \geq 0$ è uguale a $x(x-0) \geq 0$ e questa equazione perché sia maggiore o uguale a 0:
$\begin{cases}x \geq 0 \\ x \geq 1\end{cases}$ o $\begin{cases}x \leq 0 \\ x \leq 1\end{cases}$ quindi $x \geq 0$ o $x \leq 0$.
\end{example}

\begin{example}\label{ess-6}
Consideriamo $\{(x,y) \in \mathbb{R}^2 \::\: x^2 + y^2 \geq 8\}$. Se abbiamo un punto $(x,y)$ e consideriamo $x^2 + y^2$ sappiamo che $d((x,y), (0,0)) = |(x+y) - (0,0)| = |(x,y)| = \sqrt{x^2 + y^2}$. Quindi abbiamo che se $x^2 + y?^2 \geq 8 \Longleftrightarrow$ distanza di $(x,y)$ dall'origine maggiore o uguale a $\sqrt{8}$.
\end{example}

\begin{example}\label{ess-7}
In modo analogo all'esempio prima se consideriamo $\{(x,y) \in \mathbb{R}^2 \::\: x^2+y^2 \leq 8\}$ abbiamo tutti i punti interni alla circonferenza con la circonferenza inclusa.
\end{example}

\begin{figure}[h!]
\centering
\begin{subfigure}{.23\textwidth}
    \centering
    \includegraphics[width=3cm]{images/disegno-R3-ess4.png}
    \caption{Esempio \ref{ess-4}}
\end{subfigure}
\begin{subfigure}{.23\textwidth}
    \centering
    \includegraphics[width=3cm]{images/disegno-R3-ess5.png}
    \caption{Esempio \ref{ess-5}}
\end{subfigure}
\begin{subfigure}{.23\textwidth}
    \centering
    \includegraphics[width=3cm]{images/disegno-R3-ess6.png}
    \caption{Esempio \ref{ess-6}}
\end{subfigure}
\begin{subfigure}{.23\textwidth}
    \centering
    \includegraphics[width=3cm]{images/disegno-R3-ess7.png}
    \caption{Esempio \ref{ess-7}}
\end{subfigure}
\end{figure}

\begin{example}\label{ess-8}
Consideriamo l'insieme dei punti $\{(x,y) \in \mathbb{R^2} \::\: y \leq x^2 - x\}$. Sappiamo che $y = x^2 -x$ descrive una parabola con concavità verso l'alto e toccando l'asse x in 0 e 1. Questa equazione individua tutti i punti che stanno sotto la parabola inclusa appunto la parabola.
\end{example}

\begin{example}\label{ess-10}
Se andassi a considerare invece $\{(x,y) \in \mathbb{R^2} \::\: x^2 - x \leq y \leq 0\}$ come nell'esempio di prima si crea un parabola concava verso l'alto ma in questo caso dobbiamo mettere a sistema due condizioni:\\
$\begin{cases}y \geq x^2 - x & \text{ sopra la parabola } \\ y \leq 0 & \text{ sotto l'asse x }\end{cases}$
\end{example}

\begin{example}\label{ess-11}
Se andiamo a considerare $\{(x,y) \in \mathbb{R}^2 \::\:  |x| \leq 5\}$. Abbiamo che, essendo un valore assoluto, $-5 \leq x \leq 5$. Quindi andiamo a considerare che abbiamo la parte compreso fra le rette $x = -5$ e $x = 5$ comprese.\\
Mq in generale $\{|x| \leq A\}$ con A un generico numero, ha come soluzioni:
\begin{itemize}
    \item Insieme vuoto se $A < 0$.
    \item Abbiamo poi $x = 0$ se $A = 0$.
    \item Ed in fine $-A \leq x \leq A$ se $A > 0$
\end{itemize}
Analogamente la disequazione $|x| \geq A$ (con $x \in \mathbb{R}$, i valore assoluto) ha come soluzioni:
\begin{itemize}
    \item Tutto $\mathbb{R}$ se $A \leq 0$.
    \item Invece abbiamo $x \leq -A$ e $x \geq A$ se $A > 0$.
\end{itemize}
\end{example}

\begin{example}\label{ess-12}
Se ci troviamo allora a descrivere $\{(x,y) \in \mathbb{R}^2 \::\: |x| \leq 5, |y| \leq 3\}$. Siccome sono nel caso in cui abbiamo due numeri positivi abbiamo:\\
Primo caso $|x| \leq 5 \Longleftrightarrow -5 \leq x \leq 5$ e nel secondo caso $|y| \leq 3 \Longleftrightarrow -3 \leq y \leq 3$, quindi combinando tutti questi casti risulta un area specifica.
\end{example}

\begin{figure}[h!]
\centering
\begin{subfigure}{.23\textwidth}
    \centering
    \includegraphics[width=3cm]{images/disegno-R3-ess8.png}
    \caption{Esempio \ref{ess-4}}
\end{subfigure}
\begin{subfigure}{.23\textwidth}
    \centering
    \includegraphics[width=3cm]{images/disegno-R3-ess10.png}
    \caption{Esempio \ref{ess-5}}
\end{subfigure}
\begin{subfigure}{.23\textwidth}
    \centering
    \includegraphics[width=3cm]{images/disegno-R3-ess11.png}
    \caption{Esempio \ref{ess-6}}
\end{subfigure}
\begin{subfigure}{.23\textwidth}
    \centering
    \includegraphics[width=3cm]{images/disegno-R3-ess12.png}
    \caption{Esempio \ref{ess-7}}
\end{subfigure}
\end{figure}

\subsection{Curva nel piano e nello spazio}
Noi conosciamo come disegnare parabole, iperbole rette ma volgiamo poter definire anche un oggetto più generico.
\begin{definition}[Curva nel piano]
Una \textbf{curva nel piano} è una funzione $\gamma: I \to \mathbb{R}^2$ con $I \subseteq \mathbb{R}$.
\end{definition}
\hspace{-15pt}Per noi possiamo avere sia $I = (a,b)$ intervallo aperto che $I = [a,b]$ intervallo chiuso. Se $I = [a,b]$ quindi intervallo chiuso allora possiamo definire una curva chiusa.

\begin{definition}[Curva chiusa]
Dato un $I = [a,b]$, una $\gamma: I \to \mathbb{R}^2$ si dice \textbf{curva chiusa} se $\gamma(a) = \gamma(b)$ (per farlo sostituisco gli estremi nella funzione $\gamma$).
\end{definition}

\begin{note}
Notare che la funzione che abbiamo a valori in $\mathbb{R}^2$, $\gamma: I \to \mathbb{R}^2$, indico utilizzando la seguente notazioni, chiamo $t \in I$ e $\gamma(t) = (x(t), y(t)) \in \mathbb{R}^2$.
\end{note}

\begin{example}
$\gamma(t) = (t^2 + 1, 3t -2)$, con $t \in [0,3]$ è una curva che ha come intervallo $[a,]b = [0,3]$ e le componenti sono $x(t) = t^2+1, y(t) = 3t-2$ che sono due funzioni in t. Per ogni punto $t \subseteq [0,3]$ la curva $\gamma$. individua un punto del piano. Per esempio $t = 0 \to \gamma(0) = (1,-2)$ mentre $t = 3 \to \gamma(3) = (10,7)$.
\end{example}
\hspace{-15pt}Con una curva sto rappresentano un punto che si muove nel piano, perché il paramento t lo posso interpretarlo come tempo quindi $\gamma(t)$ mi dice in che punto si trova la mia curva al tempo t.

\begin{definition}[Curva nello spazio]
Una \textbf{curva nello spazio} posso definirla come una funzione $\gamma: I \to \mathbb{R}^3$ con $I \subset \mathbb{R}$
\end{definition}

\begin{example}
Se prendiamo $\gamma(t) = (t^2, \sin{t}, e^t)$, $t \in [-1, 1]$ è un curva nello spazio. Vediamo che questa funzione prende 3 variabili che saranno 3 funzioni, $\gamma(t) = (x(t), y(t), z(t))$, $x: I\to \mathbb{R}, y: I\to \mathbb{R}, <: I\to \mathbb{R}$.
\end{example}

\begin{definition}[Curva semplice]
Una curva si dice \textbf{semplice} se "non ritorna mai su se stessa" (tranne al massimo $\gamma(a) = \gamma(b)$ cioè agli estremi può tornare su se stessa ma non in altri punti).
\end{definition}

\begin{figure}[h!]
\centering
\begin{subfigure}{.3\textwidth}
    \centering
    \includegraphics[width=3.2cm]{images/curva-semplice-non-chiusa.png}
    \caption{Esempio \ref{ess-4}}
\end{subfigure}
\begin{subfigure}{.3\textwidth}
    \centering
    \includegraphics[width=3.2cm]{images/semplice-chiusa.png}
    \caption{Esempio \ref{ess-5}}
\end{subfigure}
\begin{subfigure}{.3\textwidth}
    \centering
    \includegraphics[width=3.2cm]{images/non-chiusa-non-semplice.png}
    \caption{Esempio \ref{ess-6}}
\end{subfigure}
\begin{subfigure}{.5\textwidth}
    \centering
    \vspace{20pt}
    \includegraphics[width=10cm]{images/chiusa-non-semplice.png}
    \caption{Esempio \ref{ess-7}}
\end{subfigure}
\end{figure}

\begin{definition}[Sostegno di una curva]
Si dice \textbf{sostegno} di una curva l'immagine della curva stessa, cioè la traiettoria percorsa dal punto. 
\end{definition}
\hspace{-15pt}Per esempio possiamo avere $\gamma: [a,b] \to \mathbb{R}^2$, l'immagine $\gamma([a,b]) \subseteq \mathbb{R}^2$, ci darà quini un sottoinsieme. Se poi t lo consideriamo come il tempo allora $\gamma(t)$ punto in $\mathbb{R}^2$ dove si trova la curva al punto t.

\begin{definition}[Vettore tangente alla curva]
Sia $\gamma: I \to \mathbb{R}^2$ (o $\mathbb{R}^2$) con $I$ intervallo $\subset \mathbb{R}$. Si dice \textbf{vettore tangente} alla curva il vettore $\gamma'(t) = (x'(t), y'(t))$, assumendo che $x'(t), y'(t)$ esistano.
\end{definition}

\begin{observation}
$x: \mathbb{R} \to \mathbb{R}$, $y: \mathbb{R} \to \mathbb{R}$, abbiamo che $x'(t), y'(t)$ le sappiamo calcolare, quindi possiamo definire il vettore tangente.
\end{observation}

\begin{definition}[Retta tangente ad una curva]
La \textbf{retta tangente} ad una curva in un punto è la retta che passa per quel punto ed ha come direzione il vettore tangente alla curva in quel punto stesso.
\end{definition}

\vspace{50pt}
\begin{example}
Prendiamo la curva $\gamma(t) = (\sin{t}, \cos{t})$con $t \in [0,2\pi]$, $\gamma: [0,2\pi] \to \mathbb{R}^2$ e $\gamma(t) = (x(t), y(t))$ con $x(t) = \cos{t}$ e $y(t) = \sin(t)$.
\end{example}
\begin{wrapfigure}[10]{r}{5cm}
    \vspace{-10pt}
    \centering
    \includegraphics[width=4cm]{images/ess-curva-piano.png}
\end{wrapfigure}
 Se facciamo $x^2(t) + y^2(t) = \cos{t}^2 + \sin{t}^2 = 1 \forall \: t \in [0, 2\pi]$ e quindi $x^2(t) + y^2(t) = 1$. Il sostegno di $\gamma$ è la circonferenza (di $\mathbb{R}^2$) di equazioni $x^2 + y^2 = 1$ perché $\forall \: t \in [0,2\pi] \gamma(t) \in \{(x,y) \in \mathbb{R}^2 \::\: x^2+y^2 = 1\}$.\\
Il vettore tangente invece sarà $\mathring{E}(t) = (\mathring{x}(t), \mathring{y}(t)) = (-\sin{t}, \cos{t})$, se per esempio guardiamo il punto iniziale $\gamma(0) = (\cos{0}, \sin{0}) = (1,0)$ con $\gamma: [0,2\pi]$, mente se prendo il punto finale della traiettoria $\gamma(2\pi) = (\cos{2\pi}, \sin{2\pi}) = (1,0)$ e quindi ho $\gamma(0) = \gamma(2\pi)$ e quindi ho una curva chiusa.\\
La retta tangente nel punto t a $\gamma(t)$ la scrivo come: $r: \gamma(t) + s\mathring{\gamma(t)}$ che è la forma parametrica.

\begin{example}
Prendiamo la curva $\gamma(t) = (t, t^2+t)$ con $t \in [-1,2]$, abbiamo dunque che $x(t) = t$ e $y(t) = t^2 + t$. La curva percorre il tratto della parabole $y= x^2 + x$ però con $x \in [-1,2]$.
\end{example}
\begin{wrapfigure}[6]{r}{5cm}
    \vspace{-15pt}
    \centering
    \includegraphics[width=4cm]{images/ess-curva-piano-2.png}
\end{wrapfigure}
Se prendiamo per esempio $\gamma(-1) = (1, 0)$, mentre $\gamma(2) = (2,6)$.\\
Proviamo a calcolare la retta tangente nel punto corrispondente a $t=1$, $\gamma(1) = (1,2)$ e $\gamma'(t) = (1, 2t+1)$ quindi $\gamma'(1) = (1,3)$, la retta tangente nel punto $\gamma(1) =$ la retta che passa per $\gamma(1) = (1,2)$ con direzione (1,3), al forma parametrica e dunque $(1,2) + s(1,3)$ con $s \in \mathbb{R}$. Vediamo che la curva è semplice ma non è chiusa.

\subsection{Funzioni di più variabili}
Fin ora abbiamo visto funzioni $f: \mathbb{R} \to \mathbb{R}$ o $f: A \to \mathbb{R}$ con $A \subseteq \mathbb{R}$. Mentre nell'analisi in più variabili avremo funzioni del tipo $f: \mathbb{R}^2 \to \mathbb{R}$, $f: \mathbb{R}^2 \to \mathbb{R}$ o più genericamente $f: \mathbb{R}^n \to \mathbb{R}$, queste funzioni posso anche essere $f: \Omega \to \mathbb{R}, \Omega \subseteq \mathbb{R}^n$.

\begin{definition}[Funzione, dominio, codominio]
Una \textbf{funzione} è una terna di oggetti che chiamiamo $\Omega, B, f$ dove $\Omega, B$ sono insieme e dove:
\begin{itemize}
    \item $\Omega$ si dice \textbf{dominio}, con $\Omega \subseteq \mathbb{R}^n$.
    \item $B$ si dice \textbf{codominio}, $B \subseteq \mathbb{R}$.
    \item $f$ è una legge che lega gli elementi di $\Omega$ a quelli di $B$. $f: \Omega\to B$ mette in corrispondenza ogni elemento di $\Omega$ con un solo elemento di B.
\end{itemize}
\end{definition}

\begin{example}
Una funzione in più variabili si può presentare come $f: \mathbb{R}^2 \to \mathbb{R}$, $f(x,y) = x^2 - y + xy$ con $x, y \in \mathbb{R}$ e $(x,y) \in \mathbb{R}^2$ oppure come $f: \mathbb{R}^3 \to \mathbb{R}$ e quindi nella forma $f(x,y,z) = x+y$ con $z \in \mathbb{R}$.
\end{example}

\begin{wrapfigure}[8]{l}{6cm}
\vspace{-10pt}
    \centering
    \includegraphics[width=5cm]{images/ess-funzioni-2-var.png}
\end{wrapfigure}
Ricordiamo che nell'analisi finora se $f: A \to \mathbb{R}$ con $A \subseteq \mathbb{R}$ chiamavamo il grafico di $f = \{(x,y) \in \mathbb{R}^2 \::\: x \in A, y = f(x)\}$ quindi abbiamo una linea nel piano $\mathbb{R}^2$.
Nell'analisi in 2 variabili abbiamo $f: \Omega \to \mathbb{R}$ con $\Omega \subseteq \mathbb{R}^2$ in questo caso per definire il grafico di $f = \{(x,y,z) \in \mathbb{R}^3 \::\: (x,y) \in \Omega, z = f(x,y)\}$. \\\\
Quindi il \textbf{grafico di f} è una superficie nello spazio dove $(x,y) \in \omega$ sa nel piano $xy$ mentre $z = f(x,y)$ è la quota (altezza) del punto della superficie che sta sopra a $(x,y)$.

\begin{observation}
Il dominio di $f: \mathbb{R}^n \to \mathbb{R}$ è uguale al il più grande sottoinsieme di $\mathbb{R}^n$ dove è definita (ha senso scriverla) la funzione.
\end{observation}

\hspace{-15pt}Possiamo anche prendere una funzione $f: \Omega \to \mathbb{R}$ con $\Omega \subseteq \mathbb{R}^n$ il grafico può essere generalizzato come $f= \{(x,y) \in \mathbb{R}^{n+1} \::\: x \in \Omega, y = f(x) \}$ dove $x \in \mathbb{R}^n$ è un vettore mentre $y \in \mathbb{R}$ è un numero.

\subsection{Insiemi di livello}
Per rappresentare meglio funzioni in n variabili si introduce il concetto di insiemi di livello (che nel caso di $n=2$ si dicono linee di livello). 
\begin{definition}
Sia $f: \mathbb{R}^2 \to \mathbb{R}$, e dato $\lambda \in \mathbb{R}$ (pensato come quota) \textbf{l'insieme di livello} corrispondente a $\lambda$ è il seguente insieme
$\{(x,y) \in \mathbb{R}^2 \::\: f(x,y) = \lambda\}$, è l'insieme quindi dei punti in $\mathbb{R}^2$ tale che la quota di f in questi punti è uguale a $\lambda$
\end{definition}
\hspace{-15pt}Possiamo vedere che $\{(x,y) \in \mathbb{R}^2 \::\: f(x,y) = \lambda\}$ (che viene chiamato anche insieme di livello $\lambda$ per $f$) è un sottoinsieme dello spazio di partenza tale che in questo sottoinsieme la funzione vale sempre $\lambda$.

\begin{example}\label{ess-insiemi-livello-1}
Prendiamo un $f(x,y) = x^2 + y^2$ con $f: \mathbb{R}^2 \to \mathbb{R}$, quindi con $(x,y) \to x^2 + y^2$ (quadrato della distanza di $(x,y)$ dall'origine). Gli insiemi di livello per questa funzione sono $\{(x,y) \in \mathbb{R}^2 \::\: x^2 + y^2 = \lambda\}$ per trovare questo insieme devo intersecare il grafico di $f$ con il pinao $z = \lambda$ e poi proietto sul piano $xy$.
\begin{itemize}
    \item Se $\lambda < 0 \to \O$.
    \item Se $\lambda = 0 \to (0,0)$.
    \item Se $\lambda > 0 \to$ trovo la circonferenza con centro in $(0,0)$ e raggio $\sqrt{\lambda}$.
\end{itemize}
Se scegliesti $\lambda = 1$ allora $\{(x,y) \in \mathbb{R}^2 \::\: x^2 + y^2 = 1\}$ avrei la circonferenza di raggio 1, mentre se scelgo $\lambda = 2$ allora $\{(x,y) \in \mathbb{R}^2 \:\: x^2+y^2 = 2\}$ avrei la circonferenza di raggio 2 in entrambi i casi con centro in $(0,0)$, questi sono quindi sottoinsiemi di $\mathbb{R}^2$. \\
L'insieme di livello $\lambda = \{$ punti di $\mathbb{R^2}$ tali che in questi punti la funzione vale $\lambda \}$.
\end{example}

\begin{example}\label{ess-insiemi-livello-2}
Prendiamo $f(x,y) = x\cdot y$. dato $\lambda \in \mathbb{R}$, l'insieme di livello $\lambda$ abbiamo che gli insiemi di livello sono $\{(x,y) \in \mathbb{R}^2 \::\: xy= \lambda\} \subseteq \mathbb{R}^2$, vediamo dunque di che insieme stiamo parlando:
\begin{itemize}
    \item Se $\lambda = 0 \::\: xy = 0$ questi punti sono quelli che stanno sugli assi ($x=0$ o $y=0$) allora $\{(x,y) \in \mathbb{R}^2 \::\: xy= \lambda\} = $asse y $\cup$ asse x.
    \item Se $\lambda > 0$ vuol dire che stiamo guardando $\{(x,y) \in \mathbb{R}^2 \::\: xy= \lambda\}$ con $y = \frac{\lambda}{x}$ e con $\lambda$ numero positivo, queste curve sono iperbole equilatere che sono nel 1° e nel 3° quadrante.
    \item Se $\lambda < 0$ abbiamo sempre $\{(x,y) \in \mathbb{R}^2 \::\: xy= \lambda\}$ ma questa volta con $y = \frac{\lambda}{x}$ con $\lambda < 0$ quindi saranno iperbole nel 2° e 3° quadrante.
\end{itemize}
\end{example}

\begin{example}\label{ess-insiemi-livello-3}
Prendiamo $f(x,y) = |x+y|$. Gli insiemi di livello di $\lambda$ con $\lambda \in \mathbb{R}$ in questo caso  saranno $\{(x,y) \in \mathbb{R}^2 \::\: f(x,y) = \lambda\} \subseteq \mathbb{R}^2 = \{(x,y) \in \mathbb{R}^2 \::\: |x+y| = \lambda\}$.
\begin{itemize}
    \item Se $\lambda < 0$ allora il valore assoluto non può essere mai uguale a $\lambda$ perché il valore assoluto e sempre positivo quindi $\to \O$.
    \item Se $\lambda = 0$ sto guardando $\{(x,y) \in \mathbb{R}^2 \::\: |x+y| = 0\}$ e questo è vero se $x+y = 0 \Longleftrightarrow y = -x$ quindi abbiamo la bisettrice del 2° e 4° quadrante.
    \item Se invece prendo $\lambda > 0$ sto cercando $\{(x,y) \in \mathbb{R}^2 \::\: |x+y| = \lambda\}$ ed in questo caso ci sono due possibilità, se $x+y > 0 \to y = \lambda - x$ mentre se $x+y < 0 \to -y = -\lambda-x$, quindi $\{(x,y) \in \mathbb{R}^2 \::\: |x+y| = \lambda\}$ con $\lambda > 0$ sarà $\{(x,y) \in \mathbb{R}^2 \::\: y = \lambda -x, y > -x \} \cup \{(x,y) \in \mathbb{R}^2 \::\: y = -x-\lambda, y < -x\}$.\\
    Dunque se per esempio $\lambda = 1 \to y = 1-x$ e $y = -1-x$ quindi ho l'unione di 2 rette.
\end{itemize}
\end{example}

\begin{figure}[h!]
\centering
\begin{subfigure}{.3\textwidth}
    \centering
    \includegraphics[width=5cm]{images/ess-insiemi-livello-1.png}
    \caption{Esempio \ref{ess-insiemi-livello-1}}
\end{subfigure}
\begin{subfigure}{.3\textwidth}
    \centering
    \includegraphics[width=3.7cm]{images/ess-insiemi-livello-2.png}
    \caption{Esempio \ref{ess-insiemi-livello-2}}
\end{subfigure}
\begin{subfigure}{.3\textwidth}
    \centering
    \includegraphics[width=3.3cm]{images/ess-insiemi-livello-3.png}
    \caption{Esempio \ref{ess-insiemi-livello-3}}
\end{subfigure}
\end{figure}
\newpage
\section{Limiti funzioni in più variabili}
Restringiamoci nel caso di funzioni in 2 variabili; data una funzione $f: \mathbb{R}^2 \to \mathbb{R}$, una variabile $(x,y) \in \mathbb{R^2}$ e vogliamo dare un senso a $\lim\limits_{(x,y) \to (x_0, y_0)}f(x)$ con $(x_0, y_0) \in \mathbb{R^2}$. Ci aspettiamo 4 possibilità:
\begin{enumerate}
    \item Il limite esiste ed è finito (definizione \ref{def-1}).
    \item Il limite esiste ed è uguale a $+\infty$ (definizione \ref{def-2}).
    \item Il limite esiste ed è uguale a $-\infty$ (definizione \ref{def-3}).
    \item Il limite non esiste (quindi ne il punto 1 ne il 2 ne il 3 valgono).
\end{enumerate}

\begin{definition}\label{def-1}
Si dice che $\lim\limits_{(x,y) \to (x_0, y_0)} f(x,y) = +\infty$ con una $f: \mathbb{R}^2 \to \mathbb{R}$ e con $(x_0, y_0) \in \mathbb{R^2}$ se $\forall \: M \in \mathbb{R}$ (anche molto grande) $\exists \: \delta > 0$ tale che $f(x,y) \geq M \:\forall \: (x,y) \in B_{\delta}((x_0, y_0)) \setminus \{(x_0, y_0)\}$.
\end{definition}

\begin{definition}\label{def-2}
Si dice che $\lim\limits_{(x,y) \to (x_0, y_0)} f(x,y) = -\infty$ con una $f: \mathbb{R}^2 \to \mathbb{R}$ e con $(x_0, y_0) \in \mathbb{R^2}$ se $\forall \: M \in \mathbb{R}$ (anche molto negativo) $\exists \: \delta > 0$ tale che $f(x,y) \leq M \:\forall \: (x,y) \in B_{\delta}((x_0, y_0)) \setminus \{(x_0, y_0)\}$.
\end{definition}

\begin{figure}[h!]
\centering
\begin{subfigure}{.45\textwidth}
    \centering
    \includegraphics[width=4cm]{images/limiti-piu-variabili-+inf.png}
    \caption{Definizione \ref{def-1}}
\end{subfigure}
\begin{subfigure}{.45\textwidth}
    \centering
    \includegraphics[width=4cm]{images/limiti-piu-variabili--inf.png}
    \caption{Definizione \ref{def-2}}
\end{subfigure}
\end{figure}

\begin{definition}\label{def-3}
Si dice che $\lim\limits_{(x,y) \to (x_0, y_0)} f(x,y) = l \in \mathbb{R}$ con una $f: \mathbb{R}^2 \to \mathbb{R}$ e con $(x_0, y_0) \in \mathbb{R^2}$ se $\forall \: \epsilon > 0$ (anche molto piccolo) $\exists \: \delta > 0$ tale che $|f(x,y) - l| \leq \epsilon \:\forall \: (x,y) \in B_{\delta}((x_0, y_0)) \setminus \{(x_0, y_0)\}$.
\end{definition}

\begin{definition}[Funzione continua]
Una funzione $f: \mathbb{R}^n \to \mathbb{R}$ si dice \textbf{continua} in un punto $x_0 \in \mathbb{R}^n$ (che in questo caso indica un vettore), se $\lim\limits_{x\to x_0}f(x) = f(x_0)$, ($x \in \mathbb{R}^2$ è un vettore).
\end{definition}

\begin{definition}
Una funzione $f: \mathbb{R}^n \to \mathbb{R}$ è continua in tutto $\mathbb{R}^n$ se è continua $\forall \: x \in \mathbb{R}^2$.
\end{definition}
\hspace{-15pt}In generale la regola è che ogni funzione ottenuta a partire delle funzioni elementari usando operazioni algebriche e o usando composizioni è continua dove non presenta problemi di "esistenza" (per esempio il logaritmo $\leq 0$).

\begin{observation}
Gli strumenti standard del calcolo dei limiti che studiati nell'analisi fino ad ora, continuano ad essere validi per funzioni $f: \mathbb{R}^n \to \mathbb{R}$, unicità limite, la somma il prodotto ed il quoziente del limite, il teorema dei carabinieri, la permanenza del segno, teorema del confronto ecc.
\end{observation}

\hspace{-15pt}Per fare un confronto fra analisi in una variabile e quella in più variabili facciamo uno schema:
\begin{table}[h!]
    \centering
    \setlength{\tabcolsep}{7pt}
    \renewcommand{\arraystretch}{1.7}
    \begin{tabular}{|c|c|}
    \hline
    Analisi in una variabile & Analisi in più variabili \\\hline
    \makecell{    $f: \mathbb{R}\to \mathbb{R}$, $\lim\limits_{x\to x_0}f(x) = l \in \mathbb{R}$\\
    se $\forall \epsilon > 0 \:\exists\: \gamma > 0$ tale che $\forall \:x \::\: |x - x_0|< \gamma$,\\
    $x \neq 0$ vale che $|f(x) - l| \leq \epsilon$} & \makecell{$f: \mathbb{R}^n \to \mathbb{R}, x \in \mathbb{R}^n, x_0 \in \mathbb{R}^n, \lim\limits_{x\to x_0}f(x) = l$\\ se $\forall \: \epsilon > 0 \: \exists \: \gamma > 0$ tale che $\forall \: x \::\: ||x-x_0|| < \gamma,$\\
    $x \neq x_0$ vale $|f(x) - l| \leq \epsilon$}\\\hline
    \end{tabular}
\end{table}

Possiamo vedere che se confrontiamo queste due definizioni, formalmente, sono la stessa definizione, cambia che in una variabile abbiano un valore assoluto mentre in più variabili abbiamo una norma.

\begin{observation}
Anche se la struttura della definizione di limite rimane praticamente inalterata a meno di sostituire il valore assoluto con la norma, il calcolo dei limiti per funzioni di più variabili è più complicato. Questo perché in $\mathbb{R}, x$ tende ad un punto $x_0$ lungo una direzione data, mentre in $\mathbb{R}^n, x\in \mathbb{R}^n$ può tendere ad un punto $x_0 \in \mathbb{R}^n$ con n gradi di libertà.
\end{observation}

\subsection{Dimostrazione della non esistenza di un limite}
\begin{example}
Vediamo alcuni esempi nel caso di $f: \mathbb{R}^2 \to \mathbb{R}$.\\
Consideriamo ad esempio $\lim\limits_{(x,y) \to (0,0)}\frac{xy}{x^2 + y^2}$, quindi la nostra funzione è $f(x,y) = \frac{xy}{x^2 + y^2}$.\\
Esistono "diversi modi" di avvicinarsi all'origine (0,0). Però se $\exists\:\lim\limits_{(x,y) \to (0,0)}\frac{xy}{x^2 + y^2} = l \in \mathbb{R}$ allora il limite (per sua unicità) non deve dipendere dalla direzione di avvicinamento a (0,0).\\
Prova a dire che $\nexists \:\lim\limits_{(x,y) \to (0,0)}\frac{xy}{x^2 + y^2}$ (non esiste), a seconda della direzione di avvicinamento il limite cambia, ma questo contraddice l'unicità del limite. Vediamo come possiamo avvicinarsi:
\begin{itemize}
    \item Avvicinarsi lungo l'asse x: $y=0$, quindi sto guardando la funzione $f(x,0)$ quindi la funzione $f$ solo sui punti dell'asse x. Quindi se mi restringo all'asse x ho $\lim\limits_{x\to 0}f(x,0) = 0$.
    \item Lungo l'asse y: $x=0$, quindi ho $f(0,y) = 0$ che è una funzione dove guardo solo i punti dell'asse y allora $\lim\limits_{y \to 0}f(0,y) = 0$.
    \item Lungo bisettrice $y=x$, quindi sto guardando $f(x,x)$ cioè sto guardando la funzione solo sui punti tale che $y=x$, quindi $f(x,x) = \frac{x^2}{2x^2} = \frac{1}{2}$ e quindi è costante e quindi $\lim\limits_{x\to 0}f(x,x) = \frac{1}{2}$.\\
    Questa cosa si può fare anche con la bisettrice $x = y$ che da l'ho stesso risultato ma anche con $x = -y$ e $y = -x$.
\end{itemize}
\end{example}
\begin{wrapfigure}[9]{r}{6cm}
\vspace{-20pt}
    \centering
    \includegraphics[width=5cm]{images/ess-lim-2var-asseX.png}
\end{wrapfigure}
Ho quindi trovato almeno 2 direzioni di avvicinamento (asse x e la bisettrice $y=x$) che mi danno 2 limiti diversi (0 e $\frac{1}{2}$) ma visto che il limite se esiste sarebbe unico il limite non esiste.\\
In ogni palla di centro (0,0) trovo punti in cui $f$ vale 0 e punti in cui $f$ vale $\frac{1}{2}$ allora il limite non esiste.
\hspace{-15pt}In generale possiamo dimostrare che un limite non esiste andando a trovare due direzioni di avvicinamento in cui il limite è diverso. In $\mathbb{R}^n$ capita molto spesso che i limiti non esistano, molto più spesso del caso di una singola variabile.

\begin{example}
Prendiamo ora $\lim\limits_{(x,y)\to (0,0)}f(x,y)$ con $f(x,y) = \frac{x}{||(x,y)||}$,
\end{example}
\begin{wrapfigure}[4]{r}{5cm}
\vspace{-35pt}
    \centering
    \includegraphics[width=4cm]{images/ess-lim-2var-2.png}
\end{wrapfigure}
che possiamo scrivere anche come $f(x,y) = \frac{x}{x^2 + y^2}$, dobbiamo anche in questo caso bisogna dimostrare che il limite non esiste. Per dimostrare che non esiste cerco due direzioni di avvicinamento all'origine (0,0) con due limiti diversi.
\begin{itemize}
    \item Lungo l'asse x: $y = 0$ quindi vuol dire che sto guardando la funzione $f(x,0) = \frac{x}{x^2} = \frac{1}{x}$ e quindi il $\lim\limits_{x\to 0}f(x,0) = +\infty$.
    \item Lungo l'asse y: $x=0$ vuol dire che ho $f(0,y) = \frac{0}{y^2} = 0$ e quindi $\lim\limits_{y \to 0}f(0,y) = 0$.
\end{itemize}
Bastano dunque queste due direzioni con il limite diverso per concludere che il limite non esiste.


\begin{note}
Attenzione: se provo diverse direzioni e mi viene sempre lo stesso limite questo non basta per concludere che il limite esiste perché potrebbe esserci una direzione che io non ho ancora esplorato che mi da limite diverso.
\end{note}

\begin{example}
Vediamo un esempio della casistica descritta nella nota. $f(x,y) = \frac{x^2 \cdot y}{x^4 + y^2}$ e voglio guardare $\lim\limits_{(x,y) \to (0,0)}f(x,y)$. Proviamo allora a guardare delle direzioni:
\begin{enumerate}
    \item Lungo l'asse x: $y = 0$ stiamo guardando la funzione $f(x,0) = 0$ quindi $\lim\limits_{x\to 0}f(x,0) = 0$.
    \item Lungo l'asse y: $x = 0$ quindi consideriamo al funzione $f(0,y) = 0$ quindi $\lim\limits_{y \to 0}f(0,y) = 0$.
    \item Lungo bisettrice $y= x$: in questo caso $f(x,x) = \frac{x^2 - x}{x^4 + x^2} = \frac{x^2}{x^4 + x^2} = \frac{x}{x^2 + 1}$ e quindi $\lim\limits_{x\to 0}f(x,x) = 0$.
    \item Lungo le rette del tipo $y = mx$: stiamo guardando allora $f(x,mx) = \frac{x^2 \cdot mx}{x^4 + m^2x^2} = \frac{mx}{x^2 + m^2}$, in questo caso qualunque sia $m \in \mathbb{R}^2$ posso concludere che $\lim\limits_{x\to 0}\frac{mx}{x^2 + m^2} = 0$.
\end{enumerate}
\end{example}
\begin{wrapfigure}[8]{l}{5cm}
    \vspace{-10pt}
    \centering
    \includegraphics[width=4.3cm]{images/ess-lim-2var-3.png}
\end{wrapfigure}
Nonostante abbiamo visto che lungo tutte le direzioni di avvicinamento provate il limite è 0 questo non basta a concludere che il limite esiste perché ne ho provate solo alcune. Proviamo infatti adesso ad avvicinarsi all'origine lungo la parabola $y = x^2$. Se restringiamo $f$ a questa parabola stiamo guardando $f(x, x^2) = \frac{x^2 \cdot x^2}{x^4 + x^4} = \frac{x^4}{2x^4}= \frac{1}{2}$ ma quindi $\lim\limits_{x\to 0}f(x, x^2) = \frac{1}{2} \neq 0$ e quindi abbiamo trovato un caso in cui il limite è diverso e quindi posso concludere per l'unicità del limite che non esiste.


\subsection{Calcolo limiti con teoremi}
Proviamo a studiare $\lim\limits_{(x,y)\to (0,0)}\frac{x^2 y^6}{x^2 + y^2}$ con $f: \mathbb{R}^2 \to \mathbb{R}$ come prima cosa proviamo a fare il limite classico sostituendo ma che fa si che venga $\frac{0}{0}$ che è una forma indeterminata.\\
Osservo che $\frac{x^2 y^6}{x^2 + y^2}$ è una funzione sempre positiva inoltre posso scrivere $\frac{x^2 y^6}{x^2 + y^2} = y^6 \cdot \frac{x^2}{x^2 + y^2}$ possiamo vedere che il termine $\frac{x^2}{x^2 + y^2} \leq 1$ perché denominatore $\geq$ numeratore, quindi abbiamo che $y^6 \cdot \frac{x^2}{x^2 + y^2} \leq y^6$ quindi risulta che $0 \leq f(x,y) \leq y^6$ possiamo usare il teorema dei carabinieri dove $0 \to 0$ e $y^6 \to 0$ quindi concludiamo che $\lim\limits_{(x,y)\to (0,0)}f(x,y) = 0$.

\subsection{Calcolo limiti con cambio di variabile}
Calcoliamo $\lim\limits_{(x,y) \to (0,0)}\frac{\sin(x^2 + y^2)}{(x^2 + y^2)} = \frac{0}{0}$ che è una forma indeterminata. Possiamo in questo caso usare il cambio di variabile, pongo quindi $t = x^2 + y^2$, quindi ho che quando $(x,y)\to (0,0)$ ho che $t \to 0$ quindi riscrivo il limite come $\lim\limits_{t\to 0}\frac{\sin{t}}{t} = 1 \Longrightarrow \lim\limits_{(x,y)\to (0,0)} \frac{\sin(x^2+y^2)}{x^2+y^2} = 1$.

\subsection{Regola valore assoluto}
Prendiamo ora il limite $\lim\limits_{(x,y)\to (0,0)}\frac{x^2y}{x^2 + y^2}$, in questo ci verrebbe da usare come visto prima il teorema dei carabinieri però non possiamo perché abbiamo a moltiplicare un $y$ che quando cambia segno la funzione cambia segno, dobbiamo quindi usare la regola del valore assoluto.
\begin{proposition}[Regola del valore assoluto]
Data una funzione $f: \mathbb{R}^2 \to \mathbb{R}$ se $|f(x,y)| \to 0$ allora anche $f(x,y) \to 0$
\end{proposition}
\begin{demostration}
Questo perché $-|f(x,y)| \leq f(x,y) \geq |f(x,y)|$ quindi è come dire che $-|a| \leq a \leq |a|$ e quindi se il minorane ed il maggiorante tendono a 0 per il teorema dei carabinieri anche $f(x,y)$ tende a 0.
\end{demostration}
\vspace{-5pt}
\hspace{-15pt}Applichiamolo quindi al nostro caso con $f(x,y) = \frac{x^2 \cdot y}{x^2 + y^2}$ e quindi $|f(x,y)| = \bigg| \frac{x^2 \cdot y}{x^2 + y^2} \bigg| = |y| \cdot \frac{x^2}{x^2 + y^2}$ perché $\frac{x^2}{x^2 + y^2} \geq 0$ e quindi viene influenzata solo da $y$. quindi abbiamo che $0 \leq |f(x,y)| = |y| \cdot \frac{x^2}{x^2+y^2} \leq |y|$ quindi $|y| \to 0$ e per il teorema dei carabinieri $|f(x,y)| \to 0$ e per la regola del valore assoluto $f(x,y) \to 0$.

\subsection{Metodo delle coordinate polari}
Questo è un metodo che facilità il calcolo dei limiti in $\mathbb{R}^2$. \\
Abbiamo $(x,y) \in \mathbb{R}^2$ che sono dette coordinate cartesiane ma esistono anche quelle dette coordinate polari definite come $(\rho,\Theta)$ con $\rho \in \mathbb{R}, \Theta \in \mathbb{R}$. Le coordinatore polari e le cartesiane sono legate nel seguente modo: $x = \rho \cos{\Theta}, y = \rho \sin{\Theta}$\\
Le coordinate polari hanno il seguente vantaggio: la condizione $(x,y)\to 0$ in termini di coordinate polari diventa $\rho \to 0$. Il metodo prevede di esprimete la funzioni in termini di coordinate polari, quindi ne nostro caso $f(x,y) = \frac{x^2y}{x^2 + y^2} = \frac{\rho^2\cos{\Theta}^2 \cdot \rho\sin{\Theta}}{\rho^2 \cos{\Theta}^2 + \rho^2\sin{\Theta}^2} = \frac{\rho^3 \cos{\Theta}^2 \cdot \sin{\Theta}}{\rho^2} = \rho \cos{\Theta}^2 \cdot \sin{\Theta}$ ho quindi espresso la funzione con $\rho$ e $\Theta$.\\
Vediamo ora che $\lim\limits_{(x,y)\to (0,0)}\frac{x^2y}{x^2 + y^2} = \lim\limits_{\rho \to 0}f(\rho, \Theta) = \rho \cos{\Theta}^2 \cdot \sin{\Theta}$ quindi devo fare un limite in una variabile e posso quindi procedere facendo come fatto nell'analisi ad una variabile. Vediamo quindi che $-\rho \leq \rho \cos{\Theta}^2 \cdot \sin{\Theta} \leq \rho$ visto che $\cos{\Theta}^2 \leq 1$ e $-1 \leq \sin{\Theta} \leq 1$, usano il teorema dei carabinieri abbiamo che $-\rho \to 0$, $\rho \to 0$ e quindi $\rho \cos{\Theta}^2 \cdot \sin{\Theta} \to 0$.

\begin{example}
Facciamo un altro esempio di questo metodo prendendo $f(x,y) = \frac{xy}{x^2 + y^2}$ e usandolo per dimostrare che per $(x,y)\to (0,0)$ il limite non esiste. Ricordiamo che $x = \rho\cos{\Theta}$ e $y = \rho\sin{\Theta}$ e quindi $f(\rho, \Theta) = \frac{\rho\cos{\Theta} \cdot \rho\sin{\Theta}}{\rho^2\cos{\Theta}^2 + \sin{\Theta}^2} = \frac{\rho^2\cos{\Theta} \cdot \sin{\Theta}}{\rho^2} = \cos{\Theta}\sin{\Theta}$. Quindi $\lim\limits_{(x,y)\to (0,0)}f(x,y) = \lim\limits_{\rho\to 0^+}f(\rho, \Theta) = \lim\limits_{\rho\to 0^+}\cos{\Theta}\sin{\Theta} = \cos{\Theta}\sin{\Theta}$ che quindi dipende da $\Theta$ e quindi non è un numero e quindi non esiste un limite finito in $\mathbb{R}$.
\end{example}

\begin{example}
Sia data una funzione $f: \mathbb{R}^2 \to \mathbb{R}$ definita come $f(x,y) = \begin{cases}\frac{x^4 + y^3}{x^2 + y^2} & (x,y) \neq (0,0)\\ a & (x,y) = (0,0)\end{cases}$, questo esercizio ci chiede di trovare $a \in \mathbb{R}$ tale che $f$ sia continua in $(0,0)$. Questo vuol dire che la soluzione corrisponderà a cercare $a \in \mathbb{R}$ (vedere innanzitutto se esiste) tale che $\lim\limits_{(x,y)\to (0,0)}f(x,y) = a$.\\
Quindi calcoliamo il $\lim\limits_{(x,y)\to (0,0)}\frac{x^4 + y^3}{x^2 + y^2} = \frac{0}{0}$ forma indeterminata. Questo è un caso in cui è possibile utilizzare le coordinate polari, ricordiamo che $x = \rho \cos{\Theta}, y = \rho \sin{\Theta}$, e che $\lim\limits_{(x,y)\to (0,0)} = \lim\limits_{\rho\to 0}$. \\
Scriviamo quindi la funzione con le coordinate polari: $f(\rho,\Theta) = \frac{\rho^4\cos{\Theta}^4 + \rho^3\sin{\Theta}^3}{\rho^2\cos{\Theta}^2 + \rho^2\sin{\Theta}^2} = \frac{\rho^4\cos{\Theta}^4 + \rho^3\sin{\Theta}^3}{\rho^2} = \frac{\rho^2(\rho^2\cos{\Theta}^4 + \rho\sin{\Theta}^3)}{\rho^2} = \rho^2\cos{\Theta}^4 + \rho\sin{\Theta}^3$.\\
Ora voglio calcolare $\lim\limits_{\rho \to 0}f(\rho, \Theta) = \rho^2\cos{\Theta}^4 + \rho\sin{\Theta}^3$, posso vedere che $\cos{\Theta}^4 \leq 1 \:\forall \:\Theta$ e $\sin{\Theta}^3 \leq 1 \:\forall \:\Theta$ sono due funzioni limitate per due funzioni $\rho^2, \rho \to 0$. Posso quindi affermare che $\exists\: \lim\limits_{(x,y)\to (0,0)}f(x,y) = 0$, $f$ è continua in $(0,0) \Longleftrightarrow a = 0$ abbiamo quindi concluso.
\end{example}

\subsection{Limiti che vanno a $\infty$}
Finora abbiamo visto limiti $\lim\limits_{(x,y)\to(x_0,y_0)}f(x,y)$ con $(x_0,y_0)\in \mathbb{R}^2$ fissato, quindi con un valore finito ed nel caso in cui la funzione sia definita in tutto $\mathbb{R}^2$.
\begin{observation}
Nel caso fosse definita in un sottoinsieme $f: \Omega \to \mathbb{R}$ con $\Omega \subset \mathbb{R}^2$ stessa definizione ma $(x_0,y_0)$ deve essere un punto di accumulazione per l'insieme $\Omega$.
\end{observation}
\hspace{-15pt}Ora però ci chiediamo come si estende la nozione di limite quando $\lim\limits_{(x,y)\to \infty}f(x,y) = l$. Innanzitutto dobbiamo chiederci cosa voglia dire $(x,y)\to \infty$; ricordiamo che un intorno di raggio r di $\infty$ è il complementare di $B_r((0,0))$, quindi $(x,y)\to \infty$ se la d($(x,y),0$)$\to +\infty$, che posso esprimerlo come $x^2 + y^2 \to +\infty$ oppure $\rho \to +\infty$.

\begin{definition}[Limite finito che tende a $\infty$]
Si dice che $\lim\limits_{(x,y)\to \infty}f(x,y) = l \in \mathbb{R}$ se $\forall \: \epsilon > 0$ esiste un intorno di $\infty$ tale che se $(x,y)$ appartiene all'intorno allora $|f(x) - l|\leq \epsilon$. In maniera più formale si può dire che esiste se $\forall \: \epsilon > 0 \:\exists\: \delta > 0$ tale che se $(x,y) \notin B_{\gamma}(0,0)$ allora $|f(x,y) - l| \leq 0$.
\end{definition}

\begin{definition}[Limite infinito che tende a $\infty$]
Si dice che $\lim\limits_{(x,y)\to \infty}f(x,y) = +\infty \in \mathbb{R}$ se $\forall \: M > 0$ esiste se $\exists\: \delta > 0$ tale che se $\forall \: (x,y) \notin B_{\gamma}(0,0)$ vale $f(x,y) \geq M$. Lo stesso vale per $-\infty$ ma considerando $f(x,y) \leq M$.
\end{definition}
\hspace{-15pt}Anche nel caso di $\lim\limits_{(x,y)\to \infty}f(x,y)$ valgono le seguenti cose:
\begin{itemize}
    \item Se esistono due direzioni con limite diverso allora non esiste il limite.
    \item Le coordinate polari possono semplificare il calcolo.
\end{itemize}

\begin{example}
Consideriamo $\lim\limits_{(x,y)\to +\infty}\frac{y^6}{1 + x^2 + y^2}$. Fare questo limite equivale a fare $\lim\limits_{x^2 + y^2\to +\infty} \Longleftrightarrow \lim\limits_{\rho\to +\infty} \Longleftrightarrow \lim\limits_{d((x,y),0)\to +\infty}$. Sappiamo che possiamo tendere all'infinito in diversi modi, quindi vado ora a scegliere diversi direzioni di avvicinamento:
\begin{itemize}
    \item Mi avvicino dall'asse x, quindi $y=0$ e quindi $(x,y)\to \infty$ con $y=0$ corrisponde a $x^2 \to +\infty$ il che equivale a $d((x,0),(0,0))\to +\infty$. Se quindi $y=0$ la nostra funzione $f(x,y) = f(x,0) = \frac{0}{1+x^2} \to 0$.
    \item Se scegliamo come direzione di avvicinamento ad $\infty$ l'asse y quindi $x=0$ allora $\lim\limits_{(x,y)\to +\infty}f(x,y) = \lim\limits_{y^2 \to \infty}\frac{y^6}{1 + y^2}$, ma questo limite è uguale a $\lim\limits_{y^2 \to \infty}\frac{y^6}{1 + y^2} = +\infty$.
\end{itemize}
Quindi lungo l'asse x viene 0 mentre lungo l'asse y torna $\infty$, quindi ho trovato 2 direzioni con due diversi valori del limite e quindi risulta che il limite non esiste.
\end{example}

\begin{example}
Prendiamo $\lim\limits_{(x,y)\to \infty}xye^{-(x^2 + y^2)}$. Come sempre vale che il limite vale per $\lim\limits_{x^2 + y^2\to +\infty} \Longleftrightarrow \lim\limits_{\rho\to +\infty} \Longleftrightarrow \lim\limits_{d((x,y),0)\to +\infty}$. In questo caso per calcolare il limite ci viene comodo passare alle coordinate polari, quindi $f(\rho, \Theta) = \rho\cos{\Theta}\cdot \rho\sin{\Theta} \cdot e^{-\rho^2} = \sin{\Theta}\cos{\Theta} \cdot \frac{\rho^2}{e^{\rho^2}}$.\\
Noi, essendo passati a coordinate polari, vogliamo calcolare $\lim\limits_{\rho \to +\infty}\sin{\Theta}\cos{\Theta} \cdot \frac{\rho^2}{e^{\rho^2}}$, notiamo che il fattore $\frac{\rho^2}{e^{\rho^2}}$ dipende solo da $\rho$ mentre $\sin{\Theta}\cos{\Theta}$ dipende solo da $\Theta$ ma è limitato, sappiamo che $\frac{\rho^2}{e^{\rho^2}}\to 0$ che moltiplica con una moltiplicata e quindi $f(\rho,\Theta) \to 0$, quindi $\lim\limits_{\rho \to +\infty}\sin{\Theta}\cos{\Theta} \cdot \frac{\rho^2}{e^{\rho^2}} \to 0$ e quindi $\lim\limits_{(x,y)\to \infty}xye^{-(x^2 + y^2)} \to 0$.
\end{example}

\newpage
\section{Calcolo differenziale in più variabili}
\subsection{Derivate in più variabili}
Nell'analisi finora quindi con le funzioni $f: \mathbb{R}\to \mathbb{R}$ con $x_0 \in \mathbb{R}$ si diceva che $f$ è derivabile in $x_0 \in \mathbb{R}$ se il rapporto incrementale $\lim\limits_{h\to 0}\frac{f(x_0 + h) - f(x_0)}{h}$ esiste ed è finito quindi $f'(x_0) = \lim\limits_{h\to 0}\frac{f(x_0 + h) - f(x_0)}{h}$ e la chiamiamo derivata prima di $f$ in $x_0$.\\\\
Sempre per funzioni $f: \mathbb{R}\to \mathbb{R}$ esiste la definizione che dice che $f$ è differenziabile in $x_0 \in \mathbb{R}$ se esiste un numero reale $\alpha \in \mathbb{R}$ tale che $f(x_0 + h) = f(x_0) + \alpha h + o(h)$ per $h \to 0$.\\\\
Abbiamo un un teorema per $f: \mathbb{R}\to \mathbb{R}$ che dice che $f$ è derivabile in $x_0 \Longleftrightarrow f$ è differenziabile in $x_0$ e $\alpha = f'(x_0)$. Vale anche che se $f$ è derivabile in $x_0$ allora f è continua in $x_0$.\\

\begin{wrapfigure}[5]{r}{4.5cm}
    \vspace{-25pt}
    \centering
    \includegraphics[width=4.2cm]{images/derivata-parziale.png}
\end{wrapfigure}

Passiamo ora a considerare una funzione $f: \mathbb{R}^2 \to \mathbb{R}$ e vediamo come estendere questi concetti ad una funzione in più variabili (in particolare in 2). Per fare questo introduciamo il primo concetto che è quello di \textbf{derivata parziale} che si bassa su tenere fissa una variabile in $\mathbb{R}^2$ e lasciamo variare l'altra.

\begin{definition}[Derivata parzialmente rispetto a x]
Data una funzione $f: \mathbb{R}^2 \to \mathbb{R}$ e $(x_0,y_0)\in \mathbb{R}^2$ si dice che $f$ è \textbf{derivabile parzialmente rispetto a x}  nel punto $(x_0,y_0)$ se $\lim\limits_{t\to 0}\frac{f(x_0 + t, y) - f(x_0,y_0)}{t}$ (rapporto incrementale su la sola variabile x) esiste ed è finito. Se tale limite esiste ed è finita si dice \textbf{derivata parziale rispetto a x} di $f$ in $(x_0,y_0)$ e si indica con uno delle seguente notazioni:
\vspace{-5pt}
\[\frac{\partial f}{\partial x}(x_0,y_0) \hspace{.5cm}o\hspace{.5cm} f_x(x_0,y_0) \hspace{.5cm}o\hspace{.5cm} D_x f(x_0,y_0)\]
\end{definition}
\hspace{-15pt}Analogamente possiamo definire la derivata parziale rispetto a y.
\begin{definition}[Derivata parzialmente rispetto a y]
Si dice che $f$ è \textbf{derivabile parzialmente rispetto a y}  nel punto $(x_0,y_0)$ se $\lim\limits_{t\to 0}\frac{f(x_0, y+t) - f(x_0,y_0)}{t}$ (rapporto incrementale su la sola variabile y) esiste ed è finito. Se tale limite esiste ed è finita si dice \textbf{derivata parziale rispetto a y} di $f$ in $(x_0,y_0)$ e si indica con uno delle seguente notazioni:
\vspace{-5pt}
\[\frac{\partial f}{\partial y}(x_0,y_0) \hspace{.5cm}o\hspace{.5cm} f_y(x_0,y_0) \hspace{.5cm}o\hspace{.5cm} D_y f(x_0,y_0)\]
\end{definition}

\begin{observation}
I limiti coinvolti nella definizione di derivata parziale sono limiti in una variabile (cioè in $\mathbb{R}$) infatti sono limiti del tipo $\lim\limits_{t\to 0}$.
\end{observation}

\hspace{-15pt}Possiamo vedere queste derivate geometricamente.
$f_x(x_0,y_0)$, consideriamo la retta parallela all'asse x e passante per il punto $(x_0,y-0)$, vuol dire che $y_0$ resta fissato e lascio variare x. In forma parametrica questa retta la posso scrivere come $(x_0,y_0) + t(1,0) = (x_o + t, y_0)$, guardo la funzione $f$ ristretta a questa retta, quindi $f(x_0 + t, y_0) = g(t)$ (mi muovo con il parametro t).\\
\begin{wrapfigure}[7]{l}{5.2cm}
    \vspace{-10pt}
    \centering
    \includegraphics[width=4.7cm]{images/derivata-parziale-geometricamente.png}
\end{wrapfigure}
Geometricamente $g(t)$ lo ottengo come l'intersezione del grafico di $f(x,y)$ con il piano perpendicolare al piano $xy$e contente la retta $(x_0+t,y_0)$. La derivata di $g(t)$ in $t=0$ quindi $g'(0)$ è proprio $f_x (x_0,y_0)$ perché sarebbe $\lim\limits_{t\to 0}\frac{g(t) - g(0)}{y} = \lim\limits_{t\to 0}\frac{f(x_0+t, y_0) - f(x_0,y_0)}{y} = f_x(x_0,y_0)$.\\
Analogamente per la derivata parziale $f_t(x_0,y_0)$ dove però definirò $h(t) = f(x_0, y_0+t)$, $h'(0) = f_y(x_0,y_0)$.\\\\
Nel caso di $f: \mathbb{R}^2 \to \mathbb{R}$ ho 2 derivate parziali, ma posso estendere questa definizione anche per n variabili.

\begin{definition}
Dato $f: \mathbb{R}^2 \to \mathbb{R}$ ì, un $x_0 \in \mathbb{R}^2$ dove $x_0 = (x_{01}, x_{02}, \cdots, x_{0n})$ si dice \textbf{derivata parziale} di $f$ rispetto alla variabile $x_k$ se $\lim\limits_{t\to 0 }\frac{f(x_0 + te_k) - f(x_0)}{t}$ esiste  ed è finito. Inoltre se questo limite esiste ed è finito ì, tale limite viene indicato con $\frac{\partial f}{\partial x_k}(x_0)$. 
\end{definition}

\begin{observation}
Abbiamo $x_0 \in \mathbb{R}^2$ quindi $x_0 = (x_{01}, x_{02}, \cdots, x_{on})$ fare $f(x_0 + te_k)$ vuol dire che, avendo $\{e_1, \cdots, e_n\}$ base canonica di $\mathbb{R}^n$ abbiamo che $e_1 = (1,0, \cdots, 0)$ genera l'asse $x_1$, $e_2 = (0,1,\cdots, 0)$ genera l'asse $x_2$ $e_k = (0,\cdots, 1, 0, \cdots, 0)$ genera l'asse $x_k$ e $e_n = (0,\cdots, 0,\cdots, 1)$ genera l'asse $x_n$. Quindi fare $f(x_0 + te_k)$ aggiungo t solo sulla k-esima variabile, nel caso di $\mathbb{R}^2$ fare $f(x_0+te_1)$ corrispondeva a muoversi nella direzione x e quindi calcolare la derivata parziale rispetto a x ed analogamente $(x_0+te_1)$ corrispondeva a muoversi nella direzione y e quindi calcolare la derivata parziale rispetto a y.\\
Quando $f: \mathbb{R}^n \to \mathbb{R}$ ho n direzione e quindi n vettori della base canonica, questo indica che una funzione di n variabili ha n derivata parziali ($\frac{\partial f}{\partial x_1}, \frac{\partial f}{\partial x_2}, \cdots, \frac{\partial f}{\partial x_n}$).
\end{observation}

\subsection{Derivata direzionale}
\begin{definition}[Derivata direzionale in $\mathbb{R}^2$]
Sia $f: \mathbb{R}^2 \to \mathbb{R}$, e $(x_0,y_0) \in \mathbb{R}^2$ e sia dato $v = (\alpha, \beta) \in \mathbb{R}^2$ un vettore di $\mathbb{R}^2$ non nullo (quindi $\alpha^2 + \beta^2 \neq 0$). Allora il limite $\lim\limits_{t \to 0}\frac{f(x_0 + t\alpha, y_o + t\beta) - f(x_0,y_0)}{t}$ se esiste ed è finito si dice \textbf{derivata direzionale} di $f$ in $(x_0, y_0)$ rispetto alla direzione $v = (\alpha, \beta)$ e si indica con $\frac{\partial f}{\partial v}(x_0,y_0)$
\end{definition}

\begin{wrapfigure}[7]{l}{5.7cm}
    \vspace{-10pt}
    \centering
    \includegraphics[width=5.5cm]{images/deri-direzionale.png}
\end{wrapfigure}

\hspace{-15pt}L'interpretazione geometrica di una derivata direzionale corrisponde a muoversi lungo la retta di equazione parametrica $(x_0, y_0) + t(\alpha, \beta) = (x_0 + \alpha t, y_0 + \beta t)$. \\
Scrivere quindi derivata direzionale rispetto alla direzione v significa restringere la funzione alla retta passante in ($x_0,y_0$) e con direzione $v$ e calcolarne la derivata rispetto a t nel punto $t=0$. \\\\
Definiamo $g(t) = f(x_0 + t\alpha, y_0 + t\beta)$ e $g'(0) = \frac{\partial f}{\partial v}(x_0, y_0)$ (derivata in una dimensione), e $g'(0) = \lim\limits_{t\to 0}\frac{g(t) - g(0)}{t} =\lim\limits_{t\to 0}\frac{f(x_0 + t\alpha + t\beta) - f(x_0,y_0)}{t}$ e questo corrisponde a $\frac{\partial f}{\partial v}(x_0,y_0)$ con $v=(\alpha, \beta)$.\\
Anche in questo caso, come per per derivate parziali, $g(t)$ è la funzione che ottengo intersecando il grafico di $f$ con il piano perpendicolare al piano xy e contente la retta passante per $(x_0,y_0)$ e con direzione v.\\\\
Analogamente possiamo definire al derivata direzione in n dimensioni.

\begin{definition}[Derivata direzionale in $\mathbb{R}^n$]
Sia $f: \mathbb{R}^2 \to \mathbb{R}$, e $x_0 \in \mathbb{R}^n$ ($x_0 = (x_{01}, x_{02}, \cdots, x_{0m})$ vettore) e sia dato $v \in \mathbb{R}^n$ con $v= (v_1, v_2, \cdots, v_n)$ e $v \neq 0$, definiamo $\frac{\partial f}{\partial v}(x_0)$, se esiste ed è finito, come $\lim\limits_{t\to 0}\frac{f(x_0 + tv) - f(x_0)}{t}$.
\end{definition}

\begin{observation}
Anche le derivate direzionali sono definite tramite limiti in una variabile (t paramento della retta di direzione v e passante per $x_0 \in \mathbb{R}^n$).
\end{observation}

\begin{definition}[Gradiente]
Data una $f: \mathbb{R}^n\to \mathbb{R}$, si dice \textbf{gradiente} di $f$ in $x_0 \in \mathbb{R}^n$ che ha come componenti le derivate parziali di $f$ in $x_0$ e si indica con $\nabla f(x_0) = (\frac{\partial f}{\partial x_1}(x_0), \frac{\partial f}{\partial x_2}(x_0), \cdots, \frac{\partial f}{\partial x_n}(x_0))$.
\end{definition}

\subsection{Differenziabilità di una funzione}
Ricordiamo che nel caso di una dimensione $f: \mathbb{R}\to \mathbb{R}$ si dice che $f$ è differenziabile se in $x_0$ esiste un numero reale $\alpha \in \mathbb{R}$ tale che $f(x_0 + h) = f(x_0) + \alpha \cdot h + o(h)$ con $h \in \mathbb{R}$ e $h\to 0$, dobbiamo definire una funzione differenziale in più variabili andando a generalizzare questa definizione.
\begin{definition}[Differenziabilità in $\mathbb{R}^2$]
Dato un $f: \mathbb{R}^2 \to \mathbb{R}$, un $(x_0,y_0) \in \mathbb{R}^2$ si dice che $f$ è \textbf{differenziabile} nel punto $(x_0,y_0)$ se esistono due numeri reali che chiamo $\alpha$ e $\beta$ tali che $f(x_0 + h, y_0 + k) = f(x_0, y_0) + \alpha \cdot h + \beta \cdot k + o(\sqrt{h^2 + k^2})$.
\end{definition}
\hspace{-15pt}Possiamo vedere $\alpha \cdot h + \beta \cdot k$ come il prodotto scalare $<(\alpha, \beta), (h,k)> \in \mathbb{R}$ dove $(h,k)$ è uno spostamento rispetto al punto ($x_0,y_0$) e $(\alpha, \beta)$ devono esiste per avere un differenziale di $f$, inoltre $\sqrt{h^2 + r^2}$ è uguale alla lunghezza di $(h,k)$ che quindi è uguale alla lunghezza dello spostamento, quindi l'errore in questa formula è determinata dall'o-piccolo, infatti quando scrivo $o(\sqrt{h^2 + k^2})$ significa che $\lim\limits_{(h,k)\to (0,0)}\frac{o(\sqrt{h^2 + k^2})}{\sqrt{h^2 + k^2}} = 0$ (limite in due variabili).\\\\
Da qui si può generalizzare in n dimensioni.

\begin{definition}[Differenziabilità in $\mathbb{R}^n$]
Dato un $f: \mathbb{R}^n \to \mathbb{R}$, un $x_0 \in \mathbb{R}^n$ si dice che $f$ è \textbf{differenziabile} nel punto $x_0$ se esistono un vettore $\alpha \in \mathbb{R}^n$ (devono esistere n numeri reali $\alpha_1, \cdots, \alpha_n$) tale che $f(x_0 + h) = f(x_0) + \alpha \bullet h + o(|h|)$ (norma del vettore h) per $h \to 0$ (limite in n variabili).
\end{definition}

\begin{theorem}\label{teorema-differenziabilità}
Supponiamo che $f: \mathbb{R}^n \to \mathbb{R}$ sia differenziabile in $x_0 \in \mathbb{R}^n$, allora valgono le seguenti proprietà:
\begin{enumerate}
    \item $f$ è continua in $x_0$.
    \item Esistono le derivate parziali di $f$ in $x_0$ e queste sono le componenti del vettore $\alpha = (\frac{\partial f}{\partial x_1}(x_0), \cdots, \frac{\partial f}{\partial x_n}(x_0)) = \nabla f(x_0)$.
    \item Esistono tutte le derivate direzioni di $f$ in $x_0$ e sono date da $\forall \: v \in \mathbb{R}^n \: \exists \: \frac{\partial f}{\partial v}(x_0)$ e $\frac{\partial f}{\partial v}(x_0) = \alpha \cdot v$ dove $\alpha = \nabla f(x_0)$, quindi vale che $\frac{\partial f}{\partial v}(x_0) = \nabla f(x_0) \cdot v$.
\end{enumerate}
\end{theorem}

\begin{note}
Notare bene che questo teorema vale per un solo verso e non viceversa, cioè può succedere che in $x_0$ esistano tutte le derivate parziali ma la funzione non è nemmeno continua in $x_0$ (quindi non differenziabile), quindi se $\exists$ derivata parziale non è detto che la funzione sia differenziabile.
\end{note}

\begin{example}
Consideriamo $f(x,y) = \begin{cases}(\frac{x^2 \cdot y}{x^4 + y^2})^2 & (x,y) \neq 0 \\ 0 & (x,y) = (0,0)\end{cases}$ Verifichiamo che nel punto $(x_0, y_0) = (0,0)$ esistono derivate direzionali e sono nulle, ma $f$ non è nemmeno continua.\\
Fisso $v = (\alpha, \beta) \in \mathbb{R}^2$, $(x_0, y_0) = (0,0)$ e poi calcolo la derivata direzionale $\lim\limits_{t \to 0}\frac{f(x_0 + t\alpha, y_0 + t\beta) - f(x_0,y_0)}{t} = \frac{1}{t}\bigg(\frac{t^2\alpha^2 \cdot t\beta}{2t^4\alpha^4 + t\beta^2}\bigg)^2 = t\bigg(\frac{\alpha^2\beta}{t^2\alpha^2 + \beta^2}\bigg)^2 = 0$ quindi $\frac{\partial f}{\partial v}(0,0) = 0$.
\end{example}

\begin{demostration}
Dimostriamo ora il punto (3) del teorema visto quindi che, se $f$ è differenziabile in $x_0 \in \mathbb{R}^n$ allora esistono tutte le derivate direzionali di $f$ in $x_0$, $\frac{\partial f}{\partial v}(x_0) = \alpha \cdot v$.\\\\
La nostra ipotesi è che $f$ è differenziabili in $x_0$, questo vuol dire che $\exists$ un vettore $\alpha \in \mathbb{R}^n$ tale che $f(x_o + h) = f(x_0) + \alpha \cdot h + o(|h|)$ per $h\to 0$. Fissiamo ora una direzione $v \in \mathbb{R}^n$ con $v\neq 0$, per definizione $\frac{\partial f}{\partial v}(c_0) = \lim\limits_{t\to 0}\frac{f(x_0 + tv) - f(x_0}{t}$, ora uso l'ipotesi di differenziabilità e posso usarla perché la formula della diff. vale quando $h\to 0$ e quindi scelgo $g = t \cdot v$ e posso fare questa scelta perché $v \in \mathbb{R}^n$  $(t\cdot v)\in \mathbb{R}^n$ perché pure $t \in \mathbb{R}^n$.\\\\
Quindi sostituiamo $h$ con $tv$ e viene $\lim\limits_{t\to 0}\frac{f(x_0) + t(\alpha \cdot v) + o(|tv|) - f(x_0)}{t} = \frac{t(\alpha\cdot v)}{t} + \frac{o(|tv|)}{t} = (\alpha \cdot v) + \lim\limits_{t\to 0}\frac{o(|tv|)}{t|v|} \cdot |v|$, la parte $|v| \neq 0$ è la norma di $v$, mentre $\frac{o(|tv|)}{t|v|}\to 0$ per definizione di o-piccolo, quindi questo limite fa $(\alpha \cdot v) + 0 = (\alpha \cdot v)$.\\
Quindi ho ottenuto che $\frac{\partial f}{\partial v}(c_0) = \alpha \cdot v$
\end{demostration}

\begin{observation}
Osserviamo che le derivate parziali sono casi particolari di derivate direzionali, corrispondo infatti alla scelta $v = e_k$ (vettori della base canonica).\\\\ Quindi la dimostrazione appena fatta ci mostra anche l'esistenza delle derivate parziali (2) del teorema e la formula $\frac{\partial f}{\partial x_k}(x_0) = \frac{\partial f}{\partial e_k}(x_0) = \alpha \cdot e_k = \alpha_k$ dove $e_k$ è il k-esimo vettore della base canonica di $\mathbb{R}^n$ (quindi è il vettore $0,\cdots,0, 1, 0 \cdots, 0$), e $\alpha_k$ sarà la k-esima componente di $\alpha$, ma $\alpha$ era il vettore nella formula di differenziabilità di $f$ con $\alpha_k = \frac{\partial f}{\alpha x_k} \Longrightarrow \alpha = \nabla f(x_0)$. \\
Quindi se $f$ è differenziabile in $x_0$, possiamo in conclusione scrivere che $f(x_0 + h) = f(x_0) + \nabla f(x_0) \cdot h + o(|h|)$ per $h\to 0$.
Inoltre le derivate direzionali sono date dalle formule $\frac{\partial f}{\partial v}(x_0) = \nabla f(x_0)\cdot v$.
\end{observation}

\begin{observation}
Le formule e le conclusioni del teorema valgono per qualsiasi direzione v. Nel definire le derivate direzionali possiamo in realtà limitarci alle direzioni v con $|v| = 1$, perché comunque in questo modo descrivo comunque tutte le possibili rette in cui mi muovo.
\end{observation}

\hspace{-15pt}Ora chiediamoci quale sia il significato geometrico del gradiente, per rispondere a questa domanda dobbiamo prima chiederci in quale direzione la derivata direzionale è massima o minima, per fare questo ci atteniamo all'algebra lineare che dice che $\frac{\partial f}{\partial v}(v_0) = \nabla f(x_0) \cdot v$ può essere scritto come $|\nabla f(x_0)| \cdot |v| \cdot \cos{\Theta}$ dove $\cos{\Theta}$ è l'angolo compreso tra $\nabla f$ e $v$. \\
Noi ci stiamo chiedendo $\max\limits_v \frac{\partial f}{\partial v}(x_0)$, in base all'osservazione $|v| = 1$ possiamo vedere che questo massimo lo realizzo quando massimizzo $\cos{\Theta}$ e questo è massimizzato quando $\cos{\Theta} = 1$ e questo accade quando $\Theta = 0$ e questa cosa può avvenire quando l'angolo compreso tra $\nabla f$ e $v$ è zero che è possibile se e solo se scelgo come direzione quella del gradiente stesso cioè quando $v$ è parallela a $\nabla f$.\\
Quindi riassumendo $\max\limits_{|v| = 1} \frac{\partial f}{\partial v}(c_0) = \frac{\partial f}{\partial (\nabla f(x_0))}(x_0)$.
\begin{itemize}
    \item Se voglio derivata direzionale massima devo scegliere $v =$ direzione del gradiente, cioè se $v$ è parallela a $\nabla f(x_0)$ allora ho max derivata direzionale in $x_0$.
    \item Per minimizzare $\frac{\partial f}{\partial v}(x_0)$ devo prendere $\cos{\Theta} = -1$ cioè $\Theta = \pi$ cioè $v$ parallela a $-\nabla f$ cioè $v$ dovrà avere direzione opposta a quella del gradiente. Riassumendo se $v$ parallela a $-\nabla f(x_0) \Longrightarrow$ minima derivata direzionale in $x_0$.
    \item Se scelto invece $\Theta = \pm \frac{\pi}{2}$ allora avrò che $\cos{\Theta} = 0$ ed allora $\frac{\partial f}{\partial v}(x_0) = 0$ cioè se $v$ è perpendicolare alla direzione $\nabla f(x_0) \Longrightarrow$ derivata direzionale è nulla.
\end{itemize}
In conclusione possiamo vedere il significato geometrico del gradiente. Il gradiente rappresenta la direzioni di massima pendenza della funzione che stiamo considerando, cioè è la direzione in cui muoversi per salire più in fretta possibile.\\\\
Nell'analisi finora quando si scriveva $f(x_0 + h) = f(x_0) + f'(x_0) \cdot h + o(h)$ per $h\to 0$, se chiamavamo $x_o + h = x \to x_0$ allora $f(x) = f(x_0) + f'(x_0)(x - x_0) + o(x - x_0)$ per $x \to x_0$ e in questo modo il termine $f(x_0) + f'(x_0)(x - x_0)$ rappresentava la retta tangente al grafico di $f$ nel punto $(x_0, f(x_0))$; ora invece, nell'analisi in più variabili (caso $f: \mathbb{R}^2 \to \mathbb{R}$) avviamo $f(x_0 + h, y_0 + k) = f(x_0. y_0) + \frac{\partial f}{\partial x}(x_0,y_0) \cdot h + \frac{\partial f}{\partial y}(x_0, y_0) \cdot k + o(\cdots)$, la parte $\frac{\partial f}{\partial x}(x_0,y_0) \cdot h + \frac{\partial f}{\partial y}(x_0, y_0)$ sarebbe come scrivere $\nabla f(x_0, y_0) \bullet (h,k)$, facendo come prima posso chiamare $x_0  + h = x \to x_0$ e $y_0 + k = y \to y_0$ e riscrivere $f(x,y) = f(x_0, y_0) + \frac{\partial f}{\partial x}(x_0,y_0)\cdot (x - x_0) + \frac{\partial f}{\partial y} (x_0, y_0) \cdot (y - y_0) + o(\cdots)$, questo per $x \to x_0$ e $y \to y_0$. In conclusione quello che otteniamo nell'analisi in più variabile $f(x_0, y_0) + \frac{\partial f}{\partial x}(x_0,y_0)\cdot (x - x_0) + \frac{\partial f}{\partial y} (x_0, y_0) \cdot (y - y_0)$ mi darà un piano tangente al grafico di $f$ nel punto $(x_0, y_0, f(x_0,y_0))$.

\subsection{Calcolo derivate parziali}
Innanzitutto vediamo come calcolare il vettore $\alpha$. Noi sappiamo che se la funzione è differenziabile e so calcolare le derivate parziali allora $\alpha = \nabla f(x_0, y_0)$. Se so fare calcolare le derivate parziali e se riesco a dimostrare che la funzione $f(x,y)$ è differenziabile allora $\alpha$ è il gradiente di $f$ nel punto $(x_0, y_0)$. (vettore che ha componenti le derivate parziali di $f$ in $x_0, y_0$).\\\\
Ora invece vediamo come calcolare le derivate direzione. Per loro bisogna come prima sapere se la funzione è differenziabile e calcolare le derivate parziali, allora le derivate direzionali nella direzione di $v$ $(v \in \mathbb{R}^2)$ sono date da $\frac{\partial f}{\partial v} = \nabla f \cdot v$.\\\\
Quindi bisogna sappiamo se la funzione è differenziabile e calcolare le derivate parziali sappiamo che $\frac{\partial f}{\partial v} = \nabla f \cdot v$ e che $\alpha = \nabla f(x_0, y_0)$.\\
Attenzione questo vale solo se $f$ è differenziabile perché se $f$ non è differenziabile può succedere che esistano comunque le derivate direzionali ma non saranno date dalla formula $\frac{\partial f}{\partial v} = \nabla f \cdot v$.\\\\
Campiamo ora come calcolare le derivate parziali. Per il calcolo delle derivate parziali usiamo le stesse regole dell'analisi vista fin'ora (1 dimensione), e questo lo facciamo derivando solo rispetto alla variabile che ci interessa, cioè andiamo a trattare le altre variabili come fossero costanti. Data una funzione $f: \mathbb{R}^n \to \mathbb{R}$, $\frac{\partial f}{\partial x_k}$, $f(x_1, \cdots, x_k, \cdots, x_n)$ farò $f(x_k, \cdots \: costanti)$.

\begin{example}
Consideriamo una funzione $f: \mathbb{R}^2 \to \mathbb{R}$ definita come $f(x,y) = x^2 + y^3$ (in questo caso vedo $y^3$ come costante e derivo $x^2$). Per calcolare $\frac{\partial f}{\partial x}(x,y) = f_x(x,y) = 2x$ infatti se dico che $g(x) = x^2$, $f(x,y) = g(x) + y^2$, $f_x(x,y) = g'(x) + 0$. \\
Se invece voglio calcolare $\frac{\partial f}{\partial y}(x,y) = f_y(x,y) = (x \: costante) = 3y^2$.
\end{example}

\begin{example}
Facciamo altri esempi di derivate parziali.
\begin{itemize}
    \item Prendiamo $f(x,y) = x^2 \cdot y^3$. Se faccio $f_x(x,y) = y^3 \cdot (2x) = 2x\cdot y^3$. Mentre $f_y(x,y) = x^2 \cdot 3y^2$.
    \item $f(x,y) = \sin(xy^2)$, $f (x,y) = \sin(x \cdot c)$ dove $c = y^2$ quindi rispetto a $x$ ho $f_x(x,y) = \cos{xy^2} \cdot y^2$ mentre rispetto a y ho $f_y (x,y) = \cos(xy^2)x \cdot 2y$.
    \item Facciamo un esempio in $\mathbb{R}^3$ quindi $f: \mathbb{R}^3 \to \mathbb{R}$ e con $f(x,y,z) = x \cdot e^{yz^2}$. In questo caso ci saranno allora 3 derivate parziali.\\
    $f_x(x,y,z) = e^{yz^2}$ (teniamo $y,z$ costanti), $f_y(x,y,z) = x \cdot e^{yz^2} \cdot z^2$ (teniamo $x,z$ costanti), $f_z(x,y,z) = x \cdot e^{yz^2} \cdot 2zy$ (teniamo $y,x$ costanti).
    \item $f(x,y) = e^{xy} \cdot \cos{y}$, $f_x(x,y) = ye^{xy}\cdot \cos{y}$, se invece vado a fare $f_y (x,y) = x\cos{y}e^{xy} - \sin{y} e^{xy}$.
    \item $f(x,y) = \log(1 + x^2 + y^4)$ con $f:\mathbb{R}^2 \to \mathbb{R}$ calcoliamo $f_x(x,y) = \frac{2x + y^4}{1 + x^2 + y^4}$, $f_y(x,y) = \frac{4y^3}{1 + x^2 +y^4}$.
    \item $f(x,y) = x^y$, $f_x(x,y) = yx^{y-1}$ mentre $f(x,y) = a^y = e^{\log(a)^y} = e^{y\log(a)} = e^{y \log(a)} \cdot \log(a) = a^y \cdot \log(a)$ e quindi $f_y(x,y) = x^y \cdot \log(x)$
\end{itemize}
\end{example}

\subsection{Interpretazione geometrica gradiente}
Ora vogliamo specificare la relazione fra il gradiente e gli insiemi di livello. Abbiamo detto che il gradiente ci da la direzione in cui la funzione varia più velocemente mentre gli insiemi di livello ci dice l'insieme dove la funzione ha lo stesso valore, detto questo prendiamo $f:\mathbb{R}^2 \to \mathbb{R}$, consideriamo $f(x,y) = x^2 + y^2$, gli insiemi di livello definiti da $\{(x,y) \in \mathbb{R}^2 \::\: f(x,y) = \lambda\}$, che sono circonferenze concentriche con centro dell'origini e raggio $\lambda$.\\

\begin{wrapfigure}[7]{r}{5.7cm}
    \vspace{-30pt}
    \centering
    \includegraphics[width=5cm]{images/sign-geom-gradiante.png}
\end{wrapfigure}

Calcoliamo $\nabla f(x,y) = \big( \frac{\partial f}{\partial x}(x,y), \frac{\partial f}{\partial y}(x,y) = (2x, 2y)$, $\nabla f(x,y) = 2 \cdot (x,y)$, se prendo un qualsiasi punto $(x,y) \in \mathbb{R}$ dico che in quel punto il vettore gradiente è 2 volte il vettore stesso. In generale, geometricamente il gradiente si rappresenta come "campo di vettori": in ogni punto del dominio disegno un vettore (il gradiente) che mi sta indicando la direzione per salire con la massima pendenza.

\begin{observation}
Il gradiente è sempre perpendicolare agli insiemi di livello e punta verso i $\lambda$ crescenti.
\end{observation}

\begin{example}
Consideriamo $f:\mathbb{R}^2 \to \mathbb{R}$, $f(x,y) = x^2 -xy$ calcolare $\frac{\partial f}{\partial v}$ nel punto $(x_0,y_0) = (-1,2)$ con $v = (1,3)$.
\end{example}
\begin{wrapfigure}[6]{l}{5cm}
    \vspace{-10pt}
    \centering
    \includegraphics[width=4.7cm]{images/ess-gradiante-1.png}
\end{wrapfigure}
Noi sappiamo che se so calcolare $\nabla f \Longrightarrow \frac{\partial f}{\partial v} = \nabla f(x_0, y_0) \cdot v = (\frac{\partial f}{\partial x}, \frac{\partial f}{\partial y}) = (2x - y, -x)$ sostituiamo con $(x_0,y_0)$ ed otteniamo $\nabla f(x_0,y_0) = \nabla f(-1,2) = (-4, 1)$.\\
Sappiamo che $\frac{\partial f}{\partial v}(-1,2) = \nabla f(-1,2) \cdot v = (-4,1)\cdot(1,3) = -4 + 3 = -1 < 0$ e quindi $f$ sta scendendo.\\
Geometricamente immaginiamoci un omino che si trova sul grafico di $f$ nel punto corrispondente a $(-1,2)$ (il punto sul grafico (-2,2,$f(-1,2)$) = (-1,2,3)) e si muove nella direzione $(1,3)$ si trova a scendere.

\begin{example}
Prendiamo $f(x,y) = xy$.
\end{example}
\begin{wrapfigure}[8]{l}{5cm}
    \vspace{-10pt}
    \centering
    \includegraphics[width=4.8cm]{images/ess-gradiane-2.png}
\end{wrapfigure}
Disegniamo il gradiente della funzione (che vuol dire mettersi nello spazio di partenza e per ogni punto di questo spazio disegnare un vettore che indica la direzione del gradiente, nel nostro caso ovviamente lo si fa solo per alcuni punti). \\
Iniziamo a vedere $\nabla f(x,y) = (y,x)$ (calcolo la derivata parziale per x e y). Le linee di livello sono definite come $\{(x,y)\in \mathbb{R}^2 \::\: xy = \lambda\}$. In un punto $(x_0,y_0)$ il gradiente avrà come ascissa $y_0$ e come originata $x_0$ e questo per ogni punto. 


\subsection{Teorema del differenziale totale}
Quello che ci stiamo chiedendo è come posso dimostrare che $f(x,y)$ è differenziabile in $x_0,y_0$. Per farlo si può agire in due modi, o usare la definizione, ma sconsigliato perché molto complesso, oppure utilizzare il seguente teorema.

\begin{theorem}[Teorema del differenziale totale]
Se le derivate parziali di $f$ esistono e sono continue allora $f$ è differenziabile in quel punto.
\end{theorem}
\hspace{-15pt}A livello pratico, se non si incontrano problemi nel calcolare le derivate parziali, la funzione è differenziabile.\\\\
Quindi per le domande che si eravamo posti all'inizio: come calcolare il vettore $\alpha$ e come calcolare le derivate direzione possiamo avere una risposta visto che dovevamo dove sapere se la funzione è differenziabile e calcolare le derivate parziali, ed ora possiamo per il primo punto fare come scritto nel paragrafo precedente, e per il secondo punto usare questo teorema.


\newpage
\section{Massimi e minimi per funzioni di più variabili}
\subsection{Ripasso in una variabile}
Ricordiamo che nell'analisi in una variabili abbiamo visto una serie di teoremi per trovare i massimi ed i minimi, come il teorema di Weristrass, esso diceva che se $f: [a,b]\to \mathbb{R}$ è continua allora esistono sicuramente $\max\limits_{x \in [a,b]}f$ e $\min\limits_{x \in [a,b]}f$.\\
Le ipotesi importanti in questo teorema è che a,b sono inclusi nell'intervallo e l'intervallo è chiuso, $f$ continua su $[a,b]$ e che $max, min$ sono il massimo valore assunto della funzione il minimo valore assunto della funzione. I punti in cui la funzione assume il max, min si dicono punti di max, min.\\\\
Dove cerchiamo i punti di massimo e minimo in una dimensione? 
\begin{enumerate}
    \item Possiamo cercare i punti $x_0 \in (a,b)$ tale che $f'(x_0) = 0$. Questi punti si dicono punti stazionari interni. Si dice stazionario perché è legato all'annullarsi della derivata prima, interni è perché siamo in (a,b) e quindi escludiamo gli insiemi.
    \item Sennò possiamo cercare i punti $x_0 \in (a,b)$ tali che $f'(x_0)$ non esiste, che si dicono punti singolari interni, singolari perché la derivata non esiste.
    \item L'ultima categoria sono i punti $x_0 = a$ e $x_0 = b$ che sono gli estremi dell'intervallo e si chiamano i punti di bordo o di frontiera.
\end{enumerate}
Nell'analisi ad una dimensione quindi si trovano i punti del tipo (1) (2) (3), si vanno a costituire in $f$ e si controlla dove $f$ viene massimizzato o minimizzato.\\
Quello che dobbiamo ora fare e generalizzare questo metodo a funzioni $\mathbb{R}^n \to \mathbb{R}$. Per farlo dobbiamo generalizzare:
\begin{itemize}
    \item La continuità, che però l'abbiamo già fatto infatti sappiamo quando $f: \mathbb{R}^n \to \mathbb{R}$ è continua.
    \item La chiusura, che sappiamo infatti $E \subset \mathbb{R}^n$ si dice chiuso se $E^c$ è aperto, in sintesi $E \subseteq \mathbb{R}^n$ si dice chiuso se contiene tutto il suo bordo.
    \item La limitatezza, che pure sappiamo: $E\subseteq \mathbb{R}^n$ si dice limitato se $\exists \: R > 0$ tale che $E \subseteq B_r(0)$.
\end{itemize}
Ci manca un un unica definizione per poi poter generalizzare.

\subsection{Teorema di Weristrass in n dimensioni}
\begin{definition}[Compatto]
Un insieme $A \subseteq \mathbb{R}^n$ si dice \textbf{compatto} se è limitato e chiuso.
\end{definition}
\begin{example}
Esempio di un compatto, $\mathbb{R}(n=1)[a,b]$ è un compatto di $\mathbb{R}$
\end{example}

\begin{theorem}[Teorema di Weristrass in n dimensioni]
Sia $A \subseteq \mathbb{R}^n$ un insieme compatto e sia $f: A \to \mathbb{R}$ una funzione continua. Allora esistono $\max\limits_{x \in A}f(x)$ e $\min\limits_{x \in A}f(x)$
\end{theorem}

\hspace{-15pt}I punti di minimo e massimo vanno ricercate nelle 3 categorie seguenti:
\begin{enumerate}
    \item Punti stazionari interni: che sonno i punti interni all'insieme in cui $\nabla f = 0$.
    \item Punti singolari interi: punti interni all'insieme in cui $f$ non è differenziabile.
    \item Punti di bordo: i punti del bordo in n dimensioni possono essere $\infty$.
\end{enumerate}

\begin{observation}
Non appena una le ipotesi non sono verificate, allora il massimo ed il minimo possono comunque esistere ma non necessariamente.
\end{observation}

\subsection{Calcolo massimi e minimi}
Per funzioni $f: [a,b]\to \mathbb{R}$ se $x_0$ è un punto $x_0 \in (a,b)$ di massimo o minimo e in quel punto $\exists \: f'(x_0)$ allora $f(x_0) = 0$. C'è un analogo anche per le funzioni in n variabili.
\begin{theorem}
Dato $A \subseteq \mathbb{R}^n$ e $f: A \to \mathbb{R}$, se $x_0$ è punto di massimo o minimo interno ad A se esistono le derivate parziali di $f$ in $x_0$ allora $\nabla f(x_0) = 0$.
\end{theorem}
\hspace{-15pt}Quindi $\nabla f(x_0) = 0$ è una condizione necessaria affinché un punto sia di minimo e di massimo.
\begin{demostration}
Supponiamo di avere $f: A \to \mathbb{R}$ con $A \subseteq \mathbb{R}^2$ e $(x_0,y_0) \in \mathbb{R}^2$ e che $(x_0,y_0) \in \mathring{A}$ (punti interni di A) e supponiamo che $(x_0,y_0)$ sia punto di minimo (o massimo) e supponiamo anche che $\frac{\partial f}{\partial x}$ e $\frac{\partial f}{\partial y}$ esistono in $(x_0, y_0)$.\\
Io ora voglio dimostrare che $\frac{\partial 0}{\partial x}(x_0, y_0) = 0$ e che $\frac{\partial f}{\partial y}(x_0,y_0) = 0$ (quindi che $\nabla f(x_0,y_0) = 0$). Consideriamo la funzione $g(t) = f(x_0 + t, y_0)$, poiché abbiamo ipotizzato che $(x_0,y_0)$ è punto di minimo per f $\Longrightarrow$ $g(t)$ ha un minimo per $t = 0$. g(t) però è una funzione di una variabile ed ha un minimo per il teorema in una dimensioni si può concludere che $g'(0) = 0$ che è $\frac{\partial f}{\partial x}(x_0,y_0) = 0$.\\
Analogamente, possiamo considerare la funzione $h(t) = (x_0, y_0 + t)$, anche in questo caso se vado a considerare $h'(0) = \frac{\partial f}{\partial y}(x_0, y_0) = 0$ poiché $f$ ha un minimo per $t = 0$. Se $(x_0, y_0)$ fosse stato un massimo, avrei ragionato in modo analogo. $\blacksquare$
\end{demostration}

\hspace{-15pt}A livello pratico quando cerco punti di massimo e minimo andrò a considerare i punti stazionari interi, cioè quelli dove $\nabla f(x_0,y_0) = 0$.\\
A questo punto possiamo vedere come determinare i massimi ed i minimi di una funzione $f$ continua su in insieme compatto A.
\subsubsection{Calcolo intuitivamente}
\begin{example}
Prendiamo $f(x,y) = 2x+3y$ definita su $A = [0,1] \times [0,2] \subset \mathbb{R}^2$ e $f$ continua su A compatto e quindi in A $x \geq 0$ e $y \geq 0$, quindi per il teorema di Weristrass ci dice che $\exists \: max,min$ in $f$. Intuitivamente il punto di massimo sarà il punto in cui massimizzo sia x che y, infatti il punto sarà $(1,2)$ ed il massimo sarà $f(1,2) = 8$ mentre il punto dove minimizzo $f$ quindi minimizzo sia $x$ che y quindi sarà il punto $(0,0)$ e varrà $f(0,0) = 0$.
\end{example}
\begin{example}
Prendiamo $f(x,y) = 2x-3y$ e insieme $A = [0,1] \times [0,2]$ in A $x \geq 0$ e $y \geq 0$. Anche in questo caso troviamo il massimo con la massima x ed la massima y quindi $(1,0)$ quindi $f(1,0) = 2$, mentre per il minimo si cercherà la $(x,y)$ minima possibile che è $(0,2)$ e quindi $f(0,2) = -6$. Notiamo che $\nabla f = (2,-3)$ e quindi $\nabla f \neq 0$, (1,0) e (0,2) sono punti di bordo infatti il gradiente non si annulla mai.
\end{example}
\begin{example}
$f(x,y) = x^2 - y$, prendiamo $A = [-1,2] \times [-2,3]$, $f$ continua su A, in questo caso però la $x$ e la $y$ possono cambiare segno. \\
Per trovare un punto si massimo bisogna vedere quando viene massimizzato il valore assoluto di $x$ ed in questo caso viene massimizzato con $x=2$ rispetto a y, visto che sto "togliendo" devo prendere il valore più negativo perché ci sarà un cambio di segno e si aggiungerà il valore più grande, quindi $y = -2$, il risultato è il punto $(2,-2)$, quindi $f(2,-2) = 6$.\\
Per il punto di minimo bisogna vedere quando viene minimizzato $x^2$ che è 0, e per la $y$ si cerca il valore più grande visto che abbiamo un "-y", il risultato è (0,3), e quindi $f(0,3) = -3$,
\end{example}
\subsubsection{Metodo degli insiemi di livello}
Vediamo alcuni esempi per spiegare questo metodo di calcolo.

\begin{example}
Consideriamo $f(x,y) = x$ e con $A = $ cerchio (disco) con centro in (0,0) e raggio 2 $= \{(x,y) \in \mathbb{R}^2 \::\: x^2 + y^2 \leq 4\}$. 
\end{example}
\begin{wrapfigure}[5]{r}{5.5cm}
    \vspace{-20pt}
    \centering
    \includegraphics[width=4.7cm]{images/max-min-ess-insiemi-livello-1.png}
\end{wrapfigure}
In questo caso ci aspettiamo che il punto di massimo sia $(2,0)$ e il minimo $(-2, 0)$ quindi $max(f) = 2$ e $min(f) = -2$.\\
Le linee di livello di $f(x,y)$ sono $\{(x,y) \in \mathbb{R}^2 \::\: f(x,y) = \lambda\} = \{(x,y) \in \mathbb{R^2} \::\: x = \lambda\} =$ linee di livello sono rette parallele all'asse y.
\begin{itemize}
    \item Si può osservare in questo esempio che il massimo di $f(x,y)$ in A è il più grande $\lambda$ tale che $A \cap \{(x,y) \::\: f(x,y) = \lambda\} \neq \O$, in questo caso è $\lambda = 2$ che è il massimo di f.
    \item Mentre il minimo di $f(x,y)$ in A è il più piccolo $\lambda$ tale che $A \cap \{(x,y) \:\: f(x,y) = \lambda\}\neq \O$, quindi in questo caso $\lambda = -2$ e quindi $min(f) = -2$.
\end{itemize}
\hspace{-15pt}Possiamo riassumere questo metodo dicendo che andiamo a cercare il più grande e il più piccolo $\lambda$ per i quali l'insieme di livello $\lambda$ interseca A.

\begin{example}
Dato $f(x,y) = 2x-y$ e $A =$ triangolo con vertici $(0,0), (0,1), (3,2)$. 
\end{example}

\begin{wrapfigure}[9]{l}{5.3cm}
    \vspace{-10pt}
    \centering
    \includegraphics[width=4.5cm]{images/max-min-ess-insiemi-livello-2.png}
\end{wrapfigure}
$f$ continua si A e compatto quindi per Weirstrass esiste sia massimo che minimo in $f$. In questo caso il metodo intuitivo non è fattibile quindi si usa le linee di livello, sappiamo che $\{(x,y) \in \mathbb{R}^2 \::\: f(x,y) = \lambda\} = \{(x,y) \in \mathbb{R}^2 \::\: 2x-y = \lambda\}$ linee di livello $\lambda$. Se $\lambda = 2$ ho $y = 2x -2$, con $\lambda = 3$ ho $y = 2x -3$ (in questo caso ancora la retta interseca il triangolo), con $\lambda = 4$ ho $y = 2x - 4$ per $x = 3$ ho $y= 2 \cdot 3 - 4 = 2$ retta passante per $(3,2)$, se $\lambda$ più grande per cui l'insieme dato A interseca le linee di livello $\lambda$ è $\lambda = 4$ quindi per il metodo delle linee di livello $max(f) = 4$ che il punto $(3,2)$.\\\\
Per il punto minimo si procede analogamente ma con $\lambda$ più piccoli, possiamo vedere in questo caso che $\lambda = -1$ è il pià piccolo $\lambda$ tale che $\{$linee di livello $\lambda\} \cap A \neq 0$, quindi $min(f) = -1$ ed il punto di minimo è $(0,1)$.

\begin{example}
Consideriamo la funzione $f(x,y) = x^2 + y^2$ su $A = [2,5] \times [-1,1]$, anche in questo caso f è continua su A compatto quindi esistono massimo e minimo.
\end{example}
\begin{wrapfigure}[6]{l}{5.3cm}
    \vspace{-10pt}
    \centering
    \includegraphics[width=4.5cm]{images/max-min-ess-insiemi-livello-3.png}
\end{wrapfigure}
Le linee di livello $\lambda$ sono circonferenze con centro nell'origine e raggio $\sqrt{\lambda}$. Il $\lambda$ più piccolo tale che $L_{\lambda} \cap A \neq \O$ è $\lambda = 4$ quindi $min(f) = 4$. I punti di coordinate $(5,1)$ e $(5,-1)$ sono tali che $5^2 + 1^2 = 26$ quindi $L_{26} \cap A = (5,1) \cup (5,-1)$, quindi 25 è il $\lambda$ più grande tale che $L_{\lambda} \cap A \neq \O$ quindi $max(f) = 26$ e i punti di max sono due $(5,1)$ e $(5,-1)$.\\

\subsubsection{Metodo di parametrizzazione}
Per capire questo metodo rifacciamo l'esercizio visto sopra usando appunto questa metodologia di risoluzione.
\begin{example}
Data $f(x,y) = x^2 + y^2$, $A = [2,5] \times [-1,1]$. Questo metodo ci dice che bisogna cercare i punti max e min in varie categorie.
\begin{itemize}
    \item Punti stazionari interi, interni perché $(x_0,y_0) \in (2,5) \times (-1,1)$ e $\nabla f(x_0,y_0) = 0$. Nel nostro caso $\nabla f = (2x, 2y)$ che si annulla soltanto se $(x,y) = (0,0)$, ma (0,0) non è punto interno di A quindi non ci sono punti interni stazionari.
    \item Punti singolari interi, che sono quelli in cui la funzione non è differenziabile. Nel nostro caso $f(x,y) = x^2 + y^2$ è differenziabile su tutto A, quindi non abbiamo punti singolari interni.
    \item Punti di bordo. Per cercare questi punti ci sono vari metodi, uno di questi è il \textbf{metodo della parametrizzazione}, ocn questo metodo parametrizzo i vari pezzi del bordo, e così facendo vedo la funzione come si comporta rispetto questi pezzi.
    \begin{enumerate}
        \item $\{(t,1) \:\: t \in [2,5]\}$, quindi $g_1(t) = f(t,1) = t^2 +1$.
        \item $\{(5,t) \::\: t \in [-1,1]\}$, quindi $g_2(t) = f(5,t) = t^2 + 25$.
        \item $\{(t,-1) \::\: t \in [2,5]\}$, quindi $g_3(t) = f(t,-1) = t^2 + 1$.
        \item $\{(2,t) \::\: t \in [-1,1]\}$, quindi $g_4(t) = f(2,t) = t^2 + 4$.
    \end{enumerate}
    Adesso studio le 4 funzioni sopra definite ed ho per ciascuna dei massimi e dei minimi su ogni tratto, poi li confronto e rendo il più piccolo ed il più grande che corrispondono al minimo ed al massimo.
    \begin{enumerate}
        \item Minimo per $t=2$ e massimo per $t = 5$.
        \item Minimo per $t=0$ e massimo per $t = \pm 1$.
        \item Minimo per $t=2$ e massimo per $t = 5$.
        \item Minimo per $t=0$ e massimo per $t = \pm 1$.
    \end{enumerate}
    Quindi concludo che $(2,0)$ è minimo e che $(5,-1)$ e $(5,1)$ è il punto di massimo.
\end{itemize}
\end{example}

\begin{example}
Prendiamo $f: \mathbb{R}^2 \to \mathbb{R}$, $f(x,y) = 3x^2 - y + 3$ sull'insieme $A = \{(x,y) \in \mathbb{R}^2 \::\: x^2 - 1 \leq y \leq 3\}$, A è compatto (essendo sia chiuso che limitato), $f$ è continua su un compatto quindi $f$ ammette max e min per Weirstrass.
\end{example}
\begin{wrapfigure}[6]{r}{5cm}
\vspace{-20pt}
    \centering
    \includegraphics[width=4.5cm]{images/ess-max-min-classico-R2.png}
\end{wrapfigure}
In questo caso le linee di livello tali che $3x^2 - y + 3 = \lambda$, $y = 3x^2 + (x+3)$ non sono un metodo semplice da attuare, quindi optiamo per il metodo classico:
\begin{itemize}
    \item Calcoliamo i punti stazionari interni $\nabla f = 0$, che quindi è $\nabla f(x,y) = (\frac{\partial f}{\partial x}(x,y), \frac{\partial f}{\partial y}(x,y)) = (6x, -1)$ ma $(6x,-1) \neq 0$ perché $-1$ è sempre diverso da 0, quindi non esistono punti stazionari interne quindi non esistono max e min tra i punti interni non singolari di A.
    \item Punti singolari interni non esistono quindi non esistono max e min tra i punti interni di A.
    \item Punti di bordo, andiamo ad usare la parametrizzazione. 
    \begin{enumerate}
        \item Punti di bordo sono quelli che (1) stanno sulla retta $y = 2$ quindi posso porre $x^2 -1 = 3$ quindi $ x = \pm 2$ e quindi abbiamo come punti $(-2,3)$ e $(2,3)$, per parametrizzarlo chiamo $x = t$ e $y = 2$ quindi i punti di tipo (1) sono $(x,y) = (t,3)$ con $t = [-2,2]$. Quindi la funzione $f$ rispetto a (1) possiamo chiamarla $g_1(t) = f(t,3)$ con $t \in [-2,2]$ è $g_1(t) = 3t^2$.
        \item Abbiamo poi i punti che stanno sulla parabola $y = x^2 - 1$ per $x \in [-2,2]$ quindi di tipo (2), per parametrizzare questo bordo prendiamo i punti che possono essere scritti come $(x,y) = (t, t^2-1)$ con $t \in [-2,2]$ quindi avendo come parametro $x = t$, su questo secondo bordo definirò $g_2(t) = f(t,t^2-1) = 3t^2 - t^2 + 1+4 = 2t^2+4$.
    \end{enumerate}
    Ho quindi sul bordo (1) $g_1(t) = 3t^2$ e su (2) ho $g_2(t) = 2t^2+4$ per $t \in [-2,2]$, ora dobbiamo cercare i massimi ed i minimi. Per $g_1(t) = 3t2$ ho che:
    \begin{itemize}
        \item Il massimo per $t = -2$ o per $t = 2$ è $g_1(t) = 12$.
        \item Mentre il minimo per $t = 0$ è. $g_1(t) = 0$.
    \end{itemize}
    Invece per $g_2(t) = 2t^2 + 4$ vediamo che:
    \begin{itemize}
        \item Il massimo per $t = -2$ o per $t = 2$ avrò che $g_2 (t) = 12$.
        \item Mentre il minimo come sempre ho $t=0$ con il quale ho $g_2(t) = 4$.
    \end{itemize}
    Avremo dunque come massimo $t=-2$ e $t = 2$ perché abbiamo il valore massimo in entrambi i $g_1, g_2 = 12$, il punto minimo di $f$ è $(0,3)$ e $f(0,3) = g_1(0) = 0$.
\end{itemize}

\hspace{-15pt}Ecco alcuni casi notevoli per andare a parametrizzare i bordi:
\begin{enumerate}
    \item Segmento di estremi $(a_1,b_1)$ e $(a_2,b_2)$. Questi punti possono essere parametrizzati andando a scrivere $(x,y) = (a_1,b_1) + t(a_2-a_1, b_2 - b_1)$ con $t \in [0,1]$.
    \item Il tratto del grafico $y = \varphi(x)$ con $x \in [a,b]$. In generale descriviamo $(x,y) = (t, \varphi(t))$ con $t \in [a,b]$.
    \item La circonferenza con centro in $(0,0)$ e raggio r. Questo caso si parametrizza prendendo $(x,y) = (r \cdot \cos{\Theta}, r \cdot \sin{\Theta})$ con $\Theta \in [0,2\pi]$.
    \item La circonferenza con centro in punto $(x_0, y_0)$ e raggio r. Analogo alla precedente, quindi $(x,y) = (x_0 + r \cdot \cos{\Theta}, y_0 + r\cdot \sin{\Theta})$ con $\Theta \in [0,2\pi]$.
    \item L'ellisse di equazione $ax^2 + by^2 = 1$ con $a,b > 0$. Si parametrizza simile alla circonferenza quindi con $(x,y) = (\frac{1}{\sqrt{a}}\cos{\Theta}, \frac{1}{\sqrt{b}}\sin{\Theta})$ con $\Theta \in [0,2\pi]$.
\end{enumerate}
\subsubsection{Moltiplicatori di Lagrange}
\begin{definition}[Luogo di zeri]
Data una funzione $\varphi: \mathbb{R}^2 \to \mathbb{R}$ allora l'insieme $V = \{(x,y) \in \mathbb{R}^2 \::\: \varphi(x,y) = 0\}$ si dice \textbf{luogo di zeri} della funzione $\varphi$ (linea di livello $\lambda = 0$).
\end{definition}
\hspace{-15pt}Il metodo dei moltiplicatori di Lagrande serve per trovare possibili punti di massimo e minimo di una funzione $f$ su un insieme A quando il bordo di A è un luogo di zeri di una funzione.
\begin{example}
Prendiamo $f(x,y)$ su $A = \{(x,y) \in \mathbb{R}^2 \::\: x^2 + y^2 \leq 3\}$. Il bordo di A è la circonferenza data da $x^2 + y^2 = 3$ (quindi la circonferenza che delimita) questa circonferenza la posso scrivere come $(x,y) \in \mathbb{R}^2$ tali che $x^2 + y^2 - 3 = 0$, se definisco $\varphi(x,y) = x^2 + y^2 -3$ allora il bordo di A è il luogo di zeri della funzione $\varphi$ e quindi posso usare il metodo dei moltiplicatori di Lagrange.
\end{example}

\begin{example}
Se invece prendiamo un insieme $A = \{(x,y) \in \mathbb{R}^2 \::\: 2x^2 + 3y^2 \leq 5\}$, analogamente in questo caso il bordo di A sarà il luogo di zeri della funzione $\varphi = 2x^2 + 3y^2 -5$ perché quando $\varphi (x,y) = 0$ se e solo se è il bordo di A.
\end{example}

\hspace{-15pt}Supponiamo per semplicità di essere in $\mathbb{R}^2$ e supponiamo che V sia il luogo di zeri di $\varphi(x,y)$, allora i candidati ad essere punti di minimo o massimo di $f$ in V si cercano tra le seguenti due categorie.
\begin{enumerate}
    \item Punti $(x,y) \in \mathbb{R}^2$ tali che $\begin{cases}\varphi(x,y) = 0\\ \nabla \varphi(x,y) = 0\end{cases}$ Quindi se esiste un punto che soddisfa questo sistema (1) allora tale punto è candidato a punto di massimo e minimo.\\
    In questo caso (con $\mathbb{R}^2$) ho 3 condizioni perché bisogna determinare che la derivata sia rispetto a x che y sia 0, con più variabili aumentano anche il numero di condizioni.
    \item Punti $(x,y) \in \mathbb{R}^2$ tali che $\begin{cases}\varphi(x,y) = 0 \\ \nabla f(x,y) = \lambda \nabla \varphi(x,y) \text{ per }\lambda \in \mathbb{R}\end{cases}$ Deve quindi accadere che il gradiente di $f$ è un multiplo del gradiente di $\varphi$. Anche in questo caso dobbiamo risolvere una sistema di 3 equazioni perché dobbiamo verificare sia la derivata rispetto a x che rispetto a y, cerco quindi in questo caso $\lambda, (x,y)$ tali che (2) valga, e quindi in questo caso 3 equazioni per 3 incognite che è più semplice da risolvere. (Il numero $\lambda$ si dice moltiplicatore di Lagrange)
\end{enumerate}

\begin{example}
Consideriamo la funzione $f(x,y) = x - 2y$ e come insieme $A = \{(x,y) \in \mathbb{R}^2 \::\: x^2 + y^2 \leq 3\}$, $f$ continua su A quindi esiste massimo e minimo. Proviamo come prima cosa proviamo ad utilizzare il metodo classico per la ricerca dei massimi e i minimi, quindi cerchiamo i punti stazionari interni, i punti singolari interni e i punti di bordo.
\begin{enumerate}
    \item Punti stazionari interni. Il gradiente è $\nabla f(x,y) = (1,-2)$ che è costante e quindi possiamo vedere che non ci sono punti stazionari interni.
    \item Punti singolari interni. Dal caso prima possiamo anche dire che non ci sono punti singolari interni.
    \item Punti di bordo. I punti del bordo sarà l'insieme $V = \{(x,y) \in \mathbb{R}^2 \::\: x^2 + y^2 = 3\}$, V è luogo di zeri di $\varphi = x^2 + y^2 - 3$ se $(x,y) \in V \Longleftrightarrow \varphi(x,y) = 0$, andiamo allora ad utilizzare il metodo dei moltiplicatori di Lagrange per cerare i punti di max e min sul bordo.\\
    \begin{enumerate}
        \item Come prima cosa dobbiamo trovare i punti $(x,y)$ tali che $\begin{cases}\varphi(x,y) = 0\\ \nabla \varphi(x,y) = 0\end{cases}$.\\
        Abbiamo già che $\varphi(x,y) = x^2 + y^2 -3$ quindi andiamo a calcolare il gradiente $\nabla \varphi(x,y) = (2x, 2y)$. Possiamo ora scrivere la condizione come $\begin{cases}x^2 + y^2 -3 = 0\\ 2x=0 \\ 2y = 0\end{cases}$ le ultime due però sono incompatibili con la prima e quindi non esistono soluzioni per questo sistema.
        \item Ora dobbiamo provare a trovare soluzioni nel sistema $\begin{cases}\varphi(x,y) = 0 \\ \nabla f(x,y) = \lambda \nabla \varphi(x,y)\end{cases}$, quindi cerco $\lambda,x,y$ tali che questo sistema sia verificato. Quindi risolviamo\\\\
        $\begin{cases}x^2 + y^2 - 3 = 0 \\ 1 = \lambda (2x) \\ -2 = \lambda(2y)\end{cases} = \begin{cases}x = \frac{1}{2x} \\ y = -\frac{1}{\lambda} \\ \frac{1}{4x^2} + \frac{1}{\lambda^2} - 3 = 0\end{cases} = \begin{cases}x = \frac{1}{2x} \\ y = -\frac{1}{\lambda} \\ 1 + 4 = 3 \cdot 4 \lambda^2\end{cases}$\\\\
        l'ultima equazione è uguale a $5 = 12\lambda^2 \Longrightarrow \lambda^2 = \frac{5}{12} \Longrightarrow \lambda = \pm \frac{\sqrt{5}}{2\sqrt{3}}$, allora ho le soluzioni\\
        $\begin{cases}\lambda = \frac{\sqrt{5}}{2\sqrt{3}} \\ x = \frac{\sqrt{3}}{\sqrt{5}} \\ y = -\frac{2\sqrt{3}}{\sqrt{5}}\end{cases}$ e $\begin{cases}\lambda = -\frac{\sqrt{5}}{2\sqrt{3}} \\ x = -\frac{\sqrt{3}}{\sqrt{5}} \\ y = \frac{2\sqrt{3}}{\sqrt{5}}\end{cases}$ Quindi ho 2 soluzioni al sistema (2 terne di soluzioni).\\
        I candidati punti di massimo e minimo di $f$ sul bordo $P_1 = (\frac{\sqrt{3}}{\sqrt{5}}, -\frac{2\sqrt{3}}{\sqrt{5}})$ e $P_2 = (-\frac{\sqrt{3}}{\sqrt{5}}, \frac{2\sqrt{3}}{\sqrt{5}})$, per ora sapere qualche è il massimo e quale è il minimo dovrò valutare $f(P_1) = \frac{\sqrt{3}}{\sqrt{5}} + \frac{4\sqrt{3}}{\sqrt{5}} = \frac{5 \sqrt{3}}{\sqrt{5}} = \sqrt{15}$ che è il max e $f(P_2) = -\frac{\sqrt{5}}{\sqrt{5}} - \frac{4\sqrt{5}}{\sqrt{5}} = -15$ che è il min.\\
        Quindi $P_1$ e punto di massimo e $P_2$ è punto di minimo.
    \end{enumerate}
\end{enumerate}
\end{example}
\begin{wrapfigure}[8]{l}{6cm}
\vspace{-10pt}
    \centering
    \includegraphics[width=5.5cm]{images/ess-moltip-lagrange-1.png}
\end{wrapfigure}
Possiamo ora chiederci se avessimo usato il metodo delle linee di livello avremmo ottenuto lo stesso risultato. Tramite questo metodo abbiamo la linee di livello $\{(x,y) \in \mathbb{R}^2 \::\: x - 2y = \lambda\}$ ovvero $2y = x - \lambda$ oppure $y = \frac{x}{2} - \frac{\lambda}{2}$, quindi le linee di livello sono con $\lambda = 0 \to y = \frac{x}{2}$, con $\lambda = 1 \to y = 1 \to y = \frac{x}{2} - \frac{1}{2}$ e con $\lambda = -1 \to y = \frac{x}{2} + \frac{1}{2}$, possiamo vedere come infatti esce una soluzione analoga ma più complessa. \\

\begin{example}
Consideriamo la funzione $f(x,y) = x \cdot y^2$
\end{example}

\begin{wrapfigure}[4]{r}{6cm}
    \vspace{-30pt}
    \centering
    \includegraphics[width=5.5cm]{images/ess-moltip-lagrange-2.png}
\end{wrapfigure}
\hspace{-15pt}e $A = \{(x,y) \in \mathbb{R}^2 \::\: x^2 + 3y^2 \leq 5\}$, 
A è compatta e $f$ è continua su A quindi esistono massimo e minimo.\\

Andiamo ora a cercare max e min con il metodo classico.
\begin{itemize}
    \item Punti stazionari interni, $\nabla f(x,y) = (y^2, 2xy)$ vediamo ora i punti dove il gradiente si annulla $\nabla f(x,y) = 0$ quindi $y^2 = 0$ e $2xy = 0$ che fa si che $y = 0$, in tutti questi punti $f(x,0) = 0$.
    \item Punti singolari interni non esistono perché il gradiente sull'ellisse considerata non hanno problemi.
    \item Punti di bordo. Punti in cui $\varphi(x,y) = 0$, dove $\varphi(x,y) = x^2 + 3y^2 - 5$, per risolverlo usiamo il metodo dei moltiplicatori di lagrange.
    \begin{enumerate}
        \item Prima proviamo il sistema $\begin{cases}\varphi(x,y) = 0 \\ \nabla \O = 0\end{cases} \Longleftrightarrow \begin{cases}x^2 + 3y^2 - 5 = 0 \\ 2x = 0 \\6y = 0\end{cases}$ dove possiamo vedere che $x = 0$ e $y = 0$ ma questo è incompatibile con la prima equazione quindi questo sistema non ha soluzioni.
        \item Usiamo il secondo sistema \\
        $\begin{cases}\varphi(x,y) = 0 \\ \nabla f(x,y) = \lambda \nabla \varphi(x,y)\end{cases} = \begin{cases}x^2 + 3y^2 -5 = 0\\ y ^2 = 2 \lambda x \\ 2xy = 6 \lambda y\end{cases} = \begin{cases}x^2 + 3y^2 - 5 = 0 \\ xy = 3\lambda y \\ y?2 = 2 \lambda x\end{cases}$ dalla seconda equazione posso ricavare $y (x - 3\lambda) = 0$ che mi fa ottenere $x = 3 \lambda$ e $y = 0$ con il quale ottengo $\begin{cases}x = 3 \lambda \\ y^2 = 2 \lambda \cdot (3 \lambda) = 6 \lambda^2 \\ 9\lambda^2 + 18\lambda^2 = 5\end{cases} = \begin{cases}\lambda^2 = \frac{5}{27} \to \lambda = \pm \frac{\sqrt{5}}{3\sqrt{3}}\\ x = \pm \frac{\sqrt{5}}{\sqrt{3}} \\ y = \pm \sqrt{6}\lambda\end{cases}$ con il quale ottengo 4 punti\\\\
        $P_3 = (\frac{\sqrt{5}}{\sqrt{3}}, \sqrt{6}\frac{\sqrt{5}}{3\sqrt{3}}), P_4 = (-\frac{\sqrt{5}}{\sqrt{3}}, \sqrt{6}\frac{\sqrt{5}}{3\sqrt{3}}), P_5 = (\frac{\sqrt{5}}{\sqrt{3}}, -\sqrt{6}\frac{\sqrt{5}}{3\sqrt{3}}), P_6 = (-\frac{\sqrt{5}}{\sqrt{3}}, -\sqrt{6}\frac{\sqrt{5}}{3\sqrt{3}})$\\\\
        $\begin{cases}y = 0 \\ 2\lambda x = 0 \\ x^2 + 5 = 0\end{cases} = \begin{cases}y = 0 \\ \lambda = 0 \\ x = \pm \sqrt{5}\end{cases}$. Quindi ottengo i due punti $P_1 = (\sqrt{5},0)$ e $P_2 = (-\sqrt{5},0)$.\\\\
    \end{enumerate}
    In totale o quindi 6 punti e da qui dobbiamo calcolare f in questi punti e prendere i massimo per il max ed il minimo per il min. Vediamo così che $f(\frac{\sqrt{5}}{\sqrt{3}}, \pm \frac{\sqrt{10}}{3}) = \frac{10 \sqrt{5}}{9 \sqrt{3}}$ che è il max e $f(-\frac{\sqrt{5}}{\sqrt{3}}, \pm \frac{\sqrt{10}}{3} = -\frac{10\sqrt{5}}{9\sqrt{3}}$ che è il min.
\end{itemize}
Se andassimo a vedere le linee di livello, definite dall'equazione $xy^2 = \lambda$ che possiamo scrivere come $y^2 = \frac{\lambda}{3}$, abbiamo una soluzione analoga a quella con il metodo classico, solo più complessa.


\begin{observation}
Osserviamo che nei punti di massimo o minimo le linee di livello e la linea di bordo (grafico di $\varphi$) sono tangenti. Noi però sappiamo anche che il gradiente è perpendicolare alle linee di livello, e quindi i due gradienti $\nabla f$ e $\nabla \varphi$ saranno paralleli, ecco perché nei punti di massimo e minimo risolviamo il sistema (2) in Lagrange cercando $\nabla f = \lambda \nabla \varphi$.
\end{observation}

\begin{figure}[h!]
\vspace{-10pt}
\centering
\begin{subfigure}{.45\textwidth}
    \centering
    \includegraphics[width=6cm]{images/molt-lagrange-1.png}
\end{subfigure}
\begin{subfigure}{.45\textwidth}
    \centering
    \includegraphics[width=4cm]{images/molt-lagrange-2.png}
\end{subfigure}
\end{figure}
\vspace{-15pt}
\subsubsection{Lagrange geometricamente}
Sia V il bordo di A tale che $f: A \to \mathbb{R}$ con $A \subset \mathbb{R}^2$, supponiamo che $V$ sia il luogo dei zeri di $\varphi(x,y)$, e questo vuol dire che V è linea di livello per $\varphi$.
\begin{enumerate}
    \item Nei punti di massimo e minimo le linee di livello di $f$ e la linea di livello $\lambda=0$ di $\varphi$ sono tangenti (le linee di livello sono tangenti a V nei punti di massimo e minimo).
    \item Sappiamo che $\nabla f$ è sempre perpendicolare alle linee di livello di $f$, e $\nabla \varphi$ è perpendicolare alle linee di livello di $\varphi$ $\Longrightarrow$ i due gradienti devono essere paralleli tra loro.
    \item Quindi $\nabla f$ deve essere un multiplo del $\nabla \varphi \Longrightarrow \nabla f = \lambda \nabla \varphi$.
\end{enumerate}
In conclusione, geometricamente, risolvere il sistema (2) in Lagrange equivale a cercare quei punti di V in cui le linee di livello di $f$ sono tangenti all'insieme di V stesso.\\
Gli esempi in cui il sistema (1) ha soluzioni sono casi molto particolari, vediamo alcuni esempi di questo caso.
\begin{example}
$\varphi(x,y) = x^2 - y^2$, allora $\frac{\partial \varphi}{\partial x}= 2x$ e $\frac{\partial \varphi}{\partial y} = -3y^2$, quindi il sistema (1) è uguale a $\begin{cases}x^2 - y^2 = 0 \\ 2x= 0 \\ -3y^2 = 0\end{cases}$ vediamo che $(x,y) = (0,0)$ è soluzione. Ora però chiediamoci come è fatto V, quindi come è fatto il luogo di zeri di $\varphi$. $V= \{(x,y) \in \mathbb{R}^2 \::\: \varphi(x,y) = 0\}$, $\varphi(x,y) \Longleftrightarrow y^3 = x^2 \Longleftrightarrow y = x^{\frac{2}{3}}$. Il risultato geometrico è una cuspide in cui non so definire la condizione di tangenza.
\end{example}

\begin{example}
Prendiamo $\varphi(x,y) = xy$, $\frac{\partial \varphi}{\partial x} = y$ e $\frac{\partial \varphi}{\partial y} = x$, il sistema (1) è uguale a $\begin{cases}xy = 0\\y = 0\\x=0\end{cases}$ quindi $(0,0)$ è soluzione. $V = \{(x,y) \in \mathbb{R}^2 \::\: xy = 0\}$ corrisponde hai due assi, quindi il punto (0,0) è l'incrocio dei 2 rami. 
\end{example}


\subsection{Teorema di Weristrass generalizzato}
Nell'analisi ad 1 variabile, il teorema di Weristrass generalizzato ci dice che, sia $f: \mathbb{R}\to \mathbb{R}$ continua, supponiamo che $\lim\limits_{x\to -\infty}f(x) = \lim\limits_{x\to +\infty}f(x) = +\infty$ allora esiste $\min\limits_{x \in \mathbb{R}}f(x)$. Quindi questa generalizzazione sta nel fatto che l'insieme è definito in un insieme non compatto, per colmare la mancanza di questa ipotesi dobbiamo fare un ipotesi dei limiti ad $\infty$.\\\\
Ora chiediamoci cosa succede per funzioni $f: \mathbb{R}^n \to \mathbb{R}$. Prima cosa dobbiamo chiederci cosa vuol dire andare all'$\infty$ in $\mathbb{R}^n$, questo vuol dire allontanarsi sempre di più dall'origine cioè che $|x|$ (norma) di $x$ va all'$\infty$. $\lim\limits_{(x,y) \to \infty} \Longleftrightarrow \lim\limits_{x^2 + y^2}\to +\infty \Longleftrightarrow \lim\limits_{|x| \to +\infty} \Longleftrightarrow \lim\limits_{\rho \to \infty}$. Quindi $\lim\limits_{(x,y)\to\infty}f(x,y) = +\infty$ vuol dire che $\forall \: M \in \mathbb{R} \: \exists R > 0$ tale che $f(x,y) \geq 0$ questo $\forall (x,y) \in B_R((0,0))$, quindi in generale se chiamiamo $x \in \mathbb{R}^n$ e $f: \mathbb{R}^n \to \mathbb{R}$ abbiamo che $\lim\limits_{x \to \infty}f(x) = +\infty \Longleftrightarrow \forall \: M \in \mathbb{R}$ (anche molto grande) $\exists \: R > 0$ tale che $f(x) \geq M \: \forall \: x \in B_R(0)^c$ (intorno di $\infty$).\\
Da qui possiamo riformulare il teorema di Weristrass generalizzato nel caso di funzioni in più variabili.
\begin{wrapfigure}[3]{r}{4.5cm}
    \centering
    \includegraphics[width=4.2cm]{images/weristrass-generalizzato-R2.png}
\end{wrapfigure}

\begin{definition}[Weristrass generalizzato minimo]
Sia $f: \mathbb{R}^n \to \mathbb{R}$ una funzione continua, e supponiamo che $\lim\limits_{x \to \infty}f(x) = +\infty$ (che vuol dire $\lim\limits_{|x| \to +\infty}f(x) = +\infty$) allora, esiste $\min\limits_{x\in \mathbb{R}^n}f(x)$.
\end{definition}

\begin{demostration}
Sappiamo che $\lim\limits_{|x|\to +\infty}f(x) = +\infty$, quindi da definizione di limite $\forall \: M \exists \: R > 0$ tale che $f(x) \geq M \:\forall \: x \in (B_R(0))^c$, scelgo $f: \mathbb{R}^n \to \mathbb{R}$ uguale a  $M = f(0) + 1$ allora $\exists \: R > 0$ tale che $f(x) \geq f(0) +1 \forall \: x \in (B_R(0))^c$.\\
Considero $f: \overline{B_R(0)} \to \mathbb{R}$ (la linea sopra indica la palla chiusa) con $f: \{x \in \mathbb{R}^n \::\: |x| \leq R\}$ ora questo insieme che è $\overline{B_R(0)}$ è compatto, allora so per Werstrass classico che $\exists \: \min\limits_{x \in \overline{R_R(0)}}f(x) = m$, ora vogliamo dimostra che $m$ è minimo su tutto $R^n$ e non solo su $\overline{B_R(0)}$. Ci sono due casi.
\begin{enumerate}
    \item Se $|x| \leq R$ (dentro la palla), allora $f(x) \geq m$ per definizione di $m = \min\limits_{B_R(0)}f$.
    \item Se $|x| \geq R$ (fuori la palla), allora $f(x) \geq M = f(0) + 1$ (questo per come ho scelto R) $\geq f(0) \geq m$ e questo perché 0 sta nella palla $|x| \leq R$, quindi anche in questo caso $f(x) \geq m$.
\end{enumerate}
Quindi ho dimostrato che $f(x) \geq m \forall \: x \in \mathbb{R}^n$ e per il Teorema di Weristrass classico sappiamo che $\exists \: x_0 \in B_R(0)$ tale che $f(x_0) = m$ quindi ho trovato il minimo su tutto $\mathbb{R}^n$. $\blacksquare$
\end{demostration}

\hspace{-15pt}Tutto questo può essere enunciato in maniera analoga per il massimo di una funzioni in $\mathbb{R}^n$.
\begin{definition}[Weristrass generalizzato massimo]
Sia $f: \mathbb{R}^n \to \mathbb{R}$ una funzione continua, e supponiamo che $\lim\limits_{|x| \to +\infty}f(x) = -\infty$ allora, esiste $\max\limits_{x\in \mathbb{R}^n}f(x)$.
\end{definition}


\section{Derivate parziali seconde}
Nell'analisi in una variabile per studiare localmente una funzione nell'intorno di un punto stazionario, quindi si faceva $f'(x_0) = 0$ e poi su studiava il segno della derivata seconda $f''(x)$ e concludevamo che se $f''(x_0) > 0$ c'era un minimo locale. Nell'analisi in più variabili non possiamo derivare più volte perché non abbiamo una sola variabile, quindi dobbiamo introdurre le derivate successive per funzioni di più variabili.\\\\
La prima cosa che ci verrebbe in mento è quello di derivare parzialmente più volte, quindi avere $\frac{\partial f}{\partial x_1}, \frac{\partial f}{\partial x_2}, \cdots, \frac{\partial f}{\partial x_n}$ con $f: \mathbb{R}^n \to \mathbb{R}$ che sono le derivate parziali prime e, per iniziare, come fare le derivate parziali seconde.

\begin{example}
$f(x,y) = x^2 + y^3 + x^4 \cdot y^5$, $\frac{\partial f}{\partial } = 2x + 4x^3 y^5 = g(x,y)$ e $\frac{\partial g}{\partial y} = 3y^2 + 5x^4 y^4 = h(x,y)$ da qui dobbiamo partire da $g(x,y)$, $h(x,y)$ e ricalcolare le derivate, quindi avrei per esempio per $g(x,y)$ le derivate $\frac{\partial g}{\partial x}$ e $\frac{\partial g}{\partial y}$ che però a loro volta si potrebbero scrivere come $\frac{\partial \partial f}{\partial \partial x} = \frac{\partial^2 f}{\partial x^2} = f_{xx}$ e $\frac{\partial \partial f}{\partial y \partial x} = \frac{\partial^2 f}{\partial y \partial x} = f_{xy}$. Nel caso di $h(x,y)$ invece ho $\frac{\partial h}{\partial x} = \frac{\partial \partial f}{\partial x \partial y} = \frac{\partial^2 f}{\partial x \partial y} = f_{xy}$ e $\frac{\partial h}{\partial y} = \frac{\partial \partial f}{\partial y \partial x} = \frac{\partial^2 f}{\partial y^2} = f_{yy}$.\\
Per questo esempio, ho quindi nel caso di une derivata parziali seconde 4 derivate prime. In maniera più generale $f: \mathbb{R}^n \to \mathbb{R} \Longrightarrow n \cdot n = n^2$ derivate parziali seconde.
\end{example}

\begin{observation}
In generale per una funzioni di n variabili ci sono n derivate parziali prime e $n^2$ derivate parziali seconde. Il metodo di calcolo è sempre lo stesso.
\end{observation}

\begin{example}
Tornando all'esempio scritto sopra ho che $\frac{\partial f}{\partial x} = f_x = 2x + 4x^3 y^5$ che poi va $f_{xx} = 2 + 12x^2y^5$ e $f_{xy} = 20x^3 \cdot y^4$, mentre $\frac{\partial f}{\partial y} = f_y = 3y^2 + 5^4 y^4$ mentre le derivate seconde sono $f_{yy} = 6y + 20x^4y^3$ e $f_{yx} = 20x^3y^4$.
\end{example}

\hspace{-15pt}Vediamo dall'esempio sopra che se derivo $f_{yx}$ e $f_{xy}$ abbiamo due risultati uguali, posso quindi dire che in generale questi valori sono sempre uguali. Da qui possiamo enunciare un teorema che parla dell'ordine di derivazione, enunciamo questo teorema per il caso di $f: \mathbb{R}^n \to \mathbb{R}$.

\begin{theorem}[Inversione dell'ordine di derivazione]
Se $f_{xy}$ e $f_{yx}$ esistono in un intorno del punto $(x_0, y_0) \in \mathbb{R}^n$ e sono continue nel punto $(x_0,y_0)$ allora coincidono ovvero $f_{xy}(x_0,y_0) = f_{yx}(x_0,y_0)$.
\end{theorem}
\hspace{-15pt}Con questo teorema possiamo dire che non conta l'ordine con cui derivo nel calcolare le derivate parziali successive. Questo vale anche per le derivate parziali non solo seconde, ma anche le terze, per esempio $f_{xyx} = f_{yxx}$ e $f_{xyy} = f_{xyy}$.

\subsection{Matrice Hessiana}
Sappiamo che le derivate parziali prime le possiamo derivare in $\nabla f = (f_x, f_y)$, il vettore gradiente e quello che ci da la condizione di stazionarietà di un punto perché corrisponde a $\nabla f(x_0,y_0) = 0$ di $(x_0,y_0)$, le derivate seconde allora le organizzo in una matrice che viene detta \textbf{matrice Hessiana}.
\begin{definition}[Matrice Hessiana]
Si dice \textbf{matrice Hessiana} la matrice formata dalle derivate seconde. Quindi nel caso di $f: \mathbb{R}^2\to \mathbb{R}$ e nel caso di $f: \mathbb{R}^3\to \mathbb{R}$ ho:
\[HF = \begin{bmatrix}f_{xx} & f_{xy} \\ f_{yx} & f_{yy}\end{bmatrix} \hspace{.3cm} e \hspace{.3cm} HF = \begin{bmatrix}f_{xx} & f_{xy} & f_{xz} \\ f_{yx} & f_{yy} & f_{yz} \\ f_{zx} & f_{zy} & f_{zz}\end{bmatrix}\]
\end{definition}

\begin{observation}
Se le derivate parziali seconde esistono e sono continue allora l'ordine di derivazione non importa e HF è simmetrica.
\end{observation}
\hspace{-15pt}L'idea è che tutte le condizioni che in analisi in 1 dimensione coinvolgono il segno della derivata seconda, nell'analisi in pi§ variabili coinvolgono la segnatura della matrice hessiana.

\subsection{Studio di un punto stazionario}
Sia $f: \mathbb{R}^2 \to \mathbb{R}$, ($f: \mathbb{R}^n \to \mathbb{R}$) e sia $(x_0, y_0) \in \mathbb{R}^2$ tale che $\nabla f(x_0, y_0) = 0$ ($\Longleftrightarrow (x_0, y_0)$ punto stazionario), abbiamo i seguenti criteri per decidere se questo punto è minimo o massimo locale.
\begin{enumerate}
    \item Se $Hf = \begin{bmatrix}f_{xx} & f_{xy} \\ f_{yx} & f_{yy}\end{bmatrix}$ (ricordiamo che mettiamo le derivate parziali seconde e che $Hf$ è simmetrica se $f$ è abbastanza regolare, ai sensi del Teorema di inversione di derivazione) è definita positiva allora $(x_0, y_0)$ è punto di minimo locale (una matrice è definita positiva $\Longleftrightarrow$ tutti i suoi autovalori sono positivi (++))
    \item Se $Hf$ è definita negativa (e cioè entrambi gli autovalori sono strettamente negativi $\Longleftrightarrow (--)$) allora $(x_0, y_0)$ è un punto di max locale.
    \item Se $Hf$ è indefinita (quindi se esistono due autovalori discordi $(+-)$) allora $(x_0, y_0)$ è un punto di sella, questo vuol dire che rispetto a questo punto esistono alcune direzione dove la funzione cresce ed altri dove decresce.
    \item Se $Hf$ è degenere (ovvero $det\: Hf = 0$, quindi 0 è tra gli autovalori) non posso concludere niente.
\end{enumerate}

\begin{observation}
La matrice Hessiano fornisce informazioni locali, cioè di permette di concludere l'esistenza di massimo o minimi locali (vicino al punto $(x_0,y_0)$) ma non ci da informazioni globali.
\end{observation}

\hspace{-15pt}Ricordiamo che nell'analisi in 1 dimensione i minimi ed i massimi li avevamo
\begin{itemize}
    \item Se in $f: \mathbb{R}\to \mathbb{R}$ $x_0$ è un punto di minimo locale, allora $f'(x_0) = 0$ e $f''(x_0) \geq 0$ (supponendo che in $x_0 \exists \: f',f''$)
    \item Se $x_0$ è un punto di massimo locale allora $f'(x_0) = 0$ e $f''(x_0) \leq 0$.
\end{itemize}
Nell'analisi in più variabili abbiamo un concetto analogo. Se $f: \mathbb{R}^2 \to \mathbb{R}$ e $(x_0, y_0) \in \mathbb{R}^2$ è punto di minimo locale, allora $\nabla f(x_0, y_0) = 0$ e $Hf(x_0, y_0) \geq 0$ (matrice semi definita positiva, gli autovalori sono non negativi). Nel caso di massimo locale la condizione $\nabla f(x_0, y_0) = 0$ rimane e abbiamo $Hf(x_0, y_0) \leq 0$.\\\\
Riassumendo, condizione sufficiente per un minimo locale è che $\nabla f = 0$ e $Hf > 0$, mentre avere un minimo locale implica che $\nabla f = 0$ e $Hf \geq 0$ (che è condizione necessaria). Analogamente si può dire per il massimo locale.\\
Ad esempio se la matrice hessiana ha autovalori 0,3 ($Hf \leq 0$), posso dire solamente che non è di massimo (perché $Hf \leq 0$).


\begin{figure}[h!]
\vspace{-10pt}
\centering
\begin{subfigure}{.3\textwidth}
    \centering
    \includegraphics[width=3cm]{images/min-hessiano.png}
    \caption{Minimo}
\end{subfigure}
\begin{subfigure}{.3\textwidth}
    \centering
    \includegraphics[width=3cm]{images/max-hessiano.png}
    \caption{Massimo}
\end{subfigure}
\begin{subfigure}{.3\textwidth}
    \centering
    \includegraphics[width=3cm]{images/sella-hessiano.png}
    \caption{Punto di sella}
\end{subfigure}
\end{figure}

\vspace{-10pt}
\begin{example}
Data $f(x,y) = x^2 + y^4$, $(x_0, y_0) = (0,0)$, $\nabla f(x,y) = (\frac{\partial f}{\partial x}(x,y), \frac{\partial f}{\partial y}(x,y)) = (2x, 4y^3)$.\\
$\nabla f(x,y) = 0 \Longleftrightarrow \begin{cases}2x = 0\\ 4y^3 = 0\end{cases} \Longleftrightarrow (x,y) = (0,0) \Longrightarrow (0,0)$ e punto stazionario, per studiare localmente (vicino a (0,0)) la funzione $f$ si calcoliamo la matrice hessiana. $Hf = \begin{bmatrix}f_{xx} & f_{xy} \\ f_{yx} & f_{yy}\end{bmatrix} = \begin{bmatrix}2 & 0 \\0 & 12y^2\end{bmatrix}$, $Hf(0,0) = \begin{bmatrix}2&0\\0&0\end{bmatrix}$, ha un autovalore positivo $\lambda_1 = 2$ ed un autovalore nullo $\lambda_2 = 0$, quindi $Hf$ è semi definita positiva $Hf \geq 0$, l'unica cosa che posso dire è che $(0,0)$ non è un punto di massimo locale, se $(0,0)$ fosse punto di massimo locale infatti avremmo $hf(0,0) \leq 0$. Non possiamo concludere nemmeno se è un minimo perché ci dovrebbe essere un $>$ e non un $\geq$, se però andiamo ad analizzare la funzione vediamo che $f(x) = x^2 + y^4 \geq 0$ e $f(0,0) = 0$ si vede a mano che $(0,0)$ è un punto di minimo globale quindi $f(x,y) > 0 = f(0,0)$ se $(x,y) \neq (0,0)$ con $(x,y) \in \mathbb{R}^2$.
\end{example}

\begin{example}
Consideriamo la funzione $f(x,y) = x^2 - y^4$, anche in questo caso abbiamo che $\nabla f(x,y) = (2x, -4y^3)$, $(0,0)$ è un punto stazionario quindi $\nabla f (0,0) = 0$.\\
$Hf (0,0) = \begin{bmatrix}2 & 0 \\ 0 & 0\end{bmatrix} \geq 0$, gli autovalori sono $\lambda_2 = 2$ $\lambda_2 = 0$ quindi $Hf$ è semi definita positiva in $(0,0)$, quindi il punto $(0,0)$ on è massimo locale. In questo caso l'origine $(0,0)$ non è ne un massimo ne un minimo, questo perché se $y= 0 \to f(x,0) = x^2 \geq 0$ e $x = 0 \to f(0,y) = -y^4 \leq 0$ quindi se mi muovo su l'asse y la funzione cresce mentre su l'asse y la funzione decresce. Vicino a $(0,0)$ esiste punti in cui $f(x,y) > 0$ a punti in cui $f(x,y) < 0$ quindi $(0,0)$ non è ne massimo ne minimo.
\end{example}

\begin{example}
Dato $f(x,y) = x^3 + x^2y^2 + y^4$, $f_x = 3x^2 + 2y^2$, $f_y = 2x^2y + 4y^3$, e $\nabla f = (f_x, f_y)$, quindi $\nabla f(0,0) = 0$ quindi $(0,0)$ è punto stazionario.\\
$Hf(0,0) = \begin{bmatrix}6x + 2y^2 & 4xy \\4xy & 12y^2\end{bmatrix} = \begin{bmatrix}0 & 0 \\ 0 & 0\end{bmatrix}$.
\end{example}

\newpage
\section{Superfici}
Adiamo ora a definire cos'è una superficie in $\mathbb{R}^3$. Per definire queste superfici ci sono 3 approcci, il cartesiano, l'implicito ed il parametrico.

\subsection{Superficie cartesiana}
Consideriamo $A \subset \mathbb{R}^2$ una $f: A \to \mathbb{R}$ la superficie è un grafico di una funzione quindi la possiamo definire come:
\[S = \{(x,y,z) \in \mathbb{R}^3 \::\: z = f(x,y) \: (x,y) \in A\}\]
\begin{example}
Prendiamo l'insieme $A = [-1,1] \times [-1,1] \subset \mathbb{R} =$ piano $xy$ e consideriamo $f(x,y) = x^2 + y^2$. La superficie $S = \{(x,y,z) \in \mathbb{R}^3\} = $ la parte di paraboloide che si proietta sul quadrato A.
\end{example}

\subsection{Superficie implicita}
Definire una superficie in maniera implicita vuol dire che $S$ superficie $\subset \mathbb{R}^3$, S è il luogo di zeri di una funzione di tre variabili $\varphi(x,y,z)$.
\[S = \{(x,y,z) \in \mathbb{R}^2 \::\: \varphi(x,y,z) = 0\}\]
Si dice anche che $\varphi(x,y,z) = 0$ è l'equazione della superficie, superficie "implicita", quindi non viene ricavata una variabile rispetto alle altre.

\begin{example}
$S = \{(x,y,z) \in \mathbb{R}^3 \::\: x^2 + y^2 + z^2 - 4 = 0\} = \{(x,y,z) \in \mathbb{R}^3 \::\: x^2 + y^2 + z^2 = 4\}$ sfera $\subset \mathbb{R}^3$
\end{example}

\subsection{Superficie parametrica}
Si considera un insieme $A \subseteq \mathbb{R^2}$ (insieme dove variano i parametri, che in questo caso sono 2), e consideriamo tre funzioni date che chiamiamo $x(uv), y(uv), z(u,v)$ con $(u,v) \in A$, superficie parametrica si definisce come:
\[S = \{(x,y,z) \in \mathbb{R}^3 \::\: (x,y,z) = (x(u,v), y(u,v), z(u,v)) \text{ al variare di }(u,v) \in A\}\]

\begin{example}
Un esempio di superficie in $\mathbb{R}^3$ è il piano. Un piano in $\mathbb{R}^3$ ha equazioni parametrica del tipo: $(x_0,y_0,z_0) + t(v_1,v_2,v_3) + s(w_1, w_2, w_3)$, dove $(x_0,y_0,z_0)$ sono i punti per cui passa il piano mentre $t(v_1,v_2,v_3), s(w_1, w_2, w_3$ sono i vettori che generano il piano.\\
$\begin{bmatrix}x_0 + tv_1 + sw_1\\ y_0 + tv_2 + sw_2\\ z_0 + tv_3 + sw_3\end{bmatrix} = \begin{bmatrix}x(t,s)\\y(t,s)\\z(t,s)\end{bmatrix}$ in questo caso t e s sono i parametri $(t,s) \in \mathbb{R}^2$
\end{example}

\begin{example}
Prendiamo una superficie definita come $s = \{(1 + u^2, u \cdot v, u^2 + v^2) \::\: u^2 + v^2 \leq 3\}$ con $(u,v) \in \overline{B_{\sqrt{3}}(0)} \::\: u^2 + v^2 \leq 3$.
\end{example}

\subsection{Legami tra le definizioni}
Fra questi 3 approcci di definizione di una superficie c'è un legame. Si può dire infanti che tutte le superfici cartesiane sono in realtà superfici parametriche $S = \{(x,y,z) \in \mathbb{R}^3 \::\: z = f(x,y) \text{ con }(x,y) \in A\} = \{(u,v, f(u,v)) \::\: (u,v) \in A\}$, la funzione $f(u,v)$ è definita dai primi due parametri.

\begin{example}
Facciamo un esempio partendo da un cilindro con asse lungo l'asse z e raggio
\end{example}
\begin{wrapfigure}[6]{r}{5cm}
    \vspace{-5pt}
    \centering
    \includegraphics[width=4.5cm]{images/ess-legamu-defi.png}
\end{wrapfigure}
\hspace{-15pt}di base = 1, altezza = 2, e che si appoggi sul piano xy. 
\begin{itemize}
    \item Non possiamo descriverla come superficie cartesiana.
    \item Possiamo nemmeno descrivere il nostro cilindro come superficie implicita aggiungendo una limitazione, infatti dobbiamo vedere il cilindro come $C = \{(x,y,z) \in \mathbb{R}^3 \::\: x^2 + y^2 = 1, 0\leq z \leq 2\}$, riusciamo quindi a scriverlo come superficie implicita ma con una limitazioni ad una delle variabili ($0\lq z \leq 2$).
\end{itemize} 
\newpage
\begin{itemize}
    \item Possiamo scriverla anche come superficie parametrica, per farlo usiamo come parametri $z$ ed il $\Theta$ delle coordinate polari $(\rho \equiv 1$ perché la circonferenza di base ha raggio 1), quindi scriviamo $C = \{(x,y,z) \in \mathbb{R}^3 \::\: (x,y,z) = (\cos{\Theta}, \sin{\Theta}, z)\}$ con $\{0 \leq \Theta \leq 2\pi$ e $0\leq z \leq 2\}$.
\end{itemize}


\begin{example}
Partiamo da la semisfera di raggio 3 che sta sull'asse xy. Vediamo di descrivere le superficie.\\
\end{example}
\begin{wrapfigure}[6]{l}{5cm}
    \vspace{-25pt}
    \centering
    \includegraphics[width=4.5cm]{images/ess-legamu-defi-2.png}
\end{wrapfigure}
\hspace{-15pt}Possiamo descriverla come superficie parametrica usando le coordinate sferiche e mettendo $S = \{(3\cos{\Theta} \cdot \sin{\phi}, 3\sin{\Theta}\cdot\sin{\phi}, 3\sin{\Theta}\phi) \text{ con }0\leq \Theta \leq 2\pi, 0\leq \phi \leq \frac{\pi}{2}\}$, abbiamo che $3\cos{\Theta}, 3\sin{\Theta}$ sono le coordinate polari del punto proiettato sul piano xy, le 3 coordinate che escono sono le coordinate sferiche.\\\\


\hspace{-15pt}Vediamo ora, dopo aver descritto una superficie, di definire il piano tangente ad una superficie, ed il vettore normale ad una superficie (con vettore normale si intende il vettore perpendicolare al piano tangente). Prendiamo in considerazione una superficie cartesiana decritta dall'equazione $z = f(x,y)$, quindi $S = \{(x,y,z) \in \mathbb{R}^3 \::\: z = f(x,y)\}$, data questa superficie vediamo come scrivere il piano tangente nel punto $(x_0,y_0,f(x_0,y_=))$. \\\\
Sappiamo che possiamo scrivere $f(x_0 + h, y_0 + k) = f(x_0, y_0) + f_x(x_0,y_0) \cdot h + f_y(x_0,y_0) \cdot k + o(\sqrt{h^2 + k^2})$, dove $f(x_0, y_0) + f_x(x_0,y_0) \cdot h + f_y(x_0,y_0) \cdot k$ ci da l'equazione del piano tangente, se poniamo $x_0 + h = x$ e $y_0 + k = y$ possiamo sostituire ed otteniamo $h = x - x_0$ e$k = y - y_0$, quindi otteniamo $z = f(x_0,y_0) + f_x(x_0,y_0) \cdot (x-x_0) + f_y(x_0,y_0)(y-y_0)$ che è l'equazione del piano tangente.\\

\begin{wrapfigure}[7]{r}{5.3cm}
    \vspace{-10pt}
    \centering
    \includegraphics[width=4.5cm]{images/sup-implicita.png}
\end{wrapfigure}
Se invece partiamo da una superficie implicita quindi $s = \{(x,y,z) \in \mathbb{R}^3 \::\: \varphi(x,y,z) = 0\}$, in questo caso ci accorgiamo che il luogo di zeri, cioè S, è in realtà un insieme di livello per $\varphi \Longrightarrow$ S è perpendicolare al gradiente di $\varphi$. Quindi il piano tangente ad S in un punto $(x_0,y_0,z_0)$ è il piano che passa per $(x_0,y_0,z_0)$ ed è perpendicolare a $\nabla \varphi(x_0,y_0,z_0)$. Di conseguenza il piano ha equazione $(x,y,z) \in$ piano se $<\nabla \varphi(x_0,y_0,z_0),(x,y,z)> = d$ dove d è una costante scelta in modo che il piano passi per $(x_0,y_0,z_0)$. Quindi l'equazione $<\nabla \varphi(x_0,y_0,z_0),(x,y,z)> = d$ in modo esplicito diventa $\varphi_x (x_0,y_0,z_0) \cdot x + \varphi_y(x_0,y_0,z_0) \cdot y + \varphi_z(x_0,y_0,z_0) \cdot z = \varphi_x (x_0,y_0,z_0) \cdot x_0 + \varphi_y (x_0,y_0,z_0)y_0 + \varphi_z (x_0,y_0,z_0) \cdot x_0$.
\newpage
\section{Cheatsheet}

\begin{table}[h!]
	\setlength{\tabcolsep}{7pt}
	\renewcommand{\arraystretch}{2}
	\centering
	\begin{tabular}{|c|P{180px}|}
		\hline
		\textbf{Disuguaglianza triangolare} & \multirowcell{2}{ $|a + b| \leq |a| + |b|$\\$||a| + |b|| \leq |a - b| $} \\
		\hline
		\textbf{Rapporto incrementale} & $\frac{\Delta y}{\Delta x} = \frac{f(x_1)-f(x_0)}{x_1-x_0}$ \\
		\hline
		\textbf{Funzione pari e dispari} & \multirowcell{2}{$f(-x) = f(x)$ \\ $f(-x) = -f(x)$} \\
		\hline
		\textbf{Boh} & $f(x)^{g(x)} = e^{{\log{(f(x)}^ {g(x)})}} = e^{ g(x) \cdot \log{(f(x))}}$ \\
		\hline
	\end{tabular}
	\caption{Formule varie}
\end{table}

\begin{table}[h!]
	\renewcommand{\arraystretch}{2.15}
	\centering
	\begin{tabular}{|c|P{210px}|}
		\hline
		\textbf{Zeri} &  \multirowcell{2}{\color{blue} $f: [a, b] \longrightarrow \mathbb{R}$ continua\\ \color{red}$f(a) \cdot f(b) < 0 \Longrightarrow \exists c \in (a, b) : f(c) = 0$ } \\
		\hline
		\textbf{Confronto} & \multirowcell{5}{\color{blue}$A \subset \mathbb{R}$, $x_0 \in Acc(x)$, $f,g: A \to \mathbb{R}$ \\ \color[HTML]{00A64F}$\lim\limits_{x\to x_0}f(x) = l_1 \wedge \lim\limits_{x\to x_0}g(x) = l_2 \wedge $ \\ \color[HTML]{00A64F}$\exists U \text{ intorno } x_0:x \in U \cap A \setminus \{x_0\} $ \\ \color{red}$f(x) \leq g(x) \Longrightarrow l_1 \leq l_2$ \\ \color{red}$f(x) \geq g(x) \Longrightarrow l_1 \geq l_2$}\\
		\hline
		\textbf{Weirstrass} & \multirowcell{7}{
			\color{blue}$a,b \in \overline{\mathbb{R}}$, $f: (a,b) \to \mathbb{R}$ continua:\\
			\color[HTML]{00A64F}$\exists \: \lim\limits_{x \to a}f(x) = l_1 \wedge \exists \lim\limits_{x \to b}f(x) = l_2$\\
			\color{red}$f$ lim. inf. $\Longleftrightarrow$ $l_1 \neq -\infty \wedge l_2 \neq -\infty$ \\
			\color{red}$f$ lim. sup. $\Longleftrightarrow$ $l_1 \neq +\infty \wedge l_2 \neq +\infty$\\
			\color{red}$f$ lim. $\Longleftrightarrow$ $l_1 \in \mathbb{R} \wedge l_2 \in \mathbb{R}$\\
			\color{red}$f$ ha min $\Longleftrightarrow \: \exists x_0 \in (a,b) : f(x_0) \leq min\{l_1, l_2\}$\\
			\color{red}$f$ ha max $\Longleftrightarrow \: \exists x_0 \in (a,b) : f(x_0) \geq max\{l_1, l_2\}$
		}\\
		\hline
		\textbf{Carabinieri} & Se due funzioni hanno lo stesso limite ed una è inferiore all'altra, se esiste una $g(x)$ in mezzo a queste due, avrà lo stesso limite\\
		\hline
		\textbf{Lagrange} & \\
		\hline
	\end{tabular}
	\caption{Teoremi}
\end{table}

\begin{definition}[Continuità in un punto]
	\begin{equation}
		\lim_{x \leftarrow {x_0}^+} f(x) = \lim_{x \leftarrow {x_0}^-} f(x) = f(x_0)
	\end{equation}
\end{definition}

\begin{table}[h!]
	\setlength{\tabcolsep}{7pt}
	\renewcommand{\arraystretch}{1.5}
	\centering
	\begin{tabular}{|c c c|}
		\hline
		$[1]$ $(+\infty) + (-\infty)$ & $[2]$ $(-\infty) + (+\infty)$ & $[3]$ $0 \cdot (\pm \infty)$ \\
		$[4]$ $(\pm \infty)^0$ & $[5]$ $(0^+)^0$ & $[6]$ $(1)^{\pm \infty}$\\ 
		\hline
	\end{tabular}
	\caption{Forme indeterminate}
\end{table}

\begin{itemize}
	\item Se $\lim\limits_{x\to x_0}f(x) = 0^+ \Longrightarrow \lim\limits_{x\to x_0}\frac{1}{f(x)} = +\infty$.
	\item Se $\lim\limits_{x\to x_0}f(x) = 0^- \Longrightarrow \lim\limits_{x\to x_0}\frac{1}{f(x)} = -\infty$.
	\item Se $\lim\limits_{x\to x_0}f(x) = +\infty \Longrightarrow \lim\limits_{x\to x_0}\frac{1}{f(x)} = 0^+$.
	\item Se $\lim\limits_{x\to x_0}f(x) = -\infty \Longrightarrow \lim\limits_{x\to x_0}\frac{1}{f(x)} = 0^-$.
	\item Se $\lim\limits_{x\to x_0}f(x) = l$ con $l \neq 0, \pm\infty \Longrightarrow \lim\limits_{x\to x_0}\frac{1}{f(x)} = \frac{1}{l}$.
\end{itemize}

\begin{table}[h!]
	\setlength{\tabcolsep}{7pt}
	\renewcommand{\arraystretch}{1.5}
	\centering
	\begin{tabular}{|c c|c|}
		\hline
		$\lim\limits_{x\to +\infty}x^n = +\infty$ & $\lim\limits_{x\to +\infty}\frac{1}{x^n} = \frac{1}{+\infty} = 0$ & $\lim\limits_{x\to +\infty}a^x = +\infty$ e $\lim\limits_{x\to -\infty}a^x = 0^+$ se $a \geq 1$ \\\hline
		$\lim\limits_{x\to +\infty}e^x = +\infty$ & $\lim\limits_{x\to -\infty}e^x = 0^+$ & $\lim\limits_{x\to +\infty}a^x = 1$ e $\lim\limits_{x\to -\infty}a^x = 1$ se $a = 1$  \\\hline
		$\lim\limits_{x\to 0^+}\log(x) = -\infty$ & $\lim\limits_{x\to +\infty}\log(x) = +\infty$ & $\lim\limits_{x\to +\infty}a^x = 0^+$ e $\lim\limits_{x\to -\infty}a^x = +\infty$ se $0 < a < 1$ \\
		\hline
	\end{tabular}
	\vspace{-5pt}
	\caption{Limiti fondamentali}
\end{table}

Limite di un polinomio che tende ad infinito: raccoglimento
Limite di rapporto di polinomi che tende ad infinito: raccoglimento

\begin{table}[h!]
	\centering
	\setlength{\tabcolsep}{10pt}
	\renewcommand{\arraystretch}{2.5}
	\begin{tabular}{|c|c|}
		\hline
		$\lim\limits_{x\to 0}\frac{\sin(x)}{x} = 1$ & $\lim\limits_{x\to 0} \frac{1-\cos(x)}{x^2} = \frac{1}{2}$ \\\hline
		$\lim\limits_{x\to 0}\frac{e^x-1}{x} = 1$ & $\lim\limits_{x\to 0}\frac{\log(1+x)}{x} = 1$\\
		\hline
	\end{tabular}
	\caption{Limiti notevoli}
\end{table}

\begin{itemize}
	\item $\lim\limits_{x\to +\infty}\frac{\log(x)}{x} = \frac{+\infty}{+\infty}$ forma indeterminata.\\
	Eseguiamo un cambio di variali con $y = \log(x)$ e $x = e^y$. Se $x\to +\infty \Longrightarrow y = \log(x) \to +\infty$\\\\
	Torna che $\lim\limits_{x \to +\infty}\frac{\log(x)}{x} = \lim\limits_{y\to +\infty}\frac{y}{e^y} = 0$
	\item $\lim\limits_{x\to +\infty}\frac{(\log(x))^\beta}{x^\alpha}$ con $\alpha, \beta \in \mathbb{R}$ e $\alpha, \beta > 0$\\
	Possiamo risolvere con un cambio di variabile $y = \log(x)$, $x = e^y$ e se $x \to +\infty \Longrightarrow y\to +\infty$\\\\
	Quindi $\lim\limits_{x\to +\infty}\frac{(\log(x))^\beta}{x^\alpha} = \lim\limits_{y \to +\infty}\frac{y^\beta}{(e^y)^\alpha} = \lim\limits_{y \to +\infty}\frac{y^\beta}{e^{y\cdot\alpha}} = 0$  (l'esponenziale cresce più velocemente).
	\item $\lim\limits_{x\to 0^+}x\log(x) = 0 \cdot (-\infty)$ forma indeterminata.\\
	Facciamo il cambio di variabile $y = \log(x)$, e $x = e^y$ con $x\to 0^+ \Longrightarrow y\to -\infty$.\\\\
	$\lim\limits_{x\to 0^+}x\log(x) = \lim\limits_{y\to -\infty}e^y \cdot y = 0^+ \cdot (-\infty)$ ancora indeterminata.\\
	Possiamo fare un altro cambio di varibile con $z = -y$, e $y = -z$ e se $y \to -\infty \Longrightarrow z \to +\infty$\\\\
	$\lim\limits_{y\to -\infty}e^y \cdot y = \lim\limits_{z\to +\infty}e^{-z} \cdot (-z) = \frac{-z}{e^z} = 0$
	\item $\lim\limits_{x\to 0^+}x^\alpha \cdot \log(x)$ con $\alpha > 0$.\\
	Cambio di variabile con $y = x^\alpha$, e $x = y^{\frac{1}{\alpha}}$ e con $x\to 0^+ \Longrightarrow y\to^+$\\\\
	$\lim\limits_{x\to 0^+}x^\alpha \cdot \log(x) = \lim\limits_{y\to 0^+}y \cdot \log(y^{\frac{1}{\alpha}}) = \lim\limits_{y\to 0^+}\frac{y}{\alpha} \cdot \log(y) = \frac{1}{\alpha}\lim\limits_{y\to 0^+} y \cdot \log(y) = 0$ per l'esempio sopra.
\end{itemize}

\begin{center}
	\vspace{-8pt}
	$\lim\limits_{x\to 0}\frac{f(x)}{g(x)} = 0$ allora $f(x) = o(g(x))$
\end{center}

\subsection{o-piccolo}
\begin{equation}
	f(x) = o(g(x)) \Leftrightarrow \lim_{x \rightarrow x_0} \frac{f(x)}{g(x)} = 0
\end{equation}
Dato un $A \subset \mathbb{R}$, un $x_0 \in Acc(A)$, e due funzioni $f,g: A \to \mathbb{R}$ e con tutti gli o-piccoli che si intendono per $x\to x_0$, valgono le seguenti proprietà.
\begin{enumerate}
	\item $f(x) \cdot o(g(x)) = o(f(x) \cdot g(x))$.
	\item Se $k \in \mathbb{R}$, e $k \neq 0 \Longrightarrow o(k \cdot g(x)) = o(g(x))$.
	\item $o(g) + o(g) = o(g)$. 
	\item Se $\lim\limits_{x\to x_0}f(x) = 0 \Longrightarrow f(x) \cdot g(x) = o(g(x))$.
	\item Se $\lim\limits_{x\to x_0}f(x) = 0 \Longrightarrow o(g) + o(f \cdot g) = o(g)$.
	\item $o(o(g)) = o(g)$.
	\item $o(f + g) = o(f) + o(g)$.
	\item $o(g) \cdot o(f) = o(f \cdot f)$.
\end{enumerate}

\begin{definition}[O-grande]
	Dato $A \subset \mathbb{R}$, $x_0 \in Acc(A)$, e $f,g: A \to \mathbb{R}$. Se $\exists M \in \mathbb{R}$ t.c. $|f(x)| \geq M \cdot |g(x)|    \forall x \in U \cap A \setminus \{x_0\}$ dove $U$ è un intorno di $x_0$, allora si dice che $f$ è O-grande di $g$ per $x$ che tende a $x_0$ e si scrive $f(x) = O(g(x))$ per $x\to x_0$.
\end{definition}

\begin{definition}
	Dato $A \subset \mathbb{R}$, $x_0 \in Acc(A)$, e $f,g: A \to \mathbb{R}$ infinitesime per $x\to x_0$ (cioè $\lim\limits_{x\to x_0}f(x) = 0$ e $\lim\limits_{x\to x_0}g(x) = 0$). Se esistono $L, \alpha \in \mathbb{R}$ con $L \neq 0$ t.c. $f(x) = L \cdot (g(x))^\alpha + o((g(x))^\alpha)$ per $x\to x_0$ si dice che $f$ è infinitesima di ordine $\alpha$ rispetto a $g$ con parte principali $L(g(x))^\alpha$ per x che tende a $x_0$.\\
	Stessa definizioni del caso in. cui $f$ e $g$ siano divergenti (cioè $\lim\limits_{x\to x_0}f(x) = \pm\infty$ e $\lim\limits_{x\to x_0}g(x) = \pm\infty$)
\end{definition}

Formula per l'asintoto obliquo

Derivabilità

\begin{table}[h!]
	\setlength{\tabcolsep}{5pt}
	\renewcommand{\arraystretch}{2.2}
	\centering
	\begin{tabular}{|c|c|}
		\hline
		$e^x$ & $1 + x + \frac{x^2}{2!} + \frac{x^3}{3!} + \frac{x^4}{4!} + ... + \frac{x^n}{n!} + o(x^n)$  \\
		$\log(1+x)$ & $x - \frac{x^2}{2} + \frac{x^3}{3} - \frac{x^4}{4} + \frac{x^5}{5} + ... + (-1)^{n-1}\frac{x^n}{n} + o(x^n)$ \\
		$\sin(x)$ & $x - \frac{x^3}{3!} + \frac{x^5}{5!} - \frac{x^7}{7!} + ... + (-1)^n \frac{x^{2x+1}}{(2n+1)!} + o(x^{2n+2})$ \\
		$\cos(x)$ & $1 - \frac{x^2}{2!} + \frac{x^4}{4!} - \frac{x^6}{6!} + ... + (-1)^n\frac{x^2n}{(2n)!} + o(x^{2n+1})$ \\
		$\tan(x)$ & $x + \frac{x^3}{3} + \frac{2}{15}x^5 + o(x^6)$\\
		$\arctan(x)$ & $x - \frac{x^3}{3} + \frac{x^5}{5} - \frac{x^7}{7} + ... + (-1)^n\frac{x^{2x+1}}{(2n + 1)} + o(x^{2n+2})$\\
		$\arcsin{x}$ & $x + \frac{x^3}{6} + \frac{3}{40}x^5 + o(x^6)$\\
		$\sqrt{1+x}$ & $1 + \frac{1}{2}x - \frac{1}{8}x^2 + \frac{1}{16}x^3 + o(x^3)$\\
		$(1+x)^{\alpha}$ & $1 + \alpha x + \frac{\alpha(\alpha - 1)}{2}x^2 + \frac{\alpha(\alpha - 1)(\alpha - 2)}{6}x^3 + o(x^3)$\\
		\hline
	\end{tabular}
	\caption{Formule di taylor}
\end{table}

Confronto tra infiniti e infinitesimi
\begin{tabular}{|c|}
	\hline
	$\lim_{x \rightarrow +\infty} \frac{a^x}{x^\alpha} = + \infty$ se $a>1$ \\
	\hline
	$\lim_{x \rightarrow +\infty} \frac{a^x}{x^\alpha} = 0^+$ se $0<a<1$ \\
	\hline
	$\lim_{x \rightarrow +\infty} \frac{{\log{x}}^\beta}{x^\alpha} = 0$ \\
	\hline
	$\lim_{x \rightarrow 0^+} x^\alpha \cdot \log{x} = 0$ \\
	\hline
\end{tabular}

\subsection{Teorema di Lagrange}
\begin{theorem}[Teorema di Lagrange]
	Data una $f: [a,b] \to \mathbb{R}$, continua in $[a,b]$ e derivabile in $(a,b)$. Allora $\exists c \in (a,b)$ tale che:
	\begin{center}\vspace{-5pt}
		$f'(c) = \frac{f(b) - f(a)}{b - a}$
	\end{center}
\end{theorem}

De l'Hopital

Derivate fondamentali
Inclusa composta e inversa
\end{document}
