% !TeX spellcheck = it_IT
\section{Statistica descrittiva}
La statistica si occupa dello studio dei dati, ovvero della sua \textbf{raccolta}, \textbf{analisi} ed \textbf{interpretazione}. Le risposte dipendono dai dati e dalla conoscenza pregressa del problema, quindi da eventuali ipotesi ed assunzioni.\\
\begin{itemize}
	\item Statistica \textbf{descrittiva}: quando i dati vengono analizzati senza fare assunzioni esterne per evidenziarne la struttura e rappresentarli in modo efficace
	\item \textbf{Inferenza statistica}: studia i dati utilizzando un modello probabilistico, ovvero supponendo che i dati siano valori assunti da \textit{variabili aleatorie} con una certa \textit{distribuzione di probabilità} dipendente da parametri non noti che devono essere stimati. Il modello potrà poi fare previsioni.
\end{itemize}

\subsubsection{Campioni statistici}
\begin{definition}[Popolazione]
	Insieme di oggetti o fenomeni che si vuole studiare su ognuno dei quali si può effettuare una stessa misura, ovvero un \textbf{carattere}. Può essere \textbf{ideale} o \textbf{reale}.
\end{definition}
\begin{definition}[Campione statistico]
	Un sottoinsieme della popolazione scelto per rappresentarla.
\end{definition}
\begin{definition}[Dati]
	Misure effettuate sul campione statistico.
\end{definition}
\begin{definition}[Frequenza]
	Può essere:
	\begin{itemize}
		\item \textbf{Assoluta}: il numero di volte in cui questo esito compare nei dati
		\item \textbf{Relativa}: frazione di volte in cui questo esito compare sul totale dei dati
	\end{itemize}
	In generale dipendono dai dati e quindi non coincidono su tutta la popolazione.
\end{definition}
\begin{note}
	La scelta del campione in modo che sia rappresentativo è importante ma non verrà trattata.
\end{note}
\subsubsection{Istogramma}
Consiste in una serie di colonne ognuna delle quali ha per base un intervallo numerico e per area la frequenza relativa dei dati contenuti nell'intervallo.
\begin{observation}
	La scelta delle ampiezze degli intervalli di base è cruciale. Un buon compromesso deve essere individuato sulla base della numerosità dei dati e sulla loro distribuzione.
\end{observation}
Può avere varie forme:
\begin{itemize}
	\item \textbf{Normale} se ha la forma di una \textit{campana simmetrica}
	\item \textbf{Unimodale} se si concentra su una colonna più alta o \textbf{bimodale} se su due. Può essere asimmetrica a \textit{destra} o a \textit{sinistra} in base alla concentrazione dei dati in base al picco
	\item \textbf{Platicurtica} se i dati sono concentrati in un certo intervallo o \textbf{leptocurtica} se sono composti da un gruppo centrale e da molti \textit{outliers}
\end{itemize}

\subsubsection{Indici statistici}
Dato un vettore $x=(x_1, \ldots, x_n) \in \mathbb{R}^n$ di dati numerici gli indici statistici sono quantità che riassumono alcune proprietà significative.
\begin{definition}[Media campionaria]
	La media aritmetica dei dati:
	\begin{equation}
		\bar{x} = \frac{1}{n} \sum_{i=1}^{n} x_i
	\end{equation}
\end{definition}
\begin{definition}[Mediana]
	Il dato $x_i$ tale che la metà degli altri valori è minore o uguale ad esso e l'altra metà maggiore o uguale.
\end{definition}
\begin{observation}
	La \textbf{mediana} è utile nel caso di dati molto \textbf{asimmetrici} ed è robusta rispetto alle code delle distribuzione. Al contrario la \textbf{media campionaria} viene facilmente spostata da dati molto piccoli o grandi.
\end{observation}
\begin{definition}[Varianza campionaria]
	Si usa per misurare la dispersione dei dati attorno alla media campionaria.
	\begin{equation}
		var(x) = \frac{1}{n-1}\sum_{i=1}^{n}(x_i - \bar{x})^2
	\end{equation}
	È nulla se i dati sono tutti uguali. Possiamo mappare $x$ diversamente:
	\begin{itemize}
		\item $x \mapsto x^2$ misura la media dei punti della media campionaria
		\item $x \mapsto x^3$ misura la \textbf{sample skewness}, ovvero l'asimmetria della distribuzione
		\begin{equation}
			b = \frac{1}{\sigma} \cdot \frac{1}{n} \sum_{i=1}^{n}(x_i-\bar{x})^3
		\end{equation}
		\item $x \mapsto x^4$ misura la piattezza della distribuzione dei dati, ovvero la \textbf{curtosi}
	\end{itemize}
\end{definition}
\begin{definition}[Scarto quadratico medio o deviazione standard]
	\begin{equation}
		\sigma(x)=\sqrt{var(x)}
	\end{equation}
\end{definition}
\begin{proposition}
	Dato un campione di dati $x$ ed un numero positivo $d$:
	\begin{equation}
		\frac{\#\{x_i : \lvert x_i - \bar{x}\rvert > d\}}{n-1} \leq \frac{var(x)}{d^2}
	\end{equation}
	Il termine a sinistra è la frazione di dati che differiscono dalla media campionaria più di $d$.
\end{proposition}

\subsubsection{Quantili}
\begin{definition}[Funzione di ripartizione empirica]
	Dato $x = (x_1, \ldots, x_n) \in \mathbb{R}^n$:
	\begin{equation}
		F_e(t) = \frac{\#\{i \vert x_i \leq t\}}{n}
	\end{equation}
	Per ogni $t \in \mathbb{R}$ restituisce la frequenza relativa dei dati minori o uguali a $t$. È sempre \textbf{non decrescente} e $F_e(-\infty)=0$, $F(+\infty)=1$.
\end{definition}
\begin{definition}[$\beta$-quantile]
	Il dato $x_i$ tale che:
	\begin{itemize}
		\item almeno $\beta n$ dati siano $\leq x_i$
		\item almeno $(1- \beta)n$ dati siano $\geq x_i$
	\end{itemize}
	Inoltre:
	\begin{itemize}
		\item Se $\beta n$ non è intero vale $x_{(\lceil\beta n \rceil)}$
		\item Se $\beta n$ è intero è la media aritmetica tra $x_{(\beta n)}$ e $x_{(\beta n +1)}$
	\end{itemize}
\end{definition}

\subsubsection{Dati multi-variati}
Consideriamo coppie di dati \textbf{bivariati} del tipo
\begin{equation*}
	(x,y) = ((x_1,y_1), \ldots, (x_n,y_n))
\end{equation*}
\begin{definition}[Covarianza campionaria]
	\begin{equation}
		cov(x,y) = \sum_{i=1}^{n} \frac{(x_i - \bar{x})(y_i - \bar{y})}{n-1}
	\end{equation}
\end{definition}
\begin{definition}[Coefficiente di correlazione]
	Dati $\sigma(x) \neq 0$ e $\sigma(y) \neq 0$:
	\begin{equation}
		r(x,y) = \frac{cov(x,y)}{\sigma(x)\sigma(y)}  \frac{\sum_{i=1}^{n}(x_i-\bar{x})(y_i-\bar{t})}{\sqrt{\sum_{i=1}^{n}(x_i - \bar{x})^2}\sqrt{\sum_{i=1}^{n}(y_i - \bar{y})^2}}
	\end{equation}
	Misura la presenza di una relazione lineare tra i dati $x$ e $y$ quantificata dalla \textbf{retta di regressione}.
\end{definition}
\begin{proposition}[Disuguaglianza di Cauchy-Scwarz]
	\begin{equation}
		\sum_{i=1}^{n}(x_i - \bar{x})(y_i - \bar{y}) \leq \sqrt{\sum_{i=1}^{n}(x_i - \bar{x})^2}\sqrt{\sum_{i=1}^{n}(y_i - \bar{y})^2}
	\end{equation}
	e quindi
	\begin{equation}
		\lvert r(x,y)\rvert \leq 1
	\end{equation}
\end{proposition}

La \textbf{retta di regressione} è un'approssimazione dei dati con $y_i$ con una combinazione lineare affine a $a + bx_i$, ottenuta cercando il minimo della distanza dai dati da questa retta con i quadrati degli scarti. L'obiettivo è quindi di cercare i parametri $a$ e $b$ calcolando
\begin{equation}
	\label{eq:regr}
	\inf_{a,b \in \mathbb{R}} \sum_{i=1}^{n}(y_i-a-bx_i)^2
\end{equation}

\begin{theorem}[Retta di regressione]
	Se $\sigma(x) \neq 0$ e $\sigma(y) \neq 0$, esiste un unico minimo al variare di $a,b \in \mathbb{R}$ della quantità \ref{eq:regr}, dato da:
	\begin{equation}
		b^\star = \frac{(n-1)cov(x,y)}{n \cdot var(x)} \quad\quad a^\star = -b^\star \bar{x} + \bar{y}
	\end{equation}
	e vale
	\begin{equation}
		\min_{a,b \in \mathbb{R}} \sum_{i=1}^{n}(y_i-a-bx_i)^2 = (1-r(x,y)^2)\sum_{i=1}^{n}(y_i - \bar{y})^2
	\end{equation}
\end{theorem}
Quanto più $r(x,y)$ è vicino a $1$, tanto più i valori tendono ad allinearsi con la retta. Se vale $1$ vuol dire che i punti sono tutti sulla retta. Il segno di $r(x,y)$ corrisponde al segno del coefficiente angolare. Se è prossimo a zero allora non è una buona approssimazione.