% !TeX spellcheck = it_IT
\newpage
\section{Condizionamento}
Studiare il condizionamento di un problema in forma $Ax=b$ significa chiedersi di quanto cambia la soluzione perturbando di poco $A$ e $B$.
\subsection{Metodi diretti}
Dato un sistema lineare $Ax=b$ le soluzioni possono essere trovate tramite
\begin{equation*}
	x_i = f_i(a_{11}, \ldots, a_{1n}, b1, \ldots, b_n) \quad i:1\ldots n
\end{equation*}
Si può partire studiando la forma di $A$ per cercare dei sistemi risolvibili facilmente.
\subsubsection{Matrice diagonale}
Il primo caso è quando $A$ è una matrice \textbf{diagonale}:
\begin{equation*}
	a_{ij}=0 \Longleftarrow i \neq j 
\end{equation*}
Il determinante di una matrice diagonale è $det(A)=\prod_{i=1}^{n} a_i$ e il determinante è diverso da $0$ solo se tutti gli elementi della diagonale lo sono.
Possiamo riscrivere la matrice in un sistema di equazioni lineari:
\begin{equation*}
	Ax=b \Leftrightarrow \left\{ \begin{array}{lr}
		a_1x_1 = b_1 \Leftrightarrow x1 = \frac{b1}{a1}\\
		\vdots\\
		a_nx_n = b_n \Leftrightarrow x_n=\frac{b_n}{a_n}
	\end{array} \right.
\end{equation*}
Questo sistema lo risolviamo in $O(n)$. Dato però che ridurre una matrice normale in una diagonale è un processo complesso, non conviene farlo.
\subsubsection{Matrice triangolare}
Una matrice è \textbf{triangolare inferiore} quando $a_{ij}=0$ per $j>i$, mentre è \textbf{triangolare superiore} quando $a_{ij}=0$ per $i>j$.\\
Per calcolare il determinante di una matrice triangolare superiore con Laplace facciamo:
\begin{equation*}
	A=\begin{bmatrix}
		a_{11} & 0 & \cdots & 0 \\
		\vdots & \ddots & 0 & \vdots \\
		\vdots & \ddots& \ddots & 0\\
		a_{n1} & \cdots &\cdots & a_{nn}
	\end{bmatrix}
\end{equation*}
Per risolvere $Ax=b$ con una matrice triangolare, vediamo che il sistema associato ci porta ad avere prima un equazione con tutte le incognite fino all'ultima con una sola incognita. È intuitivo che per calcolarne la soluzione è sufficiente partire dall'ultima ed eseguire una \textbf{sostituzione all'indietro}.
\begin{lstlisting}[language=Matlab]
	function[x]=backward_substitution(a,b)
		n=length(b);
		x = zeros(n,1);
		for k=n:-1:1
			s=0;
			for j=k+1:n
				s=s+a(k,j)*x(j);
			end
			x(k)=(b(k)-s)/a(k,k);
		end
\end{lstlisting}
Dato che l'operazione moltiplicativa in macchina richiede leggermente più lunga di quella della somma, è sufficiente considerare le prime per calcolare la complessità di questo programma.\\ La complessità al caso pessimo è quindi $\frac{n \cdot (n+1)}{2} = O(n^2) = \frac{n^2}{2}+O(n)$.
\begin{example}[Matrice bidiagonale superiore]
	Supponiamo di avere la seguente matrice bidiagonale superiore:
	\begin{equation*}
		A=\begin{bmatrix}
			a_{11} & a_{12} & 0 & 0\\
			0 & \ddots & \ddots & 0 \\
			0 & 0 & \ddots & a_{n-1 n} \\
			0 & 0 & 0 & a_{nn}
		\end{bmatrix}  = 
		\begin{bmatrix}
			a_1 & b_1 & 0 & 0\\
			0 & \ddots & \ddots & 0\\
			0 & 0 & \ddots & b_{n-1}\\
			0 & 0 & 0 & a_n
		\end{bmatrix}
	\end{equation*}
	In questo caso il codice si può cambiare sostituendo il secondo ciclo con l'operazione
	\begin{lstlisting}[language=Matlab]
		s=a(k,k+1)*x(k+1);
	\end{lstlisting}
	e la complessità diventa $O(n)$.
\end{example}