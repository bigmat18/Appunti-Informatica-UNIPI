% !TeX spellcheck = it_IT
\newpage
\section{Sistemi di tipi}
\subsection{Perché?}
Dato che nel lambda calcolo i programmi e i valori sono funzioni
possiamo facilmente scrivere programmi che non sono corretti rispetto
all’uso inteso dei valori. Ad esempio:
\begin{example}[Errore tipi]
	Nella seguente espressione si può applicare $0$ a \textit{False}, ottenendo quindi un risultato che però non ha alcun senso:
	\begin{equation}
		False \: 0 = (\lambda t.\lambda f.f)(\lambda z.\lambda s.z) \rightarrow \lambda f.f
	\end{equation} 
\end{example}
\noindent Analogamente per una macchina tutto è un bit: istruzioni, dati e operazioni. 
Un esempio più pratico è il seguente:
\begin{example}
	L'istruzione nel corpo dell'\textit{if} contiene un errore di tipo (stringa divisa per un intero). Se non avessimo il controllo dei tipi l'unico modo per scoprire l'errore sarebbe eseguire numerosi test per riuscire a coprire tutte le possibilità, fino ad entrare nel corpo dell'\textit{if}. Richiederebbe tempo e risorse e non ci garantisce neanche la certezza di aver provato tutti i casi possibili.
	
	\begin{lstlisting}
			if (condizione_complicata) {
				return "hello"/10;
			}
	\end{lstlisting}
\end{example}
Se in certi linguaggi di programmazione ci troveremmo davanti ad errori di esecuzione, in altri (come ad esempio JavaScript) otterremmo un errore nel risultato in quanto l'interprete proverebbe a fare un cast manuale. \\
Concludendo, la mancanza di \textbf{type safety} aumenta il numero di bug, rendendo così un software meno funzionale e più vulnerabile.

\subsection{Cosa sono i tipi?}
I \textbf{sistemi di tipo} sono meccanismi che permettono di rilevare in anticipo errori di programmazione. 
\begin{definition}[Tipo]
	Il tipo è un \textbf{attributo} di un dato che descrive come il linguaggio di programmazione permetta di usare quel particolare dato.
\end{definition}

\noindent Un tipo serve quindi a limitare i valori che un'espressione può assumere, che operazioni possono essere effettuate sui dati e in che modo questi ultimi possono essere salvati.

\begin{definition}[Sistema dei tipi]
	Un sistema dei tipi è un metodo \textbf{sintattico}, \textbf{effettivo} per dimostrare
	l'assenza di comportamenti anomali del programma \textbf{strutturando} le
	operazioni del programma in base ai tipi di valori che calcolano.
\end{definition}
\noindent Analizziamo i tre aspetti:
\begin{itemize}
	\item \textit{Sintattico}:  l'analisi viene effettuata dal punto di vista sintattico
	\item \textit{Effettivo}: si può definire un algoritmo che effettui questa analisi
	\item \textit{Strutturale}: i tipi assegnati si ottengono in maniera \textbf{composizionale} dalle sottoespressioni.
\end{itemize}

\subsection{Come funziona?}
Un sistema dei tipi associa dei tipi ai valori calcolati. Esaminando il flusso dei valori calcolati prova a dimostrare che non avvengano errori (di tipo, non in generale)facendo un controllo, che può avvenire in due modi:
\begin{itemize}
	\item \textit{Statico}: avviene in fase di compilazione, non degradando le prestazioni
	\item  \textit{Dinamico}: avviene in fase di esecuzione e aumenta il tempo di esecuzione
\end{itemize}

\subsection{Come si progetta?}
\subsubsection{Specifiche del linguaggio}
Prendiamo come esempio il seguente \textbf{linguaggio}:\\
\begin{center}
	\begin{tabular}{|c|c|c|c|}
		\hline
		\textbf{Espressioni} & \textbf{Valori} & \textbf{Valori numerici} & \textbf{Tipi} \\
		\hline
		E::= & V::= & NV::= & T::= \\
		true & true & $0 \vert 1 \vert 2 \vert \ldots$ & Bool \\
		false & false & & Nat \\
		NV & NV & &\\
		if \textit{E} then \textit{E} else \textit{E} & &  &\\
		succ \textit{E} & & &\\
		pred \textit{E} & & &\\
		isZero \textit{E} & & &\\
		\hline
	\end{tabular}
\end{center}
\subsubsection{Regole di valutazione}
Avremo le seguenti \textbf{regole di valutazione}:\\
\begin{gather}
		if \: true \: then \: E1 \: else \: E2 \rightarrow E1 \\
		if \: false \: then \: E1 \: else \: E2 \rightarrow E2 \\
		\frac{E \rightarrow E'}{if \: E \: then \: E1 \: else \: E2 \rightarrow if \: E'  \: then \: E1 \: else \: E2 \rightarrow E1} \label{eq:if_stepfw}
\end{gather}
\begin{equation}
	\frac{E \rightarrow E'}{succ \: E \rightarrow succ \: E'} \qquad
	\frac{m = n+1}{succ \: E \rightarrow succ \: E'}
\end{equation}
\begin{equation}
	\frac{E \rightarrow E'}{pred \: E \rightarrow pred \: E'} \qquad
	\frac{ n>0,\: m=n-1}{pred \: n \rightarrow m} \qquad
	pred \: 0 \rightarrow 0
\end{equation}
\begin{equation}
	\frac{E \rightarrow E'}{isZero \: E \rightarrow isZero \: E'} \qquad isZero \: 0 \rightarrow true \qquad \frac{n>0}{isZero \: n \rightarrow false}
\end{equation}
\subsubsection{Type checking}
Il \textbf{controllo di tipo} definisce una relazione binaria $(E, T)$ che associa il tipo $T$ all'espressione $E$. Questo ha due caratteristiche principali:
\begin{itemize}
	\item Utilizza il \textit{metodo sintattico}
	\item Le regole sono definite per \textit{induzione strutturale} sul programma 
\end{itemize}
Le regole sono le seguenti:
\begin{equation}
	true:Bool \qquad false:Bool \qquad n:Nat
\end{equation}
\begin{equation}
	\frac{E:Nat}{succ \: E : Nat} \qquad \frac{E:Nat}{pred \: E : Nat} \qquad \frac{E:Nat}{isZero\: E : Bool}
\end{equation}
\begin{equation}\label{eq:type_if}
	\frac{E:Bool, E1:T,E2:T}{if \: E \: then \: E1 \: else \: E2}
\end{equation}
\subsubsection{Composizionalità}
I sistemi di tipo sono \textbf{imprecisi}: non definiscono esattamente quale tipo di valore sarà restituito da ogni programma, ma solo un'\textbf{approssimazione conservativa}.
\begin{example}
	La seguente espressione:
	\begin{equation}
		if \: E \: then \: 0 \: else \: false
	\end{equation}
	potrebbe restituire come risultato sia un \textit{Bool} che un \textit{Nat} a seconda del valore di \textit{E}. Il controllo dei tipi quindi non permetterà che possano esserci due risultati diversi, riducendo la precisione ma mantenendo la sicurezza.
\end{example}
Questo avviene proprio per garantire la \textbf{composizionalità}, infatti ad esempio la regola dell'equazione \ref{eq:type_if} necessita che $E1$ ed $E2$ abbiano lo stesso tipo.

\subsection{Dimostrazione}
La \textbf{correttezza} del sistema di tipo è espressa da due proprietà:
\begin{itemize}
	\item Progresso
	\item Conservazione
\end{itemize}

\subsubsection{Progresso}
\begin{definition}[Progresso]
	Se $E:T$ allora $E$ è un valore oppure $E \rightarrow E'$ per una qualche espressione $E'$.
\end{definition}
In pratica, un'espressione ben tipata non si blocca a run-time. Può fare sempre un passo a meno che non sia un valore.
\begin{proof}
	Utilizziamo l'induzione sulla struttura di derivazione di $E:T$.\\
	I \textit{casi base} sono i seguenti:
	\begin{itemize}
		\item $true:Bool$
		\item $false:Bool$
		\item $0 \vert 1 \vert 1 \vert \ldots : Nat$
	\end{itemize}
	I \textit{casi induttivi} sono tutti molto simili, vediamo quello per la formula \ref{eq:type_if}. \\
	Per \textit{ipotesi induttiva} abbiamo due casi:
	\begin{itemize}
		\item $E1$ è un valore. In questo caso deve essere $true$ o $false$ e le regole della semantica fanno fare un \textbf{passo} del tipo $E \rightarrow E1$ o $E \rightarrow E2$
		\item  Esiste $E4$ tale che $E1 \rightarrow E4$. In questo caso si applica la regola \ref{eq:if_stepfw} e si esegue un \textbf{passo}.
	\end{itemize}
\end{proof}

\subsubsection{Conservazione}
\begin{definition}[Conservazione]
	Se $E:T$ e $E \rightarrow E'$ allora $E':T$.
\end{definition}
In pratica i tipi sono preservati dalle regole di esecuzione.
\begin{proof}
	Utilizziamo l'induzione come nella precedente dimostrazione.\\
	I \textit{casi base} sono immediati: $true$, $false$ e $0 \vert 1 \vert 2 \vert \ldots$ sono valori e di conseguenza non fanno nessun passo. \\
	Anche qui per i \textit{casi induttivi} vediamo quello per la formula \ref{eq:if_stepfw}. Per l'ipotesi induttiva abbiamo due casi:
	\begin{itemize}
		\item $E1$ è un valore: \begin{itemize}
			\item $true$: in questo caso $E \rightarrow E2$ e sappiamo già per ipotesi induttiva che $E2:T$ (sappiamo che il passo ha successo)
			\item $false$: in questo caso $E \rightarrow E3$ e $E3:T$
		\end{itemize}
		\item Esiste $E4$ tale che $E1 \rightarrow E4$. Questo implica:
		\begin{equation*}
			E = if \: E1 \: then \: E2 \: else \: E3 \rightarrow if \: E4 \: then \: E2 \: else \: E3
		\end{equation*}
		Dato che per ipotesi induttiva $E1:Bool$ abbiamo che $E4:Bool$ e, grazie alle derivazioni che valgono per ipotesi $E2:T$ e $E3:T$, possiamo derivare applicando la regola \ref{eq:type_if}.
	\end{itemize}
\end{proof}
