\newpage
\section{Introduzione ai sistemi operativi}
Inanzitutto iniziamo con una Definizione del sistema operativo: software per gestire le risorse di un computer per i suoi utenti e applicazioni. 
Sfide del sistema operativo: affidabilità, sicurezza, reattività, portabilità, ... storia del sistema operativo: a che punto siamo?
\\\\Andiamo ora a rispondere alla domanda, cos'è un sistema operativo?\\
E' software per gestire le risorse di un computer per i suoi utenti e applicazioni- Funzionalità principali: pianificazione delle risorse, memoria virtuale, comunicazione tra processi (IPC) 
e meccanismi di sincronizzazione.

\subsection{Ruoli del sistema operativo}
Abbiamo una serie di ruoli che un sistema operativo va a prendere, essi sono:
\begin{itemize}
    \item \textbf{Referee}: Allocazione delle risorse tra utenti, applicazioni, isolamento di diversi utenti, applicazioni tra loro, comunicazione tra utenti, applicazioni.
    \item \textbf{Illusionist}: Ogni applicazione sembra avere l'intera macchina tutta per sé, numero infinito di processori, quantità (quasi) infinita di memoria, archiviazione affidabile, trasporto di rete affidabile.
    \item \textbf{Glue}: Fornisce servizi comuni e standard alle applicazioni, semplifica lo sviluppo di applicazioni, librerie, widget dell'interfaccia utente, ...
\end{itemize}

\begin{example}
    Prendiamo per esempio il file system. Abbiamo che:
    \begin{itemize}
        \item Referee: Impedisci agli utenti di accedere ai file degli altri senza autorizzazione, anche dopo che un file è stato eliminato e il suo spazio è stato riutilizzato.
        \item Illusionist: I file possono diventare (quasi) arbitrariamente grandi, i file persistono anche quando la macchina va in crash nel bel mezzo di un salvataggio.
        \item Glue: Directory con nome, printf, ...
    \end{itemize}
\end{example}

\hspace{-15pt}Quindi ruoli che il sistema operativo ha creano dei pattern che possiamo rivedere
in molteplici contensti, anche online, fra questi abbiamo:
\begin{itemize}
    \item Cloud computing:
    \begin{itemize}
        \item Referee: come allocare le risorse tra le applicazioni concorrenti nel cloud?.
        \item Illusionist: le risorse di calcolo in un cloud si evolvono continuamente, come isolare le applicazioni da questa evoluzione?
        \item Glue: come fornire un accesso comune e standardizzato ai servizi cloud?
    \end{itemize}

    \item Web services: 
    \begin{itemize}
        \item Referee: assicura la reattività quando vengono aperte più schede contemporaneamente
        \item Illusionist: i servizi Web sono distribuiti geograficamente per la tolleranza ai guasti. Maschera gli errori del server agli utenti.
        \item Glue: in che modo un browser ottiene l'esecuzione portatile di script su diverse piattaforme OS e HW?
    \end{itemize}

    \item Multi-user database systems:
    \begin{itemize}
        \item Referee: come imporre l'accesso ai dati e la privacy a diversi utenti?
        \item Illusionist: come mascherare i guasti in modo che i dati rimangano coerenti e disponibili per gli utenti?
        \item Glue: quali servizi comuni allo sviluppo dei programmi?
    \end{itemize}

    \item Internet
    \begin{itemize}
        \item Referee: garantire servizi differenziati agli utenti e proteggere da DoS, spam, phishing ecc...
        \item Illusionista: internet appare come un'unica rete mondiale ma non lo è!
        \item Glue: i protocolli Internet rendono le applicazioni indipendenti dall'architettura di rete sottostante.
    \end{itemize}
\end{itemize}

\begin{example}
    Vediamo più in dettaglio il caso dei web service. \\
    Più utenti emettono richieste contemporaneamente, queste devono essere gestite contemporaneamente. 
    Molte richieste riguardano dati e calcoli, pensa ai motori di ricerca, una richiesta può comportare calcoli profondi su grandi gruppi di macchine. 
    Il server utilizza le cache per velocizzare, la cache è condivisa tra gli utenti, necessità di meccanismi di accesso sincronizzati. 
    I server inviano ai client script per la personalizzazione delle pagine, come fa il client a proteggersi dall'esecuzione di codice di terze parti che potrebbe incorporare virus/spyware?\\\\
    I siti Web devono essere aggiornati: come gestire la coerenza con le richieste di lettura simultanee? Client e server possono funzionare a velocità diverse, 
    necessità di disaccoppiamento della velocità. 
    L'hardware che supporta il sito web può essere aggiornato, come trarne vantaggio senza riscrivere il codice del web server?
\end{example}

\subsection{OS challenges}
Ci sono svide comuni che ogni sistema operativo deve affrontare per garantire un corretto funzionamento, esse sono:
\begin{itemize}
    \item \textbf{Reliability e Availability} Il sistema fa quello per cui è stato progettato? Availability, per quale parte del tempo il sistema funziona?Mean Time To Failure (MTTF), Mean Time to Repair.
    \item \textbf{Security} Il sistema può essere compromesso da un utente malintenzionato?Privacy, i dati sono accessibili solo agli utenti autorizzati.
    \item \textbf{Portability} Per i programmi: interfaccia di programmazione dell'applicazione (API), interfaccia macchina astratta. Per il sistema operativo, livello di astrazione hardware, la maggior parte dei sistemi operativi dispone di routine del kernel specifiche dell'hardware.
    \item \textbf{Perfomance} Latenza/tempo di risposta, quanto tempo richiede il completamento di un'operazione? Throughput, quante operazioni possono essere eseguite per unità di tempo? Overhead, quanto lavoro extra viene svolto dal sistema operativo? Correttezza, quanto sono uguali le prestazioni ricevute dai diversi utenti ?Prevedibilità, quanto è costante la performance nel tempo?
\end{itemize}

\subsection{Struttura OS}
Molte dipendenze tra i moduli, molte parti del sistema operativo dipendono dalle primitive di sincronizzazione, 
il sistema di memoria virtuale dipende dal supporto HW di basso livello per la traduzione degli indirizzi, mentre il 
file system e la memoria virtuale condividono blocchi di memoria fisica, il file system può dipendere da lo stack del protocollo di rete.\\\\
Nel sistema operativo è presente anche quelle che viene chiamato OS Adoption. L'adozione è al di fuori del controllo di un sistema operativo, ampia disponibilità di applicazioni, 
ampia disponibilità di HW che lo supporta. Effetto rete, app store, esempio: modello Android vs modello telefono. Sistemi vsopen proprietari, non un chiaro vincitore

