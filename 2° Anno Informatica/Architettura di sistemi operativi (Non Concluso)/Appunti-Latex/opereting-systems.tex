\newpage
\section{Introduzione ai sistemi operativi}
Inanzitutto iniziamo con una Definizione del sistema operativo: software per gestire le risorse di un computer per i suoi utenti e applicazioni. 
Sfide del sistema operativo: affidabilità, sicurezza, reattività, portabilità, ... storia del sistema operativo: a che punto siamo?
\\\\Andiamo ora a rispondere alla domanda, cos'è un sistema operativo?\\
E' software per gestire le risorse di un computer per i suoi utenti e applicazioni- Funzionalità principali: pianificazione delle risorse, memoria virtuale, comunicazione tra processi (IPC) 
e meccanismi di sincronizzazione.

\subsection{Ruoli del sistema operativo}
Abbiamo una serie di ruoli che un sistema operativo va a prendere, essi sono:
\begin{itemize}
    \item \textbf{Referee}: Allocazione delle risorse tra utenti, applicazioni, isolamento di diversi utenti, applicazioni tra loro, comunicazione tra utenti, applicazioni.
    \item \textbf{Illusionist}: Ogni applicazione sembra avere l'intera macchina tutta per sé, numero infinito di processori, quantità (quasi) infinita di memoria, archiviazione affidabile, trasporto di rete affidabile.
    \item \textbf{Glue}: Fornisce servizi comuni e standard alle applicazioni, semplifica lo sviluppo di applicazioni, librerie, widget dell'interfaccia utente, ...
\end{itemize}

\begin{example}
    Prendiamo per esempio il file system. Abbiamo che:
    \begin{itemize}
        \item Referee: Impedisci agli utenti di accedere ai file degli altri senza autorizzazione, anche dopo che un file è stato eliminato e il suo spazio è stato riutilizzato.
        \item Illusionist: I file possono diventare (quasi) arbitrariamente grandi, i file persistono anche quando la macchina va in crash nel bel mezzo di un salvataggio.
        \item Glue: Directory con nome, printf, ...
    \end{itemize}
\end{example}