% !TeX spellcheck = it_IT
\newpage
\section{PaaS}
L'approccio PaaS prevede che i fornitori esterni gestiscano sia hardware che software e che l'utente fornisca solamente i dati e l'applicativo.\\
I \textbf{vantaggi} sono:
\begin{itemize}
	\item Ridotta gestione per l'utente
	\item Manutenzione automatica
	\item Load balancing, scaling e distribuzione più efficienti
	\item Più facilità nell'adottare nuove tecnologie
\end{itemize}
Gli \textbf{svantaggi} invece:
\begin{itemize} 
	\item Disponibilità del servizio molto dipendente dal fornitore
	\item Vendor lock-in
	\item In balia di eventuali cambiamenti da parte del fornitore
\end{itemize}
\subsection{Heroku}
Heroku è una piattaforma cloud basata sulla gestione di un sistema di container, con data services integrati e un ampio ecosistema, per sviluppare ed eseguire app moderne.\\
Gli utenti usano \textit{container} chiamati \textbf{dynos} per lanciare ed eventualmente scalare le loro applicazioni. Questi sono container linux virtualizzati ed isolati progettati per eseguire codice. Ne esistono di diversi tipi (e costi) a seconda delle necessità per l'applicazione.
\subsubsection{Deployment}
Per fare il deploy di un'applicazione, Heroku necessita di:
\begin{itemize}
	\item Il \textbf{codice sorgente}
	\item Una lista di \textbf{dipendenze}
	\item Un \textbf{procfile}, ovvero un file contenente l'elenco dei comandi necessari a far partire il codice
\end{itemize}
Il sistema, una volta ricevuti questi parametri, ottiene i linguaggi e le dipendenze necessarie e produce uno \textbf{slug}, che poi verrà inviato ad un \textit{dyno}.
\subsubsection{Runtime}
Quando l'applicazione viene fatta partire, Heroku crea un numero di \textit{dyno} in base al carico previsto, ognuno caricato con la stessa configurazione fornita dall'utente.
\subsubsection{Add-ons}
Esistono più di 150 servizi di terze parti che forniscono add-ons per le applicazioni che hanno funzionalità di molti tipi diversi, tra cui monitoraggio, data store e logging.\\
Questo porta al fenomeno del \hyperref[vendor_lockin]{vendor lock in}, in quanto gli add-on potrebbero non essere disponibili al di fuori di Heroku e potrebbe quindi essere necessario riscrivere il codice per quella funzionalità.

\subsection{Altri fornitori}
Altri esempi di PaaS sono:
\begin{itemize}
	\item Microsoft Azure
	\item Open shift
	\item Google Firebase
\end{itemize}