% !TeX spellcheck = it_IT
\newpage
\section{Software services}
\subsection{REST}
REST (REpresentational State Trasfer) è uno stile architetturale sviluppato come modello astratto per il Web.\\
Ogni azione causa il passaggio dell'applicazione allo stato successivo tramite il trasferimento dello stato corrente.\\
Si basa sui seguenti principi:
\begin{itemize}
	\item \textbf{URIs (Uniform Resource Identifier)}: identificazione delle risorse in maniera \textit{univoca} e \textit{universale}, ad esempio gli indirizzi Web
	\item \textbf{Interfaccia uniforme}: ad esempio per il protocollo HTTP, l'utente invoca dei metodi predefiniti:
	\begin{itemize}
		\item \textit{POST} e \textit{PUT} per creare e aggiornare risorse
		\item \textit{DELETE} per eliminare una risorsa
		\item \textit{GET} per ottenere lo stato corrente di una risorsa
	\end{itemize}
	\item \textbf{Messaggi autoesplicativi}: ogni richiesta contiene abbastanza \textit{contesto} per riuscire a comprendere il messaggio. Inoltre le risorse sono \textit{separate} dalla loro rappresentazione in modo da poter essere accedute in diversi formati
	\item \textbf{Interazioni stateful tramite hyperlinks}: data la natura stateless, per fornire interazioni stateful si utilizzano trasferimenti espliciti di stato tramite hyperlinks
\end{itemize}
\subsubsection{Negoziazione}
Quando viene richiesta una risorsa, vanno specificati i formati accettabili. Il server poi risponde con la risorsa nel formato più appropriato o eventualmente con il codice di errore 406.
\begin{center}
	\includegraphics[scale=0.2]{negotiation.png}
\end{center}
\subsubsection{Design}
Il design deve seguire i seguenti passaggi:
\begin{enumerate}
	\item Identificare le risorse che devono essere esposte come servizi
	\item Modellare le relazioni tra le risorse tramite gli \textit{hyperlinks}
	\item Definire gli URIs seguendo le seguenti guidelines:
	\begin{itemize}
		\item Meglio i sostantivi dei verbi
		\item Brevità
		\item Utilizzare i template per costruire ed elaborare URIs parametrici
		\item Non cambiarli, nel caso utilizzare il reindirizzamento
	\end{itemize}
	\item Definire quali metodi sono applicabili alla risorsa
	\item Progettare e documentare la rappresentazione delle risorse
	\item Implementare e rilasciare il servizio
	\item Testing
\end{enumerate}
\subsubsection{Riepilogo}
L'architettura REST è \textbf{semplice} (utilizza standard famosi e infrastruttura già presente), \textbf{leggera} (sia per i protocolli utilizzati che per i messaggi) e \textbf{scalabile}. Di contro però i client possono richiedere \textbf{troppi o troppi pochi dati}, possono fare un \textbf{numero limitato di richieste} e le convenzioni per la \textbf{nomenclatura} sono inconsistenti.

\subsection{OpenAPI}
L'iniziativa OpenAPI (ideata dalla Linux Fundation Collaborative Project) ha come obiettivo quello di creare uno standard neutro ed indipendente di API di tipo REST. Formalmente chiamata \textbf{Swagger}, si basa su una descrizione in JSON degli endpoint HTTP, come vengono usati e la struttura dei loro parametri in input e output.\\
In questo modo l'applicazione può richiedere le specifiche della API per poi fare le richieste nel formato corretto.

\subsection{Microservizi}
I microservizi nascono per due motivi principali:
\begin{itemize}
	\item Diminuire il \textbf{tempo} necessario ad aggiungere funzionalità o aggiornamenti (si riduce il tempo di build)
	\item Aumentare la \textbf{scalabilità}
\end{itemize}
In particolare, l'applicazione viene vista come insieme di servizi, ognuno in esecuzione sul proprio container \textbf{indipendente} che comunica con gli altri tramite meccanismi leggeri di comunicazione. Questo garantisce che la gestione dei dati sia \textbf{decentralizzata} e che l'applicazione sia \textbf{scalabile} orizzontalmente. Garantisce infine una buona \textbf{tolleranza ai guasti}.\\
A livello di team, generalmente ce n'è uno per servizio che si occupa dell'intero lifecycle, dallo sviluppo al deploy.\\
I due lati negativi principali sono:
\begin{itemize}
	\item \textbf{Overhead di comunicazione}: dato che tutti i servizi devono comunicare tra di loro, ci sarà un grosso traffico
	\item \textbf{Complessità} del sistema
\end{itemize}