% !TeX spellcheck = it_IT
\newpage
\section{Cloud Edge Continuum}
Con l'aumentare dei dispositivi IoT nella nostra società (e.g. smart cities, power plants, AI), cresce anche la domanda per la potenza di calcolo. \\
Ognuno di questi dispositivi raccoglie dei dati con dei sensori, li processa e poi esegue delle azioni. È proprio la fase di \textbf{lavorazione} dei dati che sta diventando sempre più critica.
\subsection{Implementazioni tradizionali}
\subsubsection{Edge Computing}
I dati vengono processati ai "confini" della rete, garantendo una \textbf{latenza} molto bassa ma limitando di molto la quantità di dati che possono essere processati ed immagazzinati.
\subsubsection{IoT \& Cloud}
I dati vengono inviati e processati nel Cloud, garantendo \textbf{risorse pressoché illimitate} ma aumentando di molto la latenza e causando un overhead in essa. 
\subsection{Cloud Edge Continuum}
Questa implementazione è un ibrido tra le due precedentemente descritte con l'obiettivo di avere il meglio di entrambe: \textbf{latenza bassa}, \textbf{connettività} e \textbf{potenza di calcolo}.\\
Per farlo le applicazioni devono essere: \textbf{containerizzate} e basate su \textbf{microservizi}.
\subsubsection{Posizionamento}
Ogni \textbf{applicazione} ha diverse caratteristiche, tra cui:
\begin{itemize}
	\item Requisiti \textit{hardware}
	\item Requisiti \textit{software}
	\item \textit{QoS}
	\item \textit{Data awareness}
	\item \textit{Sicurezza e affidabilità}
\end{itemize}
mentre l'infrastruttura è \textbf{eterogenea}, \textbf{grande} e \textbf{dinamica}.\\
È necessario capire dove posizionare ogni servizio dell'applicazione, se nell'access point, nel cabinet, nel datacenter o nel cloud. Esistono tre approcci alla soluzione del problema:
\begin{itemize}
	\item \textbf{Machine learning}: prima o poi prenderà una scelta, non per forza corretta. L'infrastruttura è dinamica e difficile da spiegare.
	\item \textbf{Mixed Integer Linear Programming}: trova soluzioni sempre ottimali ma difficile da leggere e da programmare quando sono presenti dati non numerici. Inoltre è lenta da avviare.
	\item \textbf{Ragionamento dichiarativo}: si definiscono i requisiti di un determinato servizio. Il motore di inferenza cerca poi tutte le possibili soluzioni. È facile da leggere e da spiegare.
\end{itemize}

\subsection{Gestione}
Nel mondo di oggi le cose sono in continuo cambiamento. È importante \textbf{monitorare} continuamente le applicazioni e le infrastrutture in modo da avere un ragionamento continuo che permetta di intervenire velocemente.