% !TeX spellcheck = it_IT
\newpage
\section{PaaS}
L'approccio PaaS prevede che i fornitori esterni gestiscano sia hardware che software e che l'utente fornisca solamente i dati e l'applicativo.\\
I \textbf{vantaggi} sono:
\begin{itemize}
	\item Ridotta gestione per l'utente
	\item Manutenzione automatica
	\item Load balancing, scaling e distribuzione più efficienti
	\item Più facilità nell'adottare nuove tecnologie
\end{itemize}
Gli \textbf{svantaggi} invece:
\begin{itemize} 
	\item Disponibilità del servizio molto dipendente dal fornitore
	\item Vendor lock-in
	\item In balia di eventuali cambiamenti da parte del fornitore
\end{itemize}
\subsection{Heroku}
Heroku è una piattaforma cloud basata sulla gestione di un sistema di container, con data services integrati e un ampio ecosistema, per sviluppare ed eseguire app moderne.\\
Gli utenti usano \textit{container} chiamati \textbf{dynos} per lanciare ed eventualmente scalare le loro applicazioni. Questi sono container linux virtualizzati ed isolati progettati per eseguire codice 