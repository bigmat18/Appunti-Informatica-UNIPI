% !TeX spellcheck = it_IT
\newpage
\section{Introduzione}

\subsection{Service-based economy}
L'economia basata sui servizi (\textit{Everything as a service}) si fonda sulla tendenza degli ultimi 30 anni di passare dai \textit{beni} ai \textit{servizi}.
\begin{example}[Bicicletta]
	Una signora compra una bicicletta dal venditore. Questa diventa di sua proprietà e si deve occupare della manutenzione. La bicicletta diventa un \textbf{servizio} (\textit{CicloPi}) e la signora paga per usare la bici che però non è più sua (e non deve più preoccuparsi di manutenzione e furto).
\end{example}
\noindent Esempi più vicini all'informatica sono il passaggio dai supporti fisici per la musica allo streaming o i dispositivi di memorizzazione passati ai Cloud Drive.
\subsection{Service contracts}
Quando usiamo un servizio non vogliamo sapere come viene implementato. L'unica cosa che ci interessa è cosa è specificato sul \textbf{contratto di utilizzo}. Nella maggior parte dei casi l'utente non lo legge.
\subsubsection{Quality of Service}
Ci sono più fornitori che ci danno lo stesso servizio ma con qualità del servizio diverse. Dobbiamo chiederci se il prezzo più basso vale la pena del sacrificio della qualità.
\subsubsection{Service Level Agreement}
Sono i contratti di servizio che includono anche il livello di \textbf{affidabilità di servizio}. In questa situazione abbiamo tre figure:
\begin{itemize}
	\item \textit{Programmatore}
	\item \textit{Business expert}: colui che sa il livello di affidabilità in base al mercato
	\item \textit{Legale}: colui che sa come scriverlo
\end{itemize}
\begin{example}[Google SLA]
	Google Compute Engine fornisce un \textbf{Service Level Objective} (SLO) del $99.95\%$. In caso di non raggiungimento del SLO si viene rimborsati con del credito in percentuale a quanto si è distanti dal target.\\
	\begin{center}
		\begin{tabular}{|c|c|}
			\hline
			\textbf{Montlhy Uptime Percentage} & \textbf{Rimborso} \\
			\hline
			$95.00\%-<99.95\%$ & $10\%$ \\
			\hline
			$90.00\%-<95.00\%$ & $25\%$ \\
			\hline
			$<90.00\%$ & $100\%$ \\
			\hline
		\end{tabular}
	\end{center}
	Se ad esempio l'1 e il 2 Aprile dalle 8am alle 5pm (orario di lavoro) non era disponibile il servizio, a quanto ammonta il rimborso in credito?
\end{example}