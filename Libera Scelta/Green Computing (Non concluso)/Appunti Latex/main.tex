\documentclass[a4paper,10pt]{article}
\usepackage[utf8]{inputenc}

% ----  Useful packages % ---- 
\usepackage{amsmath}
\usepackage{graphicx}
\usepackage{amsfonts}
\usepackage{amsthm}
\usepackage{amssymb}
\usepackage{makecell}
% ----  Useful packages % ---- 

\usepackage{wrapfig}
\usepackage{caption}
\usepackage{subcaption}
\usepackage{hyperref}
\hypersetup{
	colorlinks,
	citecolor=black,
	filecolor=black,
	linkcolor=black,
	urlcolor=black
}

% ---- Set page size and margins replace ------
\usepackage[letterpaper,top=2cm,bottom=2cm,left=3cm,right=3cm,marginparwidth=1.75cm]{geometry}
% ---- Set page size and margins replace ------

% ------- NOTA ------
\theoremstyle{remark}
\newtheorem{note}{Note}[subsection]
% ------- NOTA ------

% ------- OSSERVAZIONE ------
\theoremstyle{definition}
\newtheorem{observation}{Osservazione}[subsection]
% ------- OSSERVAZIONE ------

% ------- DEFINIZIONE ------
\theoremstyle{plain}
\newtheorem{definition}{Definizione}[subsection]
% ------- DEFINIZIONE ------

% ------- ESEMPIO ------
\theoremstyle{definition}
\newtheorem{example}{Esempio}[subsection]
% ------- ESEMPIO ------

% ------- DIMOSTRAZIONE ------
\theoremstyle{definition}
\newtheorem{demostration}{Dimotrazione}[subsection]
% ------- DIMOSTRAZIONE ------

% ------- TEOREMA ------
\theoremstyle{definition}
\newtheorem{theorem}{Teorema}[subsection]
% ------- TEOREMA ------

% ------- COROLLARIO ------
\theoremstyle{plain}
\newtheorem{corollaries}{Corollario}[theorem]
% ------- COROLLARIO ------

% ------- PROPOSIZIONE ------
\theoremstyle{plain}
\newtheorem{proposition}{Proposizione}[subsection]
% ------- PROPOSIZIONE ------

% ---- Footer and header ---- 
\usepackage{fancyhdr}
\pagestyle{fancy}
\fancyhf{}
\fancyhead[LE,RO]{A.A 2023-2024}
\fancyhead[RE,LO]{Green Computing}
\fancyfoot[RE,LO]{\rightmark}
\fancyfoot[LE,RO]{\thepage}

\renewcommand{\headrulewidth}{.5pt}
\renewcommand{\footrulewidth}{.5pt}
% ---- Footer and header ---- 

% ----  Language setting ---- 
\usepackage[italian, english]{babel}
% ----  Language setting ---- 

\usepackage{listings}
\usepackage{color}

\definecolor{dkgreen}{rgb}{0,0.6,0}
\definecolor{gray}{rgb}{0.5,0.5,0.5}
\definecolor{mauve}{rgb}{0.58,0,0.82}

\lstset{frame=tb,
	language=C,
	aboveskip=3mm,
	belowskip=3mm,
	showstringspaces=false,
	columns=flexible,
	basicstyle={\small\ttfamily},
	numbers=none,
	numberstyle=\tiny\color{gray},
	keywordstyle=\color{blue},
	commentstyle=\color{dkgreen},
	stringstyle=\color{mauve},
	breaklines=true,
	breakatwhitespace=true,
	tabsize=3
}

\title{\textbf{Green Computing}}
\author{Realizzato da: Ghirardini Filippo}
\date{A.A. 2023-2024}

\begin{document}
	\begin{titlepage} %crea l'enviroment
	\begin{figure}[t] %inserisce le figure
		\centering\includegraphics[width=0.98\textwidth]{marchio_unipi_pant541.png}
	\end{figure}
	\vspace{20mm}
	
	\begin{Large}
		\begin{center}
			\textbf{Dipartimento di Informatica\\ Corso di Laurea Triennale in Informatica\\}
			\vspace{20mm}
			{\LARGE{Corso 3° anno - 6 CFU}}\\
			\vspace{10mm}
			{\huge{\bf Ingegneria del Software}}\\
		\end{center}
	\end{Large}
	
	
	\vspace{36mm}
	%minipage divide la pagina in due sezioni settabili
	\begin{minipage}[t]{0.47\textwidth}
		{\large{\bf Professore:}\\ \large{Prof. Jacopo Soldani}}
	\end{minipage}
	\hfill
	\begin{minipage}[t]{0.47\textwidth}\raggedleft
		{\large{\bf Autore:}\\ \large{Filippo Ghirardini}}
	\end{minipage}
	
	\vspace{25mm}
	
	\hrulefill
	
	\vspace{5mm}
	
	\centering{\large{\bf Anno Accademico 2024/2025}}
	
\end{titlepage}
	
	\tableofcontents
	\newpage
	\maketitle
	\begin{center}
		\vspace{-20pt}
		\rule{11cm}{.1pt} 
	\end{center}
	% !TeX spellcheck = it_IT
\newpage

\section{Applicazioni della Computer Grafica}
\subsection{Medicina}
Nella medicina vediamo principalmente due utilizzi:
\begin{itemize}
	\item Diagnostica: sfruttare modelli 3D creati dalle immagini della MRI e CT, aumentando la leggibilità per gli umani
	\item Telemedicina e chirurgia virtuale: ad esempio la simulazione degli interventi endoscopici o preparazione di protesi dentali
\end{itemize}
\subsection{Industria}
Sfruttare il Computer Aided Design (CAD) per progettare velocemente nell'ambito industriale ed eseguire simulazioni. Ad esempio le macchine o il vestiario.

\subsection{Intrattenimento}
\subsubsection{Film}
Inizialmente effetti visivi realizzati grazie alla CGI. Si noti che bisogna distinguere tra effetti:
\begin{itemize}
	\item \emph{Visivi}: fatti in post produzione 
	\item \emph{Speciali}: fatti veri registrati dalla videocamera, ad esempio stunts o esplosioni
\end{itemize}
Poi con il tempo si è passati alla produzione di cortometraggi completamente in CGI ed infine anche lungometraggi.\\
La maggior parte sono comunque non realistici dal punto di vista degli umani, e i pochi tentativi di fare ciò sono stati fallimentari. Questo principio è rappresentato dal fenomeno dell'Uncanny Valley:
\begin{center}
	\includegraphics[scale=0.3]{uncanney_valley.png}
\end{center}
\subsubsection{Videogiochi}
Settore importantissimo per la computer grafica con un grande sviluppo.

\subsection{Beni culturali}
Un ambito relativamente nuovo per la computer grafica. Può servire per la \textbf{presentazione} (museale ad esempio), per \textbf{archiviare} pezzi artistici tridimensionali e per \textbf{studiare} (restaurazione, simulazione fisica, visualizzazione scientifica). \\
Il primo esempio di applicazione della CG è la scansione del David di Michelangelo. Una volta eseguita la scansione della statua si è creato il modello 3d dai dati grezzi e si è poi passati ad esempio al restauro.

\section{Paradigmi di renderizzazione}
Un \textbf{algoritmo di rendering} è una serie di passi che trasforma la descrizione digitale di una scena e di quattro parametri in un'\textbf{immagine raster}.\\
\begin{figure}[h]
	\includegraphics[scale=0.25]{rendering_algorithm.png}
	\centering
\end{figure}

\subsection{Ray Tracing}
L'idea alla base è quella di "sparare" un \textbf{raggio} dal punto di partenza e controllare se ha colpito qualche oggetto.
\begin{lstlisting}[mathescape=true]
	for each pixel p;
		make a ray r (viewpoint to p)
		for each primitive o in scene:
			find intersect(r,o)
		keep the closest intersection $o_j$
		find color of $o_j$ at p
\end{lstlisting}
Non tutti gli oggetti della scena però saranno illuminati, quindi quando un raggio interseca il punto della scena si fa partire un altro raggio che va verso la fonte di luce.\\
Se quest'ultimo incontra un oggetto vuol dire che l'oggetto è in ombra e non è raggiunto dalla luce.\\
C'è da tenere conto che la luce rimbalza un certo numero di volte, tramite il \textbf{reflection ray}. Ovviamente questo costa risorse, più si fa rimbalzare la luce e più l'immagine è realistica e costosa.\\
C'è poi la \textbf{rifrazione} di un oggetto che consiste sempre nel far partire altri raggi una volta che uno raggiunge un oggetto (ad esempio una bottiglia d'acqua).
\subsubsection{Costo}
Dipende da quanti rimbalzi ($N$) facciamo e da quante intersezioni con gli oggetti abbiamo:
\begin{equation}
	RTCost(r)=N \sum_{\forall o \in S} Int(r,o)
\end{equation}
Si noti che $Int(r,o)$ rappresenta il costo dell'intersezione del raggio $r$ con l'oggetto $o$.
\subsubsection{Primitive}
Tutto ciò che riesco facilmente ad intersecare con raggi:
\begin{itemize}
	\item Triangoli, quadrilateri, etc...
	\item Superfici implicite: sfera, geometria solida costruttiva, etc...
\end{itemize}

\subsection{Rasterizzazione}
Nella rasterizzazione (\textit{Transform \& Lighting}) proietto le primitive della scena sul mio schermo, ovvero prendo ogni \textbf{vertice} di ogni primitiva, lo proietto verso il viewpoint e vedo dove interseca la mia finestra. La linea che segna è il \textbf{proiettore}. A partire dalla proiezione dei soli vertici saprò quali pixel selezionare.\\
Il vantaggio principale è che è sufficiente proiettare pochi vertici per rasterizzare gran parte dello schermo.
\begin{lstlisting}
	for each primitive t:
		find where t falls on screen
		rasterize the 2D shape
		for each produced pixel p:
			find the color for t
			color p with it
\end{lstlisting}
\subsubsection{Primitive}
Tutto ciò che so proiettare da 3 dimensioni a 2 e rasterizzare:
\begin{itemize}
	\item Punto
	\item Segmento
	\item Triangolo
\end{itemize}
\subsubsection{Pipeline}
\begin{enumerate}
	\item Identifico le proiezioni delle primitive sullo schermo
	\item Trovo l'area da rasterizzare
	\item La computo
\end{enumerate}
\begin{figure}[h]
	\includegraphics[scale=0.25]{rasterization_pipeline.png}
	\centering
\end{figure}
\subsubsection{Punto}
\subsubsection{Linea}
Conoscendo i due vertici trovare tutti i pixel che la coinvolgono. Utilizzando l'equazione della rette si applica il seguente codice:
\begin{lstlisting}
	RasterizeH(x0,y0,x1,y1){
		m = (y1-y0)/(x1-x0)
		x=x0;
		y=y0;
		do{
			pixel(x,y)=0
		}
	}
\end{lstlisting}
\subsubsection{Triangoli}
Dati i tre vertici, se i lati passano per il centro del pixel allora questo viene illuminato. Questo ci garantisce che non ci sia ambiguità in caso di triangoli adiacenti, poiché non "litigheranno" per lo stesso pixel.
\subsubsection{Costi}
È composto dal costo della trasformazione dei vertici e il costo di rasterizzare la primitiva p in proporzione alla sua dimensione sullo schermo:
\begin{equation}
	%TODO
\end{equation}
\subsection{Confronto}
Ancora ad oggi il metodo della \textit{rasterizzazione} è il più popolare.\\
I vantaggi del \textit{ray-tracing} sono i seguenti:
\begin{itemize}
	\item Algoritmo più semplice concettualmente
	\item Ottimo per effetti grafici complessi di \textbf{alta qualità}
\end{itemize}
mentre quelli della \textit{rasterizzazione}:
\begin{itemize}
	\item Complessità più controllabile in quanto non serve tutta la scena in ogni momento della renderizzazione: ogni primitiva lavora per conto suo ed è quindi più facilmente parallelizzabile
	\item Funziona meglio con i dati dinamici (oggetti che si muovono sulla scena)
	\item Più \textbf{controllabile} e \textbf{veloce}
\end{itemize}
Uno dei motivi principali per cui ancora oggi la rasterizzazione è più popolare è per come affronta il problema dell'\textbf{Hidden Surface Removal}.
\subsubsection{Combinazione}
La soluzione vincente è quella di usare entrambi gli approcci: la \textbf{rasterizzazione} per disegnare gli oggetti della scena e il \textbf{ray-tracing} per elaborare luci ed ombre.
	% !TeX spellcheck = it_IT
\newpage
\section{Green Computing}
In generale, il green computing può aiutare le organizzazioni a ridurre l'impatto ambientale e a risparmiare sui costi energetici e di gestione.

\begin{definition}[Green Computing]
	Il green computing tratta la \textbf{progettazione}, la \textbf{realizzazione} e l'\textbf{utilizzo} di \underline{sistemi ICT}, \underline{computer} e \underline{dispositivi elettronici}\footnote{Tutti quei dispositivi che si appoggiano all'informatica per funzionare, e.g. aspirapolvere} in modo responsabile e sostenibile dal punto di vista ambientale, considerando in particolare il \textbf{consumo energetico} e \textbf{impronta di carbonio}.
\end{definition}

\begin{definition}[$CO_2$-eq]
	L'anidride carbonica equivalente è una misura che esprime l'impatto di una certa quantità di gas serra rispetto alla stessa quantità di anidride carbonica.
\end{definition}

\begin{definition}{Energy Star}
	Il progetto Energy Star nasce negli anni '90 ed è stata una delle prime iniziative relative al green computing per dare un indicatore dell'efficienza energetica. Il problema principale è che è \textbf{facoltativo}.
\end{definition}

\subsection{Approccio olistico}
Per funzionare segue un approccio \textbf{olistico}, analizzando tutto il \textbf{ciclo di vita} di un sistema, sia \emph{vericalmente} che \emph{orizzontalmente}.

\begin{itemize}
	\item \textbf{Progetto}: progettare in modo sostenibile computer, server, sistemi di raffreddamento e software a basso consumo e alta efficienza.
	\item \textbf{Produzione}: attenzione a non sprecare risorse limitate, ridurre gli scarti di fabbricazione e utilizzare fonti rinnovabili per la produzione.
	\item \textbf{Trasporto}: cercare di ridurre e ammortizzare l'uso di carburanti fossili sostituendoli con veicoli elettrici o ibridi e facendo spedizioni accorpate.
	\item \textbf{Uso}: utilizzare i sistemi cercando di ridurre il consumo con politiche di risparmio (e.g. ibernazione)
	\item \textbf{Dismissione}: lo smaltimento di dispositivi elettronici attraverso il riciclo
\end{itemize}

\subsection{Pilastri fondamentali}
\begin{center}
	\includegraphics[scale=0.5]{green_pilastri.png}
\end{center}
\subsubsection{Ingegneria del software sostenibile}
È possibile fare in modo che i programmi consumino meno energia e che il loro dispiegamento nelle varie fasi del ciclo di vita produca minori gas inquinanti. In particolare, programmare \emph{sfruttando le peculiarità di linguaggi e hardware} che possano rendere il software più disponibile.
\subsubsection{Hardware ad alta efficienza energetica}
\subsubsection{Cloud computing e virtualizzazione}
\subsubsection{Gestione adattiva dell'energia}
\subsubsection{Energia da fonti rinnovabili}
\subsubsection{Riciclo, smaltimento, riuso}
È possibile \textbf{sensibilizzare} l'utente sul giusto uso dei mezzi a sua disposizione e quindi della loro conseguente fine di utilizzo. Ottimizzare l'impiego dei dispositivi porta una determinante longevità, minimizzando quindi il rifiuto.
\begin{itemize}
	\item Inoltre, per ridurre la produzione di rifiuti è fondamentale \textbf{riutilizzare} (ad esempio rivendendo) i dispositivi elettronici ancora validi. In molti casi è sufficiente sostituire componenti degradati (e.g. le batterie) e mantenere il resto. Oltretutto molti dispositivi possono essere considerati obsoleti per certi scopi ma ancora ottimi per altri (e.g. server).
	\item È fondamentale ingegnerizzare il processo di \textbf{smaltimento} in modo da permettere il \textbf{riciclo} di parte dei componenti. Ad esempio dalle schede stampate si possono recuperare metalli preziosi come l'oro. La legislazione italiana necessita il corretto trattamento dei rifiuti per ridurre l'inquinamento. Di conseguenza anche la scelta di macchinari e strumenti mirati allo smaltimento è fondamentale per fare in modo che un'azienda possa essere ritenuta green.
	\item La tecnologia stessa può essere uno strumento potente per \textbf{sensibilizzare} il consumatore su queste tematiche e per fargli conoscere le aziende green.
\end{itemize}
\subsection{Applicazioni green}
Sfruttare i sistemi ICT per l'\textbf{ottimizzazione} di processi che sfruttano risorse limitate (e.g. combustibili fossili nel trasporto, energia elettrica nel riscaldamento, acqua potabile nell'irrigazione) è un aspetto importante del green computing.
	% !TeX spellcheck = it_IT
\subsection{Ebook reader}
Entro il 2025 si prevede che gli e-reader rappresenteranno circa il $75\%$ del mercato totale, anche se allo stesso tempo il numero di libri cartacei prodotti e  venduti è in continuo aumento.

\subsubsection{Ciclo di produzione}
Vediamo il ciclo di vita di un libro tradizionale cartaceo
\begin{center}
	\includegraphics[scale=0.3]{books_lifecycle.png}
\end{center}
Le materie prime necessarie sono, per un libro a copertina morbida, $150-300$g di carta e $7.5$lt di acqua. Sono necessari $2$KWh  e la loro distribuzione (assumendo che non si usi la macchina per comprarlo) produce circa 10 volte quella della produzione. L'utilizzo è trascurabile dal punto di vista energetico in quanto al massimo serve una luce per leggere.
\begin{center}
	\includegraphics[scale=0.3]{ereader_lifecycle.png}
\end{center}
Per quanto riguarda invece gli e-book reader, sono necessari circa $15$Kg di materie prime (metalli rari, sabbia, etc...) e $300$lt di acqua (batterie, chip, oro dei circuiti). Sono necessari $100$KWh per la produzione e assumiamo i costi di distribuzione di un  \href{https://www.icao.int/environmental-protection/Carbonoffset/Pages/default.aspx}{\color{blue}volo Milano-Roma}.

\newpage
\subsubsection{Confronto}
Considerando i dati precedenti:
\begin{enumerate}
	\item Quanti libri si producono con le materie prime necessarie per produrre un e-book reader?\\
	\begin{equation*}
		\frac{15Kg}{0.150Kg} = 100 \quad \frac{15Kg}{0.300Kg}=50
	\end{equation*}
	\item Quanti libri si producono con l'acqua necessaria per produrre un e-book reader?\\
	\begin{equation*}
		\frac{300lt}{7.5lt} = 40
	\end{equation*}
	\item Quanti libri si producono con l'energia necessaria per produrre un e-book reader?\\
	\begin{equation*}
		\frac{100KWh}{2KWh} = 50
	\end{equation*}
	\item Quanti libri serve produrre e trasportare per inquinare quanto per la produzione e il trasporto di un e-book reader?\\
	\begin{equation*}
		\begin{split}
			&\text{Produzione e-book reader}=0.319\frac{g}{Kw/h} \cdot 100Kw/h = 31.9Kg \quad \text{Distribuzione e-book reader}=41.8 Kg \\
			&\text{Totale e-book reader}=31.9Kg + 41.8Kg = 73.7 Kg \\
			&\text{Produzione libro}=0.319 \frac{g}{Kw/h} \cdot 2Kw/h = 0.638Kg \quad  \text{Distribuzione libro}= 0.638Kg \cdot 10 = 6.380Kg\\
			&\text{Totale libro}=6,380Kg + 0.638 Kg = 7.018Kg \\
			& \text{\textbf{Libri per e-book reader}}=\frac{73.7Kg}{7.018Kg}=10.5
		\end{split}
	\end{equation*}
	\item Qual'è la media dei valori delle risposte precedenti (quanti libri vale un e-book reader)?
	\begin{equation*}
		\frac{\frac{100+50}{2} + 40 + 50 + 10.5}{5} = 43.9
	\end{equation*}
	\item Quanti libri bisogna leggere all'anno per ammortizzare un e-book reader su 5 anni di vita media?
	\begin{equation*}
		\frac{43.9}{5} = 8.8
	\end{equation*}
\end{enumerate}

\subsubsection{Salute}
La produzione di libri ed e-book reader produce ossidi di azoto e zolfo che entrano in profondità nei polmoni, peggiorando l'asma, causando la tosse cronica e aumentando il rischio di morte prematura. Un e-book reader produce $70$ volte questi prodotti rispetto che ad un libro cartaceo.

\subsubsection{Dismissione}
\begin{table}[h]
	\begin{tabular}{|c|c|}
		\hline
		Libro & E-book reader \\
		\hline
		\multirowcell{2}{La \textbf{decomposizione} può generare il doppio \\delle emissioni e degli impatti tossici sulle \\ falde acquifere rispetto alla sua intera produzione} & \multirowcell{2}{In caso di \textbf{smaltimento illegale} in uno \\ dei paesi in via di sviluppo, i lavoratori\\ (spesso bambini) saranno esposti all'impatto\\ tossico di alcune sostanze smantellate. }\\
		\hline
		\multirowcell{2}{Può essere prestato, regalato, donato \\ad una biblioteca oppure correttamente riciclato. }& \multirowcell{2}{Se correttamente riciclato, molti materiali \\si potranno recuperare o smaltire correttamente.}\\
		\hline
	\end{tabular}
\end{table}

\subsection{Blockchain}
Bitcoin nasce nel 2008 come prima tecnologia basata sulla blockchain.
\begin{definition}[Blockchain]
	Blockchain è un libro mastro distribuito in grado di registrare e validare transazioni in assenza di un’entità centrale (e.g. banca).
\end{definition}
In particolare una blockchain ha le seguenti caratteristiche:
\begin{itemize}
	\item \textbf{Distribuita}: tutti i nodi partecipanti ne conservano una copia per trasparenza
	\item \textbf{Immutabile}: i record nella catena non possono essere né modificati né cancellati
	\item \textbf{Marcata temporalmente}: ogni transazione ha un timestamp
	\item \textbf{Unanime}: tutti i nodi partecipanti devono riconoscere la validità delle transazioni
	\item \textbf{Anonima}: l’identità dei partecipanti non è rivelata
	\item \textbf{Sicura}: tutti i record vengono criptati individualmente
	\item \textbf{Programmabile} per mezzo di SmartContracts
\end{itemize}

\subsubsection{Proof of Work}
La blockchain si basa sul concetto per cui ogni blocco, composto dalla transazione e dal riferimento a quella precedente, possa essere aggiunto solo quando viene fornita una \textbf{proof of work} da parte dei minatori, che risolvono problemi difficili (e.g. scomposizione in fattori primi).\\
Quando viene richiesta una transazione si crea un blocco che viene distribuito a tutti i partecipanti.\\
La difficoltà della proof of work aumenta con l'aumentare delle capacità computazionali dei nodi che scrivono nella blockchain, in modo tale da equilibrare:
\begin{itemize}
	\item \textbf{Sicurezza}: ad esempio evitando attacchi di doppia-spesa, o aggiunta di blocchi falsi 
	\item \textbf{Velocità di esecuzione} delle transazioni (stabilita attorno ai 10 minuti)
\end{itemize}

\subsubsection{Hardaware}
La potenza hardware per Bitcoin si misura in \textbf{GigaHash} al secondo (un hash è un calcolo da risolvere). L'hardware necessario si è evoluto con il tempo:
\begin{itemize}
	\item 2008 - \textbf{CPU}, $0.01GH/s$ con un consumo di $2.5Wh/GH$
	\item 2009 - \textbf{GPU}, $0.2-2GH/s$
\end{itemize}

\subsubsection{Minatori}
Possiamo suddividere le categorie dei minatori in:
\begin{itemize}
	\item \textbf{Piccoli}, il $15\%$ del totale, con un consumo fino a $0.1MW$ per $0.9PH/s$
	\item \textbf{Medi}, il $19\%$ del totale, con un consumo tra $0.1MW$ e $1MW$ per $9PH/s$
	\item \textbf{Grandi}, il $66\%$ del totale, con un consumo maggiore di $1MW$ per oltre $9PH/s$
\end{itemize}
I minatori si dividono in \textbf{pool} dove condividono il potere di calcolo:
\begin{center}
	\includegraphics[scale=0.3]{bitcoin_pools.png}
\end{center}

\subsubsection{Analisi}
Consideriamo che al 2019 il consumo dell'hardware più efficiente era di $1.4 \cdot 10^{-5} Wh/GH$ e che per raffreddarlo veniva utilizzato il $5\%$ del consumo. Il numero di hash eseguiti in un'ora a novembre del 2019 era di $3.56 \cdot 10^{11} TH$. La localizzazione geografica dei minatori era:
\begin{itemize}
	\item \textbf{Cina} con il $68\%$ ad un costo di $0.55 \frac{kgCO_2-eq}{kWH}$
	\item \textbf{EU} con l'$11\%$ ad un costo di $0.28 \frac{kgCO_2-eq}{kWH}$
	\item \textbf{Privati} con il $21\%$ ad un costo di $0.475 \frac{kgCO_2-eq}{kWH}$
\end{itemize}
Considerando queste informazioni
\begin{enumerate}
	\item Qual è un limite inferiore al consumo energetico annuo di Bitcoin?
	\begin{equation*}
		\begin{split}
			&\text{Consumo per hash}=1.4 \cdot 10^{-5} \frac{Wh}{GH} + 5\% = 1.47 \cdot 10^{-5} \frac{Wh}{GH} \\
			&\text{Consumo per ora}=1.47 \cdot 10^{-5} \frac{Wh}{GH}  \cdot 3.56 \cdot 10^{14} GH = 5.2332 \cdot 10^9 W = 5.2332 \cdot 10^6 kW \\
			&\textbf{Consumo annuo}=5.2332 \cdot 10^6 kWh * 8760h = 4.5842832 \cdot 10^{10} kWh
		\end{split}
	\end{equation*}
	\item Quante emissioni di carbonio vengono prodotte all'anno se si utilizza quel limite inferiore come stima?
	\begin{equation*}
		\begin{split}
			&\text{Consumo \textbf{Cina}}=4.5842832 \cdot 10^{10} kWh \cdot 0.68 \cdot 0.55 \frac{kgCO_2-eq}{kWH} = 1.7145219168 \cdot 10^{10} kgCO_2-eq\\
			&\text{Consumo \textbf{EU}}=4.5842832 \cdot 10^{10} kWh \cdot 0.11 \cdot 0.28 \frac{kgCO_2-eq}{kWH} = 1.4119592256 \cdot 10^{9} kgCO_2-eq\\
			&\text{Consumo \textbf{privati}}=4.5842832 \cdot 10^{10} kWh \cdot 0.21 \cdot 0.475 \frac{kgCO_2-eq}{kWH} = 4.572822492 \cdot 10^{9} kgCO_2-eq
		\end{split}
	\end{equation*}
\end{enumerate}

\subsubsection{Oggi}
Nella primavera del 2021 alcuni stati come la Cina proibiscono il mining di Bitcoin e questo ha aumentato l'intensità del mining del $43\%$ rispetto al 2019. Si noti che al momento la Cabon Footprint del Bitcoin è di $77.42$ Mega Tonnellate di $CO_2$ ogni anno. In pratica una transazione con Bitcoin equivale a $1,000,000$ transazioni VISA.
\begin{center}
	\includegraphics[scale=0.5]{bitcoin-energy-consumpti.png}
\end{center}
\begin{center}
	\includegraphics[scale=0.5]{bitcoin-energy-consumption_country.png}
\end{center}

Ad oggi esiste il \textbf{Crypto Climate Accord} che ha come obiettivo quello di contribuire a raggiungere gli Accordi di Parigi tramite l'utilizzo di energie rinnovabili entro il 2030.

\subsubsection{Conclusione}
La blockchain è una tecnologia all'avanguardia che potrebbe avere un impatto molto grande su molti settori. È importante eseguire un'analisi di costi e benefici per valutare se conviene o meno:
\begin{enumerate}
	\item \textbf{Emissioni} di carbonio
	\item Rischi di \textbf{centralizzazione}: se qualcuno ottenesse il $51\%$ della computing power avrebbe il controllo della blockchain
	\item Possibilità di \textbf{controllo} per evitare traffici illegali
\end{enumerate}

\subsubsection{Proof of Stake}
Per affrontare le problematiche indicate ai punti 1 e 2 si vorrebbe introdurre la \textbf{proof of stake}, dove l'abilità di minare è determinata in base alla quantità di moneta che un utente possiede. Il minatore non viene premiato con la moneta al completamento del calcolo ma con degli interessi. In questo modo si evita anche l'attacco del $51\%$ poiché si rende necessario avere il $51\%$ della moneta (più difficile).
\end{document}
