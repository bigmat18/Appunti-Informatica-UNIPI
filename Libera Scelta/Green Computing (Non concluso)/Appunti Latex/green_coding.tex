% !TeX spellcheck = it_IT
\newpage
\section{Green coding}
\subsection{Strutture dati}
Le strutture dati forniscono un modo per organizzare e processare le informazioni. Una struttura \textbf{CRUD} consente di fare quattro operazioni:
\begin{itemize}
	\item \textit{Create}: aggiunta di un elemento
	\item \textit{Read}: lettura di un elemento
	\item \textit{Update}: aggiornamento di un valore
	\item \textit{Delete}: eliminazione di un dato
\end{itemize}
In particolare Java fornisce strutture dati come \textit{List}, \textit{Set} e \textit{Map}. Noi utilizzeremo le prime due. Queste hanno diverse implementazioni con diverse complessità:
\begin{table}[!h]
	\centering
	\begin{tabular}{|c|c|c|c|c|}
		\hline
		\textbf{Struttura} & \textit{Create} & \textit{Read} & \textit{Update} & \textit{Delete} \\
		\hline
		ArrayList & $O(1)$ & $O(1)$ & $O(N)$ & $O(N)$ \\
		LinkedList & $O(N)$ & $O(N)$ & $O(N)$ & $O(N)$\\
		HashSet & $O(1)$ & $O(1)$ & $O(1)$ & $O(1)$\\
		TreeSet & $O(\log N)$ & $O(\log N)$ & $O(\log N)$ & $O(\log N)$\\
		\hline
	\end{tabular}
\end{table}
\begin{note}
	La differenza tra \textit{lista} e \textit{dizionario} è che la prima può contenere duplicati mentre il secondo no.
\end{note}

\subsection{Analisi}
Le strutture di tipo \textit{List} sono meno efficienti di quelle \textit{Set}, però possono contenere duplicati. Se si prevede un mix bilanciato di operazioni \textit{ArrayList} è migliore di \textit{LinkedList} tranne per la modifica e l'eliminazione. Alla fine la struttura più efficiente è teoricamente \textit{HashSet}, ma ad esempio se è necessario tenere gli elementi ordinati conviene \textit{TreeSet}.\\
È quindi fondamentale scegliere una struttura dati coerente con il caso di utilizzo.