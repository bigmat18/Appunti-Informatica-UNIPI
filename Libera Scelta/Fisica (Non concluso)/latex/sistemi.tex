\newpage
\section{Sistemi di punti matariali}
Consideriamo un insieme ("sistema") di punti materiali. Deduciamo delle equazioni del moto e leggi di conservazione 
come conseguenza delle leggi di Newton.
\begin{itemize}
    \item Sistemi di punti molto molto numerosi possono dare origine a fenomeni emergenti (es. sistemi biologici) \textbf{"More is different"}
    \item Concetti utili sono introdotti da teorie non riduzionistiche.
\end{itemize}
Punti $\{\vec{r}_{\alpha}\}^N_{\alpha=1}$ con masse $\{m_{\alpha}\}^N_{\alpha = 1}$ etc. Risultate delle forze su ogni punto $\{\vec{F}_{\alpha}\}^N_{\alpha = 1}$.
Quantità definite per il sistema:
\begin{itemize}
    \item \textbf{Massa} $$M = \sum_{\alpha = 1}^{N}m_{\alpha}$$
    \item \textbf{Vettore quantià di moto} $$\vec{P}(t) = \sum_{\alpha = 1}^{N} \vec{P}_{\alpha}(t)$$
    \item \textbf{Vettore momento angolare rispetto ad un polo P} $$\vec{L}_p(t) = \sum_{\alpha = 1}^{N}\vec{L}_{p, \alpha}(t) \:\:\: \text{stesso P per ogni }\alpha$$
    \item \textbf{Energia cinetica} %finisci
    \item \textbf{Energia meccanica} %finisci
\end{itemize}
Nel caso di sistemi estrmamente numerosi (es. atomi di gas, elettroni in un metallo, granelli di sabbia in una duna), può convenire
considerare il sistema \textbf{continuo}.
\begin{itemize}
    \item $n(\vec{r})$: numero di punti in un volumetto $V$ attorno a $\vec{r}$, diviso per $V$ $[n] = m^{-3}$
    \item $g(\vec{r})$: massa in un volumetto V attorno a $\vec{r}$ diviso per V (\textbf{"densità"}) $[g] = kg \cdot m^{-3}$
\end{itemize}
\begin{observation}
    Il risultato non deve dipendere dal valore preciso di V, ma l'ordine di grandezza dipende dal problema (es. densità della terra...?)
\end{observation}
\begin{definition}
    Definisco \textbf{centro di massa} del sistema di punti di materiali il punto gemotrico 
    $$\vec{r}_{CM} = \frac{1}{M}\sum_{\alpha = 1}^{N} m_{\alpha}\vec{r}_{\alpha}(t)$$
\end{definition}
\begin{example}
    $\vec{r}_1(t) = v_0 t \hat{x} + y_0 \hat{y} \:\:\: m_1 = m \hspace{10pt} \vec{r}_2(t) = -2v_0 t \hat{x} + 2y_0 \hat{y} \:\:\: m_2 = 3m$
    $$\vec{r}_{CM} = \frac{1}{m + 3m}[(mv_0t - 6mv_0t)\hat{x} + (my_0 + 6my_0)\hat{y}] = \frac{1}{4m}[-5mv_0t\hat{x} + 7my_0 \hat{y}] = -\frac{5}{4} v_0t\hat{x} + \frac{7}{4}y_0\hat{y}$$
\end{example}
\begin{observation}
    Il CM non è un punto fisico, può trovare fuori dalla regione spaziale occupata dal sistema.
\end{observation}
Posso indicare ogni punto materiale con la posizione relativa al CM:
$$\vec{r}_{\alpha}'(t) = \vec{r}_{\alpha} - \vec{r}_{CM}(t)$$
Osservo che 
$$\sum_{N}^{\alpha = 1}m_{\alpha}\vec{r}'_{\alpha}(t) = \sum_{\alpha = 1}^{N}[n_{\alpha}\vec{r}_{\alpha}(t) - m_{\alpha}\vec{r}_{CM}(t)] = M\vec{r}_{CM}(t) - (\sum_{\alpha = 1}^{N}m_{\alpha})\vec{r}_{CM}(t) = 0$$
Esprimo il vettore quantità di moto:
$$\vec{p}(t) = \sum_{\alpha = 1}^{N}\dot{\vec{r}}_{\alpha}(t) = \sum_{\alpha = 1}^{N}m_{\alpha}[\dot{\vec{r}}_{CM} + \dot{\vec{r}}'_{\alpha}(t)]
= \Big(\sum_{\alpha = 1}^{N}m_{\alpha}\Big)\dot{\vec{r}}_{CM}(t) = \frac{d}{dt} \Big(\sum_{\alpha=1}^{N}m_{\alpha}\vec{r}_{\alpha}'(t)\Big) = M\dot{\vec{r}}_{CM}(t)
$$
ma non è un punto materiale.