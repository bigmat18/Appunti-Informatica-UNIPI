\newpage
\section{Sistemi di punti matariali}
Consideriamo un insieme ("sistema") di punti materiali. Deduciamo delle equazioni del moto e leggi di conservazione 
come conseguenza delle leggi di Newton.
\begin{itemize}
    \item Sistemi di punti molto molto numerosi possono dare origine a fenomeni emergenti (es. sistemi biologici) \textbf{"More is different"}
    \item Concetti utili sono introdotti da teorie non riduzionistiche.
\end{itemize}
Punti $\{\vec{r}_{\alpha}\}^N_{\alpha=1}$ con masse $\{m_{\alpha}\}^N_{\alpha = 1}$ etc. Risultate delle forze su ogni punto $\{\vec{F}_{\alpha}\}^N_{\alpha = 1}$.
Quantità definite per il sistema:
\begin{itemize}
    \item \textbf{Massa} $$M = \sum_{\alpha = 1}^{N}m_{\alpha}$$
    \item \textbf{Vettore quantià di moto} $$\vec{P}(t) = \sum_{\alpha = 1}^{N} \vec{P}_{\alpha}(t)$$
    \item \textbf{Vettore momento angolare rispetto ad un polo P} $$\vec{L}_p(t) = \sum_{\alpha = 1}^{N}\vec{L}_{p, \alpha}(t) \:\:\: \text{stesso P per ogni }\alpha$$
    \item \textbf{Energia cinetica} %finisci
    \item \textbf{Energia meccanica} %finisci
\end{itemize}
Nel caso di sistemi estrmamente numerosi (es. atomi di gas, elettroni in un metallo, granelli di sabbia in una duna), può convenire
considerare il sistema \textbf{continuo}.
\begin{itemize}
    \item $n(\vec{r})$: numero di punti in un volumetto $V$ attorno a $\vec{r}$, diviso per $V$ $[n] = m^{-3}$
    \item $g(\vec{r})$: massa in un volumetto V attorno a $\vec{r}$ diviso per V (\textbf{"densità"}) $[g] = kg \cdot m^{-3}$
\end{itemize}
\begin{observation}
    Il risultato non deve dipendere dal valore preciso di V, ma l'ordine di grandezza dipende dal problema (es. densità della terra...?)
\end{observation}
\begin{definition}
    Definisco \textbf{centro di massa} del sistema di punti di materiali il punto geometrico 
    $$\vec{r}_{CM} = \frac{1}{M}\sum_{\alpha = 1}^{N} m_{\alpha}\vec{r}_{\alpha}(t)$$
\end{definition}
\begin{example}
    $\vec{r}_1(t) = v_0 t \hat{x} + y_0 \hat{y} \:\:\: m_1 = m \hspace{10pt} \vec{r}_2(t) = -2v_0 t \hat{x} + 2y_0 \hat{y} \:\:\: m_2 = 3m$
    $$\vec{r}_{CM} = \frac{1}{m + 3m}[(mv_0t - 6mv_0t)\hat{x} + (my_0 + 6my_0)\hat{y}] = \frac{1}{4m}[-5mv_0t\hat{x} + 7my_0 \hat{y}] = -\frac{5}{4} v_0t\hat{x} + \frac{7}{4}y_0\hat{y}$$
\end{example}
\begin{observation}
    Il CM non è un punto fisico, può trovare fuori dalla regione spaziale occupata dal sistema.
\end{observation}
Posso indicare ogni punto materiale con la posizione relativa al CM:
$$\vec{r}_{\alpha}'(t) = \vec{r}_{\alpha} - \vec{r}_{CM}(t)$$
Osservo che 
$$\sum_{N}^{\alpha = 1}m_{\alpha}\vec{r}'_{\alpha}(t) = \sum_{\alpha = 1}^{N}[n_{\alpha}\vec{r}_{\alpha}(t) - m_{\alpha}\vec{r}_{CM}(t)] = M\vec{r}_{CM}(t) - (\sum_{\alpha = 1}^{N}m_{\alpha})\vec{r}_{CM}(t) = 0$$
Esprimo il vettore quantità di moto:
$$\vec{p}(t) = \sum_{\alpha = 1}^{N}\dot{\vec{r}}_{\alpha}(t) = \sum_{\alpha = 1}^{N}m_{\alpha}[\dot{\vec{r}}_{CM} + \dot{\vec{r}}'_{\alpha}(t)]
= \Big(\sum_{\alpha = 1}^{N}m_{\alpha}\Big)\dot{\vec{r}}_{CM}(t) = \frac{d}{dt} \Big(\sum_{\alpha=1}^{N}m_{\alpha}\vec{r}_{\alpha}'(t)\Big) = M\dot{\vec{r}}_{CM}(t)
$$ 
ma non è un punto materiale. \\
Esprimo il vettore momento angolare rispetto a P:
\begin{equation*}
    \begin{split}
        \vec{L}_p(t) & = \sum_{\alpha=1}^{N}m_{\alpha}(\vec{r}_{\alpha} - \vec{r}_p(t)) \times \dot{\vec{r}}_{\alpha}(t) = \sum_{\alpha=1}^{N} m_{\alpha}(\vec{r}_{CM}(t) + \vec{r}'_{\alpha}(t) - \vec{r}_p(t)) \times (\dot{\vec{r}}_{CM} + \dot{\vec{r}}_{\alpha}'(t))\\
                     & = \sum_{\alpha=1}^{N}m_{\alpha}\big\{(\vec{r}_{CM}(t) - \vec{r}_p(t))\times \dot{\vec{r}}_{CM}(t) + \vec{r}_{\alpha}'\times \dot{\vec{r}}_{\alpha}'(t) + (\vec{r}_{CM}(t) - \vec{r}_p(t)) \times \dot{\vec{r}}_{\alpha}'(t) + \vec{r}_{\alpha}'(t) \times \dot{\vec{r}}_{CM}(t) \big\}\\
                     & = \bigg(\sum_{\alpha=1}^{N}m_{\alpha}(\vec{r}_{CM}(t) - \vec{r}_p(t)) \times \dot{\vec{r}}_{CM}(t) + \sum_{\alpha=1}^{N}m_{\alpha}\vec{r}_{\alpha}' \times \dot{\vec{r}}_{\alpha}'(t)\bigg) + (\vec{r}_{CM}(t) - \vec{r}_p(t)) \times \frac{d}{dt}(\sum_{\alpha=1}^{N} m_{\alpha} \vec{r}_{\alpha}'(t))\\
                     & + (\sum_{\alpha=1}^{N} m_{\alpha}\vec{r}_{\alpha}'(t)) \times \dot{\vec{r}}_{CM}(t) = M(\vec{r}_{CM}(t) - \vec{r}_p(t)) \times \dot{\vec{r}}_{CM}(t) + \sum_{\alpha=1}^{N}m_{\alpha}\vec{r}_{\alpha}'(t) \times \dot{\vec{r}}_{\alpha}'(t)
    \end{split}
\end{equation*}
$M(\vec{r}_{CM}(t) - \vec{r}_p(t)) \times \dot{\vec{r}}_{CM}(t)$ del CM. $\sum_{\alpha=1}^{N}m_{\alpha}\vec{r}_{\alpha}'(t) \times \dot{\vec{r}}_{\alpha}'(t)$ rispetto al CM. Ma non è un punto materiale.\\
Esprimo l'energia cinetica:
\begin{equation*}
    \begin{split}
        K & = \sum_{\alpha=1}^{N}\frac{1}{2}m_{\alpha}||\dot{\vec{r}}_{\alpha}(t)||^2 = \sum_{\alpha=1}^{N}\frac{1}{2}m_{\alpha}(\dot{\vec{r}}_{CM}(t) + \dot{\vec{r}}_{\alpha}'(t)) \cdot (\dot{\vec{r}}_{CM}(t) + \dot{\vec{r}}_{\alpha}'(t))\\
          & = \sum_{\alpha=1}^{N}\frac{1}{2}m_{\alpha}[||\dot{\vec{r}}_{CM}(t)||^2 + ||\dot{\vec{r}}_{\alpha}'(t)|| + 2\dot{\vec{r}}_{\alpha}'(t) \cdot \dot{\vec{r}}_{CM}(t)] = \frac{1}{2}(\sum_{\alpha=1}^{N}m_{\alpha}) ||\dot{\vec{r}}_{CM}||^2 \\
          & + \sum_{\alpha=1}^{N}\frac{1}{2}m_{\alpha}||\dot{\vec{r}}_{\alpha}(t)||^2 + 2(\sum_{\alpha=1}^{N}m_{\alpha}\dot{\vec{r}}_{\alpha}'(t)) \cdot \dot{\vec{r}}_{CM}(t) = \frac{1}{2}M ||\dot{\vec{r}}_{CM}(t)||^2 + \sum_{\alpha=1}^{N}\frac{1}{2}m_{\alpha}||\dot{\vec{r}}_{\alpha}'(t)||^2
    \end{split}
\end{equation*}
$\frac{1}{2}M ||\dot{\vec{r}}_{CM}(t)||^2$ dal CM e $\sum_{\alpha=1}^{N}\frac{1}{2}m_{\alpha}||\dot{\vec{r}}_{\alpha}'(t)||^2$ è rispetto al CM. Ma non è un punto materiale.

\subsection{Forze interne}
Chiamiamo \textbf{forze interne} quelle esercitate su un punto materiale su resto del sistema e \textbf{forze esterne} le altre. Ogni
risultante diventa:
$$\vec{F}_{\alpha}(t) = \sum_{\beta=1}^{N} \vec{F}_{\alpha\beta}(t) + \vec{F}_{\alpha}^{E}(t)$$
\begin{observation}
    Per la terza legge di Newton $\vec{F}_{\alpha\beta}(t) = -\vec{F}_{\beta\alpha}(t)$
\end{observation}
\hspace{-15pt}Definisco \textbf{vettore momento di una forza applicata in $\vec{r}(t)$ rispetto al polo p} la quantità:
$$\vec{M}_p(t) = (\vec{r}(t) - \vec{r}_p) \times \vec{F}(t) \hspace{15pt}[\vec{M}_p] = N \cdot m \text{ (non si usa J)}$$
\begin{observation}
    Il momento totale delle forze interne è nullo.
    \begin{equation*}
        \begin{split}
            \sum_{\alpha=1}^{N} & = (\vec{r}_{\alpha}(t) - \vec{r}_p(t)) \times \sum_{\beta=1}^{N}\vec{F}_{\alpha\beta}(t) = \sum_{\alpha=1}^{N} \sum_{\beta=1}^{N} (\vec{r}_{\alpha}(t) - \vec{r}_p(t)) \times \vec{F}_{\alpha\beta}(t)\\
                                & = \frac{1}{2}\sum_{\alpha=1}^{N}\sum_{\beta=1}^{N} \text{ (dummy indices) }[(\vec{r}_{\alpha}(t) - \vec{r}_p(t)) \times \vec{F}_{\alpha\beta}(t) + (\vec{r}_{\beta}(t) - \vec{r}_p(t)) \times \vec{F}_{\beta\alpha}(t)]\\
                                & = \frac{1}{2}\sum_{\alpha=1}^{N}\sum_{\beta=1}^{N}[(\vec{r}_{\alpha}(t) - \vec{r}_p(t)) \times \vec{F}_{\alpha\beta}(t) - (\vec{r}_{\beta}(t) - \vec{r}_p(t)) \times \vec{F}_{\alpha\beta}(t) \text{ (per la terza legge) }]\\
                                & = \frac{1}{2}\sum_{\alpha=1}^{N}\sum_{\beta=1}^{N}[(\vec{r}_{\alpha}(t) - \vec{r}_{\beta}(t)) \times \vec{F}_{\alpha\beta}(t)] = 0
        \end{split}
    \end{equation*}
    Per la simmetria $\vec{F}_{\alpha\beta}(t)$ deve essere parallela al vettore che congiunge $\vec{r}_{\alpha}(t)$ con $\vec{r}_{\beta}(t)$ quindi il prodotto vettoriale è nullo.
\end{observation}
\hspace{-15pt}Le equazioni del moto per la \textbf{quantità di moto}:
$$\dot{\vec{p}}(t) = \sum_{\alpha=1}^{N} m_{\alpha}\ddot{\vec{r}}_{\alpha}(t) = \sum_{\alpha=1}^{N}[\sum_{\beta=1}^{N}\vec{F}_{\alpha\beta}(t) + \vec{F}_{\alpha}^E(t)]$$
$$M\ddot{\vec{r}}_{CM}(t) = \frac{1}{2}\sum_{\alpha=1}^{N}\sum_{\beta=1}^{N}[\vec{F}_{\alpha\beta}(t) + \vec{F}_{\beta\alpha}(t)] + \sum_{\alpha=1}^{N}\vec{F}_{\alpha}^E(t)$$
Abbiamo che $\sum_{\alpha=1}^{N}\sum_{\beta=1}^{N}$ è il dummy indices, mentre $[\vec{F}_{\alpha\beta}(t) + \vec{F}_{\beta\alpha}(t)] = 0$ come visto sopra, quindi la \textbf{prima equazione cardinale} diventa:
$$M\ddot{\vec{r}}_{CM}(t) = \sum_{\alpha=1}^{N}\vec{F}_{\alpha}^E(t)$$

\begin{example}
    $x_{CM}(t) = (m_1 x_1(t) + m_2x_2(t))/(m_1 + m_2) \hspace{15pt} (m_1 + m_2)\ddot{x}_{CM}(t) = F - F_{a1} + F_{a2}$ 
    Sommo tutte le forze applicate al sistema ($F - F_{a1} + F_{a2}$).
\end{example}

\begin{example}
    $\dot{x}_2(t_0) = v_0 \hspace{10pt} \dot{x}_1(t_0) = 0 \hspace{10pt} \dot{c}_{CM} = m_2\dot{x}_2(t_0)/(m_1 + m_2)$\\
    $(m_1 + m_2)\ddot{x}_{CM}(t) = 0 \Rightarrow \dot{x}_{CM}(t) = \dot{x}_{CM}(t_0)$\\
    Le due masse oscillano ma il CM compie un moto rettilineo uniforme.
\end{example}

\subsection{Equazione del moto per il momento angolare}
\begin{equation*}
    \begin{split}
        \dot{\vec{L}}_p(t) & = \sum_{\alpha=1}^{N}m_{\alpha}[(\dot{\vec{r}}_{\alpha}(t) - \dot{\vec{r}}_p(t)) \times \dot{\vec{r}}_{\alpha}(t) + (\vec{r}_{\alpha}(t) - \vec{r}_p(t)) \times \ddot{\vec{r}}_{\alpha}(t)]\\
                           & = -\dot{\vec{r}}_p(t) \times (\sum_{\alpha=1}^{N}m_{\alpha}\dot{\vec{r}}_{\alpha}(t)) + \sum_{\alpha=1}^{N}(\vec{r}_{\alpha}(t) - \vec{r}_p(t)) \times (m_{\alpha}\ddot{\vec{r}_{\alpha}}(t))\\
                           & = -\dot{\vec{r}}_p(t) \times \vec{p}(t) + \sum_{\alpha=1}^{N} (\vec{r}_{\alpha} - \vec{r}_p(t)) \times \vec{F}_{\alpha}(t)
    \end{split}
\end{equation*}
Siccome il momento totale delle forze interne è nullo abbiamo che:
$$\dot{\vec{L}}_p(t) = \sum_{\alpha=1}^{N}(\vec{r}_{\alpha}(t) - \vec{r}_p(t)) \times \vec{F}_{\alpha}^E(t) - \dot{\vec{r}}_p(t) \times \vec{P}(t)$$
$$\dot{\vec{L}}_p(t) = \vec{M}_p^E(t) - \dot{\vec{r}}_p(t) \times \vec{P}(t) \text{ Seconda equazione cardinale }$$
\begin{example}
    Esempio del \textbf{campo centrale}. $\vec{F}(\vec{r}) = -\frac{k}{||\vec{r}||}\cdot \hat{r} \hspace{15pt} polo: 0$
    $$\vec{L}_0(t) = m(\vec{r}(t) - \vec{r}_0) \times \dot{\vec{r}}(t) = mr(t) \hat{r} \times (\hat{r}(t) \hat{r} + r(t)\dot{\Theta}(t)\hat{\Theta}) = mr(t)^2\dot{\Theta}(t)\hat{z}$$
    $\dot{\vec{L}}_0(t) = (\vec{r}(t) - \vec{r}_0) \times \vec{F} = r(t)\hat{r} \times (-\frac{k}{||\vec{r}||}\hat{r}) = 0$\\
    Il momento angolare è costante (\textbf{"conservativo"}) $\Rightarrow r(t)^2 \dot{\Theta}(t) = L_0(t_0) \hspace{15pt} \dot{\Theta}(t) = L_0(t_0)/r(t)^2$. Questo vuol dire che la \textbf{la velocità
    angolare aumenta quando $r(t)$ decresce}
\end{example}